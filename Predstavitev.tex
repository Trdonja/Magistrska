\documentclass{beamer}
\usepackage[slovene]{babel}
\usepackage[utf8]{inputenc}
\usetheme{Goettingen} %Warsaw

\usepackage{amsmath,amssymb,amsfonts,amsthm,mathrsfs,mathtools,stmaryrd,graphicx}
\SetSymbolFont{stmry}{bold}{U}{stmry}{m}{n}
\usepackage[mathcal]{euscript}
\usepackage{verbatim}

% ukazi
\newcommand{\R}{\mathbb R} % množica realnih števil
\newcommand{\E}{\mathscr E} % Evklidski točkovni prostor
\newcommand{\V}{\mathscr V} % trirazsežni Evklidski vektorski prostor
\newcommand{\W}{\mathscr W} % normirani vektorski prostor
\renewcommand{\L}{\mathscr L} % prostor linearnih preslikav
\newcommand{\B}{\mathcal B} % materialno telo
\newcommand{\U}{\mathcal U} % množica U v E
\newcommand{\vek}[1]{\boldsymbol #1} % vektor
\newcommand{\ten}[1]{\mathbf #1} % tenzor drugega reda
\newcommand{\grad}{\mathrm{grad}} % gradient
\renewcommand{\div}{\mathrm{div}} % divergenca
\newcommand{\Grad}{\mathrm{Grad}} % Gradient
\newcommand{\Div}{\mathrm {Div}} % Divergenca
\newcommand{\tr}{\mathrm{tr}} % sled
\newcommand{\bsi}{\bar{\sigma}} % bar sigma
\newcommand{\rdva}{I\! I} % rimska 2
\newcommand{\rtri}{I\! I\! I} % rimska 3
\newcommand{\bottop}[3]{{#1}_{#2}^{\phantom{#2}{#3}}}
\newcommand{\topbot}[3]{{#1}^{#2}_{\phantom{#2}{#3}}}
\newcommand{\cs}[3]{{\mit\Gamma}_{#1\phantom{#2}#3}^{\phantom{#1}#2}} % Christoffelov simbol prve vrste
\newcommand{\ks}[2]{{\mit\Gamma}^{#1}_{\phantom{#1}#2}} % Christoffelov simbol druge vrste
\newcommand{\at}[2]{#1\Big|_{#2}} % izvršeno v točki


\title{Hamiltonov princip v mehaniki kontinuuma}
\author{Domen Mo\v{c}nik}


\begin{document}


\begin{frame}%[plain]
\titlepage
\begin{center}Mentor: izr.~prof.~dr.~Igor Dobov\v{s}ek\end{center}
\end{frame}


\begin{frame}
\frametitle{Konfiguracije}

\begin{block}{Konfiguracija materialnega telesa}
	\[ \kappa\colon \B\to\E,\qquad \kappa\colon X\mapsto x=\kappa(X). \]
\end{block}

\begin{figure}[h]
 \begin{center}
	\scalebox{.5}{\begin{picture}(360,300)


\put(20,30){\circle*{2}}
\put(20,20){$O$}

\put(20,30){\vector(-1,-2){12}}
\put(20,30){\vector(1,0){30}}
\put(20,30){\vector(0,1){30}}

\put(12,0){$\vek{E}_1$}
\put(51,20){$\vek{E}_2$}
\put(2,55){$\vek{E}_3$}

\put(80,90){\circle*{2}}
\put(20,30){\vector(1,1){60}}
\put(82,90){$X$}
\put(50,50){$\vek{X}$}

\put(70,70){\line(-4,1){32}}
\put(38,78){\line(-1,1){18}}
\put(20,96){\line(0,1){12}}
\put(20,108){\line(3,2){18}}
\put(38,120){\line(4,1){32}}
\put(70,128){\line(1,0){12}}
\put(82,128){\line(4,-1){24}}
\put(106,122){\line(3,-2){24}}
\put(130,106){\line(1,-1){12}}
\put(142,94){\line(-1,-4){3}}
\put(139,82){\line(-4,-1){48}}
\put(91,70){\line(-1,0){21}}

\put(43,125){$\B$}


\put(150,95){\vector(4,1){30}}
\put(150,80){\vector(4,0){125}}


%---------------

\put(220,30){\circle*{2}}
\put(220,20){$o$}

\put(220,30){\vector(-1,-2){12}}
\put(220,30){\vector(1,0){30}}
\put(220,30){\vector(0,1){30}}

\put(212,0){$\vek{e}_1$}
\put(251,20){$\vek{e}_2$}
\put(202,55){$\vek{e}_3$}

\put(240,110){\circle*{2}} % 230, 90
\put(220,30){\vector(1,4){20}}
\put(242,110){$x_1$}
%\put(260,70){$\vek{x}=\iota_o(x)$}

\put(314,77){\circle*{2}}
\put(220,30){\vector(2,1){94}}
\put(316,77){$x_2$}
%\put(260,70){$\vek{x}=\iota_o(x)$}

\put(230,90){\line(-4,1){32}} % +160, +20
\put(198,98){\line(-1,1){18}}
\put(180,116){\line(0,1){12}}
\put(180,128){\line(3,2){18}}
\put(198,140){\line(4,1){32}}
\put(230,148){\line(1,0){12}}
\put(242,148){\line(4,-1){24}}
\put(266,142){\line(3,-2){24}}
\put(290,126){\line(1,-1){12}}
\put(302,114){\line(-1,-4){3}}
\put(299,102){\line(-4,-1){48}}
\put(251,90){\line(-1,0){21}}

\put(198,149){$\B_{t_1}$}

\put(285,80){\line(-1,3){8}} % +160, +20
\put(277,104){\line(1,2){6}}
\put(283,116){\line(2,1){24}}
\put(307,128){\line(4,1){12}}
\put(319,131){\line(4,-1){24}}
\put(343,125){\line(1,-1){12}}
\put(355,113){\line(1,-3){5}}
\put(360,98){\line(0,-1){18}}
\put(360,80){\line(-1,-6){5}}
\put(355,50){\line(-2,-3){12}}
\put(343,32){\line(-4,-1){12}}
\put(331,29){\line(-3,2){18}}
\put(313,41){\line(-3,4){28}}
\put(285,79){\circle{1}}

\put(343,22){$\B_{t_2}$}


\end{picture}}
 \end{center}
\end{figure}

\end{frame}


\begin{frame}
\frametitle{Gibanje}

\begin{block}{Gibanje materialnega telesa \tiny{v časovnem intervalu $I=[t_1,t_2]$}}
	\begin{equation*}
		\chi\colon\B\times I\to\E,\qquad \chi\colon(X,t)\mapsto x
	\end{equation*}
\end{block}

\begin{block}{Trenutna konfiguracija telesa ob času $t\in I$}
	\[ \chi_t\colon\B\to\E,\qquad \chi_t(x):=\chi(X,t) \]
	\tiny{\[\B_t:=\chi_t(\B)\]}
\end{block}

\begin{figure}[h]
 \begin{center}
	\scalebox{.5}{\begin{picture}(360,300)


\put(20,30){\circle*{2}}
\put(20,20){$O$}

\put(20,30){\vector(-1,-2){12}}
\put(20,30){\vector(1,0){30}}
\put(20,30){\vector(0,1){30}}

\put(12,0){$\vek{E}_1$}
\put(51,20){$\vek{E}_2$}
\put(2,55){$\vek{E}_3$}

\put(80,90){\circle*{2}}
\put(20,30){\vector(1,1){60}}
\put(82,90){$X$}
\put(50,50){$\vek{X}$}

\put(70,70){\line(-4,1){32}}
\put(38,78){\line(-1,1){18}}
\put(20,96){\line(0,1){12}}
\put(20,108){\line(3,2){18}}
\put(38,120){\line(4,1){32}}
\put(70,128){\line(1,0){12}}
\put(82,128){\line(4,-1){24}}
\put(106,122){\line(3,-2){24}}
\put(130,106){\line(1,-1){12}}
\put(142,94){\line(-1,-4){3}}
\put(139,82){\line(-4,-1){48}}
\put(91,70){\line(-1,0){21}}

\put(43,125){$\B$}


\put(150,95){\vector(4,1){30}}
\put(150,80){\vector(4,0){125}}


%---------------

\put(220,30){\circle*{2}}
\put(220,20){$o$}

\put(220,30){\vector(-1,-2){12}}
\put(220,30){\vector(1,0){30}}
\put(220,30){\vector(0,1){30}}

\put(212,0){$\vek{e}_1$}
\put(251,20){$\vek{e}_2$}
\put(202,55){$\vek{e}_3$}

\put(240,110){\circle*{2}} % 230, 90
\put(220,30){\vector(1,4){20}}
\put(242,110){$x_1$}
%\put(260,70){$\vek{x}=\iota_o(x)$}

\put(314,77){\circle*{2}}
\put(220,30){\vector(2,1){94}}
\put(316,77){$x_2$}
%\put(260,70){$\vek{x}=\iota_o(x)$}

\put(230,90){\line(-4,1){32}} % +160, +20
\put(198,98){\line(-1,1){18}}
\put(180,116){\line(0,1){12}}
\put(180,128){\line(3,2){18}}
\put(198,140){\line(4,1){32}}
\put(230,148){\line(1,0){12}}
\put(242,148){\line(4,-1){24}}
\put(266,142){\line(3,-2){24}}
\put(290,126){\line(1,-1){12}}
\put(302,114){\line(-1,-4){3}}
\put(299,102){\line(-4,-1){48}}
\put(251,90){\line(-1,0){21}}

\put(198,149){$\B_{t_1}$}

\put(285,80){\line(-1,3){8}} % +160, +20
\put(277,104){\line(1,2){6}}
\put(283,116){\line(2,1){24}}
\put(307,128){\line(4,1){12}}
\put(319,131){\line(4,-1){24}}
\put(343,125){\line(1,-1){12}}
\put(355,113){\line(1,-3){5}}
\put(360,98){\line(0,-1){18}}
\put(360,80){\line(-1,-6){5}}
\put(355,50){\line(-2,-3){12}}
\put(343,32){\line(-4,-1){12}}
\put(331,29){\line(-3,2){18}}
\put(313,41){\line(-3,4){28}}
\put(285,79){\circle{1}}

\put(343,22){$\B_{t_2}$}


\end{picture}}
 \end{center}
\end{figure}

\end{frame}


\begin{frame}{Količine, povezane z gibanjem}

\begin{itemize}
	\item Deformacijski gradient
	\[
		\ten{F}\colon \B\times I\to\L(\V),\ \ten{F}(X,t)=\Grad\vek{\chi}(X,t)=\Grad\vek{\chi}_t(X)
	\]
	\item Jacobijan $J=\det\ten{F}$
	\item Hitrost in pospešek
	\begin{align*}
		\vek{v}\colon \B\times I\to \V \qquad & \vek{v}(X,t) = \frac{\partial\vek{\chi}}{\partial t}(X,t) \\
		\vek{a}\colon \B\times I\to \V \qquad & \vek{a}(X,t) = \frac{\partial^2\vek{\chi}}{\partial t^2}(X,t)
	\end{align*}
\end{itemize}

\end{frame}


\begin{frame}{Materialni in prostorski opis}

\[ \Omega=\{(x,t)\;;\ x\in\chi_t(\B),\ t\in I\}\subset\E\times I \]

\begin{block}{Prostorski opis polja $\vek{f}\colon \B\times I\to\W$}
	\[
		\bar{\vek{f}}\colon\Omega\to\W,\qquad
		\bar{\vek{f}}(x,t):=\vek{f}(\chi^{-1}(x,t),t)=\vek{f}(X,t)
	\]
	\tiny{\[ \chi^{-1}(x,t):=\bottop{\chi}{t}{-1}(x) \]}
\end{block}

\begin{block}{Materialni opis polja $\vek{f}\colon\Omega\to\W$}
	\begin{equation*}
		\hat{\vek{f}}\colon \B\times I\to\W,\qquad
		\hat{\vek{f}}(X,t):=\vek{f}(\chi(X,t),t)=\vek{f}(x,t)
	\end{equation*}
\end{block}

\end{frame}


\begin{frame}{Diferencialni operatorji v materialnem in prostorskem opisu}

\[ \Grad\vek{f}:=\nabla\hat{\vek{f}},\qquad \Div\vek{f}:=\div\hat{\vek{f}}, \]
\[ \grad\vek{f}:=\nabla\bar{\vek{f}},\qquad \div\vek{f}:=\div\bar{\vek{f}} \]

\begin{block}{Zveza med gradienti}
	$\phi$ skalarno, $\vek{u}$ pa vektorsko polje
	\begin{equation*}
		\Grad\phi=\ten{F}^{T}\grad\phi,\qquad \Grad\vek{u}=(\grad\vek{u})\ten{F}
	\end{equation*}
\end{block}

\begin{block}{Materialni časovni odvod}
	\begin{equation*}
		\frac{d\vek{f}}{dt}=\frac{\partial\vek{f}}{\partial t}+(\grad\vek{f})(\vek{v})
	\end{equation*}
\end{block}

\end{frame}


\begin{frame}{Površinski in prostorninski element}

\begin{equation*}
	\vek{n}\,da=J\ten{F}^{-T}\vek{N}\,dA
\end{equation*}
\begin{equation*}
	dv=J\,dV
\end{equation*}

\begin{align*}
	\int_{\mathcal{S}_t}\vek{f}[\vek{n}]\,da=\int_{\mathcal{S}}\vek{f}[J\ten{F}^{-T}\vek{N}]\,dA\\
	\int_{\mathcal{S}}\vek{f}[\vek{N}]\,dA=\int_{\mathcal{S}_t}\vek{f}[J^{-1}\ten{F}^{T}\vek{n}]\,da
\end{align*}

\begin{equation*}
	\int_{\mathcal{P}_t}\vek{f}\,dv=\int_{\mathcal{P}}\vek{f}J\,dV \qquad
	\int_{\mathcal{P}}\vek{f}\,dV=\int_{\mathcal{P}_t}\vek{f}J^{-1}\,dv
\end{equation*}

\end{frame}


\begin{frame}{Masa in masna gostota}

\[ M(\mathcal{P})=\int_{\mathcal{P}}\rho_R\,dV,\quad M(\mathcal{P})=\int_{\chi_t(\mathcal{P})}\rho\,dv \]
\[ \rho_R=J\rho \]

\begin{block}{Zakon o ohranitvi mase}
	\[ \frac{d}{dt}\int_{\chi_t(\mathcal{P})}\rho(x,t)\,dv = 0\quad\textrm{za vsak}\ t\in I \]
\end{block}

\begin{itemize}
	\item \[ \int_{\mathcal{P}}\vek{f}\rho_R\,dV=\int_{\chi_t(\mathcal{P})}\vek{f}\rho\,dv \]
	\item \[ \dot{\rho}+\rho\div\vek{v}=0 \]
	\item \[
		\frac{d}{dt}\int_{\chi_t(\mathcal{P})}\vek{f}\rho\,dv=
		\int_{\chi_t(\mathcal{P})}\dot{\vek{f}}\rho\,dv.
	\]
\end{itemize}

\end{frame}


\begin{frame}{Dopustno gibanje}

\[ \mathcal{X}=C^2(\B\times I,\E) \]
\begin{block}{Dopustno gibanje}
	$\chi\in\mathcal{X}$, ki ustreza začetnim in robnim pogojem\\
	Množica vseh dopustnih gibanj: $\mathcal{A}$
\end{block}

\begin{block}{Variacija gibanja}
	$\vek{\eta}\in C^2(\B\times I,\V)$,
	\[
		\vek{\eta}(X,t_1)=\vek{0}\quad \textrm{in} \quad
		\vek{\eta}(X,t_2)=\vek{0}\quad \textrm{za vse}\ X\in \B\quad\textrm{ter}
	\]
	\[
		\vek{\eta}(X,t)=\vek{0}\quad \textrm{za vse}\ (X,t)\in \partial_1 \B\times I
	\]
	Množica vseh variacij gibanja: $\mathcal{T}$
\end{block}

\begin{block}{Bližnje gibanje}
	$\chi+\varepsilon\vek{\eta}$, $\varepsilon\in\R$
\end{block}

\end{frame}


\begin{frame}{Variacije}

$U^{\mathrm{odp}}\subseteq\mathcal{A}$, $F\colon U\to\mathcal{Y}$
\[
	\delta F\colon U\to\L(\mathcal{T},\mathcal{Y}),\qquad
	\delta F(\chi)(\vek{\eta})=\at{\frac{d}{d\varepsilon}F(\chi+\varepsilon\vek{\eta})}{\varepsilon=0}
\]
\[ F(\chi)=Y \]
\[ Y^*=F(\chi+\varepsilon\vek{\eta}) \]
\[ \delta Y = \at{\Big(\frac{dY^*}{d\varepsilon}\Big)}{\varepsilon=0} \]

\end{frame}


\begin{frame}{Variacije osnovnih količin}

\begin{itemize}

\item
\[ F(\chi)=d\vek{\chi}/dt=\dot{\vek{\chi}}=\vek{v} \]
\[ \vek{v}^*=\frac{d}{dt}(\vek{\chi}+\varepsilon\vek{\eta})=\vek{v}+\varepsilon\dot{\vek{\eta}} \]
\[ \delta\vek{v}=\dot{\vek{\eta}} \]

\item
$F(\chi)=\Grad{\vek{\chi}}=\ten{F}$, $\delta\ten{F}=\Grad\vek{\eta}$

\item
$J=\det(\Grad\vek{\chi})$, $J^*=\det(\Grad(\vek{\chi}+\varepsilon\vek{\eta}))=\det(\ten{F}^*)$,
$\delta J=J\div\vek{\eta}$

\item
$\rho^*=\frac{\rho_R}{J^*}$, $\delta\rho=-\rho\div\vek{\eta}$

\end{itemize}

\end{frame}


\begin{frame}{Kinetična energija}

\begin{equation*}
	T = \int_{\B}\frac{1}{2}\langle\vek{v},\vek{v}\rangle\rho_R\, dV =
	\int_{\B_t}\frac{1}{2}\langle\vek{v},\vek{v}\rangle\rho\, dv
\end{equation*}

\[
	\mathcal{I}=\int_{t_1}^{t_2} T\,dt=\int_{t_1}^{t_2}
	\int_{\B}\frac{1}{2}\langle\vek{v},\vek{v}\rangle\rho_R\, dV \,dt
\]

\begin{equation*}
	\delta \mathcal{I} =
	-\int_{t_1}^{t_2}\int_{\B}\langle\rho_R\vek{a},\vek{\eta}\rangle\, dV\, dt=
	-\int_{t_1}^{t_2}\int_{\B_t}\langle\rho\vek{a},\vek{\eta}\rangle\, dv\, dt
\end{equation*}

\end{frame}


\begin{frame}{Virtualno delo zunanjih sil}

\[
	\vek{f}(t)=\vek{f}_b(t)+\vek{f}_s(t),
\]
\[
	\vek{f}_b(t)=\int_{\B_t}\vek{b}\rho\, dv=\int_{\B}\vek{b}\rho_R\, dV
\]
\[
	\vek{f}_s(t)=\int_{\partial \B_t}\vek{t}\, da=\int_{\partial \B}\vek{t}_R\, dA
\]
\[ \vek{b}(x,t)=-\grad\phi(x,t),\qquad \vek{t}_R(x,t)=-\grad\psi(x,t) \]

\begin{equation*}
	W = \int_{\B}\phi\rho_R\, dV + \int_{\partial_2 \B}\psi\, dA
\end{equation*}

\begin{align*}
	\delta W &= -\int_{\B}\langle\rho_R\vek{b},\vek{\eta}\rangle\, dV
	- \int_{\partial_2 \B}\langle\vek{t}_R,\vek{\eta}\rangle\, dA\\
	&=-\int_{\B_t}\langle\rho\vek{b},\vek{\eta}\rangle\, dv
	- \int_{\partial_2 \B_t}\langle\vek{t},\vek{\eta}\rangle\, da
\end{align*}

\end{frame}


\begin{frame}{Hamiltonov princip}

\begin{block}{Hamiltonov funkcional}
	gibanju $\chi\in \mathcal{A}$ s kinetično energijo $T$ in potencialno
	energijo $U+W$ priredi realno število
	\[
		H=\int_{t_1}^{t_2}(T-U-W)\,dt.
	\]
\end{block}

\begin{block}{Hamiltonov princip}
	Za pravo gibanje telesa velja $J>0$, Hamiltonov funkcional pa pri pravem gibanju
	zavzame lokalni minimum.
\end{block}

\begin{block}{Trditev}
	Če funkcional $F\colon \mathcal{D}\subseteq\mathcal{A}\to\R$ zavzame lokalni minimum pri dopustnem gibanju $\chi\in\mathcal{D}$, potem velja
	\[ \delta F(\chi)(\vek{\eta})=0\quad\textrm{za vsak}\ \vek{\eta}\in\mathcal{T}. \]
\end{block}

\end{frame}


\begin{frame}{Potreben pogoj za pravo gibanje}

\begin{itemize}
	\item $J>0$
	\item \[ \delta H = \int_{t_1}^{t_2}(\delta T-\delta U-\delta W)\,dt = 0 \]
\end{itemize}

\end{frame}


\begin{frame}{Elastični fluid}

\begin{block}{Notranja energija}
	\[ U=\int_{\B}\rho_R e(\rho)\,dV=\int_{\chi_t(\B)}\rho e(\rho)\,dv \]
	$e\colon[0,\infty)\to\R$ gostota notranje energije
\end{block}

\begin{align*} 
	\delta U &=-\int_{\B}\rho_R e'(\rho)\rho\div\vek{\eta}\,dV \\
	&= -\int_{\chi_t(\B)}\rho^2 e'(\rho)\div\vek{\eta}\,dv.
\end{align*}

\begin{block}{Robni pogoj napetosti}
	\[ \vek{t}=-p\vek{n}\ \textrm{na}\ \{ (x,t)\,;\ x\in\chi_t(\mathcal{R}),\ t\in[t_1,t_2] \} \]
\end{block}

\end{frame}


\begin{frame}{Elastični fluid}

\begin{multline*}
	\int_{t_1}^{t_2}\bigg( \int_{\chi_t(\B)}\Big<-\rho\vek{a}-
	\grad\big(\rho^2 e'(\rho)\big)+\rho\vek{b},\vek{\eta}\Big>\,dv+ \\
	+\int_{\chi_t(\partial \B)}\big<\big(\rho^2 e'(\rho)-p\big)\vek{n},\vek{\eta}\big>\,da\bigg)\,dt=0
\end{multline*}

\begin{equation*}
	\rho\vek{a}=\grad\big(\rho^2 e'(\rho)\big)+\rho\vek{b}\quad
	\textrm{na}\ \Omega
\end{equation*}

\begin{equation*}
	\rho^2 e'(\rho)=p \quad\textrm{na}\ \{ (x,t)\,;\ x\in\chi_t(\mathcal{R}),\ t\in[t_1,t_2] \}
\end{equation*}

\end{frame}


\begin{frame}{Hiperelastično trdno telo}

\begin{block}{Notranja energija}
	\[ U=\int_{\B}\rho_R \sigma(\ten{F})\,dV=\int_{\chi_t(\B)}\rho\sigma(\ten{F})\,dv \]
	$\sigma\colon\L(\V)\to\R$ funkcija shranjene energije ali gostota deformacijske energije
\end{block}

\begin{align*} 
	\delta U &=\int_{\B}\langle\ten{S},\Grad\vek{\eta}\rangle\,dV \\
	&= \int_{\chi_t(\B)}\langle\ten{T},\grad\vek{\eta}\rangle\,dv
\end{align*}

\[ \ten{S}=\rho_RD\sigma(\ten{F}),\qquad \ten{T}=\frac{1}{J}\ten{S}\ten{F}^{T}=\rho D\sigma(\ten{F})\ten{F}^{T} \]

\end{frame}


\begin{frame}{Hiperelastično trdno telo}

\begin{multline*}
	\int_{t_1}^{t_2}\bigg( \int_{\B}\langle -\rho_R\vek{a}+
	\Div\ten{S}+\rho_R\vek{b},\vek{\eta}\rangle \,dV+ \\
	+\int_{\partial_2 \B}\langle\vek{t}_R-\ten{S}\vek{N},\vek{\eta}\rangle\,dA\bigg)\,dt=0
\end{multline*}
\begin{multline*}
	\int_{t_1}^{t_2}\bigg( \int_{\chi_t(\B)}\langle -\rho\vek{a}+
	\div\ten{T}+\rho\vek{b},\vek{\eta}\rangle\,dv+ \\
	+\int_{\chi_t(\partial_2 \B)}\langle\vek{t}-\ten{T}\vek{n},\vek{\eta}\rangle\,ds\bigg)\,dt=0
\end{multline*}

\end{frame}


\begin{frame}{Hiperelastično trdno telo}

\begin{equation*}
	\rho_R\vek{a}=\Div\ten{S}+\rho_R\vek{b}\quad\textrm{na}\ \B\times[t_1,t_2]
\end{equation*}
\begin{equation*}
	\ten{S}\vek{N}=\vek{t}_R\quad\textrm{na}\ \mathcal{R}\times[t_1,t_2]
\end{equation*}
\\
\begin{equation*}
	\rho\vek{a}=\div\ten{T}+\rho\vek{b}\quad\textrm{na}\ \Omega
\end{equation*}
\begin{equation*}
	\ten{T}\vek{n}=\vek{t}\quad\textrm{na}\ \{(x,t)\,;\ x\in\chi_t(\mathcal{R}),\ t\in[t_1,t_2] \}
\end{equation*}

\end{frame}



\end{document}