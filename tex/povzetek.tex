\begin{center}
{\bf Hamiltonov princip v mehaniki kontinuuma}\\[3mm]
{\sc Povzetek}
\end{center}

Namen mehanike kontinuuma je raziskovanje gibanja materialnih teles pod
vplivom sil v skladu z zakoni narave. Klasičen pristop za iskanje enačb gibanja
vključuje uporabo Eulerjevih in Newtonovih zakonov. Alternativni pristop
je variacijski, s pomočjo Hamiltonovega principa. Pri tem pristopu se pravo
gibanje materialnega telesa išče kot stacionarno točko Hamiltonovega
energijskega funkcionala. V klasičnih teorijah fluidov in deformabilnih teles je problem
iskanja pravega gibanja
mogoče postaviti kot primer variacijske naloge na prostoru vseh dopustnih gibanj,
enačbe gibanja, ki izhajajo iz potrebnih pogojev za nastop stacionarne točke
funkcionala, pa je mogoče poiskati s pomočjo ustrezno definiranih testnih funkcij.

\vfill
\begin{center}
{\bf Hamilton's principle in continuum mechanics}\\[3mm]
{\sc Abstract}
\end{center}

The purpose of continuum mechanics is investigation of motion of material bodies
under the influence of forces according to the laws of nature. Finding the equations of motion
by classical approach includes Euler and Newton laws. Alternative approach is
variational, by using Hamilton's principle. With this principle the true motion
is found as a stationary point of Hamilton's energy functional. For classical
theories of fluid and solid mechanics, searching of true motion can be put
as an example of variational task on the space of all admissible motions.
Equations of motion, which come from necessary conditions for occurrence
of sationary point of the functional, can be found with properly defined test functions.


\vfill\noindent
{\bf Math.~Subj.~Class. (2010):} 49S05, 70H25, 74A05, 74A10, 74A20  \\[1mm]  
{\bf Klju"cne besede:} Hamiltonov princip, Mehanika kontinuuma \\[1mm]  
{\bf Keywords:} Hamilton's principle, Continuum mechanics