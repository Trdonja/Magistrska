\chapter{Sredstva iz analize in mehanike kontinuuma}


\section{Tenzorska analiza}


\subsection{Evklidski prostor}


V klasični mehaniki opisujemo dogodke v \emph{Newtonovem prostor-času}, kar je produkt
trirazsežnega Evklidskega prostora ter prostora realnih števil $\R$. Evklidski prostor nam služi
za opis položaja in geometrije objektov, prostor $\R$ pa predstavlja časovno os.

\begin{definicija} \label{d:ep}
	Množica točk $\E$ je \emph{Evklidski točkovni prostor} in trirazsežni Evklidski vektorski prostor $V$ je
	\emph{translacijski prostor} za $\E$, če poljubnemu paru točk $p,q\in\E$ pripada vektor iz $V$,
	ki ga zapišemo kot $q-p$, tako da velja:
	\begin{enumerate}
		\item Za vsak $p\in\E$ je $p-p=\vek{0}$, ničelni vektor.
		\item Za vsak $p\in\E$ in vsak $\vek{v}\in V$ obstaja natanko ena točka $q\in\E$, da je
		$\vek{v}=q-p$. Pišemo: $q=p+\vek{v}$.
		\item Za vse $p,q,r\in\E$ velja $(q-p)+(r-q)=(r-p)$.
	\end{enumerate}
\end{definicija}

Razdalja med točkama $p,q\in\E$ je definirana kot $d(p,q)=\|q-p\|$, kjer $\|.\|$ označuje normo na
vektorskem prostoru $V$, porojeno iz standardnega skalarnega produkta. $(\E,d)$ je metrični prostor.

V prostoru $\E$ si izberimo točko, jo označimo z $o$ in jo imenujmo \emph{izhodišče}.
V skladu z aksiomi iz definicije pripada vsaki točki $p\in\E$ glede na izhodišče $o$ \emph{krajevni vektor}
$\vek{r}(p)=p-o\in V$. Naj bo $\{\vek{e}_1,\vek{e}_2,\vek{e}_3\}$ neka ortonormirana baza prostora $V$,
ki je desnosučna, tj.~$\vek{e}_3=(\vek{e}_1\times\vek{e}_2)$. Krajevni vektor $\vek{r}(p)$
ima enoličen razvoj po bazi, $\vek{r}(p)=y_k\vek{e}_k$, koeficientom $y_k$ tega razvoja pa rečemo
\emph{kartezijeve koordinate}. Te so odvisne od izbire izhodišča in ortonormirane baze.

Včasih si želimo prostor $\E$ opremiti s kakšnim drugim koordinatnim sistemom, ki ga podamo
s koordinatno transformacijo kartezijevih koordinat
\begin{equation}\label{e:kt}
	x^j = \hat{x}^j(y_1,y_2,y_3) \ \Leftrightarrow \ y_k=\hat{y}_k(x^1,x^2,x^3), \quad j,k=1,2,3.
\end{equation}
Za koordinatno transformacijo zahtevamo, da je bijektivna in gladka preslikava iz
$D^{\mathrm{odp}}\subseteq\R^3$ v $U^{\mathrm{odp}}\subseteq\R^3$ s prav tako gladkim inverzom.

Imamo torej bijektivno korespondenco med naslednjimi objekti:
\begin{itemize}
	\item točka $p\in\E$,
	\item krajevni vektor $\vek{r}(p)=y_k\vek{e}_k\in V$,
	\item koordinate $(x^1,x^2,x^3)\in\R^3$.
\end{itemize}
Zato bomo točke iz prostora $\E$ v bodoče identificirali z njihovimi krajevnimi vektorji ali pa z njihovimi koordinatami.

Newtonov prostor čas $\mathcal{N}=\E\times\R$ lahko naredimo za vektorski prostor s skalarnim produktom
\[ (\vek{x},t_1)\cdot(\vek{y},t_2) = \vek{x}\cdot\vek{y} + t_1 t_2. \]
Iz skalarnega produkta dobimo tudi normo in metriko.


\subsection{Tenzorska polja, gradient in divergenca}


V tem razdelku bomo navedli nekaj bistvenih pojmov in rezultatov iz tenzorske analize. Predpostavlja
se, da bralec že pozna osnove tenzorske analize. Razdelek služi bolj predstavitvi oznak, ki jih
bomo uporabljali v nadaljevanju.

Funkciji $f\colon\mathcal{D}\subseteq\E\to W$, kjer je $W$ neki normirani vektorski prostor, rečemo
\emph{tenzorsko polje}. V posebnem primeru, ko je $W=\R$, rečemo funkciji $f$ \emph{skalarno},
v primeru $W=V$ pa \emph{vektorsko polje}. Tenzorskim poljem bomo včasih kot argument namesto točke raje
podali njen krajevni vektor ali pa njene koordinate, ne da bi pri tem spremenili oznako za tenzorsko polje.

\begin{definicija}
	Naj bo $f:\mathcal{D}^{\mathrm{odp}}\subseteq\E\to W$ tenzorsko polje. Funkcija $f$ je \emph{odvedljiva}
	v točki $x\in\mathcal{D}$, če obstaja taka linearna preslikava $\nabla f(x):V\to W$, da za vsak $\vek{h}\in V$ velja
	\begin{equation*}
		f(x+\vek{h})=f(x)+\nabla f(x)[\vek{h}]+o(\vek{h}),
	\end{equation*}
	kjer je $o(\vek{h})\in W$ količina, za katero je
	\[ \lim_{\| h\|\to 0}\frac{\|o(\vek{h})\|}{\|\vek{h}\|}=0. \]
	Če $\nabla f(x)$ obstaja, ji rečemo \emph{gradient} ali pa \emph{krepki} oz.~\emph{Fréchetov odvod} polja $f$ v točki $x$
	in ga običajno označimo z $\grad f(x)$.
\end{definicija}
Gradient, če obstaja, je tudi tenzorsko polje, vrednosti pa zavzema v prostoru $\L(V,W)$.

Gradient skalarnega polja $f$ zavzema vrednosti iz prostora $\L(V,\R)$, kar so linearni funkcionali.
S pojmom \emph{gradient} in oznako $\grad f$ se v tem primeru označuje polje vektorjev, ki pripadajo polju
linearnih funkcionalov po Riezsovem izreku o reprezentaciji, tako da velja
\[ \nabla f[\vek{h}]=\grad f \cdot \vek{h} \quad \forall\, \vek{h}\in V. \]

Krepke odvode računamo s pomočjo \emph{šibkega} oz.~\emph{smernega} (včasih tudi \emph{Gâteauxovega}) \emph{odvoda}:
\begin{equation*}
	\delta f(x)[\vek{h}]=\lim_{s\to 0}\frac{1}{s}\Big(f(x+s\vek{h})-f(x)\Big)=
	\at{\frac{d}{ds}f(x+s\vek{h})}{s=0}.
\end{equation*}
Znano je, da če obstaja krepki odvod,
potem obstaja tudi šibki odvod in sta enaka: $\nabla f(x)[\vek{h}]=\delta f(x)[\vek{h}]$ za vsak $\vek{h}\in V$.
Za krepke in šibke odvode veljajo enaki izreki, kot veljajo za preslikave med normiranimi prostori, ki jih
poznamo iz matematične analize: izrek o posrednem odvajanju, izrek o odvajanju produkta\footnote{
Izrek o odvajanju produkta velja za katerikoli produkt, ki je bilinearna preslikava. Med njimi so
npr. skalarni in vektorski produkt vektorjev, produkt tenzorja s skalarjem, tenzorski produkt,
delovanje tenzorja na vektorju itd.},
izrek o totalnem odvodu itd.

Medtem ko gradient viša red tenzorskega polja, ga divergenca niža.
\emph{Divergenca vektorskega polja} $\vek{u}$ je skalarno polje
\begin{equation} \label{e:div1}
	\div\vek{u}=\tr(\nabla\vek{u}).
\end{equation}
\emph{Divergenca tenzorskega polja} $S\colon\mathcal{D}\to\L(V)$, je vektorsko polje $\div S$ z lastnostjo,
da za vsako konstantno vektorsko polje $\vek{v}$ velja
\begin{equation} \label{e:div2}
	\vek{v}\cdot\div S = \div(S^{\,T}\vek{v}).
\end{equation}
Podajmo nekaj lastnosti gradienta in divergence, ki jih bomo potrebovali v nadaljevanju.
\begin{trditev} \label{t:divprop}
	Za diferenciabilno skalarno polje $\phi$ ter diferenciabilni vektorski polji $\vek{u}$ in $\vek{v}$ velja
	\begin{enumerate}
		\item $\nabla(\phi\vek{v})=\vek{v}\otimes\nabla\phi+\phi\nabla\vek{v},$
		\item $\nabla(\vek{u}\cdot\vek{v})=(\nabla\vek{u})^T\vek{v}+(\nabla\vek{v})^T\vek{u},$
		\item $\div(\phi\vek{v})=\vek{v}\cdot\nabla\phi+\phi\div\vek{v}$,
		\item $\div(\vek{u}\otimes\vek{v})=(\nabla\vek{u})\vek{v}+\vek{u}\div\vek{v}$.
	\end{enumerate}
\end{trditev}
\proof
	Za poljuben vektor $\vek{h}$ je
	\begin{align*}
		\nabla(\phi\vek{v})[\vek{h}]&=(\nabla\phi[\vek{h}])\vek{v}+\phi(\nabla\vek{v})[\vek{h}]=
		(\nabla\phi\cdot\vek{h})\vek{v}+\phi(\nabla\vek{v})\vek{h}=\\
		&=(\vek{v}\otimes\nabla\phi)\vek{h}+\phi(\nabla\vek{v})\vek{h}=
		\big(\vek{v}\otimes\nabla\phi+\phi\nabla\vek{v}\big)[\vek{h}],
	\end{align*}
	iz česar sledi prva enakost, ter
	\begin{align*}
		\nabla(\vek{u}\cdot\vek{v})[\vek{h}]&=(\nabla\vek{u})[\vek{h}]\cdot\vek{v}+\vek{u}\cdot(\nabla\vek{v})[\vek{h}]=
		(\nabla\vek{u})^T\vek{v}\cdot\vek{h}+(\nabla\vek{v})^T\vek{u}\cdot\vek{h}=\\
		&=\big((\nabla\vek{u})^T\vek{v}+(\nabla\vek{v})^T\vek{u}\big)[\vek{h}],
	\end{align*}
	iz česar sledi druga enakost. Tretjo enakost dokažemo neposredno:
	\begin{align*}
		\div(\phi\vek{v})&=\tr\big(\nabla(\phi\vek{v})\big)=\tr(\vek{v}\otimes\nabla\phi+\phi\nabla\vek{v})=\\
		&=\tr(\vek{v}\otimes\nabla\phi)+\phi\tr(\nabla\vek{v})=\vek{v}\cdot\nabla\phi+\phi\div\vek{v}.
	\end{align*}
	Pri tem smo uporabili definicijo divergence (\ref{e:div1}), prvo enakost trditve in dejstvi, da je sled
	linearen operator ter da je sled tenzorskega produkta dveh vektorjev enaka skalarnemu produktu teh
	dveh vektorjev.
	
	Za poljubno konstantno vektorsko polje $\vek{w}$ je
	\begin{align*}
		\vek{w}\cdot\div(\vek{u}\otimes\vek{v})&=\div\big((\vek{u}\otimes\vek{v})^T\vek{w}\big)=
		\div\big((\vek{v}\otimes\vek{u})\vek{w}\big)=\\
		&=\div\big((\vek{u}\cdot\vek{w})\vek{v}\big)=\vek{v}\cdot\nabla(\vek{u}\cdot\vek{w})+(\vek{u}\cdot\vek{w})\div\vek{v}=\\
		&=(\nabla\vek{u})\vek{v}\cdot\vek{w}+\vek{u}\div\vek{v}\cdot\vek{w}=\big((\nabla\vek{u})\vek{v}+\vek{u}\div\vek{v}\big)\cdot\vek{w},
	\end{align*}
	iz česar sledi četrta enakost. Pri tem smo uporabili definicijo divergence (\ref{e:div2}) ter
	tretjo in drugo enakost trditve, pri čemer smo upoštevali, da je $\nabla\vek{w}=\ten{0}$.
\endproof

Če je polje $f$ časovno odvisno, tj. $f$ je preslikava
\[ f\colon\mathcal{D}\times I\to W, \]
kjer je $\mathcal{D}\subseteq\E$ odprta množica, $I=(t_1,t_2)\subseteq\R$ (časovni) interval, $W$ pa neki normirani prostor,
potem je definiran še \emph{časovni odvod}
\[ \frac{\partial}{\partial t}f(x,t) = \lim_{s\to 0}\frac{1}{s}\Big(f(x,t+s)-f(x,t)\Big), \]
kar je zopet časovno odvisno tenzorsko polje z vrednostmi iz prostora $W$. $n$-ti
časovni odvod polja $f$ označimo z $\partial^n f/\partial t^n$.


\subsection{Integralski izreki}


\begin{trditev}\label{t:oiz}
	Naj bo $\mathcal{D}\subseteq\E$ odprta množica ter $f\colon\mathcal{D}\to W$ zvezno tenzorsko polje.
	Če za vsako podmnožico $\mathcal{N}\subseteq\mathcal{D}$ velja
	\[ \int_{\mathcal{N}}f\,dv=0, \]
	potem je $f(x)=0$ za vsak $x\in\mathcal{D}$.
\end{trditev}
\proof
	Recimo, da obstaja $x_0\in\mathcal{D}$, da je $f(x_0)\neq 0$. Ker je $f$ zvezno, obstaja okolica
	$\mathcal{N}\subseteq\mathcal{D}$ točke $x_0$ z volumnom $v(\mathcal{N})>0$, tako da je $f(x)\neq 0$
	za vsak $x\in\mathcal{N}$. Po izreku o povprečni vrednosti iz analize obstaja točka $\xi\in\mathcal{N}$, da je
	\[ \int_{\mathcal{N}}f\,dv=v(\mathcal{N})f(\xi)\neq 0, \]
	kar je v protislovju z začetno predpostavko.
\endproof

\begin{izrek}[Divergenčni izrek]
	Naj bo $\mathcal{D}$ omejeno območje v $\E$, katerega rob $\partial\mathcal{D}$ je sestavljen
	iz končnega števila gladkih ploskev. Naj bodo $\phi\colon\overline{\mathcal{D}}\to\R$,
	$\vek{u}\colon\overline{\mathcal{D}}\to V$ ter $\ten{S}\colon\overline{\mathcal{D}}\to\L(V)$
	zvezna polja, diferenciabilna v notranjosti območja $\mathcal{D}$, ter $\vek{n}\colon\partial\mathcal{D}\to V$
	polje zunanjih enotskih normal. Potem velja
	\begin{align}
		\int_{\partial\mathcal{D}} \phi\vek{n}\,ds &= \int_{\mathcal{D}} \grad\phi\,dv, \nonumber \\
		\int_{\partial\mathcal{D}} \vek{u}\cdot\vek{n}\,ds &= \int_{\mathcal{D}} \div\vek{u}\,dv, \\
		\int_{\partial\mathcal{D}} \ten{S}\vek{n}\,ds &= \int_{\mathcal{D}} \div\ten{S}\,dv. \nonumber
	\end{align}
\end{izrek}
\proof
	Druga enakost je splošno znan \emph{Gaussov izrek} iz vektorske analize, zato ga tukaj ne bomo dokazovali.
	Naj bo $\vek{w}$ poljubno konstantno vektorsko polje. Potem je
	\begin{align*}
		\vek{w}\cdot\int_{\partial\mathcal{D}}\phi\vek{n}\,ds &= \int_{\partial\mathcal{D}}\phi\vek{w}\cdot\vek{n}\,ds=
		\int_{\mathcal{D}}\div(\phi\vek{w})\,dv = \\
		&= \int_{\mathcal{D}}\vek{w}\cdot\grad\phi \,dv = \vek{w}\cdot\int_{\mathcal{D}}\grad\phi \,dv,
	\end{align*}
	s čimer smo dokazali prvo enakost.
	Pri tem smo uporabili Gaussov izrek ter tretjo enakost iz trditve (\ref{t:divprop}), pri čemer
	smo upoštevali, da je $\div\vek{w}=0$.
	
	Zopet naj bo $\vek{w}$ poljubno konstantno vektorsko polje. Potem je
	\begin{align*}
		\vek{w}\cdot\int_{\partial\mathcal{D}}\ten{S}\vek{n}\,ds &= \int_{\partial\mathcal{D}}\vek{w}\cdot\ten{S}\vek{n}\,ds=
		\int_{\partial\mathcal{D}}\ten{S}^T\vek{w}\cdot\vek{n}\,ds=
		\int_{\mathcal{D}}\div(\ten{S}^T\vek{w})\,dv = \\ &=\int_{\mathcal{D}}\vek{w}\cdot\div\ten{S}\,dv=
		\vek{w}\cdot\int_{\mathcal{D}}\div\ten{S}\,dv,
	\end{align*}
	kjer smo zopet uporabili Gaussov izrek in definicijo divergence (\ref{e:div2}). S tem smo dokazali še tretjo enakost.
\endproof


\section{Materialno telo}


Materialno telo je določeno z izbrano množico točk $\B\subseteq\E$. Tej določitvi rečemo
\emph{referenčna konfiguracija} materialnega telesa in služi zgolj označbi telesa; ni nujno,
da se telo dejansko kadarkoli nahaja v tem položaju. Oznaka $\B$ bo odslej vedno prestavljala
referenčno konfiguracijo telesa.


\subsection{Gibanje}


\begin{definicija}
	\emph{Gibanje} materialnega telesa v časovnem
	intervalu $I=(t_0,t_1)$ je zvezno, časovno odvisno vektorsko polje
	\begin{equation}\label{e:chi}
		\chi\colon\B\times I\to\E,\qquad \chi\colon (\vek{X},t)\mapsto\vek{x},
	\end{equation}
	za katerega je za vsak $t\in I$ preslikava
	\[ \chi_t\colon\B\to\B_t, \qquad \chi_t\colon\vek{X}\mapsto\chi(\vek{X},t)=\vek{x} \]
	bijektivna. Slika $\B_t$ se imenuje \emph{trenutna konfiguracija} materialnega
	telesa \emph{ob času} $t$.
\end{definicija}
Gibanje (\ref{e:chi}) ni bijektivna preslikava, zato nima inverza. Kljub temu na smiselen
način definiramo \emph{inverzno gibanje}
\[ \chi^{-1}(\vek{x},t)\equiv\bottop{\chi}{t}{-1}(\vek{x}), \]
za katerega zahtevamo, da je zvezno na domeni $\{\B_t\times\{t\}\,|\,t\in I\}$.
Dodatno bomo predpostavili, da sta gibanje in inverzno gibanje razreda $C^1$, razen če ne bo navedeno drugače.

%V predstavitvi gibanja (\ref{e:chi}) se krajevni vektor $\vek{X}$ imenuje \emph{materialni vektor},
%točka $X$, ki jo $\vek{X}$ predstavlja, se imenuje \emph{materialna točka} in njene koordinate
%$(X^1,X^2,X^3)$ se imenujejo \emph{materialne koordinate}. Krajevni vektor $\vek{x}$ se imenuje
%\emph{prostorski vektor}, točka $x$, ki jo predstavlja, se imenuje \emph{prostorska točka},
%njene koordinate $(x^1,x^2,x^3)$ pa se imenujejo \emph{prostorske koordinate}.

\emph{Hitrost} $\vek{v}$ in \emph{pospešek} $\vek{a}$ gibanja $\chi$ v časovnem intervalu $I$ sta vektorski polji
\begin{align}
	\vek{v}\colon\B\times I\to V \qquad & \vek{v} = \frac{\partial}{\partial t}\chi(\vek{X},t), \label{e:v} \\
	\vek{a}\colon\B\times I\to V \qquad & \vek{a} = \frac{\partial^2}{\partial t^2}\chi(\vek{X},t). \label{e:a}
\end{align}
Pri tem mora biti $\chi$ vsaj dvakrat odvedljivo po času. \emph{Deformacijski gradient} gibanja $\chi$ je
tenzorsko polje
\begin{equation} \label{e:F} F=\Grad\chi(\vek{X},t). \end{equation}
Za determinanto $J=\det F$ se predpostavi, da je pozitivna.


\subsection{Materialni in prostorski opis}


Materialno telo spremljajo različne fizikalne lastnosti, katerih vrednosti predstavimo z elementi nekega
normiranega prostora $W$. Vrednosti fizikalnih količin
se med gibanjem v časovnem intervalu $I\subseteq\R$ spreminjajo, opišemo pa jih lahko na
dva različna načina.
\begin{definicija}
	\emph{Materialni} ali \emph{Lagrangejev opis} je predstavitev fizikalne količine $f$, ki spremlja materialno telo
	med gibanjem $\chi$, s tenzorskim poljem
	\[ \hat{f}\colon\B\times I \to W, \qquad \hat{f}\colon (\vek{X},t)\mapsto w. \]
	\emph{Prostorski} ali \emph{Eulerjev opis} je pri vsakem $t\in I$ predstavitev taiste fizikalne količine s tenzorskim poljem
	\[ \bar{f}(\cdot,t)\colon\B_t\to W, \qquad \bar{f}\colon (\vek{x},t)\mapsto w. \]
\end{definicija}
Med materialnim in prostorskim opisom veljata zvezi
\begin{equation}\label{e:mpo-zveza}
	\hat{f}(\vek{X},t) = \bar{f}\big(\chi(\vek{X},t),t\big),\qquad \bar{f}(\vek{x},t) = \hat{f}\big(\chi^{-1}(\vek{x},t),t\big).
\end{equation}
Kasneje bomo strešico ali črtico v oznaki za polje izpustili, če bo jasno iz konteksta, kateri opis uporabljamo.
Veliki $\vek{X}$ v argumentu polja bo vedno ponazarjal materialni opis, mali $\vek{x}$ pa prostorski opis.
Do dvomov lahko pride, kadar ne pišemo argumentov polja, še posebej, če so vključeni odvodi. Temu dvomu se izognemo tako,
da uporabljamo različno notacijo za odvode. V materialnem opisu pišemo oznaki za gradient in divergenco z
veliko začetnico
\[ \Grad f\equiv\grad \hat{f}(\vek{X},t),\qquad \Div f \equiv \div \hat{f}(\vek{X},t), \]
v prostorskem opisu pa z malo
\[ \grad f\equiv\grad \bar{f}(\vek{x},t),\qquad \div f \equiv \div \bar{f}(\vek{x},t). \]
Če je $\phi$ skalarno, $\vek{u}$ pa vektorsko polje, potem je zveza med gradientoma
\begin{equation}\label{e:gz}
	\Grad\phi=F^{\,T}\grad\phi,\qquad \Grad\vek{u}=(\grad\vek{u})F.
\end{equation}
Res, če je $\vek{w}$ poljubno vektorsko polje, s posrednim odvajanjem dobimo
\begin{align*}
	&\Grad\phi\cdot\vek{w}=\grad\phi\cdot(\Grad\chi)\vek{w}=
	\grad\phi\cdot F\vek{w}=F^{\,T}\grad\phi\cdot\vek{w}, \\
	&(\Grad\vek{u})\vek{w}=(\grad\vek{u})(\Grad\chi)\vek{w}
	=(\grad\vek{u})F\vek{w}.
\end{align*}

\begin{definicija}
	Časovni odvod
	\[ \dot{f}=\frac{d}{dt}f \]
	fizikalne količine $f$ imenujemo \emph{materialni odvod}.
\end{definicija}
Materialno odvod meri naglost spreminjanja fizikalne količine v dani materialni točki tekom
časa. Od tod tudi tako poimenovanje. V prostorskem opisu fizikalne količine $f$ je materialni
odvod kar parcialni odvod po času
\[ \dot{\hat{f}}=\frac{\partial}{\partial t}\hat{f}(\vek{X},t). \]
Ker v prostorskem opisu določeno točko prostora tekom časa zavzemajo različne
materialne točke, dobimo materialni odvod v prostorskem opisu s totalnim odvodom po času,
\[
	\dot{\bar{f}} = \frac{\partial}{\partial t}\bar{f}(\vek{x},t) +
	\grad \bar{f}(\vek{x},t)\,[\bar{\vek{v}}(\vek{x},t)].
\]
Pri izpeljavi te enakosti se uporabi še pravilo za posredno odvajanje,
definicijo hitrosti (\ref{e:v}) ter zvezo (\ref{e:mpo-zveza}).
\begin{primer}
	Hitrost in pospešek sta prvi in drugi materialni odvod krajevnega vektorja,
	\[ \vek{v}=\dot{\vek{x}},\qquad \vek{a}=\ddot{\vek{x}}. \]
	Pospešek je materialni odvod hitrosti in se v prostorskem opisu izraža kot
	\[ \vek{a}=\dot{\vek{v}}=\frac{\partial}{\partial t}\vek{v}+L\vek{v}, \]
	kjer je $L=\grad\vek{v}$ t.~i.~\emph{hitrostni gradient}. Iz $\Grad\vek{v}=\Grad{\dot{\vek{x}}}=\dot{F}$ in
	zveze (\ref{e:gz}) dobimo
	\begin{equation} \label{e:L} L=\grad\vek{v}=\dot{F}F^{-1}. \end{equation}
\end{primer}


\subsection{Površinski in prostorninski element}


Gladka krivulja znotraj telesa je podana kot slika gladke preslikave
\[\vek{C}\colon \Theta\to\Br, \quad \vek{C}\colon\alpha\mapsto\vek{C}(\alpha),\]
kjer je $\Theta$ odprti interval v $\R$.
V trenutni konfiguraciji telesa se ta krivulja nahaja na lokaciji, ki je določena s sliko pripadajoče preslikave
\[ \vek{c}\colon \Theta\to\B_t,\quad \vek{c}\colon\alpha\mapsto\chi\big(\vek{C}(\alpha),t\big). \]
S posrednim odvajanjem dobimo zvezo
\begin{equation} \label{e:3101}
	\vek{c}'(\alpha) = \big( \Grad\chi\big(\vek{C}(\alpha),t\big)\big)\vek{C}'(\alpha).
\end{equation}
\begin{definicija}
	Naj bosta $\vek{C}$ in $\vek{c}$ definirana kot v prejšnjem odstavku.
	Pri danem $\vek{X}=\vek{C}(\alpha_0)$ in $\vek{x}=\vek{c}(\alpha_0)$, $\alpha_0\in\Theta$, imenujemo
	infinitezimalna tangentna vektorja
	\[ d\vek{X}=\vek{C}'(\alpha_0)d\alpha \quad \mathrm{in} \quad d\vek{x}=\vek{c}'(\alpha_0)d\alpha \]
	\emph{materialni dolžinski element} v referenčni oziroma trenutni konfiguraciji.
\end{definicija}

\begin{definicija}
	Če sta $d\vek{X}_1$, $d\vek{X}_2$ in $d\vek{x}_1$, $d\vek{x}_2$ materialna dolžinska elementa,
	potem infinitezimalna vektorja
	\[ d\vek{S}= d\vek{X}_1\times d\vek{X}_2\quad\mathrm{in}\quad d\vek{s}= d\vek{x}_1\times d\vek{x}_2 \]
	imenujemo \emph{materialni površinski element} v referenčni oz. trenutni konfiguraciji.
	Za materialne dolžinske elemente $d\vek{X}_1$, $d\vek{X}_2$, $d\vek{X}_3$
	in $d\vek{x}_1$, $d\vek{x}_2$, $d\vek{x}_3$ se infinitezimalni števili
	\[ dV= d\vek{X}_1\times d\vek{X}_2\cdot d\vek{X}_3\quad\mathrm{in}\quad dv= d\vek{x}_1\times d\vek{x}_2\cdot d\vek{x}_3 \]
	imenujeta \emph{materialni prostorninski element} v referenčni oziroma trenutni konfiguraciji.
\end{definicija}
\begin{trditev}
	Za pare materialnih dolžinskih, površinskih in volumskih elementov $(d\vek{X},d\vek{x})$, $(d\vek{S},d\vek{s})$ in $(dV,dv)$ velja
	\begin{equation}\label{e:dxdX}
		d\vek{x}=\ten{F}d\vek{X},\quad d\vek{s}=J\ten{F}^{-T}d\vek{S}\quad\mathrm{in}\quad
		dv=JdV,
	\end{equation}
	kjer je $\ten{F}$ deformacijski gradient (\ref{e:F}) z determinanto $J>0$.
\end{trditev}
\proof
	Prva enakost sledi neposredno iz (\ref{e:3101}). Tretja enakost sledi iz lastnosti determinante,
	\begin{align*}
		dv&=d\vek{x}_1\times d\vek{x}_2\cdot d\vek{x}_3=
		\ten{F}d\vek{X}_1\times \ten{F}d\vek{X}_2\cdot \ten{F}d\vek{X}_3=\\
		&=Jd\vek{X}_1\times d\vek{X}_2\cdot d\vek{X}_3 = JdV.
	\end{align*}
	Za poljuben vektor $\vek{u}$ je
	\begin{align*}
		d\vek{s}\cdot\vek{u}&=\ten{F}d\vek{X}_1\times\ten{F}d\vek{X}_2\cdot\vek{u}=
		\ten{F}d\vek{X}_1\times\ten{F}d\vek{X}_2\cdot\ten{F}(\ten{F}^{-1}\vek{u})=\\
		&=J d\vek{X}_1\times d\vek{X}_2 \cdot \ten{F}^{-1}\vek{u} =J d\vek{S}\cdot\ten{F}^{-1}\vek{u}=\\
		&=J\ten{F}^{-T}d\vek{S}\cdot\vek{u},
	\end{align*}
	iz česar sledi druga enakost. 
\endproof

Površinski element se običajno piše kot
\[ d\vek{S}=\vek{N}dS \quad\mathrm{oziroma}\quad d\vek{s}=\vek{n}ds, \]
kjer sta
\[
	\vek{N}=\frac{d\vek{X}_1\times d\vek{X}_2}{\|d\vek{X}_1\times d\vek{X}_2\|} \quad\mathrm{in}\quad
	\vek{n}=\frac{d\vek{x}_1\times d\vek{x}_2}{\|d\vek{x}_1\times d\vek{x}_2\|}
\]
\emph{enotski normali} ter
\[ dS=\|d\vek{X}_1\times d\vek{X}_2\| \quad\mathrm{in}\quad ds=\|d\vek{x}_1\times d\vek{x}_2\| \]
\emph{ploščinska elementa}.


\subsection{Transportni izrek}


V tem razdelku bomo podali transportni izrek, ki je v bistvu posplošitev izreka o odvajanju
integrala s parametrom, ki ga poznamo iz matematične analize. S transportnim izrekom
bomo dobili enačbo za časovni odvod integrala po območju trenutne konfiguracije materialnega telesa.
Še prej pa potrebujemo formulo za materialni odvod determinante deformacijskega gradienta.

Najprej poiščimo odvod za determinanto $\det\colon\L(V)\to\R$, kjer naj bo tokrat $V$
realni $n$-razsežni vektorski prostor. Naj bo $\omega\colon V^n\to\R$ netrivialna
alternirajoča $n$-linearna forma. Spomnimo, za determinanto in sled velja
\begin{align*}
	&\omega(A\vek{u}_1,\dots,A\vek{u}_n)=(\det A)\,\omega(\vek{u}_1,\dots,\vek{u}_n), \\
	&\sum_{j=1}^n\omega(\vek{u}_1,\dots,A\vek{u}_j,\dots,\vek{u}_n)=(\tr A)\,\omega(\vek{u}_1,\dots,\vek{u}_n).
\end{align*}
Po definiciji krepkega odvoda na Banachovih prostorih velja za odvod determinante
\[ (\partial_{A}\det A)[S] = \det(A+S) - \det A + o(S). \]
%kjer je $o(S)$ količina, za katero velja \[ \lim_{\|S\|\to 0}\frac{|o(S)|}{\|S\|}=0. \]
Torej imamo
\begin{multline*}
	(\partial_{A}\det A)[S]\,\omega(\vek{u}_1,\dots,\vek{u}_n)=\\
	\omega\big((A+S)\vek{u}_1,\dots,(A+S)\vek{u}_n\big) -
	\omega(A\vek{u}_1,\dots,A\vek{u}_n) + o(S).
\end{multline*}
Nadalje uporabimo linearnost forme $\omega$, člene, v katerih nastopa $S$ vsaj dvakrat,
pa vključimo v izraz $o(S)$, in tako iz desne strani zadnje enakosti dobimo
\begin{align*}
	&= \sum_{j=1}^n\omega(A\vek{u}_1,\dots,S\vek{u}_j,\dots,A\vek{u}_n) + o(S) = \\
	&= \sum_{j=1}^n\omega(A\vek{u}_1,\dots,AA^{-1}S\vek{u}_j,\dots,A\vek{u}_n) + o(S) = \\
	&= (\det A)\sum_{j=1}^n\omega(\vek{u}_1,\dots,A^{-1}S\vek{u}_j,\dots,\vek{u}_n) + o(S) = \\
	&= (\det A\big)\tr(A^{-1}S)\,\omega(\vek{u}_1,\dots,\vek{u}_n) + o(S).
\end{align*}
V limiti, ko gre $\|S\|$ proti 0, dobimo
\[
	(\partial_{A}\det A)[S] = (\det A)\tr(A^{-1}S) = (\det A)A^{-T}\cdot S=(\det A)A^{-T}[S],
\]
iz česar sledi formula za odvod determinante
\[ \partial_{A}\det A=(\det A)A^{-T}. \]
Z uporabo verižnega pravila za odvajanje, pravkar izpeljane enakosti za odvod determinante
ter enačbe (\ref{e:L}) dobimo pravilo za materialni odvod determinante deformacijskega gradienta:
\begin{align}
	\dot{J}&=(\det\ten{F})\dot{}=(\partial_{\ten{F}}\det\ten{F})\big[\dot{\ten{F}}\big]=J\ten{F}^{-T}\cdot\dot{\ten{F}}=
	\nonumber \\ &= J\tr(\ten{F}^{-1}\dot{\ten{F}}) = J\tr(\grad\vek{v})=J\div\vek{v}. \label{e:dotJ}
\end{align}

\begin{izrek}[Transportni izrek]
	Naj bo $\mathcal{P}\subseteq\B$ del materialnega telesa v referenčni konfiguraciji in naj
	$\mathcal{P}_t=\chi(\mathcal{P},t)\subseteq\B$ označuje njegovo trenutno konfiguracijo ob času $t$.
	Naj bosta $\hat{f}\colon\mathcal{P}\to W$ in $\bar{f}\colon\mathcal{P}_t\to W$
	materialni oziroma prostorski opis neke fizikalne količine, ki sta razreda $C^1$. Potem velja
	\begin{equation*}
		\frac{d}{dt}\int_{\mathcal{P}_t}\bar{f}\,dv =
		\int_{\mathcal{P}_t}(\dot{\bar{f}}+\bar{f}\div\vek{v})\,dv.
	\end{equation*}
	Nadalje, če je območje $\mathcal{P}_t$ omejeno ter je njegov rob sestavljen iz končnega števila
	gladkih ploskev in če sta $\hat{f}$ in $\bar{f}$ skalarno ali pa vektorsko polje
	razreda $C^1$ na zaprtju $\overline{\mathcal{P}}$ oz.~$\overline{\mathcal{P}_t}$, potem velja
	\begin{equation*}
		\frac{d}{dt}\int_{\mathcal{P}_t}\bar{f}\,dv =
		\int_{\mathcal{P}_t}\frac{\partial\bar{f}}{\partial t}\,dv +
		\int_{\partial\mathcal{P}_t}(\vek{v}\cdot\vek{n})\bar{f}\,ds.
	\end{equation*}
\end{izrek}
\proof
	Z uporabo (\ref{e:dxdX})${}_3$ in (\ref{e:dotJ}) pridemo do
	\begin{align*}
		\frac{d}{dt}\int_{\mathcal{P}_t}\bar{f}\,dv &= \frac{d}{dt}\int_{\mathcal{P}}\hat{f}J\,dV =
		\int_{\mathcal{P}}\frac{d}{dt}(\hat{f}J)\,dV = \int_{\mathcal{P}}(\dot{\hat{f}}J+\hat{f}\dot{J})\,dV =\\
		&=\int_{\mathcal{P}}(\dot{\hat{f}}+\hat{f}\div\vek{v})J\,dV = \int_{\mathcal{P}_t}(\dot{\bar{f}}+\bar{f}\div\vek{v})\,dv.
	\end{align*}
	Pri tem smo na drugem koraku smeli zamenjati vrstni red odvajanja in integriranja, ker se referenčna konfiguracija
	s časom ne spreminja. Formalno to utemeljimo tako, da odvod zapišemo kot limito diferenčnega kvocienta,
	dejstvo, da se v obeh integralih v diferenčnem kvocientu integrira po enakem območju, izkoristimo za uporabo
	linearnosti integrala in tako celoten diferenčni kvocient spravimo pod en sam integral, nazadnje pa še zamenjamo
	limito in integral.
\endproof


\subsection{Zakon o ohranitvi mase}


Če je $\mathcal{P}\subseteq\B$ poljuben del materialnega telesa, ki v trenutni konfiguraciji pri času $t$
zavzema območje $\mathcal{P}_t=\chi(\mathcal{P},t)\subseteq\B_t$, potem je \emph{masa} tega déla telesa
ob času $t$ podana z integralom
\[ m(\mathcal{P},t)=\int_{\mathcal{P}_t}\rho\,dv, \]
kjer je
\[ \rho(\cdot,t)\colon\B_t\to(0,\infty) \]
pozitivno integrabilno skalarno polje, imenovano
\emph{masna gostota trenutne konfiguracije}. V klasični mehaniki se prepostavi naslednji zakon.
\begin{aksiom}[Zakon o ohranitvi mase]
	Masa katerega koli déla telesa $\mathcal{P}\subseteq\B$ se z gibanjem telesa ne spreminja, t.j.
	\[ \frac{d}{dt}m(\mathcal{P},t)=\frac{d}{dt}\int_{\mathcal{P}_t}\rho\,dv=0. \]
\end{aksiom}
V skladu z zakonom o ohranitvi mase pripada materialnemu telesu \emph{masna gostota referenčne konfiguracije}
\[ \rho_{\r}\colon\B\to(0,\infty), \]
tako da velja
\[ m(\mathcal{P},t)\equiv m(\mathcal{P})=\int_{\mathcal{P}}\rho_{\r}\,dV=\int_{\mathcal{P}_t}\rho\,dv. \]
Ker to velja za kateri koli del telesa $\mathcal{P}\subseteq\B$, lahko iz lastnosti (\ref{e:dxdX})${}_3$ in trditve \ref{t:oiz}
za masni gostoti $\rho_{\r}$ in $\rho$, če sta zvezni, dobimo zvezo
\[ \rho_{\r}(\vek{X})=J(\vek{X},t)\hat{\rho}(\vek{X},t). \]






