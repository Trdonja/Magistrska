\appendix

\chapter{Konstitucijske enačbe za izotropno elastično trdno telo}


\section{Sprememba opazovališča in objektivnost}


V razdelku \ref{s:evklipro} smo podali definicijo opazovališča za Evklidski
prostor $\E$. Nato smo v poglavju \ref{chp:kinkon} omenili, da so določene
fizikalne količine lahko odvisne od izbire opazovališča. V tem razdelku bomo
na kratko analizirali spremembo opazovališča.

\emph{Sprememba opazovališča} za $\E$ je prehod iz opazovališča $\iota_o$ v opazovališče
$\iota^*_{o^*}$. Pri tem točka $x\in\E$, ki ima v opazovališču $\iota_o$ krajevni vektor
$\vek{x}=\iota_o(x)$, dobi nov krajevni vektor $\vek{x}^*=\iota^*_{o^*}(x)$ glede na opazovališče
$\iota^*_{o^*}$. Vsaka sprememba opazovališča porodi transformacijo
\[
	\iota^*_{o^*}\circ\iota_o^{-1}\colon\V\to\V,\qquad
	\vek{x}\mapsto\vek{x}^*=\iota^*_{o^*}(\iota_o^{-1}(\vek{x})).
\]

Razdalja oz.~metrika na $\E$ je odvisna od izbire opazovališča. \emph{Izmerjena razdalja} med točkama
$x_1,x_2\in\E$ v opazovališču $\iota_o$ je
$d(x_1,x_2)=\|\iota_o(x_2)-\iota_o(x_1)\|=\|\iota(x_1,x_2)\|$,
kjer je $\|\cdot\|$ norma prostora $\V$. Nas bodo zanimale take
spremembe opazovališča, ki ohranjajo izmerjene razdalje med točkami, kar pomeni,
da mora za vsaki točki $x_1,x_2\in\E$ veljati
$\|\iota^*(x_1,x_2)\|=\|\iota(x_1,x_2)\|$.
Iz linearne algebre vemo, da so izometrije na prostoru $\V$ natanko ortogonalne preslikave,
torej mora veljati
\begin{equation} \label{e:vektrans}
	\iota^*(x_1,x_2)=\ten{Q}\iota(x_1,x_2)\quad\forall x_1,x_2\in\E,
\end{equation}
kjer je
\[
	\ten{Q}\in\mathscr{O}(\V)=\{\ten{R}\in\L(\V)\;;\ \ten{R}\ten{R}^T=\ten{R}^T\ten{R}=\ten{1}\}
\]
ortogonalna preslikava. Množica $\mathscr{O}(\V)$ vseh ortogonalnih preslikav na prostoru
$\V$ je za operacijo kompozitum grupa, imenovana tudi \emph{ortogonalna grupa}. Če je $x\in\E$ in
je $\vek{x}=\iota_o(x)$ ter $\vek{x}^*=\iota^*_{o^*}(x)$, potem je
\begin{equation} \label{e:toga}
	\vek{x}^*=\iota^*(o^*,x)=\iota^*(o^*,o)+\iota^*(o,x)=\vek{c}+\ten{Q}\vek{x}.
\end{equation}
Pri tem smo označili $\vek{c}=\iota^*(o^*,o)$. Transformaciji oblike (\ref{e:toga})
se reče tudi \emph{toga transformacija}. Odselj bomo izraz \textit{sprememba opazovališča}
uporabljali zgolj za take spremembe opazovališča, ki porodijo togo transformacijo.

Za vsak vektor $\vek{u}\in\V$ po aksiomih Evklidskega točkovnega prostora obstajata
točki $x_1,x_2\in\E$, da je $\vek{u}=\iota(x_1,x_2)$. Vektorju $\vek{u}$ pripada
vektor $\vek{u}^*=\iota^*(x_1,x_2)$, za katerega po (\ref{e:vektrans}) velja
\begin{equation} \label{e:travek}
	\vek{u}^*=\ten{Q}\vek{u}.
\end{equation}
Linearni preslikavi $\ten{A}\in\L(\V)$ z razvojem $\ten{A}=A_{ij}\,\vek{e}_i\otimes\vek{e}_j$
pripada linearna preslikava
\begin{align}
	\ten{A}^*&=A_{ij}\,\vek{e}^*_i\otimes\vek{e}^*_j=A_{ij}\,\ten{Q}\vek{e}_i\otimes\ten{Q}\vek{e}_j
	=\ten{Q}A_{ij}\,\vek{e}_i\otimes\vek{e}_j\ten{Q}^T \nonumber \\
	&=\ten{Q}\ten{A}\ten{Q}^T. \label{e:traten}
\end{align}
Za vektorsko količino $\vek{u}$ oz.~tenzorsko količino $\ten{A}$
rečemo, da je \emph{objektivna} ali pa \emph{neodvisna od opazovališča},
če se z vsako spremembo opazovališča iz $\iota_o$ v $\iota^*_{o^*}$
transformira po pravilu (\ref{e:travek}) oz.~(\ref{e:traten}).
Skalarna količina je \emph{objektivna} ali \emph{neodvisna od opazovališča},
če ob vsaki spremembi opazovališča ostane nespremenjena.

Za prostor $\E_R$, ki služi zgolj določitvi materialnega telesa, naj velja,
da ima vseskozi enako opazovališče. Spremembe opazovališča pridejo v upoštev zgolj
za prostor $\E$, kjer potekajo fizikalni dogodki. Naj bo $\chi\colon\B\times I\to\E$
gibanje s pripadajočima vektorskima poljema $\vek{\chi}=\iota_o\circ\chi$ in
$\vek{\chi}^*=\iota^*_{o^*}\circ\chi$. Po (\ref{e:toga}) velja
\[
	\vek{\chi}^*(X,t)=\ten{Q}\vek{\chi}(X,t)+\vek{c}.
\]
Naj bosta $\ten{F}=\Grad\vek{\chi}$ in $\ten{F}^*=\Grad\vek{\chi}^*$ pripadajoča
deformacijska gradienta gibanja. Iz zgornje relacije dobimo
\[
	\ten{F}^*=\ten{Q}\ten{F}.
\]
Deformacijski gradient torej ni objektiven oz.~neodvisen od opazovališča, saj se ne
transformira po pravilu (\ref{e:traten}). Jacobijan je objektivna skalarna
količina, saj velja
\[
	J^*=|\det(\ten{F}^*)|=|\det(\ten{Q}\ten{F})|=|\det(\ten{Q})||\det(\ten{F})|
	=1\cdot|\det(\ten{F})|=J.
\]
Posledično je tudi masna gostota trenutne konfiguracije objektivna skalarna količina,
\[ \rho^*=\frac{\rho_R}{J^*}=\frac{\rho_R}{J}=\rho. \]

Za gostoto notranje energije $\sigma(\ten{F})$, ki smo jo predstavili v primeru
elastičnega trdnega telesa, zahtevamo, da je kot skalarna količina neodvisna
od opazovališča, torej velja
\begin{equation} \label{e:objesig}
	\sigma(\ten{F}^*)=\sigma(\ten{Q}\ten{F})=\sigma(\ten{F})
\end{equation}
za vse $(X,t)\in\B\times I$.
Dokažimo, da je tedaj Cauchyev napetostni tenzor (\ref{e:napetostni})
\[
	\ten{T}=\mathcal{T}(\ten{F})=\rho D\sigma(\ten{F})\ten{F}^T
	=\frac{\rho_R}{|\det\ten{F}|} D\sigma(\ten{F})\ten{F}^T
\]
tudi objektiven. Če izraz $\sigma(\ten{Q}\ten{F})$ posredno odvajamo po $\ten{F}$,
dobimo za poljuben $\ten{H}\in\L(\V)$
\[
	D(\sigma\circ(\ten{X}\mapsto\ten{Q}\ten{X}))(\ten{F})(\ten{H})=
	D\sigma(\ten{Q}\ten{F})\cdot(\ten{Q}\ten{H})=\ten{Q}^T D\sigma(\ten{Q}\ten{F})\cdot\ten{H}.
\]
Z odvajanjem enakosti (\ref{e:objesig}) po $\ten{F}$ torej pridemo do
\[
	\ten{Q}^T D\sigma(\ten{Q}\ten{F}) = D\sigma(\ten{F})\qquad\textrm{oziroma}
	\qquad D\sigma(\ten{Q}\ten{F}) = \ten{Q}D\sigma(\ten{F}).
\]
Z upoštevanjem pravkar dobljene enakosti in objektivnosti masne gostote dobimo
\begin{align*}
	\ten{T}^*&=\rho^* D\sigma(\ten{F}^*)\ten{F}^{*T}
	=\rho D\sigma(\ten{Q}\ten{F})\ten{F}^{T}\ten{Q}^{T}
	=\rho \ten{Q} D\sigma(\ten{F})\ten{F}^{T}\ten{Q}^{T} \nonumber \\
	&=\ten{Q}\ten{T}\ten{Q}^{T}
\end{align*}
oziroma
\[
	\mathcal{T}(\ten{Q}\ten{F})=\ten{Q}\mathcal{T}(\ten{F})\ten{Q}^{T}\qquad\forall\;\ten{Q}\in\mathscr{O}(\V),
\]
kar dokazuje, da je $\ten{T}$ neodvisen od opazovališča.


\subsection{Materialne simetrije}


\emph{Sprememba referenčne konfiguracije} je bijektivna gladka preslikava $\kappa\colon\B\to\E_R$.
Pri tem je $\B_{\kappa}=\kappa(\B)$ nova referenčna konfiguracija telesa. Gibanje
$\chi\colon\B\times I\to\E$ lahko enakovredno predstavimo s preslikavo
\begin{equation} \label{e:chgrek}
	\chi_{\kappa}\colon\B_{\kappa}\times I\to\E, \qquad
	\chi_{\kappa}=\chi\circ\kappa^{-1}.
\end{equation}
Če označimo $\ten{F}=\Grad\vek{\chi}$, $\ten{F}_{\kappa}=\Grad\vek{\chi}_{\kappa}$
in $\ten{P}=\Grad\vek{\kappa}$, potem z odvajanjem enakosti (\ref{e:chgrek})
dobimo zvezo med gradienti
\[
	\ten{F}_{\kappa}=\ten{F}\ten{P}^{-1}.
\]

Naj bosta $\kappa\colon\B\to\E_R$ in $\hat{\kappa}\colon\B\to\E_R$ referenčni konfiguraciji.
V točki $x=\chi(X,t)\in\B_t$ je vrednost Cauchyjevega napetostnega tenzorja
\begin{equation} \label{e:responf}
	\ten{T}(x,t)=\mathcal{T}_{\kappa}(\ten{F}_{\kappa}(\kappa(X),t))\quad\textrm{oz.}\quad
	\ten{T}(x,t)=\mathcal{T}_{\hat{\kappa}}(\ten{F}_{\hat{\kappa}}(\hat{\kappa}(X),t)).
\end{equation}
Pri tem sta $\ten{F}_{\kappa}=\Grad\vek{\chi}_{\kappa}$ in
$\ten{F}_{\hat{\kappa}}=\Grad\vek{\chi}_{\hat{\kappa}}$ (glej zvezo (\ref{e:chgrek})).
V splošnem sta \emph{odzivni funkciji} $\mathcal{T}_{\kappa}$ in $\mathcal{T}_{\hat{\kappa}}$
različni. Iz diagrama REF vidimo, da velja $\chi_{\kappa}=\chi_{\hat{\kappa}}\circ\hat{\kappa}\circ\kappa^{-1}$.
Če označimo s
\begin{equation} \label{e:pupa}
	\ten{G}=\Grad(\hat{\vek{\kappa}}\circ\vek{\kappa}^{-1}),
\end{equation}
potem je $\ten{F}_{\kappa}=\ten{F}_{\hat{\kappa}}\ten{G}$. Po primerjanju relacij (\ref{e:responf})
sedaj vidimo, da med odzivnima funkcijama velja zveza
\begin{equation} \label{e:dolgnoc}
	\mathcal{T}_{\hat{\kappa}}(\ten{F}_{\hat{\kappa}})
	=\mathcal{T}_{\kappa}(\ten{F}_{\hat{\kappa}}\ten{G}).
\end{equation}

Materialno telo lahko poseduje določene simetrije, zaradi katerih ni mogo\-če razločiti
med določenimi referenčnimi konfiguracijami. Na primer, materialno telo iz
kubične kristalne strukture je nemogoče razločiti pred in po rotaciji za
$90^{\circ}$ okoli ene od kristalnih osi.

\begin{definicija}
	Referenčni konfiguraciji $\kappa\colon\B\to\E_R$ in $\hat{\kappa}\colon\B\to\E_R$
	sta \emph{materialno nerazločljivi}, če velja
	\begin{equation} \label{e:grarefc}
		\mathcal{T}_{\hat{\kappa}}(\cdot)=\mathcal{T}_{\kappa}(\cdot).
	\end{equation}
\end{definicija}

Če uporabimo oznako (\ref{e:pupa}) in upoštevamo enakosti (\ref{e:dolgnoc}),
potem vidimo, da je pogoj (\ref{e:grarefc}) ekvivalenten pogoju
\begin{equation} \label{e:pogir}
	\mathcal{T}_{\kappa}(\ten{F})=\mathcal{T}_{\kappa}(\ten{F}\ten{G})\qquad
	\forall\;\ten{F}.
\end{equation}

\begin{definicija}
	Če za linearno preslikavo $\ten{G}$ velja pogoj (\ref{e:pogir}), potem ji rečemo
	\emph{materialna simetrijska transformacija} gle\-de na konfiguracijo $\kappa$.
\end{definicija}

Materialna nerazločljivost v fizikalnem smislu pomeni, da se odziv materiala
glede na konfiguracijo $\kappa$ ne da razločiti od odziva glede na konfiguracijo
$\hat{\kappa}$ z nobenim mogočim eksperimentom.

Predpostavili bomo, da vse materialne simetrijske transformacije ohranjajo prostornino telesa,
saj bi se sicer materialno telo lahko poljubno razši\-rilo, ne da bi se to poznalo na odzivu,
kar pa je fizikalno nesprejemljivo. Zato je potrebna predpostavka, da je
\[
	\ten{G}\in\mathscr{U}(\V)=\{\ten{A}\in\L(\V)\;;\ |\det\ten{A}|=1 \}.
\]
Množica $\mathscr{U}(\V)$ je za operacijo kompozitum grupa, imenovana \emph{unimodularna grupa}.
Grupa $\mathscr{O}(\V)$ je podgrupa grupe $\mathscr{U}(\V)$.

\begin{izrek}
	Množica $\mathcal{G}_{\kappa}$ vseh materialnih simetrijskih transformacij glede
	na konfiguracijo $\kappa$ je podgrupa unimodularne grupe.
\end{izrek}

\proof
	Dovolj je dokazati, da je $\mathcal{G}_{\kappa}$ grupa. Za poljubni $\ten{G}_1,\ten{G}_2\in\mathcal{G}_{\kappa}$
	z upoštevanjem relacije (\ref{e:pogir}) velja
	\[
		\mathcal{T}_{\kappa}(\ten{F})=\mathcal{T}_{\kappa}(\ten{F}\ten{G}_1)
		=\mathcal{T}_{\kappa}((\ten{F}\ten{G}_1)\ten{G}_2)=\mathcal{T}_{\kappa}(\ten{F}(\ten{G}_1\ten{G}_2)),
	\]
	torej je $\ten{G}_1\ten{G}_2\in\mathcal{G}_{\kappa}$. Očitno je tudi $\ten{1}\in\mathcal{G}_{\kappa}$.
	Naj bo $\ten{G}\in\mathcal{G}_{\kappa}$. Ker je $|\det\ten{G}|=1$, obstaja $\ten{G}^{-1}$ in je
	\[
		\mathcal{T}_{\kappa}(\ten{F}\ten{G}^{-1})=\mathcal{T}_{\kappa}((\ten{F}\ten{G}^{-1})\ten{G})
		=\mathcal{T}_{\kappa}(\ten{F}),
	\]
	kar dokazuje, da je tudi $\ten{G}^{-1}\in\mathcal{G}_{\kappa}$. Torej je $\mathcal{G}_{\kappa}$ grupa.
\endproof

$\mathcal{G}_{\kappa}$ se imenuje \emph{materialna simetrijska grupa} za $\mathcal{T}$ glede
na referenčno konfiguracijo $\kappa$. Očitno je, da je $\mathcal{G}_{\kappa}$ odvisna od
referenčne konfiguracije $\kappa$. Če je $\hat{\kappa}$ še ena referenčna konfiguracija
s pripadajočo materialno simetrijsko grupo $\mathcal{G}_{\hat{\kappa}}$
in je $\ten{G}$ kot v (\ref{e:pupa}), potem za poljuben
$\ten{H}\in\mathcal{G}_{\kappa}$ dobimo iz (\ref{e:dolgnoc})
\begin{align*}
	\mathcal{T}_{\hat{\kappa}}(\ten{F})&=\mathcal{T}_{\kappa}(\ten{F}\ten{G})
	=\mathcal{T}_{\kappa}((\ten{F}\ten{G})\ten{H})
	=\mathcal{T}_{\kappa}(\ten{F}(\ten{G}\ten{H}\ten{G}^{-1})\ten{G}) \\
	&=\mathcal{T}_{\hat{\kappa}}(\ten{F}(\ten{G}\ten{H}\ten{G}^{-1})),
\end{align*}
iz česar sledi, da je $\ten{G}\ten{H}\ten{G}^{-1}\in\mathcal{G}_{\hat{\kappa}}$.
S tem smo dokazali naslednji izrek.

\begin{izrek}[Nollovo pravilo]
	Naj bosta $\kappa$ ter $\hat{\kappa}$ konfiguraciji in naj bo
	$\ten{G}=\Grad(\hat{\vek{\kappa}}\circ\vek{\kappa}^{-1})$. Potem
	med materialnima simetrijskima grupama velja zveza
	\[
		\mathcal{G}_{\hat{\kappa}}=\ten{G}\mathcal{G}_{\kappa}\ten{G}^{-1}
		=\{\ten{G}\ten{H}\ten{G}^{-1}\;;\ \ten{H}\in\mathcal{G}_{\kappa}\}.
	\]
\end{izrek}

V zgornjem izreku je $\ten{G}$ gradient spremembe konfiguracije in ni nujno,
da pripada unimodularni grupi.

\begin{definicija}
	\begin{enumerate}
		\item Material je \emph{trdno telo}, če obstaja referenčna konfiguracija $\kappa$, da je
		$\mathcal{G}_{\kappa}$ podgrupa ortogonalne grupe $\mathscr{O}(\V)$.
		Taka konfiguracija se imenuje \emph{neizkrivljena}.
		\item Material je \emph{fluid}, če je za neko referenčno konfiguracijo $\kappa$
		pripadajoča simetrijska grupa kar cela unimodularna grupa,
		t.j.~$\mathcal{G}_{\kappa}=\mathscr{U}(\V)$.
		\item Če material ni niti trdno telo niti fluid, potem je \emph{fluidni kristal}.
	\end{enumerate}
\end{definicija}

Po Nollovem pravilu je za fluide $\mathcal{G}_{\kappa}=\mathcal{G}_{\hat{\kappa}}$
za poljubni referenčni konfiguraciji $\kappa$ in $\hat{\kappa}$. Za fluid so torej
vse referenčne konfiguracije enakovredne.

\begin{definicija}
	Materialno telo je \emph{izotropno}, če obstaja referenčna konfiguracija $\kappa$,
	da je $\mathscr{O}(\V)\subseteq\mathcal{G}_{\kappa}$. Taka konfiguracija
	se imenuje \emph{izotropna}.
\end{definicija}

Grupa $\mathscr{O}(\V)$ je maksimalna podgrupa v grupi $\mathscr{U}(\V)$, kar pomeni,
da če je $\mathcal{G}$ taka grupa, da velja $\mathscr{O}(\V)\subseteq\mathcal{G}\subseteq\mathscr{U}(\V)$,
potem je bodisi $\mathcal{G}=\mathscr{O}(\V)$ bodisi $\mathcal{G}=\mathscr{U}(\V)$.\footnote{
Izrek je brez dokaza naveden v \cite[str.~88]{liu}.}
Izotropno materialno telo je torej bodisi fluid, $\mathcal{G}_{\kappa}=\mathscr{U}(\V)$,
bodisi izotropno trdno telo, $\mathcal{G}_{\kappa}=\mathscr{O}(\V)$.

\begin{izrek}[Polarni razcep]
	Za vsako obrnljivo linearno preslikavo $\ten{F}\in\L(\V)$, $\det\ten{F}\neq 0$,
	obstajata simetrični pozitivno definitni linearni preslikavi $\ten{V}$ in $\ten{U}$
	ter ortogonalna linearna preslikava $\ten{R}$, tako da velja
	\[
		\ten{F}=\ten{R}\ten{U}=\ten{V}\ten{R}.
	\]
	Pri tem so $\ten{U}$, $\ten{V}$ in $\ten{R}$ z zgornjim razcepom enolično določene.
\end{izrek}

\proof
	Takoj se vidi, da sta preslikavi $\ten{F}\ten{F}^T$ in $\ten{F}^T\ten{F}$
	simetrični, hkrati pa tudi pozitivno definitni, saj za poljuben $\vek{v}\in\V$,
	$\vek{v}\neq\vek{0}$, velja
	\[ \vek{v}\cdot\ten{F}^T\ten{F}\vek{v}=\ten{F}\vek{v}\cdot\ten{F}\vek{v}>0, \]
	saj je $\ten{F}$ nesingularna.
	
	Iz spektralnega izreka iz linearne algebre vemo, da se da vsako simetrično
	linearno preslikavo $\ten{S}$ zapisati kot
	\[ \ten{S}=\sum_{j=1}^3 a_j\vek{u}_j\otimes\vek{u}_j, \]
	kjer so $a_j$ lastne vrednosti, $\vek{u}_j$ pa pripadajoči lastni vektorji.
	Pri tem je mogoče lastne vektorje izberati tako, da so medsebojno ortogonalni.
	Vse lastne vrednosti simetrične preslikave so vedno realne. Za vsako
	simetrično linearno preslikavo $\ten{S}$ obstaja kvadratni koren, t.j.~enolično določena linearna
	preslikava $\ten{G}$, da je $\ten{G}^2=\ten{S}$. Velja
	\[ \ten{G}=\sqrt{\ten{S}}=\sum_{j=1}^3 \sqrt{a_j}\,\vek{u}_j\otimes\vek{u}_j. \]
	
	Definirajmo
	\[
		\ten{U}=\sqrt{\ten{F}^T\ten{F}},\quad\ten{R}=\ten{F}\ten{U}^{-1},\quad
		\ten{V}=\ten{R}\ten{U}\ten{R}^T.
	\]
	Po definiciji je $\ten{U}$ simetrična pozitivno definitna in $\ten{R}$ je ortogonalna, saj je
	\begin{align*}
		\ten{R}\ten{R}^T&=\ten{F}\ten{U}^{-1}(\ten{F}\ten{U}^{-1})^T
		=\ten{F}\ten{U}^{-1}\ten{U}^{-T}\ten{F}^T \\ &=\ten{F}\ten{U}^{-2}\ten{F}^T
		=\ten{F}(\ten{F}^T\ten{F})^{-1}\ten{F}^T=\ten{1}.
	\end{align*}
	Poleg tega je
	\[
		\ten{V}^2=(\ten{R}\ten{U}\ten{R}^T)(\ten{R}\ten{U}\ten{R}^T)
		=(\ten{R}\ten{U})(\ten{R}\ten{U})^T=\ten{F}\ten{F}^T,
	\]
	torej je $\ten{V}$ kvadratni koren od $\ten{F}\ten{F}^T$ in je zato
	simetrična in pozitivno definitna. Enoličnost sledi iz enoličnosti obstoja
	kvadratnega korena.
\endproof