\appendix

\chapter{Konstitucijske enačbe za izotropno hiperelastično trdno telo}


Namen tega dodatka je izpeljava konstitucijske enačbe za Cauchyjev napetostni tenzor,
ki se pojavi v primeru hiperelastičnega trdnega telesa.


\section{Sprememba opazovališča in objektivnost}


V razdelku \ref{s:evklipro} smo podali definicijo opazovališča za Evklidski
prostor $\E$. Določene fizikalne količine, definirane v poglavju \ref{chp:kinkon},
so odvisne od izbire opazovališča. V tem razdelku bomo na kratko analizirali spremembo opazovališča
in kako to vpliva na spremembo nekaterih fizikalnih količin.

Da se bomo izognili morebitnim dvomom, omenimo, da bodo imele
v tem poglavju oznake z $*$ drug pomen, kot pa ga imajo v poglavju \ref{pog:ham}.

\emph{Sprememba opazovališča} za $\E$ je prehod iz opazovališča $\iota_o$ v opazovališče
$\iota^*_{o^*}$. Pri tem točka $x\in\E$, ki ima v opazovališču $\iota_o$ krajevni vektor
$\vek{x}=\iota_o(x)$, dobi nov krajevni vektor $\vek{x}^*=\iota^*_{o^*}(x)$ glede na opazovališče
$\iota^*_{o^*}$. Vsaka sprememba opazovališča porodi transformacijo
\[
	\iota^*_{o^*}\circ\iota_o^{-1}\colon\V\to\V,\qquad
	\vek{x}\mapsto\vek{x}^*=\iota^*_{o^*}(\iota_o^{-1}(\vek{x})).
\]

Razdalja oz.~metrika na $\E$ je odvisna od izbire opazovališča. \emph{Izmerjena razdalja} med točkama
$x_1,x_2\in\E$ v opazovališču $\iota_o$ je
\[ d(x_1,x_2)=\|\iota_o(x_2)-\iota_o(x_1)\|=\|\iota(x_1,x_2)\|, \]
kjer je $\|\cdot\|$ norma prostora $\V$. Nas bodo zanimale take
spremembe opazovališča, ki ohranjajo izmerjene razdalje med točkami, kar pomeni,
da mora za vsaki točki $x_1,x_2\in\E$ veljati
$\|\iota^*(x_1,x_2)\|=\|\iota(x_1,x_2)\|$.
Iz linearne algebre vemo, da so izometrije na prostoru $\V$ natanko ortogonalne preslikave,
torej mora veljati
\begin{equation} \label{e:vektrans}
	\iota^*(x_1,x_2)=\ten{Q}\iota(x_1,x_2)\quad\forall\, x_1,x_2\in\E,
\end{equation}
kjer je $\ten{Q}\in\mathscr{O}(\V)$ ortogonalna preslikava. Če je $x\in\E$ in
je $\vek{x}=\iota_o(x)$ ter $\vek{x}^*=\iota^*_{o^*}(x)$, potem je
\begin{equation} \label{e:toga}
	\vek{x}^*=\iota^*(o^*,x)=\iota^*(o^*,o)+\iota^*(o,x)=\vek{c}+\ten{Q}\vek{x}.
\end{equation}
Pri tem smo označili $\vek{c}=\iota^*(o^*,o)$. Transformaciji oblike (\ref{e:toga})
se reče tudi \emph{toga transformacija}. Odselj bomo izraz \textit{sprememba opazovališča}
uporabljali zgolj za take spremembe opazovališča, ki porodijo togo transformacijo.
V splošnem sta lahko $\ten{Q}$ in $\vek{c}$ časovno odvisna in v takem primeru
gre za časovno odvisno spremembo opazovališča, vendar se bomo mi omejili na
take spremembe opazovališč, kjer sta $\ten{Q}$ in $\vek{c}$ konstantna.

Za vsak vektor $\vek{u}\in\V$ po aksiomih Evklidskega točkovnega prostora obstajata
točki $x_1,x_2\in\E$, da je $\vek{u}=\iota(x_1,x_2)$. Vektorju $\vek{u}$ pripada
vektor $\vek{u}^*=\iota^*(x_1,x_2)$, za katerega po (\ref{e:vektrans}) velja
\begin{equation} \label{e:travek}
	\vek{u}^*=\ten{Q}\vek{u}.
\end{equation}
Linearni preslikavi $\ten{A}\in\L(\V)$ z razvojem $\ten{A}=A_{ij}\,\vek{e}_i\otimes\vek{e}_j$
pripada linearna preslikava
\begin{align}
	\ten{A}^*&=A_{ij}\,\vek{e}^*_i\otimes\vek{e}^*_j=A_{ij}\,\ten{Q}\vek{e}_i\otimes\ten{Q}\vek{e}_j
	=\ten{Q}A_{ij}\,\vek{e}_i\otimes\vek{e}_j\ten{Q}^T \nonumber \\
	&=\ten{Q}\ten{A}\ten{Q}^T. \label{e:traten}
\end{align}
Za vektorsko količino $\vek{u}$ oz.~tenzorsko količino $\ten{A}$
rečemo, da je \emph{objektivna} ali pa \emph{neodvisna od opazovališča},
če se z vsako spremembo opazovališča iz $\iota_o$ v $\iota^*_{o^*}$
transformira po pravilu (\ref{e:travek}) oz.~(\ref{e:traten}).
Skalarna količina je \emph{objektivna} ali \emph{neodvisna od opazovališča},
če ob vsaki spremembi opazovališča ostane nespremenjena.

Za prostor $\E_R$, ki služi zgolj določitvi materialnega telesa, naj velja,
da ima vseskozi enako opazovališče. Spremembe opazovališča pridejo v upoštev zgolj
za prostor $\E$, kjer potekajo fizikalni dogodki. Naj bo $\chi\colon\B\times I\to\E$
gibanje s pripadajočima vektorskima poljema $\vek{\chi}=\iota_o\circ\chi$ in
$\vek{\chi}^*=\iota^*_{o^*}\circ\chi$. Po (\ref{e:toga}) velja
\[
	\vek{\chi}^*(X,t)=\ten{Q}\vek{\chi}(X,t)+\vek{c}.
\]
Naj bosta $\ten{F}=\Grad\vek{\chi}$ in $\ten{F}^*=\Grad\vek{\chi}^*$ pripadajoča
deformacijska gradienta gibanja. Iz zgornje relacije dobimo
\[
	\ten{F}^*=\ten{Q}\ten{F}.
\]
Deformacijski gradient torej ni objektiven oz.~neodvisen od opazovališča, saj se ne
transformira po pravilu (\ref{e:traten}). Absolutna vrednost jacobijana je objektivna skalarna
količina, saj velja
\[
	|J^*|=|\det(\ten{F}^*)|=|\det(\ten{Q}\ten{F})|=|\det\ten{Q}||\det\ten{F}|
	=|\det\ten{F}|=|J|.
\]
Masna gostota mora biti pozitivna, saj bi sicer lahko obstajali deli telesa
z negativno ali ničelno maso. Zato v primeru, ko je $J<0$, definiramo masno
gostoto trenutne konfiguracije kot $\rho=\rho_R/|J|$ in je tudi objektivna skalarna količina,
saj velja
\[ \rho^*=\frac{\rho_R}{|J^*|}=\frac{\rho_R}{|J|}=\rho. \]

Za funkcijo shranjene energije $\sigma=\sigma(\ten{F})$, ki smo jo predstavili v primeru
hiperelastičnega trdnega telesa, zahtevamo, da je kot skalarna količina neodvisna
od opazovališča, torej velja
\begin{equation} \label{e:objesig}
	\sigma^*=\sigma(\ten{F}^*)=\sigma(\ten{Q}\ten{F})=\sigma(\ten{F})
\end{equation}
za vse nesingularne $\ten{F}\in\L(\V)$ in vse $\ten{Q}\in\mathscr{O}(\V)$.
Če deformacijski gradient zapišemo v polarnem razcepu $\ten{F}=\ten{R}\ten{U}$ (izrek \ref{i:polraz})
in v (\ref{e:objesig}) za $\ten{Q}$ uporabimo $\ten{R}^T$, dobimo
\begin{equation} \label{e:siuonly}
	\sigma(\ten{F})=\sigma(\ten{U}),
\end{equation}
torej je $\sigma$ odvisna od deformacijskega gradienta $\ten{F}$ le preko desnega
razteznostnega tenzorja $\ten{U}$ in nič od rotacije $\ten{R}$.

Dokažimo še, da je tedaj Cauchyev napetostni tenzor (\ref{e:napetostni})
\begin{equation} \label{e:Tdrugi}
	\ten{T}=\rho D\sigma(\ten{F})\ten{F}^T
	%=\frac{\rho_R}{|\det\ten{F}|} D\sigma(\ten{F})\ten{F}^T
\end{equation}
tudi objektiven. Če izraz $\sigma(\ten{Q}\ten{F})$ odvajamo po $\ten{F}$,
dobimo za poljuben $\ten{H}\in\L(\V)$
\[
	%D(\sigma\circ(\ten{X}\mapsto\ten{Q}\ten{X}))(\ten{F})(\ten{H})=
	%D\sigma(\ten{Q}\ten{F})\cdot\ten{H}
	\frac{d}{d\varepsilon}\at{\sigma(\ten{Q}(\ten{F}+\varepsilon\ten{H}))}{\varepsilon=0}
	=\langle D\sigma(\ten{Q}\ten{F}),\ten{Q}\ten{H}\rangle=\langle\ten{Q}^T D\sigma(\ten{Q}\ten{F}),\ten{H}\rangle.
\]
Z odvajanjem zadnje enakosti iz (\ref{e:objesig}) po $\ten{F}$ torej pridemo do
\[
	\ten{Q}^T D\sigma(\ten{Q}\ten{F}) = D\sigma(\ten{F})\qquad\textrm{oziroma}
	\qquad D\sigma(\ten{Q}\ten{F}) = \ten{Q}D\sigma(\ten{F}).
\]
Z upoštevanjem pravkar dobljene enakosti in objektivnosti masne gostote dobimo
\begin{align*}
	\ten{T}^*&=\rho^* D\sigma(\ten{F}^*)\ten{F}^{*T}
	=\rho D\sigma(\ten{Q}\ten{F})\ten{F}^{T}\ten{Q}^{T}
	=\rho \ten{Q} D\sigma(\ten{F})\ten{F}^{T}\ten{Q}^{T} \nonumber \\
	&=\ten{Q}\ten{T}\ten{Q}^{T}\qquad\forall\;\ten{Q}\in\mathscr{O}(\V),
\end{align*}
%oziroma
%\[
%	\mathcal{T}(\ten{Q}\ten{F})=\ten{Q}\mathcal{T}(\ten{F})\ten{Q}^{T}\qquad\forall\;\ten{Q}\in\mathscr{O}(\V),
%\]
kar dokazuje, da je $\ten{T}$ neodvisen od opazovališča.


\section{Materialne simetrije}


\emph{Sprememba referenčne konfiguracije} je difeomorfizem $\kappa\colon\B\to\E_R$.
Pri tem je $\B_{\kappa}=\kappa(\B)$ nova referenčna konfiguracija telesa. Gibanje
$\chi\colon\B\times I\to\E$ lahko enakovredno predstavimo s preslikavo
\begin{equation} \label{e:chgrek}
	\chi_{\kappa}\colon\B_{\kappa}\times I\to\E, \qquad
	\chi_{\kappa}=\chi\circ\kappa^{-1}.
\end{equation}
Če označimo $\ten{F}=\Grad\vek{\chi}$, $\ten{F}_{\kappa}=\Grad\vek{\chi}_{\kappa}$
in $\ten{P}=\Grad\vek{\kappa}$, potem z odvajanjem enakosti (\ref{e:chgrek})
dobimo zvezo med gradienti
\[
	\ten{F}_{\kappa}=\ten{F}\ten{P}^{-1}.
\]

Naj bosta $\kappa\colon\B\to\E_R$ in $\widehat{\kappa}\colon\B\to\E_R$ novi referenčni konfiguraciji.
Naj bo $\chi\colon\B\times I\to\E_R$ gibanje in naj bosta
$\chi_{\kappa}=\chi\circ\kappa^{-1}$ in $\chi_{\widehat{\kappa}}=\chi\circ\widehat{\kappa}^{-1}$
opisa istega gibanja glede na referenčni konfiguraciji $\kappa(\B)$ in $\widehat{\kappa}(\B)$.
\begin{figure}[h] \begin{center}
	\begin{picture}(290,130)


\put(20,60){$\B$}
\put(122,5){$\B_{\widehat{\kappa}}$}
\put(122,115){$\B_{\kappa}$}

\put(35,75){\vector(2,1){80}}
\put(60,100){$\kappa$}
\put(35,55){\vector(2,-1){80}}
\put(60,25){$\widehat{\kappa}$}

\put(125,25){\vector(0,1){80}}
\put(88,40){$\kappa\circ\widehat{\kappa}^{-1}$}
\put(130,105){\vector(0,-1){80}}
\put(133,85){$\widehat{\kappa}\circ\kappa^{-1}$}

\put(40,65){\vector(1,0){210}}
\put(190,70){$\chi$}
\put(140,15){\vector(3,1){120}}
\put(200,25){$\chi_{\widehat{\kappa}}$}
\put(140,115){\vector(3,-1){120}}
\put(200,100){$\chi_{\kappa}$}

\put(270,60){$\B_t$}

\put(70,5){$\E_R$}
\put(260,25){$\E$}


\end{picture}
	\caption{Sprememba referenčne konfiguracije.}
	\label{pic:diagram}
\end{center} \end{figure}
Označimo pripadajoča deformacijska gradienta z
\[
	\ten{F}_{\kappa}=\Grad\vek{\chi}_{\kappa}\quad \mathrm{in}\quad
	\ten{F}_{\widehat{\kappa}}=\Grad\vek{\chi}_{\widehat{\kappa}}.
\]
Jasno je, da imata referenčni konfiguraciji $\kappa$ in $\widehat{\kappa}$ vsaka
svojo funkcijo shranjene energije, $\sigma_{\kappa}$ in $\sigma_{\widehat{\kappa}}$,
ki sta v splošnem različni, vendar pa mora v vsaki točki $X\in\B$ veljati
\begin{equation} \label{e:responf}
	\sigma_{\kappa}(\ten{F}_{\kappa}(\kappa(X),t))
	=\sigma_{\widehat{\kappa}}(\ten{F}_{\widehat{\kappa}}(\widehat{\kappa}(X),t)).
\end{equation}
Iz slike \ref{pic:diagram} vidimo, da velja $\chi_{\kappa}=\chi_{\widehat{\kappa}}\circ\widehat{\kappa}\circ\kappa^{-1}$.
Če označimo s
\begin{equation} \label{e:pupa}
	\ten{P}=\Grad(\widehat{\vek{\kappa}}\circ\vek{\kappa}^{-1}),
\end{equation}
potem je $\ten{F}_{\kappa}=\ten{F}_{\widehat{\kappa}}\ten{P}$. Iz relacije (\ref{e:responf})
sedaj vidimo, da med funkcijama shranjene energije velja zveza
\begin{equation} \label{e:dolgnoc}
	\sigma_{\widehat{\kappa}}(\ten{F}_{\widehat{\kappa}})
	=\sigma_{\kappa}(\ten{F}_{\widehat{\kappa}}\ten{P}).
\end{equation}

Materialno telo lahko poseduje določene materialne simetrije, zaradi katerih ni mogoče razločiti
med določenimi referenčnimi konfiguracijami. Na primer, materialno telo iz
kubične kristalne strukture je nemogoče razločiti pred in po rotaciji za
$90^{\circ}$ okoli ene od kristalnih osi.

\begin{definicija}
	Referenčni konfiguraciji $\kappa\colon\B\to\E_R$ in $\widehat{\kappa}\colon\B\to\E_R$
	sta \emph{materialno nerazločljivi}, če velja
	\begin{equation} \label{e:grarefc}
		\sigma_{\widehat{\kappa}}(\ten{F})=\sigma_{\kappa}(\ten{F})
		\qquad\forall\:\ten{F}\in\mathrm{Inv}(\V).
	\end{equation}
\end{definicija}

Odziv materiala je odvisen od napetostnega tenzorja $\ten{T}$, ta pa je odvisen
od funkcije shranjene energije.
Materialna nerazločljivost v fizikalnem smislu pomeni, da se odziv materiala
glede na konfiguracijo $\kappa$ ne da razločiti od odziva glede na konfiguracijo
$\widehat{\kappa}$ z nobenim mogočim eksperimentom.

Če uporabimo oznako (\ref{e:pupa}) in upoštevamo enakosti (\ref{e:dolgnoc}),
potem vidimo, da je pogoj (\ref{e:grarefc}) ekvivalenten pogoju
\begin{equation*}
	\sigma_{\kappa}(\ten{F})=\sigma_{\kappa}(\ten{F}\ten{P})\qquad
	\forall\:\ten{F}\in\mathrm{Inv}(\V).
\end{equation*}

\begin{definicija}
	Če za linearno preslikavo $\ten{G}\in\mathscr{U}(\V)$ velja pogoj
	\begin{equation} \label{e:pogir}
		\sigma_{\kappa}(\ten{F})=\sigma_{\kappa}(\ten{F}\ten{G})\qquad
		\forall\:\ten{F}\in\mathrm{Inv}(\V),
	\end{equation}
	potem ji rečemo \emph{materialna simetrijska transformacija} za $\sigma$ glede na konfiguracijo $\kappa$.
\end{definicija}

Pogoj $\ten{G}\in\mathscr{U}(\V)$ sodi k predpostavki,
da vse materialne simetrijske transformacije ohranjajo prostornino telesa,
saj bi se sicer materialno telo lahko poljubno razši\-rilo, ne da bi se to poznalo na odzivu,
kar pa je fizikalno nesprejemljivo.

\begin{izrek}
	Množica $\mathcal{G}_{\kappa}$ vseh materialnih simetrijskih transformacij za $\sigma$ glede
	na konfiguracijo $\kappa$ je podgrupa unimodularne grupe.
\end{izrek}

\proof
	Dovolj je dokazati, da je $\mathcal{G}_{\kappa}$ grupa, saj je $\mathcal{G}_{\kappa}\subseteq\mathscr{U}(\V)$.
	Za poljubni $\ten{G}_1,\ten{G}_2\in\mathcal{G}_{\kappa}$
	z upoštevanjem relacije (\ref{e:pogir}) velja
	\[
		\sigma_{\kappa}(\ten{F})=\sigma_{\kappa}(\ten{F}\ten{G}_1)
		=\sigma_{\kappa}((\ten{F}\ten{G}_1)\ten{G}_2)=\sigma_{\kappa}(\ten{F}(\ten{G}_1\ten{G}_2)),
	\]
	torej je $\ten{G}_1\ten{G}_2\in\mathcal{G}_{\kappa}$. Očitno je tudi $\ten{1}\in\mathcal{G}_{\kappa}$.
	Naj bo $\ten{G}\in\mathcal{G}_{\kappa}$. Ker je $|\det\ten{G}|=1$, obstaja $\ten{G}^{-1}$ in je
	\[
		\sigma_{\kappa}(\ten{F}\ten{G}^{-1})=\sigma_{\kappa}((\ten{F}\ten{G}^{-1})\ten{G})
		=\sigma_{\kappa}(\ten{F}),
	\]
	kar dokazuje, da je tudi $\ten{G}^{-1}\in\mathcal{G}_{\kappa}$. Torej je $\mathcal{G}_{\kappa}$ grupa.
\endproof

$\mathcal{G}_{\kappa}$ se imenuje \emph{materialna simetrijska grupa} za $\sigma$ glede
na referenčno konfiguracijo $\kappa$. Očitno je, da je $\mathcal{G}_{\kappa}$ odvisna od
referenčne konfiguracije $\kappa$. Če je $\widehat{\kappa}$ še ena referenčna konfiguracija
s pripadajočo materialno simetrijsko grupo $\mathcal{G}_{\widehat{\kappa}}$
in je $\ten{P}$ kot v (\ref{e:pupa}), potem za poljuben
$\ten{G}\in\mathcal{G}_{\kappa}$ dobimo iz (\ref{e:dolgnoc})
\begin{align*}
	\sigma_{\widehat{\kappa}}(\ten{F})&=\sigma_{\kappa}(\ten{F}\ten{P})
	=\sigma_{\kappa}((\ten{F}\ten{P})\ten{G})
	=\sigma_{\kappa}(\ten{F}(\ten{P}\ten{G}\ten{P}^{-1})\ten{P}) \\
	&=\sigma_{\widehat{\kappa}}(\ten{F}(\ten{P}\ten{G}\ten{P}^{-1})),
\end{align*}
iz česar sledi, da je $\ten{P}\ten{G}\ten{P}^{-1}\in\mathcal{G}_{\widehat{\kappa}}$.
Očitno je preslikava $\phi\colon\mathcal{G}_{\kappa}\to\mathcal{G}_{\widehat{\kappa}}$,
$\phi(\ten{G})=\ten{P}\ten{G}\ten{P}^{-1}$, homomorfizem grup in očitno je injektivna,
saj za $\ten{G}_1,\ten{G}_2\in\mathcal{G}_{\kappa}$ iz
$\ten{P}\ten{G}_1\ten{P}^{-1}=\ten{P}\ten{G}_2\ten{P}^{-1}$ nemudoma sledi $\ten{G}_1=\ten{G}_2$.
S podobnim postopkom, kot zgoraj, se dokaže še, da za poljuben $\ten{H}\in\mathcal{G}_{\widehat{\kappa}}$
velja $\ten{P}^{-1}\ten{H}\ten{P}\in\mathcal{G}_{\kappa}$, torej je
$\phi(\ten{P}^{-1}\ten{H}\ten{P})=\ten{H}$, kar pomeni, da je $\phi$ tudi surjektivna.
$\phi$ je torej izomorfizem grup, zato smo dokazali naslednji izrek.

\begin{izrek}[Nollovo pravilo]
	Naj bosta $\kappa$ ter $\widehat{\kappa}$ konfiguraciji in naj bo
	$\ten{P}=\Grad(\widehat{\vek{\kappa}}\circ\vek{\kappa}^{-1})$. Potem
	med materialnima simetrijskima grupama velja zveza
	\[
		\mathcal{G}_{\widehat{\kappa}}=\ten{P}\mathcal{G}_{\kappa}\ten{P}^{-1}
		=\{\ten{P}\ten{G}\ten{P}^{-1}\;;\ \ten{G}\in\mathcal{G}_{\kappa}\}.
	\]
\end{izrek}

V zgornjem izreku je $\ten{P}$ gradient spremembe konfiguracije in ni nujno,
da pripada unimodularni grupi.


\section{Izotropna trdna hiperelastična telesa}


\begin{definicija}
	Trdno hiperelastično materialno telo je \emph{izotropno}, če obstaja referenčna
	konfiguracija $\kappa$, da je $\mathcal{G}_{\kappa}=\mathscr{O}(\V)$. Taka konfiguracija
	se imenuje \emph{izotropna}.
\end{definicija}

\begin{opomba}
Podobno, kot smo definirali materialno simetrijsko grupo za $\sigma$, bi lahko
definirali tudi materialno simetrijsko grupo za $\ten{T}$. Za elastične
materiale velja definicija, da je materialno telo \emph{trdno}, če obstaja
referenčna konfiguracija telesa, tako da je materialna simetrijska grupa za $\ten{T}$
glede na to konfiguracijo podgrupa ortogonalne grupe $\mathscr{O}(V)$.
Taki referenčni konfiguraciji se reče \emph{naravna referenčna konfiguracija}. V \cite[str.~310]{truesdell} je dokazano,
da sta pri hiperelastičnih trdnih telesih materialni simetrijski grupi za $\sigma$ in za $\ten{T}$
glede na isto referenčno konfiguracijo enaki.
\end{opomba}

Izotropnost v fizikalnem smislu pomeni, da nobena ortogonalna transformacija izotropne
konfiguracije ne spremeni odziva v materialu.

K zgornji definiciji sodi še ta opomba: Če je $\widehat{\kappa}$ taka konfiguracija, da je
$\mathcal{G}_{\widehat{\kappa}}\subseteq\mathscr{O}(\V)$, potem je $\mathcal{G}_{\widehat{\kappa}}$
podgrupa grupe $\mathcal{G}_{\kappa}=\mathscr{O}(\V)$. Po Nollovem pravilu sta
$\mathcal{G}_{\widehat{\kappa}}$ in $\mathcal{G}_{\kappa}$ izomorfni, zato $\mathcal{G}_{\widehat{\kappa}}$
ne more biti prava podgrupa grupe $\mathcal{G}_{\kappa}$, ampak je
$\mathcal{G}_{\widehat{\kappa}}=\mathcal{G}_{\kappa}$.

V nadaljevanju naj bo $\sigma$ funkcija shranjene energije glede na neko izotropno referenčno konfiguracijo
in $\mathcal{G}$ naj bo pripadajoča materialna simetrijska grupa.

Naj bo $\ten{F}=\ten{V}\ten{R}$ polarni razcep. Ker za izotropni material velja $\mathcal{G}=\mathscr{O}(\V)$,
lahko v (\ref{e:pogir}) za $\ten{G}$ uporabimo $\ten{R}^T$ in dobimo $\sigma(\ten{F})=\sigma(\ten{V})$.
Za izotropne materiale torej skupaj z (\ref{e:siuonly}) velja
\[
	\sigma(\ten{F})=\sigma(\ten{U})=\sigma(\ten{V}).
\]
V praktični uporabi je namesto levega razteznostnega tenzorja $\ten{V}$ bolj primeren
levi Cauchy-Greenov deformacijski tenzor $\ten{B}=\ten{V}^2=\ten{F}\ten{F}^T$, ker
ga je lažje izračunati. Zato predstavimo funkcijo $\bsi$, definirano kot
\begin{equation*} %\label{e:varisi}
	\bar{\sigma}(\ten{B})=\bar{\sigma}(\ten{V}^2)=\sigma(\ten{V})=\sigma(\ten{F}).
\end{equation*}

Poiščimo zvezo med odvodoma za $\sigma$ in $\bsi$. Najprej pokažimo, da je $D\bsi(\ten{B})$
simetričen teznor. $\ten{B}$ je simetričen, $\ten{B}=\ten{B}^{T}$, in
za poljuben $\ten{A}\in\L(\V)$ je
\begin{align*}
	\langle D\bsi(\ten{B}),\ten{A}\rangle &=\langle D\bsi(\ten{B}^T),\ten{A}\rangle=
	\frac{d}{d\varepsilon}\at{\bsi((\ten{B}+\varepsilon\ten{A})^T)}{\varepsilon=0}\\
	&=\langle D\bsi(\ten{B}^T),\ten{A}^T\rangle=\langle D\bsi(\ten{B}),\ten{A}^T\rangle\\
	&=\langle(D\bsi(\ten{B}))^T,\ten{A}\rangle,
\end{align*}
torej je $D\bsi(\ten{B})=(D\bsi(\ten{B}))^T$. Če odvajamo $\bsi(\ten{B})=\bsi(\ten{F}\ten{F}^T)$
po $\ten{F}$, dobimo
\begin{align*}
	\frac{d}{d\varepsilon}\at{\bsi((\ten{F}+\varepsilon\ten{A})(\ten{F}+\varepsilon\ten{A})^T)}{\varepsilon=0}&=
	\langle D\bsi(\ten{F}\ten{F}^T),\ten{F}\ten{A}^T+\ten{A}\ten{F}^T\rangle\\
	&=\langle (D\bsi(\ten{B}))^T,\ten{A}\ten{F}^T\rangle+\langle D\bsi(\ten{B}),\ten{A}\ten{F}^T\rangle \\
	&=2\langle D\bsi(\ten{B}),\ten{A}\ten{F}^T\rangle=2\langle D\bsi(\ten{B})\ten{F},\ten{A}\rangle,
\end{align*}
pri čemer smo na predzadnjem koraku upoštevali simetričnost $D\bsi(\ten{B})$.
Če ta rezultat upoštevamo pri odvajanju zveze $\sigma(\ten{F})=\bar{\sigma}(\ten{F}\ten{F}^T)$
po $\ten{F}$, dobimo
\begin{equation*}
	D\sigma(\ten{F})=2D\bsi(\ten{B})\ten{F}.
\end{equation*}
Cauchyev napetostni tenzor (\ref{e:Tdrugi}) lahko sedaj zapišemo v obliki
\begin{equation} \label{e:tmalodrugace}
	\ten{T}=\rho D\sigma(\ten{F})\ten{F}^T=2\rho D\bsi(\ten{B})\ten{F}\ten{F}^T=2\rho D\bsi(\ten{B})\ten{B}.
\end{equation}

Za izotropno hiperelastično trdno materialno telo lahko pogoja (\ref{e:objesig})
in (\ref{e:pogir}) združimo v en sam ekvivalenten pogoj:
\[
	\sigma(\ten{F})=\sigma(\ten{Q}\ten{F}\ten{Q}),\qquad\forall\:\ten{Q}\in\mathscr{O}(\V)
	\ \textrm{in}\ \forall\:\ten{F}\in\mathrm{Inv}(\V).
\]
Če v splošnem katera koli tenzorska funkcija zadošča temu pogoju, potem rečemo,
da je \emph{izotropna}. Funkcija $\bsi(\ten{B})$ je tudi izotropna na svoji domeni:
\[
	\bsi(\ten{B})=\sigma(\ten{F})=\sigma(\ten{Q}\ten{F}\ten{Q})=
	\bsi(\ten{Q}\ten{F}\ten{Q}(\ten{Q}\ten{F}\ten{Q})^T)=\bsi(\ten{Q}\ten{B}\ten{Q}^T),
\]
kar velja za vsak simetričen pozitivno definiten $\ten{B}$ in vsak $\ten{Q}\in\mathscr{O}(\V)$.

\begin{izrek} \label{i:prinva}
	Naj bo $\phi\colon\mathrm{Sym}(\V)\to\R$ tenzorska funkcija.
	Potem je $\phi$ izotropna natanko tedaj, ko se jo da zapisati v obliki
	\[
		\phi(\ten{A})=f(a_1,a_2,a_3),
	\]
	kjer je $f$ poljubna skalarna funkcija, odvisna od lastnih vrednosti $a_1,a_2,a_3$
	preslikave $\ten{A}$.
\end{izrek}

\proof
	Naj bo $\phi$ izotropna in naj imata $\ten{A},\ten{B}\in\mathrm{Sym}(\V)$
	enaki množici lastnih vrednosti, $\{a_1,a_2,a_3\}=\{b_1,b_2,b_3\}$.
	Po spektralnem izreku za simetrične linearne preslikave
	obstajata ortonormirani bazi $\{\vek{d}_j\}$ in $\{\vek{e}_j\}$, da je
	\[
		\ten{A}=\sum_{j=1}^3 a_j\vek{d}_j\otimes\vek{d}_j\quad\textrm{in}\quad
		\ten{B}=\sum_{j=1}^3 b_j\vek{e}_j\otimes\vek{e}_j.
	\]
	Obstaja ortogonalna preslikava $\ten{Q}\in\mathscr{O}(\V)$,
	da velja $\ten{Q}\vek{e}_j=\vek{d}_j$, $j=1,2,3$, zato je
	\begin{align*}
		\ten{Q}\ten{B}\ten{Q}^T&=\sum_{j=1}^3 b_j\ten{Q}(\vek{e}_j\otimes\vek{e}_j)\ten{Q}^T
		=\sum_{j=1}^3 b_j(\ten{Q}\vek{e}_j)\otimes(\ten{Q}\vek{e}_j) \\
		&=\sum_{j=1}^3 a_j\vek{d}_j\otimes\vek{d}_j=\ten{A}.
	\end{align*}
	Ker je $\phi$ po predpostavki izotropna, je
	\[
		\phi(\ten{A})=\phi(\ten{Q}\ten{B}\ten{Q}^T)=\phi(\ten{B}).
	\]
	Dokazali smo, da je $\phi$ odvisna le od lastnih vrednosti simetrične matrike,
	torej je $\phi(\ten{A})=f(a_1,a_2,a_3)$.
	
	Obratno, če velja $\phi(\ten{A})=f(a_1,a_2,a_3)$, potem je $\phi(\ten{A})=\phi(\ten{Q}\ten{A}\ten{Q}^T)$,
	saj imata linearni preslikavi $\ten{A}$ in $\ten{Q}\ten{A}\ten{Q}^T$ enake lastne vrednosti,
	kar se najhitreje vidi tako, da imata $\ten{A}$ in $\ten{Q}\ten{A}\ten{Q}^T$ enak karakteristični polinom.
	Če je namreč $p_{ten{A}}$ karakteristični polinom za $\ten{A}$, potem je po Cayley-Hamiltonovem izreku
	$p_{ten{A}}=0$, hitro pa lahko vidimo, da je tudi $p_{ten{A}}(\ten{Q}\ten{A}\ten{Q}^T)=0$.
\endproof

Karakteristični polinom linearne preslikave $\ten{A}\in\L(\V)$ je
\[
	\det(\ten{A}-\lambda\ten{1})=-\lambda^3+I_{\ten{A}}\lambda^2+\rdva_{\ten{A}}\lambda+\rtri_{\ten{A}},
\]
pri tem so $I_{\ten{A}}, \rdva_{\ten{A}}, \rtri_{\ten{A}}$ \emph{glavne invariante} za $\ten{A}$.
Če so $a_1,a_2,a_3$ lastne vrednosti od $\ten{A}$, potem je
\begin{align*}
	I_{\ten{A}}&=a_1+a_2+a_3=\tr\ten{A},\\
	\rdva_{\ten{A}}&=a_1a_2+a_1a_3+a_2a_3=\frac{1}{2}\big((\tr\ten{A})^2-\tr(\ten{A}^2)\big),\\
	\rtri_{\ten{A}}&=a_1a_2a_3=\det\ten{A}.
\end{align*}
Množici $\{a_1,a_2,a_3\}$ in $\{I_{\ten{A}},\rdva_{\ten{A}},\rtri_{\ten{A}}\}$ sta ekvivalentni
v smislu, da ena enoli\-čno določa drugo. Zato lahko v izreku \ref{i:prinva} nadomestimo
funkcijo $f(a_1,a_2,a_3)$ z neko drugo funkcijo $\tilde{f}(I_{\ten{A}},\rdva_{\ten{A}},\rtri_{\ten{A}})$.
Funkcijo $\bsi$ lahko torej predstavimo kot
\begin{equation} \label{e:equinox}
	\bsi(\ten{B})=w(I_{\ten{B}},\rdva_{\ten{B}},\rtri_{\ten{B}}).
\end{equation}
Na spremenljivke $I_{\ten{B}},\rdva_{\ten{B}},\rtri_{\ten{B}}$ v funkciji $w$ lahko gledamo kot
na funkcije $I_{\ten{B}}=I(\ten{B})$, itd. Z odvajanjem enakosti (\ref{e:equinox}) po $\ten{B}$ dobimo
\begin{equation} \label{e:dbshy}
	D\bsi(\ten{B})=\frac{\partial w}{\partial I_{\ten{B}}}DI(\ten{B})+
	\frac{\partial w}{\partial \rdva_{\ten{B}}}D\rdva(\ten{B})+
	\frac{\partial w}{\partial \rtri_{\ten{B}}}D\rtri(\ten{B}).
\end{equation}
$D\rtri(\ten{B})=D\det(\ten{B})$ smo že izračunali v (\ref{e:odvodet}). Izračunajmo še
odvoda preostalih dveh glavnih invariant. Za poljuben $\ten{S}$ imamo
\begin{align*}
	DI(\ten{B})(\ten{S})&=D\tr(\ten{B})(\ten{S})=
	\frac{d}{d\varepsilon}\at{\tr(\ten{B}+\varepsilon\ten{S})}{\varepsilon=0}=
	\tr(\ten{S})=\langle\ten{1},\ten{S}\rangle,\\
	D\rdva(\ten{B})(\ten{S})&=
	\frac{1}{2}\frac{d}{d\varepsilon}\at{\Big(
	\big(\tr(\ten{B}+\varepsilon\ten{S})\big)^2-\tr\big((\ten{B}+\varepsilon\ten{S})^2\big)
	\Big)}{\varepsilon=0}\\
	&=(\tr\ten{B})(\tr\ten{S})-\frac{1}{2}\tr(\ten{B}\ten{S})-\frac{1}{2}\tr(\ten{S}\ten{B})\\
	&=(\tr\ten{B})\langle\ten{1},\ten{S}\rangle-\frac{1}{2}\langle\ten{B}^T,\ten{S}\rangle
	-\frac{1}{2}\langle\ten{S},\ten{B}^T\rangle\\
	&=\big<(\tr\ten{B})\ten{1}-\ten{B}^T,\ten{S}\big>.
\end{align*}
Odvodi glavnih invariant so torej
\[
	DI(\ten{B})=\ten{1},\quad D\rdva(\ten{B})=I_{\ten{B}}\ten{1}-\ten{B}^T,\quad
	D\rtri(\ten{B})=\rtri_{\ten{B}}\ten{B}^{-T}.
\]
Če to upoštevamo v (\ref{e:dbshy}), rezultat pa nato v (\ref{e:tmalodrugace}), dobimo
izraz za Cauchyjev napetostni tenzor $\ten{T}$ za izotropno hiperelastično trdno materialno telo,
ki je
\begin{equation}
	\ten{T}=2\rho\bigg(
	\rtri_{\ten{B}}\frac{\partial w}{\partial \rtri_{\ten{B}}}\ten{1}+
	\Big(\frac{\partial w}{\partial I_{\ten{B}}}+I_{\ten{B}}\frac{\partial w}{\partial \rdva_{\ten{B}}}\Big)\ten{B}-
	\frac{\partial w}{\partial \rdva_{\ten{B}}}\ten{B}^2
	\bigg).
\end{equation}