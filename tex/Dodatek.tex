\appendix

\chapter{Konstitucijske enačbe za izotropno elastično trdno telo}


\section{Sprememba opazovališča in objektivnost}


V razdelku \ref{s:evklipro} smo podali definicijo opazovališča za Evklidski
prostor $\E$. Nato smo v poglavju \ref{chp:kinkon} omenili, da so določene
fizikalne količine lahko odvisne od izbire opazovališča. V tem razdelku bomo
na kratko analizirali spremembo opazovališča.

\emph{Sprememba opazovališča} za $\E$ je prehod iz opazovališča $\iota_o$ v opazovališče
$\iota^*_{o^*}$. Pri tem točka $x\in\E$, ki ima v opazovališču $\iota_o$ krajevni vektor
$\vek{x}=\iota_o(x)$, dobi nov krajevni vektor $\vek{x}^*=\iota^*_{o^*}(x)$ glede na opazovališče
$\iota^*_{o^*}$. Vsaka sprememba opazovališča porodi transformacijo
\[
	\iota^*_{o^*}\circ\iota_o^{-1}\colon\V\to\V,\qquad
	\vek{x}\mapsto\vek{x}^*=\iota^*_{o^*}(\iota_o^{-1}(\vek{x})).
\]

Razdalja oz.~metrika na $\E$ je odvisna od izbire opazovališča. \emph{Izmerjena razdalja} med točkama
$x_1,x_2\in\E$ v opazovališču $\iota_o$ je
$d(x_1,x_2)=\|\iota_o(x_2)-\iota_o(x_1)\|=\|\iota(x_1,x_2)\|$,
kjer je $\|\cdot\|$ norma prostora $\V$. Nas bodo zanimale take
spremembe opazovališča, ki ohranjajo izmerjene razdalje med točkami, kar pomeni,
da mora za vsaki točki $x_1,x_2\in\E$ veljati
$\|\iota^*(x_1,x_2)\|=\|\iota(x_1,x_2)\|$.
Iz linearne algebre vemo, da so izometrije na prostoru $\V$ natanko ortogonalne preslikave,
torej mora veljati
\begin{equation} \label{e:vektrans}
	\iota^*(x_1,x_2)=\ten{Q}\iota(x_1,x_2)\quad\forall x_1,x_2\in\E,
\end{equation}
kjer je
\[
	\ten{Q}\in\mathscr{O}(\V)=\{\ten{R}\in\L(\V)\;;\ \ten{R}\ten{R}^T=\ten{R}^T\ten{R}=\ten{1}\}
\]
ortogonalna preslikava. Množica $\mathscr{O}(\V)$ vseh ortogonalnih preslikav na prostoru
$\V$ je za operacijo kompozitum grupa, imenovana tudi \emph{ortogonalna grupa}. Če je $x\in\E$ in
je $\vek{x}=\iota_o(x)$ ter $\vek{x}^*=\iota^*_{o^*}(x)$, potem je
\begin{equation} \label{e:toga}
	\vek{x}^*=\iota^*(o^*,x)=\iota^*(o^*,o)+\iota^*(o,x)=\vek{c}+\ten{Q}\vek{x}.
\end{equation}
Pri tem smo označili $\vek{c}=\iota^*(o^*,o)$. Transformaciji oblike (\ref{e:toga})
se reče tudi \emph{toga transformacija}. Odselj bomo izraz \textit{sprememba opazovališča}
uporabljali zgolj za take spremembe opazovališča, ki porodijo togo transformacijo.

Za vsak vektor $\vek{u}\in\V$ po aksiomih Evklidskega točkovnega prostora obstajata
točki $x_1,x_2\in\E$, da je $\vek{u}=\iota(x_1,x_2)$. Vektorju $\vek{u}$ pripada
vektor $\vek{u}^*=\iota^*(x_1,x_2)$, za katerega po (\ref{e:vektrans}) velja
\begin{equation} \label{e:travek}
	\vek{u}^*=\ten{Q}\vek{u}.
\end{equation}
Linearni preslikavi $\ten{A}\in\L(\V)$ z razvojem $\ten{A}=A_{ij}\,\vek{e}_i\otimes\vek{e}_j$
pripada linearna preslikava
\begin{align}
	\ten{A}^*&=A_{ij}\,\vek{e}^*_i\otimes\vek{e}^*_j=A_{ij}\,\ten{Q}\vek{e}_i\otimes\ten{Q}\vek{e}_j
	=\ten{Q}A_{ij}\,\vek{e}_i\otimes\vek{e}_j\ten{Q}^T \nonumber \\
	&=\ten{Q}\ten{A}\ten{Q}^T. \label{e:traten}
\end{align}
Za vektorsko količino $\vek{u}$ oz.~tenzorsko količino $\ten{A}$
rečemo, da je \emph{objektivna} ali pa \emph{neodvisna od opazovališča},
če se z vsako spremembo opazovališča iz $\iota_o$ v $\iota^*_{o^*}$
transformira po pravilu (\ref{e:travek}) oz.~(\ref{e:traten}).
Skalarna količina je \emph{objektivna} ali \emph{neodvisna od opazovališča},
če ob vsaki spremembi opazovališča ostane nespremenjena.

Za prostor $\E_R$, ki služi zgolj določitvi materialnega telesa, naj velja,
da ima vseskozi enako opazovališče. Spremembe opazovališča pridejo v upoštev zgolj
za prostor $\E$, kjer potekajo fizikalni dogodki. Naj bo $\chi\colon\B\times I\to\E$
gibanje s pripadajočima vektorskima poljema $\vek{\chi}=\iota_o\circ\chi$ in
$\vek{\chi}^*=\iota^*_{o^*}\circ\chi$. Po (\ref{e:toga}) velja
\[
	\vek{\chi}^*(X,t)=\ten{Q}\vek{\chi}(X,t)+\vek{c}.
\]
Naj bosta $\ten{F}=\Grad\vek{\chi}$ in $\ten{F}^*=\Grad\vek{\chi}^*$ pripadajoča
deformacijska gradienta gibanja. Iz zgornje relacije dobimo
\[
	\ten{F}^*=\ten{Q}\ten{F}.
\]
Deformacijski gradient torej ni objektiven oz.~neodvisen od opazovališča, saj se ne
transformira po pravilu (\ref{e:traten}). Jacobijan je objektivna skalarna
količina, saj velja
\[
	J^*=|\det(\ten{F}^*)|=|\det(\ten{Q}\ten{F})|=|\det(\ten{Q})||\det(\ten{F})|
	=1\cdot|\det(\ten{F})|=J.
\]
Posledično je tudi masna gostota trenutne konfiguracije objektivna skalarna količina,
\[ \rho^*=\frac{\rho_R}{J^*}=\frac{\rho_R}{J}=\rho. \]

Za gostoto notranje energije $\sigma(\ten{F})$, ki smo jo predstavili v primeru
elastičnega trdnega telesa, zahtevamo, da je kot skalarna količina neodvisna
od opazovališča, torej velja
\begin{equation} \label{e:objesig}
	\sigma(\ten{F}^*)=\sigma(\ten{Q}\ten{F})=\sigma(\ten{F})
\end{equation}
za vse $(X,t)\in\B\times I$.
Dokažimo, da je tedaj Cauchyev napetostni tenzor (\ref{e:napetostni})
\[
	\ten{T}=\mathcal{T}(\ten{F})=\rho D\sigma(\ten{F})\ten{F}^T
	=\frac{\rho_R}{|\det\ten{F}|} D\sigma(\ten{F})\ten{F}^T
\]
tudi objektiven. Če izraz $\sigma(\ten{Q}\ten{F})$ posredno odvajamo po $\ten{F}$,
dobimo za poljuben $\ten{H}\in\L(\V)$
\[
	D(\sigma\circ(\ten{X}\mapsto\ten{Q}\ten{X}))(\ten{F})(\ten{H})=
	D\sigma(\ten{Q}\ten{F})\cdot(\ten{Q}\ten{H})=\ten{Q}^T D\sigma(\ten{Q}\ten{F})\cdot\ten{H}.
\]
Z odvajanjem enakosti (\ref{e:objesig}) po $\ten{F}$ torej pridemo do
\[
	\ten{Q}^T D\sigma(\ten{Q}\ten{F}) = D\sigma(\ten{F})\qquad\textrm{oziroma}
	\qquad D\sigma(\ten{Q}\ten{F}) = \ten{Q}D\sigma(\ten{F}).
\]
Z upoštevanjem pravkar dobljene enakosti in objektivnosti masne gostote dobimo
\begin{align*}
	\ten{T}^*&=\rho^* D\sigma(\ten{F}^*)\ten{F}^{*T}
	=\rho D\sigma(\ten{Q}\ten{F})\ten{F}^{T}\ten{Q}^{T}
	=\rho \ten{Q} D\sigma(\ten{F})\ten{F}^{T}\ten{Q}^{T} \nonumber \\
	&=\ten{Q}\ten{T}\ten{Q}^{T}
\end{align*}
oziroma
\[
	\mathcal{T}(\ten{Q}\ten{F})=\ten{Q}\mathcal{T}(\ten{F})\ten{Q}^{T}\qquad\forall\;\ten{Q}\in\mathscr{O}(\V),
\]
kar dokazuje, da je $\ten{T}$ neodvisen od opazovališča.


\subsection{Materialne simetrije}


\emph{Sprememba referenčne konfiguracije} je bijektivna gladka preslikava $\kappa\colon\B\to\E_R$.
Pri tem je $\B_{\kappa}=\kappa(\B)$ nova referenčna konfiguracija telesa. Gibanje
$\chi\colon\B\times I\to\E$ lahko enakovredno predstavimo s preslikavo
\[
	\chi_{\kappa}\colon\B_{\kappa}\times I\to\E, \qquad
	\chi_{\kappa}=\chi\circ\kappa^{-1}.
\]
Če označimo $\ten{F}=\Grad\vek{\chi}$, $\ten{F}_{\kappa}=\Grad\vek{\chi}_{\kappa}$
in $\ten{P}=\Grad\vek{\kappa}$, potem z odvajanjem enakosti $\chi=\chi_{\kappa}\circ\kappa$
dobimo zvezo med gradienti
\[
	\ten{F}=\ten{F}_{\kappa}\,\ten{P}.
\]

\begin{definicija}
	Referenčni konfiguraciji $\kappa\colon\B\to\E_R$ in $\hat{\kappa}\colon\B\to\E_R$
	sta \emph{materialno nerazločljivi}, če velja
	\begin{equation} \label{e:grarefc}
		\mathcal{T}(\ten{F}_{\kappa})=\mathcal{T}(\ten{F}_{\hat{\kappa}}),
	\end{equation}
	pri čemer je $\ten{F}_{\kappa}=\Grad\vek{\chi}_{\kappa}$ in
	$\ten{F}_{\hat{\kappa}}=\Grad\vek{\chi}_{\hat{\kappa}}$.
\end{definicija}

Če označimo z $\ten{G}=\Grad(\vek{\kappa}\circ\hat{\vek{\kappa}}^{-1})$,
potem je pogoj (\ref{e:grarefc}) ekvivalenten pogoju
\[
	a
\]

\begin{izrek}[Polarni razcep]
	Za vsako obrnljivo linearno preslikavo $\ten{F}\in\L(\V)$, $\det\ten{F}\neq 0$,
	obstajata simetrični pozitivno definitni linearni preslikavi $\ten{V}$ in $\ten{U}$
	ter ortogonalna linearna preslikava $\ten{R}$, tako da velja
	\[
		\ten{F}=\ten{R}\ten{U}=\ten{V}\ten{R}.
	\]
	Pri tem so $\ten{U}$, $\ten{V}$ in $\ten{R}$ z zgornjim razcepom enolično določene.
\end{izrek}

\proof
	Takoj se vidi, da sta preslikavi $\ten{F}\ten{F}^T$ in $\ten{F}^T\ten{F}$
	simetrični, hkrati pa tudi pozitivno definitni, saj za poljuben $\vek{v}\in\V$,
	$\vek{v}\neq\vek{0}$, velja
	\[ \vek{v}\cdot\ten{F}^T\ten{F}\vek{v}=\ten{F}\vek{v}\cdot\ten{F}\vek{v}>0, \]
	saj je $\ten{F}$ nesingularna.
	
	Iz spektralnega izreka iz linearne algebre vemo, da se da vsako simetrično
	linearno preslikavo $\ten{S}$ zapisati kot
	\[ \ten{S}=\sum_{j=1}^3 a_j\vek{u}_j\otimes\vek{u}_j, \]
	kjer so $a_j$ lastne vrednosti, $\vek{u}_j$ pa pripadajoči lastni vektorji.
	Pri tem je mogoče lastne vektorje izberati tako, da so medsebojno ortogonalni.
	Vse lastne vrednosti simetrične preslikave so vedno realne. Za vsako
	simetrično linearno preslikavo $\ten{S}$ obstaja kvadratni koren, t.j.~enolično določena linearna
	preslikava $\ten{G}$, da je $\ten{G}^2=\ten{S}$. Velja
	\[ \ten{G}=\sqrt{\ten{S}}=\sum_{j=1}^3 \sqrt{a_j}\,\vek{u}_j\otimes\vek{u}_j. \]
	
	Definirajmo
	\[
		\ten{U}=\sqrt{\ten{F}^T\ten{F}},\quad\ten{R}=\ten{F}\ten{U}^{-1},\quad
		\ten{V}=\ten{R}\ten{U}\ten{R}^T.
	\]
	Po definiciji je $\ten{U}$ simetrična pozitivno definitna in $\ten{R}$ je ortogonalna, saj je
	\begin{align*}
		\ten{R}\ten{R}^T&=\ten{F}\ten{U}^{-1}(\ten{F}\ten{U}^{-1})^T
		=\ten{F}\ten{U}^{-1}\ten{U}^{-T}\ten{F}^T \\ &=\ten{F}\ten{U}^{-2}\ten{F}^T
		=\ten{F}(\ten{F}^T\ten{F})^{-1}\ten{F}^T=\ten{1}.
	\end{align*}
	Poleg tega je
	\[
		\ten{V}^2=(\ten{R}\ten{U}\ten{R}^T)(\ten{R}\ten{U}\ten{R}^T)
		=(\ten{R}\ten{U})(\ten{R}\ten{U})^T=\ten{F}\ten{F}^T,
	\]
	torej je $\ten{V}$ kvadratni koren od $\ten{F}\ten{F}^T$ in je zato
	simetrična in pozitivno definitna. Enoličnost sledi iz enoličnosti obstoja
	kvadratnega korena.
\endproof