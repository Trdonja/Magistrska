% ----- Transportni izrek v splošni obliki -----

\begin{izrek}[Transportni izrek]
	Naj bo $S(t)$ singularna ploskev glede na polje $f\colon\Omega\to\W$. Velja
	\begin{equation*}
		\frac{d}{dt}\int_B \hat{f}\,dV = \int_B \frac{\partial \hat{f}}{\partial t}\,dV -
		\int_{S_{\r}(t)} \llbracket\hat{f}\rrbracket(\vek{w}_{\r}\cdot\vek{N})\,dS\quad\textrm{in}
	\end{equation*}
	\[
		\frac{d}{dt}\int_{B_t} f\,dV = \int_{B_t} \frac{\partial f}{\partial t}\,dV +
		\int_{\partial B_t} f(\vek{v}\cdot\vek{n})\,dS - \int_{S(t)} \llbracket f\rrbracket(\vek{w}\cdot\vek{n})\,dS.
	\]
\end{izrek}
\proof
	Izrek bomo dokazali za primer, ko je gibanje singularne ploskve $S(t)$ nepovratno, tj.~parametrizacija
	$\omega$ je injektivna preslikava. Izrek velja tudi sicer, le dokaz je bolj zapleten.
	Najprej dokažimo prvo enakost. Opazimo, da velja
	\[
		\frac{d}{dt}\int_B \hat{f}(X,t)\,dV = \frac{d}{dt}\left(
		\int_{B^-(t)} \hat{f}(X,t)\,dV + \int_{B^+(t)} \hat{f}(X,t)\,dV \right).
	\]
	Posvetimo se le odvajanju prvega integrala, odvajanja drugega integrala poteka na enak način.
	Po definiciji odvoda je
	\begin{gather*}
		\frac{d}{dt}\int_{B^-(t)} \hat{f}(X,t)\,dV =\\ =\lim_{h\to 0} \frac{1}{h}\left(
		\int_{B^-(t+h)} \hat{f}(X,t+h)\,dV - \int_{B^-(t)} \hat{f}(X,t)\,dV \right) \\
		=\lim_{h\to 0} \frac{1}{h}\left( \int_{B^-(t+h)} \hat{f}(X,t+h)\,dV - \int_{B^-(t)} \hat{f}(X,t+h)\,dV \right) + \\
		+ \lim_{h\to 0} \frac{1}{h}\left( \int_{B^-(t)} \hat{f}(X,t+h)\,dV - \int_{B^-(t)} \hat{f}(X,t)\,dV \right) \\
		= \lim_{h\to 0} \frac{1}{h}\int_{B^-(t+h)\setminus B^-(t)} \hat{f}(X,t+h)\,dV +
		\int_{B^-(t)} \frac{\partial\hat{f}}{\partial t}(X,t)\,dV.
	\end{gather*}
	Območje $B^-(t)\setminus B^-(t+h)$ je po začetni predpostavki o nepovratnosti singularne ploskve prazno,
	zato smo integral po tem območju izpustili. Limito bomo računali po komponentah. Polje $\hat{f}$
	lahko namreč zapišemo v komponentni obliki glede na neko fiksno bazo $\{b_k\}_{k=1}^m$ prostora $\W$ kot
	\[ \hat{f}(X,t)=\sum_{k=1}^m \hat{f}_k(X,t)b_k. \]
	Zaradi začetne predpostavke lahko
	območje $B^-(t+h)\setminus B^-(t)$ regularno parametriziramo kar s skrčeno bijektivno parametrizacijo
	\[
		\omega(\theta,\tau)\colon\Theta_{\tau}\times [t,t+h]\to B^-(t+h)\setminus B^-(t),\quad
		\Theta_{\tau}=\{\theta\in \Theta\: ;\: \omega(\theta,\tau)\in B\},
	\]
	zato je
	\begin{multline*}
		F_k(t)=\lim_{h\to 0} \frac{1}{h}\int_{B^-(t+h)\setminus B^-(t)} \hat{f}_k(X,t+h)\,dV=\\
		=\lim_{h\to 0} \frac{1}{h}\int_{t}^{t+h}\iint_{\Theta_{\tau}}\hat{f}_k\big(\omega(\theta,\tau),t+h\big)
		\left|\frac{\partial\omega}{\partial\theta^1}\times\frac{\partial\omega}{\partial\theta^2}\cdot\frac{\partial\omega}{\partial\tau}\right|
		(\theta,\tau)\,d\theta^1 d\theta^2 d\tau.
	\end{multline*}
	Po Lagrangevem izreku o srednji vrednosti obstaja $\xi\in [t,t+h]$, da je
	\[
		F_k(t)=\lim_{h\to 0} \frac{1}{h}h\iint_{\Theta_{\xi}}\hat{f}_k\big(\omega(\theta,\xi),t+h\big)
		\left|\frac{\partial\omega}{\partial\theta^1}\times\frac{\partial\omega}{\partial\theta^2}\cdot\frac{\partial\omega}{\partial\tau}\right|
		(\theta,\xi)\,d\theta^1 d\theta^2.
	\]
	Ko gre $h$ proti $0$, gre $\xi$ proti $t$ in
	\[ \hat{f}_k\big(\omega(\theta,\xi),t+h\big)\to\hat{f}^-_k\big(\omega(\theta,t),t\big). \]
	Upoštevamo še
	\[
		\left|\frac{\partial\omega}{\partial\theta^1}\times\frac{\partial\omega}{\partial\theta^2}\cdot
		\frac{\partial\omega}{\partial\tau}\right|\,d\theta^1 d\theta^2=
		\vek{N}\cdot\vek{w}_{\r}
		\left\|\frac{\partial\omega}{\partial\theta^1}\times\frac{\partial\omega}{\partial\theta^2}\right\|\,d\theta^1 d\theta^2=
		\vek{N}\cdot\vek{w}_{\r}\,dS,
	\]
	pri čemer je skalarni produkt $\vek{w}_{\r}\cdot\vek{N}$ pozitiven, zato smo absolutno vrednost izspustili.
	Hitrost $\vek{w}_{\r}$ in normala $\vek{N}$ namreč zaradi začetne predpostavke vedno kažeta v isti polprostor v $\E$,
	določen s tangentno ravnino, katere normala je $\vek{N}$. Tako dobimo
	\[ F_k(t)=\int_{S_{\r}(t)}\hat{f}_k^-(\vek{w}_{\r}\cdot\vek{N})\,dS \]
	in če vsak $F_k(t)$ pomnožimo z $b_k$, nato pa tvorimo vsoto po vseh $k$, dobimo
	\[
		\lim_{h\to 0} \frac{1}{h}\int_{B^-(t+h)\setminus B^-(t)} \hat{f}(X,t+h)\,dV =
		\int_{S_{\r}(t)}\hat{f}^-(\vek{w}_{\r}\cdot\vek{N})\,dS.
	\]
	Dokazali smo
	\begin{equation} \label{e:vmrez}
		\frac{d}{dt}\int_{B^-(t)} \hat{f}\,dV =
		\int_{S_{\r}(t)}\hat{f}^-(\vek{w}_{\r}\cdot\vek{N})\,dS +
		\int_{B^-(t)} \frac{\partial\hat{f}}{\partial t}\,dV.
	\end{equation}
	Na popolnoma enak način dobimo še
	\begin{equation*}
		\frac{d}{dt}\int_{B^+(t)} \hat{f}\,dV =
		-\int_{S_{\r}(t)}\hat{f}^+(\vek{w}_{\r}\cdot\vek{N})\,dS +
		\int_{B^+(t)} \frac{\partial\hat{f}}{\partial t}\,dV.
	\end{equation*}
	Ploskev $S_{\r}(t)$ je pri tem orientirana nasprotno, kot v prejšnjem primeru,
	zato je njena normala $-\vek{N}$ in posledično je ploskovni integral v tem primeru
	nasprotno predznačen. Če obe dobljeni enačbi seštejemo, dobimo prvo enačbo
	iz izreka.
	
	Drugo enakost bomo dokazali za primer, ko je $f$ skalarno ali pa vektorsko polje.
	Tudi tu integral najprej razcepimo na vsoto dveh integralov,
	enega po območju $B_t^-$ in drugega po območju $B_t^+$. Zopet se posvetimo le odvajanju
	prvega integrala, saj pri računanju odvoda drugega integrala uporabimo povsem enake metode.
	Z uporabo (\ref{e:dxdX})${}_3$ in malo prej izpeljane enakosti (\ref{e:vmrez}) dobimo
	\begin{align}
		\frac{d}{dt}\int_{B_t^-}f\,dV&=\frac{d}{dt}\int_{B^-(t)}\hat{f}J\,dV \nonumber \\
		&=\int_{B^-(t)}\frac{\partial(\hat{f}J)}{\partial t}\,dV + \int_{S_{\r}(t)}\hat{f}^-J^-(\vek{w}_{\r}\cdot\vek{N})\,dS. \label{e:vr2}
	\end{align}
	Integrand v prvem integralu lahko s pomočjo enačbe (\ref{e:dotJ}) za odvod determinante $J$ razpišemo kot
	\[ \frac{\partial(\hat{f}J)}{\partial t}=\left(\frac{\partial\hat{f}}{\partial t}+\hat{f}\widehat{\div\vek{v}}\right)J, \]
	integral pa potem s pomočjo (\ref{e:dxdX}) in (\ref{e:matodv}) transformiramo nazaj v
	\[
		\int_{B^-(t)}\frac{\partial(\hat{f}J)}{\partial t}\,dV =
		\int_{B_t^-}\left(\frac{\partial f}{\partial t}+(\nabla_{x}f)[\vek{v}]+f\div\vek{v}\right)\,dV.
	\]
	Sedaj uporabimo tretjo oz.~četrto enačbo iz trditve \ref{t:divprop} ter nato še drugo oz.~tretjo enakost
	iz divergenčnega izreka \ref{i:divtheo}, odvisno od tega, ali je $f$ skalarno ali
	vektorsko polje, ter dobimo
	\begin{equation} \label{e:vr3}
		\int_{B^-(t)}\frac{\partial(\hat{f}J)}{\partial t}\,dV =
		\int_{B_t^-}\frac{\partial f}{\partial t}\,dV + \int_{\partial B_t^-}f^-(\vek{v}^-\cdot\vek{n})\,dS.
	\end{equation}
	Prevedimo še preostali integral:
	\begin{multline} \label{e:vr4}
		\int_{S_{\r}(t)}\hat{f}^-J^-(\vek{w}_{\r}\cdot\vek{N})\,dS =
		\int_{S_{\r}(t)}\hat{f}^-\big((\ten{F}^{-1})^-\ten{F}^-\vek{w}_{\r}\cdot J^-\vek{N}\big)\,dS= \\
		=\int_{S_{\r}(t)}\hat{f}^-\big(\ten{F}^-\vek{w}_{\r}\cdot J^-(\ten{F}^{-T})^-\vek{N}\big)\,dS
		=\int_{S(t)}f^-\big((\vek{w}-\vek{v}^-)\cdot\vek{n}\big)\,dS.
	\end{multline}
	Tu smo na zadnjem koraku uporabili (\ref{e:dxdX})${}_2$ in $(\ref{e:w})$. V enačbi (\ref{e:vr3})
	lahko ploskovni integral po območju $\partial B_t^-$ zapišemo kot vsoto dveh ploskovnih integralov,
	enega po ploskvi $\partial B_t^-\setminus S(t)$ in drugega po ploskvi $S(t)$.
	Iz (\ref{e:vr2}), (\ref{e:vr3}) in (\ref{e:vr4}) sedaj dobimo
	\[
		\frac{d}{dt}\int_{B_t^-}f\,dV=
		\int_{B_t^-}\frac{\partial f}{\partial t}\,dV + \int_{\partial B_t^-\setminus S(t)}f(\vek{v}\cdot\vek{n})\,dS +
		\int_{S(t)}f^-(\vek{w}\cdot\vek{n})\,dS.
	\]
	Na enak način pridemo do
	\[
		\frac{d}{dt}\int_{B_t^+}f\,dV=
		\int_{B_t^+}\frac{\partial f}{\partial t}\,dV + \int_{\partial B_t^+\setminus S(t)}f(\vek{v}\cdot\vek{n})\,dS -
		\int_{S(t)}f^+(\vek{w}\cdot\vek{n})\,dS.
	\]
	Pri tem je ploskev $S(t)$ orientirana nasprotno, kot v prejšnjem primeru, zato pride ploskovni integral po
	ploskvi $S(t)$ negativno predznačen. Če sedaj seštejemo dobljeni enakosti, dobimo drugo enakost iz izreka.
\endproof