\section{Rezerve}

Včasih si želimo prostor $\E$ opremiti s kakšnim drugim koordinatnim sistemom.
\begin{definicija}\label{d:ks}
	Naj bosta $\mathcal{D}\subseteq\E$ in $U\subseteq\R^3$ odprti množici. Bijektivno preslikavo
	$\psi\colon\mathcal{D}\to U$ z lastnostjo, da sta preslikavi $\psi\circ\kappa$ in njen inverz
	$\kappa^{-1}\circ\psi^{-1}$ gladki (na svojih domenah), imenujemo \emph{koordinatni sistem}
	za $\mathcal{D}\subseteq\E$.
\end{definicija}
Inverz od $\psi$ bomo označili s $\varphi=\psi^{-1}$.
Koordinatni sistem lahko podamo kot koordinatno transformacijo kartezijevih koordinat $y_k$ s funkcijami
\begin{equation*}
	x_k = \psi_k(y_1,y_2,y_3) \ \Leftrightarrow \ y_k=\varphi_k(x_1,x_2,x_3), \quad k=1,2,3.
\end{equation*}

\[
	(x_1,x_2,x_3)\in \R^3\ \xrightleftharpoons[\ \psi\ ]{\ \varphi\ }\ 
	x\in\E\ \xrightleftharpoons[\ \phantom{\varphi}\ ]{\ \phantom{\psi}\ }\ 
	\vek{r}(x)=y_i\vek{e}_i \in V
\]

\subsection{Zapis v naravni bazi}

Označimo z $y_k$ kartezijeve, z $x_k$ pa krivočrtne koordinate točke $x\in\E$.
Preko koordinatne transformacije (\ref{e:kt}) lahko definiramo vektorje
\begin{align*}
	\vek{g}_k(x)=\frac{\partial}{\partial x^k}\hat{y}_j(x^1,x^2,x^3)\,\vek{e}_j, \qquad
	\vek{g}^k(x)=\frac{\partial}{\partial y_j}\hat{x}^k(y_1,y_2,y_3)\,\vek{e}_j.
\end{align*}
$\vek{g}_k$ imenujemo \emph{tangentni}, $\vek{g}^k$ pa \emph{gradientni vektorji}.
Množici $\{\vek{g^k(x)}\}$ in $\{\vek{g_k(x)}\}$ tvorita bazi za $V$, imenovani \emph{naravni bazi} v točki $x$,
in sta si dualni, tj.~$\vek{g}^i(x)\cdot\vek{g}_j(x)=\topbot{\delta}{i}{j}$.
Če uporabljamo kartezijev koordinatni sistem, potem je naravna baza kar $\{\vek{e}_k\}$, ki je dualna sama sebi
in je neodvisna od točke $x$.
\emph{Metrični koeficienti} so definirani kot
\begin{equation}\label{e:g}
	g^{ij}(x)=\vek{g}^i(x)\cdot\vek{g}^j(x) \quad\mathrm{in}\quad g_{ij}(x)=\vek{g}_i(x)\cdot\vek{g}_j(x).
\end{equation}

Vektorsko polje običajno zapišemo v komponentni obliki glede na naravno bazo kot
\[ \vek{u}=u^j\vek{g}_j=u_k\vek{g}^k. \]
Tenzorsko polje z vrednostmi iz prostora $\L(V)$ pa običajno zapišemo v komponentni obliki
glede na eno od baz prostora $\L(V)$, ki je izpeljana iz naravnih baz za $V$:
\[
	S=\bottop{S}{i}{j}\,\vek{g}^i\otimes\vek{g}_j=\topbot{S}{i}{j}\,\vek{g}_i\otimes\vek{g}^j=
	S^{ij}\,\vek{g}_i\otimes\vek{g}_j=S_{ij}\,\vek{g}^i\otimes\vek{g}^j.
\]

Za polje $f$ rečemo, da \emph{je razreda $C^n$} na neki odprti domeni, če je obstaja $n$-ti odvod polja $f$
na tej domeni in je zvezen.
Polje je razreda $C^n$ na zaprtju neke domene, če je razreda $C^n$ v notranjosti te domene in obstaja zvezna razširitev
$n$-tega odvoda na zaprtje domene.

\[
	\frac{\partial^n}{\partial t^n}f = \underbrace{\frac{\partial}{\partial t}
	\bigg(\frac{\partial}{\partial t}\bigg(\cdots \bigg(\frac{\partial}{\partial t}}_{n-\mathrm{krat}}f\bigg)\cdots\bigg)\bigg).
\]
Za tako polje $f$ pravimo, da \emph{je razreda $C^N$} na neki odprti domeni prostora $\E\times\R$, če na tej domeni obstajajo odvodi
\begin{equation}\label{e:Nodv}
	\frac{\partial^m}{\partial t^m}D^n f,\quad \textrm{za vse}\quad 0\leq m,n\leq N,\quad m+n=N,
\end{equation}
in so zvezni. Polje je razreda $C^n$ na zaprtju neke domene v $\E\times\R$, če je razreda $C^N$
v notranjosti te domene in obstajajo zvezne razširitve polj (\ref{e:Nodv}) na zaprtje domene.

\subsection{Gradient in divergenca}

Naj bo $f$ skalarno polje. V dani točki $x\in\E$ je $Df$ linearna preslikava iz $V$ v $\R$, torej linearen
funkcional, ki mu po Riezsovem izreku pripada enolično določen vektor $\nabla f$, imenovan
\emph{gradient skalarnega polja} $f$ v točki $x$, tako da velja
\[
	Df[\vek{h}]=\nabla f \cdot \vek{h}.
\]
V komponentni obliki je
\[
	\nabla f = \frac{\partial f}{\partial x^k}\,\vek{g}^k.
\]
\emph{Gradient vektorskega polja} $\vek{u}$ v točki $x\in\E$ je zgolj drugo ime za linearno preslikavo
$D\vek{u}\in\L(V)$ v točki $x$. Označili ga bomo z $\nabla\vek{u}$. Gradiente vektorjev iz naravne baze
v dani točki zapišemo v komponentni obliki glede na naravno bazo:
\begin{align*}
	\nabla \vek{g}_i&=\Gamma_i=\cs{i}{j}{k}\,\vek{g}_j\otimes\vek{g}^k\\
	\nabla \vek{g}^i&=\Gamma^i=\topbot{\Gamma}{i}{jk}\,\vek{g}^j\otimes\vek{g}^k.
\end{align*}
Koeficienti $\cs{i}{j}{k}$ in $\topbot{\Gamma}{i}{jk}$ se imenujejo \emph{Christoffelovi simboli} in
ne predstavljajo koeficientov tenzorja tretjega reda. Navedimo nekaj lastnosti Christoffelovih
simbolov (izpeljavo lahko najdemo npr. v \cite[str. 275--277]{liu} ali \cite[str. 59]{haupt}):
\begin{equation*}
	\cs{j}{i}{k}=-\topbot{\Gamma}{i}{jk},\quad \topbot{\Gamma}{i}{jk}=\topbot{\Gamma}{i}{kj},\quad
	\cs{j}{i}{k}=\cs{k}{i}{j},
\end{equation*}
\begin{equation*}
	\cs{i}{j}{k}=\frac{1}{2}\,g^{jl}\left(\frac{\partial g_{li}}{\partial x^k}+
	\frac{\partial g_{lk}}{\partial x^i}-\frac{\partial g_{ik}}{\partial x^l}\right).
\end{equation*}
Komponentna oblika gradienta vektorskega polja $\vek{u}$ je
\begin{equation*}
	\nabla\vek{u}= \topbot{u}{j}{,k}\:\vek{g}_j\otimes\vek{g}^k,\quad
	\topbot{u}{j}{,k}=\frac{\partial u^j}{\partial x^k}+u^i\,\cs{i}{j}{k}.
\end{equation*}
Izraz $\topbot{u}{j}{,k}$ označuje t.~i.~\emph{kovariantni odvod} komponente $u^j$
po koordinati $x^k$. V kartezijevih koordinatah so vsi Christoffelovi simboli enaki 0 in
kovariantni odvod je kar običajni parcialni odvod.

Medtem ko gradient viša red tenzorskega polja, ga divergenca niža.
\emph{Divergenca vektorskega polja} $\vek{u}$ je skalarno polje
\begin{equation*}
	\div\vek{u}=\tr(\nabla\vek{u})=\topbot{u}{j}{,j}=\frac{\partial u^j}{\partial x^j}+u^k\,\cs{k}{j}{j}.
\end{equation*}
\emph{Divergenca tenzorskega polja} $S\in\L(V)$ je vektorsko polje $\div S$ z lastnostjo,
da za vsako konstantno vektorsko polje $\vek{v}$ velja
\[ \vek{v}\cdot\div S = \div(S^T\vek{v}). \]
V komponentni obliki je
\[ \div S = \topbot{S}{ij}{,j}\,\vek{g}_i, \]
kjer je $\topbot{S}{ij}{,j}$ kovariantni odvod komponente $S^{ij}$ po koordinati $x^j$,
\[ \topbot{S}{ij}{,j} = \frac{\partial S^{ij}}{\partial x^j}+S^{ik}\,\cs{k}{j}{j}. \]

\emph{Deformacijski gradient} gibanja $\chi$ je
\[ F\colon\B\times I\to \L(V) \qquad F=\nabla\chi(\vek{X},t). \]
Glede na mešano bazo $\{\vek{g}_j(\vek{x})\otimes\vek{g}^k(\vek{X})\}$ prostora $\L(V)$,
kjer je $\vek{x}=\chi(\vek{X},t)$, ima $F$ preprosto komponentno obliko
\[ F(\vek{X},t) = \frac{\partial x^j}{\partial X^k}\vek{g}_j(\vek{x})\otimes\vek{g}^k(\vek{X}). \]