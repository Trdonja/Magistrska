\chapter*{Zaključek}
\addcontentsline{toc}{chapter}{Zaključek}


Pri izpeljavi enačb gibanja za kontinuum nas Hamiltonov princip pripelje do enakega rezltata,
kot običajne, direktne metode. Hamiltonov princip torej zares predstavlja alternativo običajnim
metodam, vendar ga veliko monografij kljub temu ne omenja. Morda zato, ker je zanj
vseeno potrebno nekaj več matematičnega predznanja, kot je recimo variacijski račun
in nekaj osnov funkcionalne analize.

Za uporabo Hamiltonovega principa nam ni bilo potrebno predhodno definirati
zakonov dinamike (Eulerjev prvi in drugi zakon), zato se nam tudi ni bilo
potrebno ukvarjati z inercialnimi opazovališči. Prav tako smo se lahko
izognili nekaterim dinamičnim količinam, kot sta recimo gibalna in vrtilna količina,
definirati smo morali le osnovne
kinematične količine, kot so hitrost, pospešek in deformacijski gradient.
Ni nam bilo potrebno predhodno definirati Cauchyjevega napetostnega tenzorja
na običajen način, preko Cauchyjevega izreka. Po drugi strani pa iz Hamiltonovega
principa nismo dobili enačbe za ohranitev vrtilne količine.
Kljub temu, da smo s pomočjo Hamiltonovega principa dobili enačbo za ohranitev vztrajnostnega momenta,
smo morali lokalno obliko zakona o ohranitvi mase vseeno izpeljati neodvisno.

Poleg podanih primerov (elastični fluid in hiperelastično trdno telo) je
mogoče Hamiltonov princip uporabiti za splošno materialno telo iz
zveznega medija, tudi če je iz neelastičnega materiala. Pri tem se
izraz za notranjo energijo definira v splošni obliki (\cite[str.~45]{bedford}) in iz Hamiltonovega
principa ravno tako dobimo enačbo za ohranitev linearnega momenta, v
kateri pa nastopajo neznane konstitucijske količine. Te je potrebno določiti
na podoben način, kot smo to storili v dodatku za hiperelastično trdno telo,
kjer se upošteva princip materialne objektivnosti (neodvisnost od opazovališča)
ter materialne simetrije. Nekaj končnih parametrov je potrebno določiti
še eksperimentalno.

Prvotno je bilo načrtovano, da v to delo vključim še Hamiltonov princip
za materiale s singularnimi ploskvami, vendar se je izkazalo, da bi bila
stroga matematična formulacija, kot je predstavljena v tem délu za gladka polja,
za materiale s singularnimi ploskvami mnogo bolj zapletena in za študijo
tega področja bi porabil mnogo več časa.

Možne razširitve Hamiltonovega principa v mehaniki kontinuma je veliko.
Poleg pri prej omenjenih materialov s singularnimi ploskvami se ga uporablja
še za mešanice in materiale z mikrostrukturami. Zanimivo bi bilo raziskati
tudi možnosti za uporabo v termodinamiki in piezoelektriki.