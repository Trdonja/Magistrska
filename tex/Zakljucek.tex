\chapter*{Zaključek}
\addcontentsline{toc}{chapter}{Zaključek}


Pri izpeljavi enačb gibanja za kontinuum nas Hamiltonov princip pripelje do enakega rezultata,
kot običajne, direktne metode. Hamiltonov princip torej zares predstavlja alternativo običajnim
metodam, vendar ga veliko monografij kljub temu ne omenja. Morda zato, ker je zanj
vseeno potrebno nekaj več matematičnega predznanja, kot je recimo variacijski račun
in nekaj osnov funkcionalne analize.

Za uporabo Hamiltonovega principa nam ni bilo potrebno predhodno definirati
zakonov dinamike (Eulerjev prvi in drugi zakon), zato se nam tudi ni bilo
potrebno ukvarjati z inercialnimi opazovališči. Prav tako smo se lahko
izognili nekaterim dinamičnim količinam, kot sta recimo gibalna in vrtilna količina,
definirati smo morali le osnovne
kinematične količine, kot so hitrost, pospešek in deformacijski gradient.
Ni nam bilo potrebno predhodno definirati Cauchyjevega napetostnega tenzorja
na običajen način, preko Cauchyjevega izreka.
Lokalno obliko zakona o ohranitvi mase smo izpeljali neodvisno.

Poleg podanih primerov (elastični fluid in hiperelastično trdno telo) je
mogoče Hamiltonov princip uporabiti za splošno materialno telo iz
zveznega medija, tudi če je iz neelastičnega materiala. Pri tem se
izraz za notranjo energijo definira v splošni obliki (\cite[str.~45]{bedford}) in iz Hamiltonovega
principa ravno tako dobimo enačbo za ohranitev gibalne količine, v
kateri pa nastopajo neznane konstitucijske količine. Te je potrebno določiti
na podoben način, kot smo to storili v dodatku za hiperelastično trdno telo,
kjer se upošteva princip materialne objektivnosti (neodvisnost od opazovališča)
ter materialne simetrije. Nekaj končnih parametrov je potrebno določiti
še eksperimentalno.

Možnih razširitev Hamiltonovega principa v mehaniki kontinuuma je veliko.
Zanimivo bi bilo raziskati tudi možnosti za uporabo v termodinamiki, piezoelektriki
in pri drugih oblikah materialov s kompleksno strukturo.