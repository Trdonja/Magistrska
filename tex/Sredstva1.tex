\chapter{Predhodna sredstva}


V tem poglavju bomo navedli nekaj bistvenih pojmov in rezultatov iz matematične analize
in mehanike kontinuuma, ki so potrebni za razumevanje nadaljnje vsebine. Predpostavlja
se, da bralec te pojme in rezultate že pozna, zato večine izrekov in trditev
tu ne bomo dokazovali. Razdelek služi bolj predstavitvi oznak, ki jih
bomo uporabljali v nadaljevanju.


\section{Odvod na Banachovem prostoru}


\begin{definicija} \label{def:odvod}
	Naj bosta $(V,\|\cdot\|_V)$, $(W,\|\cdot\|_W)$ Banachova prostora, $U\subseteq V$ odprta
	množica in $F\colon U\to W$ preslikava. $F$ je \emph{diferenciabilna} oz.~\emph{odvedljiva}
	(v smislu Frécheta) v točki $u\in U$, če obstaja omejena (ekvivalentno: zvezna) linearna preslikava
	\[ DF(u)\colon V\to W, \]
	da za vsak $v\in V$ velja
	\begin{equation} \label{e:frede}
		\lim_{v\to 0}\frac{\|F(u+v)-F(u)-DF(u)(v)\|_W}{\|v\|_V}=0.
	\end{equation}
	$F$ je \emph{diferenciabilna na $U$}, če je diferenciabilna v vsaki točki $u\in U$.
	$DF(u)$ imenujemo \emph{(krepki ali Fréchetov) odvod} preslikave $F$.
	
	$F$ \emph{je razreda $C^1$} na $U$ (pišemo tudi: $F\in C^1(U,W)$ ali pa $F\in C^1(U)$,
	če je kodomena jasna iz konteksta), če je preslikava $u\mapsto DF(u)$ zvezna.
	$F$ \emph{je razreda $C^2$} na $U$ ($F\in C^2(U,W)$ ali $F\in C^2(U)$),
	če je preslikava $u\mapsto D^2F(u)=D(DF)(u)$ zvezna. Induktivno definiramo še $D^rF(u)$ in pojem
	$F$ \emph{je razreda $C^r$} za poljuben $r\in\mathbb{N}$. $F$ je \emph{gladka} na $U$,
	če je $F\in C^r(U,W)$ za vsak $r\in\mathbb{N}$.
\end{definicija}

Če odvod odbstaja, je enoličen. Diferenciabilna preslikava je hkrati tudi zvezna.
Linearna preslikava med končnorazsežnima Banachovima prostoroma je vedno omejena
oz.~zvezna. Pogoj (\ref{e:frede}) je ekvivalenten pogoju
\begin{equation*} \label{e:odekvi}
	F(u+v)=F(u)+DF(u)(v)+o(v)\quad\textrm{in}\quad
	\lim_{v\to 0}\frac{\|o(v)\|_W}{\|v\|_V}=0.
\end{equation*}

\begin{definicija}
	Naj bosta $V$, $W$ Banachova prostora, $U\subseteq V$ odprta
	množica in $F\colon U\to W$ preslikava. \emph{Šibki odvod} (tudi: \emph{Gâteauxov} ali
	\emph{smerni odvod}) preslikave $F$ v $u\in U$ in v smeri $v\in V$ je
	\[
		\delta F(u)(v) =
		\lim_{\varepsilon\to 0}\frac{F(u+\varepsilon v)-F(u)}{\varepsilon}
		=\at{\frac{d}{d\varepsilon}F(u+\varepsilon v)}{\varepsilon = 0},
	\]
	če obstaja.
\end{definicija}

\begin{trditev}
	Naj bo $F$ kot v prejšnjih definicijah. Če je $F$ odvedljiva v $u\in U$,
	potem obstaja šibki odvod preslikave $F$ v $u$ v smeri vsakega vektorja $v\in V$
	in velja
	\[
		\delta F(u)(v)=DF(u)(v).
	\]
\end{trditev}

\begin{izrek}[Odvod kompozituma]
	Naj bodo $V$, $W$, $Y$ Banachovi prostori, $U\subseteq V$ in $\Omega\subseteq W$ odprti
	množici in $F\colon U\to W$ ter $G\colon\Omega\to Y$ odvedljivi preslikavi,
	tako da je $F(U)\subseteq\Omega$.
	Potem je $G\circ F\colon U\to Z$ odvedljiva in za vsak $u\in U$ velja
	\begin{equation} \label{e:odkompo}
		D(G\circ F)(u)=DG(F(u))\circ DF(u).
	\end{equation}
\end{izrek}
Če je $v\in V$, potem (\ref{e:odkompo}) pomeni
\[
	D(G\circ F)(u)(v)=DG(F(u))(DF(u)(v)).
\]
Ta izrek je poznan tudi pod imenom \emph{verižno pravilo}.

\begin{izrek}[Odvod produkta]
	Naj bodo $V$, $W_1$, $W_2$, $Y$ \textcolor[rgb]{1,0,0}{končnorazsežni} Banachovi prostori,
	$U\subseteq V$ odprta množica,
	\[
		F\colon U\to W_1,\qquad G\colon U\to W_2
	\]
	odvedljivi preslikavi in
	\[ \pi\colon W_1\times W_2\to Y \]
	bilinearna preslikava. Potem je produkt
	\[
		H\colon U\to Y, \qquad H(u)=\pi(F(u),G(u))
	\]
	odvedljiv v vsakem $u\in U$ in za vsak $v\in V$ velja
	\[
		DH(u)(v)=\pi\big(DF(u)(v),G(u)\big)+\pi\big(F(u),DG(u)(v)\big).
	\]
\end{izrek}
Pri tem bilinearna preslikava $\pi$ predstavlja katerikoli produkt, ki je bilinearen,
npr.~produkt vektorja s skalarjem, skalarni produkt vektorjev, vektorski produkt vektorjev,
tenzorski produkt vektorjev, produkt linearne prelikave in vektorja (aplikacija), itd.

\begin{izrek}[Izrek o inverzni funkciji]
	Naj bosta $V$, $W$ Banachova prostora, $U\subseteq V$ odprta in $x_0\in U$. Naj bo
	$F\colon U\to W$ razreda $C^1$ in predpostavimo, da je $DF(x_0)$ obrnljiva in
	ima zvezni inverz. Potem obstajata okolici $U_1\subseteq U$ točke $x_0$ in
	$U_2\subseteq W$ točke $y_0=F(x_0)$, tako da $F$ preslika $U_1$ bijektivno na
	$U_2$, inverz $\phi:=F^{-1}\colon U_2\to U_1$ je \textcolor[rgb]{1,0,0}{zvezno?} odvedljiv
	in velja
	\[ D\phi(y_0)=(DF(x_0))^{-1}. \]
\end{izrek}

\begin{definicija}
	Naj bosta $V$, $W$ Banachova prostora, $U\subseteq V$ odprta in $F\colon U\to W$ preslikava. 
	$F$ je razreda $C^r$ ($r\in\mathbb{N}$) na zaprtju $\overline{U}$, če je razreda $C^r$ na $U$ in
	obstaja zvezna razširitev preslikave $x\mapsto D^rF(x)$ na $\overline{U}$.
\end{definicija}

\begin{comment}
Naj bosta $(V_1,\|\cdot\|_1)$, $(V_2,\|\cdot\|_2)$ Banachova prostora. $V=V_1\times V_2$
je vektorski prostor, z ustrezno definirano normo $\|\cdot\|_V$, porojeno iz norm $\|\cdot\|_1$ in $\|\cdot\|_2$,
pa postane Banachov prostor. Naj bo $F\colon U\to W$, kjer je $U\subseteq V$ odprta množica, in naj bosta
\[ U_1=\{x\in V_1\;;\ (x,y_0)\in U\}\quad\textrm{ter}\quad U_2=\{y\in V_2\;;\ (x_0,y)\in U\}. \]
Če v točki $(x_0,y_0)\in U$ obstajata
\begin{itemize}
\item odvod preslikave $F(\cdot,y_0)\colon U_1\to W$, ki ga označimo z $\partial_1 F(x_0,y_0)$ in
\item odvod preslikave $F(x_0,\cdot)\colon U_2\to W$, ki ga označimo z $\partial_2 F(x_0,y_0)$
\end{itemize}
in sta oba zvezna, potem je $F$ odvedljiva v $(x_0,y_0)$ in velja
\[ DF(x_0,y_0)(u,v)=\partial_1 F(x_0,y_0)(u)+\partial_2 F(x_0,y_0)(v). \]
\end{comment}


\section{Tenzorska analiza}


\subsection{Evklidski prostor}


V klasični mehaniki opisujemo dogodke v \emph{Newtonovem prostor-času}, kar je produkt
trirazsežnega Evklidskega prostora ter prostora realnih števil $\R$. Evklidski prostor nam služi
za opis položaja in geometrije objektov, prostor $\R$ pa predstavlja časovno os.

\begin{definicija} \label{d:ep}
	Množica točk $\E$ je \emph{Evklidski točkovni prostor} in trirazsežni Evklidski vektorski prostor $\V$ je
	\emph{translacijski prostor} za $\E$, če poljubnemu paru točk $x,y\in\E$ pripada vektor iz $V$,
	ki ga zapišemo kot $y-x$, tako da velja:
	\begin{enumerate}
		\item Za vsak $x\in\E$ je $x-x=\vek{0}$, ničelni vektor.
		\item Za vsak $x\in\E$ in vsak $\vek{v}\in V$ obstaja natanko ena točka $y\in\E$, da je
		$\vek{v}=y-x$. Pišemo: $y=x+\vek{v}$.
		\item Za vse $x,y,z\in\E$ velja $(y-x)+(z-y)=(z-x)$.
	\end{enumerate}
\end{definicija}

Razdalja med točkama $x,y\in\E$ je definirana kot $d(x,y)=\|y-x\|$, kjer $\|.\|$ označuje normo na
vektorskem prostoru $V$, porojeno iz standardnega skalarnega produkta. $(\E,d)$ je metrični prostor.

V prostoru $\E$ si izberimo točko, jo označimo z $o$ in jo imenujmo \emph{izhodišče}.
V skladu z aksiomi iz definicije pripada vsaki točki $p\in\E$ glede na izhodišče $o$ \emph{krajevni vektor}
$\vek{x}=x-o\in V$. Naj bo $\{\vek{e}_1,\vek{e}_2,\vek{e}_3\}$ neka ortonormirana baza prostora $V$,
ki je desnosučna, tj.~$\vek{e}_3=(\vek{e}_1\times\vek{e}_2)$. Krajevni vektor $\vek{x}$
ima enoličen razvoj po bazi, $\vek{x}=x_i\vek{e}_i$, koeficientom $x_i$ tega razvoja pa rečemo
\emph{kartezijeve koordinate}. Te so odvisne od izbire izhodišča in ortonormirane baze.

Z izbiro izhodišča postane $\E$ vektorski prostor, preslikava
\[
	\iota\colon\E\to\V,\qquad\iota\colon x\mapsto x-o=\vek{x}
\]
pa je izomorfizem vektorskih prostorov. $\E$ torej lahko smatramo kot $\V\cong\R^3$.

Včasih si želimo prostor $\E$ opremiti s kakšnim drugim koordinatnim sistemom, ki ga podamo
s koordinatno transformacijo kartezijevih koordinat
\begin{equation}\label{e:kt}
	\theta^j = \theta^j(x_1,x_2,x_3) \ \Leftrightarrow \ x_i=x_i(\theta^1,\theta^2,\theta^3), \quad i,j=1,2,3.
\end{equation}
Koordinatnim sistemom se bomo bolj podrobno posvetili v razredlku \ref{s:koordinate}.

Imamo torej bijektivno korespondenco med naslednjimi objekti:
\begin{itemize}
	\item točka $x\in\E$,
	\item krajevni vektor $\vek{x}=x_i\vek{e}_i\in V$,
	\item koordinate $(x_1,x_2,x_3)$ ali $(\theta^1,\theta^2,\theta^3)\in\R^3$.
\end{itemize}
Zato bomo točke iz prostora $\E$ v bodoče identificirali z njihovimi krajevnimi
vektorji ali pa z njihovimi koordinatami.
Vsako preslikavo $F\colon U\to \E$, kjer je $U\subseteq V$ in $V$ neki Banachov prostor,
bomo, še posebej, kadar bomo imeli opravka z odvodi, zaradi izomorfnosti prostorov
$\E$ in $\V$ obravnavali kot preslikavo $F\colon U\to\V$.

Newtonov prostor čas $\mathcal{N}=\E\times\R$ lahko naredimo za vektorski prostor s skalarnim produktom
\[ (x,t_1)\cdot(y,t_2) = \vek{x}\cdot\vek{y} + t_1 t_2. \]
kjer sta $\vek{x}$ in $\vek{y}$ krajevna vektorja točk $x$ in $y$.
Iz skalarnega produkta dobimo tudi normo in metriko.

\begin{definicija}
	\begin{itemize}
		\item
		Za krivuljo ali ploskev v prostoru $\E$ bomo rekli, da je razreda $C^r$, $r\in\mathbb{N}$, če se jo da
		parametrizirati s preslikavo, ki je razreda $C^r$.
		\item
		Območje $B\subset\E$ je \emph{regularno}, če je omejeno in je njegov rob $\partial B$
		orientabilna ploskev, sestavljena iz končnega števila ploskev razreda $C^r$ za neki $r\geq 1$.
		\item
		Točka na ploskvi v prostoru $\E$ je \emph{regularna}, če obstaja enotska normala
		na to ploskev v tej točki, sicer je \emph{neregularna}.
	\end{itemize}
\end{definicija}
Množica vseh neregularnih točk na robu regularnega območja ima ploščin\-sko mero oz.~ploščino enako 0.


\subsection{Tenzorska polja, gradient in divergenca}


Rezervirajmo oznako $\W$ za poljuben končnorazsežen normiran vektorski
prostor nad obsegom $\R$. Običajno bo
\[ \W\in\big\{ \R,\V,\L(\V),\L(\V,\L(\V)) \big\}. \]
Elemente oz.~vektorje prostora $\V$ bomo vedno označevali s krepkimi poševni\-mi simboli,
npr.~$\vek{u}$, $\vek{\mu}$, $\vek{F}$. Elemente prostora $\L(\V)$, t.j.~linearne
preslikave iz $\V$ v $\V$, bomo vedno označevali s krepkimi pokončnimi simboli, npr.~$\ten{F}$,
in jih včasih imenovali tudi \emph{tenzorji drugega reda} ali pa na kratko kar \emph{tenzorji}.
Elemente prostora $\W$ bomo običajno označevali s krepkimi poševnimi simboli, torej kot vektorje.
Funkcije, katerih kodomena je $\V$, $\L(\V)$ ali $\W$, bomo označevali kot elemente teh prostorov.

Na vektorskem prostoru $\L(\V)$ definiramo še običajen skalarni produkt: Za $\ten{A},\ten{S}\in\L(\V)$ je
\[ \ten{A}\cdot\ten{S}=\tr(\ten{A}^{T}\ten{S}). \]
Hitro se lahko prepričamo, da ta predpis res ustreza vsem pogojem za skalarni produkt.

Naj bo $U\subseteq\E$, običajno odprta. Funkciji $\vek{f}\colon U\to\W$ rečemo \emph{tenzorsko polje} ali pa
\emph{tenzorska funkcija}. V posebnem primeru, ko je $\W=\R$, ji rečemo \emph{skalarno polje},
v primeru $\W=\V$ pa \emph{vektorsko polje}.

Naj bo $\psi\colon U\to\R^3$ bijektivna preslikava razreda $C^1$, katere inverz je prav tako
razreda $C^1$. $\psi$ priredi točki v $U$ njene koordinate, torej gre za koordinatni sistem.
Funkciji $\vek{f}\circ\psi^{-1}$ in $\vek{f}\circ\iota^{-1}$ bomo z zlorabo notacije pogosto
označevali enako, kot $\vek{f}$, torej
\[
	\vek{f}(x_1,x_2,x_3):=\vek{f}(\psi^{-1}(x_1,x_2,x_3)),\qquad
	\vek{f}(\vek{x}):=\vek{f}(\iota^{-1}(\vek{x})).
\]

\begin{definicija}
	\emph{Gradient} tenzorskega polja $\vek{f}$ je odvod funkcije $\vek{f}\circ\iota^{-1}$,
	in ga pišemo kot $\nabla\vek{f}$ ali $\grad\vek{f}$.
\end{definicija}
Uporabljali bomo obe oznaki. V skladu s to definicijo je gradient preslikava
$\nabla\vek{f}\colon\iota(U)\to\L(\V,\W)$, vendar ga dojemamo tudi kot tenzorsko polje
$\nabla\vek{f}\colon U\to\L(\V,\W)$ na očiten način: za $x\in U$,
$\vek{x}=\iota(x)$ in $\vek{v}\in\V$ je
\[
	\nabla\vek{f}(x)(\vek{v})=D(\vek{f}\circ\iota^{-1})(\vek{x})(\vek{v}).
\]

Če je $f\colon U\to\R$ skalarno polje, potem v $x\in U$ vrednost gradienta $\nabla f(x)$
pripada prostoru $\L(\V,\R)$, kar je linearni funkcional na prostoru $\V$. Po Riezsovem
izreku o reprezentaciji mu enolično pripada vektor $\vek{u}\in\V$, da za vsak
$\vek{v}\in\V$ velja $\nabla f(x)(\vek{v})=\vek{u}\cdot\vek{v}$. V tem primeru velja
dogovor, da identificiramo $\nabla f$ s pripadajočim vektorskim poljem in pišemo
\[ \nabla f(x)(\vek{v})=\nabla f(x)\cdot\vek{v}. \]

Če je $\vek{f}\colon U\to\V$ vektorsko polje, potem je v $x\in U$ vrednost gradienta $\nabla\vek{f}(x)$
linearna preslikava iz $\V$ v $\V$ in pišemo
\[ \nabla\vek{f}(x)(\vek{v})=\nabla\vek{f}(x)\vek{v}, \]
torej oklepaj opustimo, kot je to tudi sicer v navadi za linearne preslikave.

Oznaka $I$ bo praviloma služila za odprti interval $I=(t_1,t_2)\subset\R$. Tudi funkciji
$\vek{f}\colon U\times I\to\W$ bomo rekli \emph{(časovno odvisno) tenzorsko polje}.
Gradient časovno odvisnega tenzorskega polja je definiran kot gradient polja
$\vek{f}(\cdot,t)\colon U\to\W$, v katerem je spremenljivka $t$ fiksna. Označimo
ga enako, torej $\nabla\vek{f}=\nabla\vek{f}(x,t)$ ali $\grad\vek{f}=\grad\vek{f}(x,t)$.
Odvod funkcije $\vek{f}(x,\cdot)\colon U\to\W$, v kateri je spremenljivka $x$ fiksna,
bomo imenovali \emph{časovni odvod} in ga bomo označili kot parcialni odvod
$\partial\vek{f}/\partial t$.

Medtem ko gradient viša red tenzorskega polja, ga divergenca niža.
\begin{definicija} \label{def:div}
	\emph{Divergenca vektorskega polja} $\vek{u}\colon U\to\V$ je skalarno polje
	\begin{equation} \label{e:div1}
		\div\vek{u}=\tr(\nabla\vek{u}).
	\end{equation}
	\emph{Divergenca tenzorskega polja} $\ten{S}\colon U\to\L(\V)$ je vektorsko polje $\div\ten{S}$ z lastnostjo,
	da za vsako konstantno vektorsko polje $\vek{v}$ velja
	\begin{equation} \label{e:div2}
		\vek{v}\cdot\div\ten{S} = \div(\ten{S}^{T}\vek{v}).
	\end{equation}
\end{definicija}

Podajmo nekaj lastnosti gradienta in divergence, ki jih bomo potrebovali v nadaljevanju.
\begin{trditev} \label{t:divprop}
	Za diferenciabilno skalarno polje $\phi$ ter diferenciabilni vektorski polji $\vek{u}$ in $\vek{v}$ velja
	\begin{enumerate}
		\item $\nabla(\phi\vek{v})=\vek{v}\otimes\nabla\phi+\phi\nabla\vek{v},$
		\item $\nabla(\vek{u}\cdot\vek{v})=(\nabla\vek{u})^T\vek{v}+(\nabla\vek{v})^T\vek{u},$
		\item $\div(\phi\vek{v})=\vek{v}\cdot\nabla\phi+\phi\div\vek{v}$,
		\item $\div(\vek{u}\otimes\vek{v})=(\nabla\vek{u})\vek{v}+\vek{u}\div\vek{v}$.
	\end{enumerate}
\end{trditev}
\proof
	Za poljubno konstantno vektorsko polje $\vek{h}$ je
	\begin{align*}
		\nabla(\phi\vek{v})\vek{h}&=
		(\nabla\phi\cdot\vek{h})\vek{v}+\phi(\nabla\vek{v})\vek{h}=
		=(\vek{v}\otimes\nabla\phi)\vek{h}+\phi(\nabla\vek{v})\vek{h}\\
		&=\big(\vek{v}\otimes\nabla\phi+\phi\nabla\vek{v}\big)\vek{h},
	\end{align*}
	iz česar sledi prva enakost, ter
	\begin{align*}
		\nabla(\vek{u}\cdot\vek{v})\vek{h}&=((\nabla\vek{u})\vek{h})\cdot\vek{v}+\vek{u}\cdot(\nabla\vek{v})\vek{h}=
		(\nabla\vek{u})^T\vek{v}\cdot\vek{h}+(\nabla\vek{v})^T\vek{u}\cdot\vek{h}\\
		&=\big((\nabla\vek{u})^T\vek{v}+(\nabla\vek{v})^T\vek{u}\big)\vek{h},
	\end{align*}
	iz česar sledi druga enakost. Tretjo enakost dokažemo neposredno:
	\begin{align*}
		\div(\phi\vek{v})&=\tr\big(\nabla(\phi\vek{v})\big)=\tr(\vek{v}\otimes\nabla\phi+\phi\nabla\vek{v})\\
		&=\tr(\vek{v}\otimes\nabla\phi)+\phi\tr(\nabla\vek{v})=\vek{v}\cdot\nabla\phi+\phi\div\vek{v}.
	\end{align*}
	Pri tem smo uporabili definicijo divergence (\ref{e:div1}), prvo enakost trditve in dejstvi, da je sled
	linearen operator ter da je sled tenzorskega produkta dveh vektorjev enaka skalarnemu produktu teh
	dveh vektorjev.
	
	Za poljubno konstantno vektorsko polje $\vek{w}$ je
	\begin{align*}
		\vek{w}\cdot\div(\vek{u}\otimes\vek{v})&=\div\big((\vek{u}\otimes\vek{v})^T\vek{w}\big)=
		\div\big((\vek{v}\otimes\vek{u})\vek{w}\big)\\
		&=\div\big((\vek{u}\cdot\vek{w})\vek{v}\big)=\vek{v}\cdot\nabla(\vek{u}\cdot\vek{w})+(\vek{u}\cdot\vek{w})\div\vek{v}\\
		&=(\nabla\vek{u})\vek{v}\cdot\vek{w}+\vek{u}\div\vek{v}\cdot\vek{w}=\big((\nabla\vek{u})\vek{v}+\vek{u}\div\vek{v}\big)\cdot\vek{w},
	\end{align*}
	iz česar sledi četrta enakost. Pri tem smo uporabili definicijo divergence (\ref{e:div2}) ter
	tretjo in drugo enakost trditve, pri čemer smo upoštevali, da je $\nabla\vek{w}=\ten{0}$
	(ničelna linearna preslikava).
\endproof


\subsection{Integralski izreki}


Ploščinsko mero objektov v prostoru $\E$ bomo označili z $s$ ali $S$, volumsko
mero pa z $v$ ali $V$. Pripadajoča infinitezimalna elementa sta $ds$ in $dv$ oz.~$dS$
in $dV$.

\begin{trditev}\label{t:oiz}
	Naj bo $U\subseteq\E$ odprta množica ter $f\colon U\to \R$ zvezno skalarno polje.
	Če za vsako podmnožico $N\subseteq U$ velja
	\[ \int_{N} f\,dv=0, \]
	potem je $f(x)=0$ za vsak $x\in U$.
\end{trditev}
\proof
	Recimo, da obstaja $x_0\in U$, da je $f(x_0)> 0$. Če je $f(x_0)< 0$, potem gledamo polje $-f$.
	Ker je $f$ zvezno, obstaja okolica $N\subseteq U$ točke $x_0$ z volumnom $v(N)>0$, tako da je $f(x)>0$
	za vsak $x\in N$. Po izreku o povprečni vrednosti iz analize obstaja točka $\xi\in N$, da je
	\[ \int_{N}f\,dv=v(N)f(\xi)> 0, \]
	kar nasprotuje predpostavki iz izreka, torej take točke $x_0\in U$ ni.
\endproof

\begin{izrek}[Divergenčni izrek] \label{i:divtheo}
	Naj bo $B$ regularno območje v $\E$ in $\vek{n}(x)\in\V$ zunanja enotska normala
	na rob $\partial B$ v regularni točki $x\in\partial B$. Naj bodo $\phi\colon\overline{B}\to\R$,
	$\vek{u}\colon\overline{B}\to V$ ter $\ten{S}\colon\overline{B}\to\L(V)$
	zvezna polja, diferenciabilna v notranjosti območja $B$. Potem velja
	\begin{align*}
		\int_{\partial B} \phi\vek{n}\,ds &= \int_{B} \grad\phi\,dv, \\
		\int_{\partial B} \vek{u}\cdot\vek{n}\,ds &= \int_{B} \div\vek{u}\,dv, \\
		\int_{\partial B} \ten{S}\vek{n}\,ds &= \int_{B} \div\ten{S}\,dv.
	\end{align*}
\end{izrek}
\proof
	Druga enakost je splošno znan \emph{Gaussov izrek} iz vektorske analize, zato ga tukaj ne bomo dokazovali.
	Naj bo $\vek{w}$ poljubno konstantno vektorsko polje. Potem je
	\begin{align*}
		\vek{w}\cdot\int_{\partial B}\phi\vek{n}\,ds &= \int_{\partial B}\phi\vek{w}\cdot\vek{n}\,ds=
		\int_{B}\div(\phi\vek{w})\,dv \\
		&= \int_{B}\vek{w}\cdot\grad\phi \,dv = \vek{w}\cdot\int_{B}\grad\phi \,dv,
	\end{align*}
	s čimer smo dokazali prvo enakost.
	Pri tem smo uporabili Gaussov izrek, tretjo enakost iz trditve (\ref{t:divprop}) ter
	upoštevali, da je $\div\vek{w}=0$.
	
	Zopet naj bo $\vek{w}$ poljubno konstantno vektorsko polje. Potem je
	\begin{align*}
		\vek{w}\cdot\int_{\partial B}\ten{S}\vek{n}\,ds &= \int_{\partial B}\vek{w}\cdot\ten{S}\vek{n}\,ds=
		\int_{\partial B}\ten{S}^T\vek{w}\cdot\vek{n}\,ds=
		\int_{B}\div(\ten{S}^T\vek{w})\,dv \\ &=\int_{B}\vek{w}\cdot\div\ten{S}\,dv=
		\vek{w}\cdot\int_{B}\div\ten{S}\,dv,
	\end{align*}
	kjer smo zopet uporabili Gaussov izrek in definicijo divergence (\ref{e:div2}). S tem smo dokazali še tretjo enakost.
\endproof