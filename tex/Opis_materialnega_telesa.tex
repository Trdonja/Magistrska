\chapter{Matematični opis materialnega telesa}

\section{Materialno telo}

V tem razdelku bomo podali abstraktno definicijo materialnega telesa. Ker bomo kasneje
potrebovali variacijski račun, moramo materialno telo definirati kot trirazsežno gladko
mnogoterost, da bomo na voljo dobili vsa sredstva iz matematične analize.
V klasični mehaniki se materialno telo umesti v Evklidski prostor. Ta povezava bo opisana v razdelku
\ref{sec-ep}.

\begin{definicija}
	\emph{Materialno telo} $\B$ je abstraktna množica točk $\P$, za katero
	obstaja množica $\mathscr{K}$ bijektivnih preslikav $\chi$
	\begin{align*}
		\chi \colon \B &\rightarrow \chi [\B] \subseteq \R^3 \\
		\P &\mapsto \chi(\P) = (x^1,x^2,x^3)\ \Leftrightarrow \ \P = \chi^{-1}(x^1,x^2,x^3),
	\end{align*}
	imenovanih \emph{konfiguracije}, tako da je za
	poljubni konfiguraciji
	\begin{equation*}
		\begin{array}{rlcrl}
		\chi_1 \colon \B &\rightarrow \chi_1 [\B] \subseteq \R^3 &\ \mathrm{in}\ &
		\chi_2 \colon \B &\rightarrow \chi_2 [\B] \subseteq \R^3 \\
		\P &\mapsto \chi_1(\P) = (x^1,x^2,x^3)&&
		\P &\mapsto \chi_2(\P) = (y^1,y^2,y^3)
		\end{array}
	\end{equation*}
	kompozitum
	\begin{align*}
		\chi_2\circ\chi_1^{-1} \colon\phantom{xxx} \chi_1 [\B] &\rightarrow \chi_2 [\B] \\
		(x^1,x^2,x^3) &\mapsto (y^1,y^2,y^3) = \chi_2\left(\chi_1^{-1}(x^1,x^2,x^3)\right)
	\end{align*}
	gladka preslikava.
	Točke $\P\in\B$ imenujemo tudi \emph{delci} ali pa \emph{materialne točke}.
\end{definicija}
Vsaka konfiguracija je parametrizacija materialnega telesa: vsaki materialni točki enolično priredi trojico
realnih števil -- \emph{koordinate}.
\begin{figure}[ht]
	\centering
	\begin{picture}(300,185)
		\put(150,163){\circle*{2}} \put(145,168){$\P\in\B$}
		\put(145,160){\vector(-1,-1){100}} \put(57,110){$\chi_1(\P)$}
		\put(155,160){\vector(1,-1){100}} \put(211,110){$\chi_2(\P)$}
		\put(45,55){\vector(1,0){210}} \put(100,38){$y^k=\chi_2\left(\chi_1^{-1}(x^l)\right)$}
		\put(40,55){\circle*{2}} \put(15,35){$(x^1,x^2,x^3)$}
		\put(260,55){\circle*{2}} \put(235,35){$(y^1,y^2,y^3)$}
		\thinlines
		\qbezier(10,13)(0,60)(10,90)
		\qbezier(8,13)(40,30)(70,20)
		\qbezier(68,18)(75,50)(90,85)
		\qbezier(92,83)(50,95)(8,88)
		\put(30,3){$\chi_1[\B]\subseteq \R^3$}
		\qbezier(210,20)(250,30)(300,20)
		\qbezier(298,18)(285,50)(298,85)
		\qbezier(300,83)(260,70)(210,85)
		\qbezier(212,87)(225,55)(212,18)
		\put(230,3){$\chi_2[\B]\subseteq \R^3$}
		\put(150,163){\oval(90,43)} \put(90,175){$\B$}
	\end{picture}
	\caption{Konfiguracije}
\end{figure}
\begin{definicija}
	\emph{Sklicna konfiguracija} je poljubno izbrana, a fiksna konfiguracija $\r\in\mathscr{K}$,
	\begin{align*}
		\r \colon \B &\rightarrow \r [\B] \subseteq \R^3 \\
		\P &\mapsto \r(\P) = (X^1,X^2,X^3)\ \Leftrightarrow \ \P = \r^{-1}(X^1,X^2,X^3).
	\end{align*}
\end{definicija}
\begin{definicija}
	\emph{Gibanje} materialnega telesa v časovnem intervalu $I\subseteq\R$ je enoparametrična
	družina konfiguracij $\{\chi_t \, ; \ t\in I \} \subset \mathscr{K}$,
	\begin{align*}
		\chi_t \colon \B &\rightarrow \chi_t [\B] \subseteq \R^3 \\
		\P &\mapsto \chi_t(\P) = \left(x^1(t),x^2(t),x^3(t)\right)\ \Leftrightarrow \ 
		\P = \chi_t^{\phantom{t}-1}(x^1,x^2,x^3),
	\end{align*}	
	z lastnostjo, da je za vsak $\P\in\B$ preslikava $t\mapsto\chi_t(\P)$
	gladka na celotnem intervalu $I$. Konfiguracijo $\chi_t$ iz take družine imenujemo
	\emph{trenutna konfiguracija} pri času $t$.
\end{definicija}
Gibanje običajno predstavimo s preslikavo
\begin{align}
	&\chi_{\r}\colon\r[\B]\times I\rightarrow\chi_t[\B]\subseteq\R^3 \nonumber \\
	&\chi_{\r}(X^1,X^2,X^3,t) = \chi_t\left(\r^{-1}(X^1,X^2,X^3)\right),\label{eq-chir}
\end{align}
ki se jo da izraziti s tremi realnimi funkcijami štirih spremenljivk
\begin{equation}\label{eq-xk}
	x^k=\chi_{\r}^{\phantom{\r}k}(X^1,X^2,X^3,t)\quad\mathrm{za}\ k=1,2,3.
\end{equation}
Pri opisu gibanja na tak način se znebimo abstraktnih točk $\P$, na voljo pa dobimo vsa sredstva iz analize.
\begin{definicija}
	V predstavitvi (\ref{eq-xk}) se številska trojica $(X^1,X^2,X^3)$ imenuje \emph{materialne},
	trojica $(x^1,x^2,x^3)$ pa \emph{prostorske koordinate} materialne točke
	\[ \P=\r^{-1}(X^1,X^2,X^3)=\chi_t^{\phantom{t}-1}(x^1,x^2,x^3). \]
\end{definicija}
\begin{figure}[ht]
	\centering
	\begin{picture}(320,185)
		\put(150,163){\circle*{2}} \put(145,168){$\P\in\B$}
		\put(145,160){\vector(-1,-1){100}} \put(57,110){$\r(\P)$}
		\put(155,160){\vector(1,-1){100}} \put(211,110){$\chi_t(\P)$}
		\put(45,55){\vector(1,0){210}} \put(105,38){$x^k=\chi_{\r}^{\phantom{\r}k}\left(X^L,t\right)$}
		\put(40,55){\circle*{2}} \put(10,35){$(X^1,X^2,X^3)$}
		\put(260,55){\circle*{2}} \put(218,35){$\left(x^1(t),x^2(t),x^3(t)\right)$}
		\thinlines
		\qbezier(10,13)(0,60)(10,90)
		\qbezier(8,13)(40,30)(80,20)
		\qbezier(78,18)(85,50)(90,85)
		\qbezier(92,83)(50,95)(8,88)
		\put(30,3){$\r[\B]\subseteq \R^3$}
		\qbezier(210,20)(250,30)(320,20)
		\qbezier(318,18)(310,50)(318,85)
		\qbezier(320,83)(270,70)(210,85)
		\qbezier(212,87)(220,55)(212,18)
		\put(230,3){$\chi_t[\B]\subseteq \R^3$}
		\put(150,163){\oval(90,43)} \put(90,175){$\B$}
	\end{picture}
	\caption{Sklicna in trenutna konfiguracija, materialne in prostorske koordinate.}
\end{figure}
\begin{opomba}\label{op-chirinv}
	Preslikava $\chi_{\r}$ nima inverza, saj ni bijektivna. Če nanjo gledamo kot na časovno odvisno
	preslikavo iz $\r[\B]$ v $\chi_t[\B]$, ta potem ima časovno odvisni inverz, zato na smiselen način definiramo
	\begin{align*}
		&\chi_{\r}^{\phantom{\r}-1}\colon\chi_t[\B]\times I\rightarrow \r[\B]\subseteq\R^3 \\
		&\chi_{\r}^{\phantom{\r}-1}(x^1,x^2,x^3,t) = \r\left(\chi_t^{\phantom{t}-1}(x^1,x^2,x^3)\right).
	\end{align*}
\end{opomba}
Z enačbami (\ref{eq-xk}) je podana časovno odvisna koordinatna transformacija,
njena Jacobijeva matrika je
\begin{equation*}
	J = \left[\begin{array}{ccc}
			\frac{\partial x^1}{\partial X^1} & \frac{\partial x^1}{\partial X^2} & \frac{\partial x^1}{\partial X^3} \\[10pt]
			\frac{\partial x^2}{\partial X^1} & \frac{\partial x^2}{\partial X^2} & \frac{\partial x^2}{\partial X^3} \\[10pt]
			\frac{\partial x^3}{\partial X^1} & \frac{\partial x^3}{\partial X^2} & \frac{\partial x^3}{\partial X^3}
			\end{array}\right].
\end{equation*}
Ker so konfiguracije gladke bijektivne preslikave, je determinanta matrike $J$ različna od 0.


\section{Konfiguracije v Evklidskem prostoru}\label{sec-ep}

V klasični mehaniki konfiguracijam materialnega telesa pripisujemo enakovredne
konfiguracije v Evklidskem prostoru. Na tak način opremimo materialno telo
z Evklidsko geometrijo, kar nam omogoča merjenje razdalij, kotov ter
ostalih izpeljanih fizikalnih količin, kot je npr. deformacija.

\begin{definicija}
	\emph{Evklidski prostor} dimenzije $n$ je množica točk $\E$, za katero obstaja
	realen $n$-razsežen vektorski prostor $V$, imenovan \emph{translacijski prostor},
	tako da vsakemu paru točk $x,y$ iz $\E$ enolično pripada vektor iz $V$, ki ga označimo
	z $\vec{xy}$, tako da veljajo naslednje lastnosti:
	\begin{enumerate}
		\item $\vec{xy}=-\vec{yx} \quad \forall\: x,y\in\E$,
		\item $\vec{xy}+\vec{yz}=\vec{xz} \quad \forall\: x,y,z\in\E$,
		\item če je $o$ poljubna, a fiksna točka iz $\E$, potem za vsako točko $x\in\E$ obstaja
		enolično določen vektor $\vek{v}\in V$, da je $\vek{v}=\vec{ox}$.
	\end{enumerate}
\end{definicija}
Točki $o$ iz tretjega aksioma rečemo \emph{izhodišče}, vektorju $\vek{v}$ pa \emph{krajevni
vektor} točke $x$ glede na izhodišče $o$. Odslej bomo oznako $\E$ uporabljali za trirazsežen
Evklidski prostor, $V$ pa za njegov translacijski prostor, ki je dodatno opremljen
s standardnim skalarnim produktom, kar nam bo omogočalo računanje geometrijskih količin.
\begin{definicija}
	Množico $V_x=\{\vec{xy}\ |\ y\in\E\}$ imenujemo \emph{tangentni prostor} Evklidskega prostora
	$\E$ v točki $x\in\E$.
\end{definicija}
$V_x$ je vektorski prostor, ki je kopija prostora $V$. Če sta $V_x$ in $V_y$ tangentna prostora,
je $V_y$ translacija prostora $V_x$ za vektor $\vec{xy}$. Vektorje iz različnih tangentnih prostorov
lahko med sabo seštevamo, kot da bi bili iz istega vektorskega prostora.

\begin{definicija}\label{def-ks}
	Naj bosta $\mathcal{D}\subseteq \E$ in $U\subseteq\R^3$ odprti množici. Bijektivno gladko preslikavo
	$\Psi\colon\mathcal{D}\rightarrow U$, katere inverz $\Psi^{-1}\colon U\rightarrow\mathcal{D}$
	je tudi gladka preslikava, imenujemo \emph{koordinatni sistem} za $\mathcal{D}\subseteq \E$.
\end{definicija}

V prostoru $\E$ si izberemo izhodišče $o$, $\{\vek{e}_1,\vek{e}_2,\vek{e}_3\}$ pa naj bo ortonormirana
baza prostora $V_o\cong V$. Poljubni točki $x$ iz $\E$ pripada glede na izhodišče $o$ krajevni
vektor $\vek{u}_x$ z enoličnim
razvojem po bazi, $\vek{u}_x=y_k\vek{e}_k$. Preslikava $x\mapsto (y_1,y_2,y_3)$, ki točki $x$ priredi
trojico koeficientov razvoja njenega krajevnega vektorja po bazi $\{\vek{e}_k\}$, je očitno koordinatni sistem,
ki ga imenujemo \emph{Kartezijev koordinatni sistem}. 
Povezavo med konfiguracijo $\chi$ in njej enakovredno konfiguracijo v Evklidskem prostoru
podaja koordinatni sistem $\psi \colon\mathcal{D}^{\mathrm{odp}}\subseteq\E\rightarrow U\subseteq\R^3$,\label{ks}
ki je podan s koordinatno transformacijo Kartezijevih koordinat,
\begin{equation*}
	x^k = \psi^k(y_1,y_2,y_3) \ \Leftrightarrow \ y_k=\varphi_k(x^1,x^2,x^3), \quad k=1,2,3.
\end{equation*}
Inverz od $\psi$ bomo označili s $\varphi=\psi^{-1}$. Če je torej $\chi\in\mathscr{K}$ konfiguracija,
je njena pripadajoča konfiguracija v Evklidskem prostoru definirana kot
\begin{align*}
	\varphi\circ\chi\colon\B&\rightarrow\varphi\big[\chi[\B]\big]=\body\subseteq\E\\
	\P&\mapsto\varphi\big(\chi(\P)\big).
\end{align*}
\begin{opomba}
Običajno je, da se za konfiguracijo $\varphi\circ\chi$ z zlorabo notacije uporabi kar oznako $\chi$. V takem primeru
moramo biti pozorni, ko naletimo na oznako za $\chi$, ali gre za preslikavo 
$\chi\colon\B\rightarrow\R^3$ ali za $\chi\colon\B\rightarrow\E$.
\end{opomba}
{\color[rgb]{1,0,0}SLIKA $\B$ levo, $\body$ desno, $\chi[\B]$ spodaj, kartezijev KS zgoraj, puščice vmes.}

Gibanje v Evklidskem prostoru lahko tako predstavimo s preslikavo, ki jo, prav tako z zlorabo notacije,
označimo enako, kot v (\ref{eq-chir}), in je
\begin{align}
	\chi_{\r} \colon\body_{\r}\times I&\rightarrow \body_{t} \nonumber \\
	(X,t) &\mapsto x=\varphi\big(\chi_{\r}(\psi(X),t)\big), \label{eq-chire}
\end{align}
kjer je
\begin{equation*}
	\body_{\r}=\varphi\big[\r[\B]\big]\subseteq\E \quad\mathrm{in}\quad \body_{t}=\varphi\big[\chi_t[\B]\big]\subseteq\E.
\end{equation*}
%V predstavitvi $x=\chi_{\r}(X,t)$ rečemo $x$ \emph{prostorske}, $X$ pa \emph{materialne točke}.
Podobno, kot v opombi \ref{op-chirinv}, tudi tukaj na smiselen način definiramo
\begin{align}
	\bottop{\chi}{\r}{-1} \colon\body_{t}\times I&\rightarrow \body_{\r} \nonumber \\
	(x,t) &\mapsto X=\varphi\big(\bottop{\chi}{\r}{-1}(\psi(x),t)\big). \label{eq-chirinve}
\end{align}


\section{Tenzorska polja}

Funkciji $f\colon\mathcal{D}\times I\rightarrow W$, kjer je $\mathcal{D}\subseteq\E$, $I\subseteq\R$ časovni
interval, $W$ pa nek normiran vektorski prostor, rečemo \emph{tenzorsko polje}. V posebnem primeru, ko je $W=\R$,
rečemo $f$ \emph{skalarno}, v primeru $W=V$ pa \emph{vektorsko polje}. Tenzorsko polje $f$ lahko
preko koordinatnega sistema definiramo tudi na podmnožici $\psi[\mathcal{D}]$ v $\R^3$, in sicer 
\begin{equation*}
	\tilde{f}\colon\psi[\mathcal{D}]\times I\rightarrow W, \quad
	\tilde{f}\colon(x^1,x^2,x^3,t)\mapsto f\big(\varphi(x^1,x^2,x^3),t\big).
\end{equation*}
Zopet je običajno, da se $\tilde{f}$ označi kar z $f$.

\subsection{Zapis v naravni bazi}

Tenzorje, ki so elementi nekega vektorskega prostora $W$, lahko predstavimo v komponentni obliki
glede na neko bazo prostora $W$. Posebno pomemben je dualni par baz, ki ga v točki
\[
	(x^1,x^2,x^3)\in \R^3\ \xrightleftharpoons[\ \psi\ ]{\ \varphi\ }\ 
	x\in\E\ \xrightleftharpoons[\ \phantom{\varphi}\ ]{\ \phantom{\psi}\ }\ 
	\vek{x}=y_i\vek{e}_i \in V_o
\]
definira koordinatni sistem\footnote{Za oznake glej stran \pageref{ks}.}:
\begin{align*}
	\vek{g}_k(x)&=\frac{\partial}{\partial x^k}\varphi(x^1,x^2,x^3)=
	\frac{\partial}{\partial x^k}\varphi_j(x^1,x^2,x^3)\,\vek{e}_j \\
	\vek{g}^k(x)&=\grad \psi^k(y_1,y_2,y_3)=
	\frac{\partial}{\partial y_j}\psi^k(y_1,y_2,y_3)\,\vek{e}_j.
\end{align*}
$\vek{g}_k$ imenujemo \emph{tangentni}, $\vek{g}^k$ pa \emph{gradientni vektorji}.
\begin{trditev}\label{tr-nb}
	Množici $\{\vek{g^k(x)}\}$ in $\{\vek{g_j(x)}\}$ tvorita bazo tangentnega prostora $V_x$.
	Bazi sta si dualni, tj. $\vek{g}^i(x)\cdot\vek{g}_j(x)=\topbot{\delta}{i}{j}$.
\end{trditev}
Trditev je splošno znana iz tenzorske analize, zato dokaza tukaj ne bomo navajali; bralec ga lahko
najde npr. v \cite[str.~273]{liu}. Bazi iz trditve \ref{tr-nb} imenujemo \emph{naravni bazi} v točki
$x\in\E$ glede na koordinatni sistem $\psi$. Definirajmo še \emph{metrične koeficiente}:
\begin{equation*}
	g^{ij}(x)=\vek{g}^i(x)\cdot\vek{g}^j(x) \quad\mathrm{in}\quad g_{ij}(x)=\vek{g}_i(x)\cdot\vek{g}_j(x).
\end{equation*}

Tenzorska polja običajno zapisujemo v komponentni obliki glede na naravno bazo. Če je
$\vek{u}$ vektorsko polje z vrednostmi v $V$, $S$ pa tenzorsko polje z vrednostmi v $V\otimes V$,
potem ju v točki $x\in\E$ pri nekem danem času v naravni bazi zapišemo kot
\begin{align*}
	\vek{u}&=u_j\,\vek{g}^j=u^k\,\vek{g}_k,\\
	S&=\bottop{S}{i}{j}\,\vek{g}^i\otimes\vek{g}_j=\topbot{S}{i}{j}\,\vek{g}_i\otimes\vek{g}^j=
	S^{ij}\,\vek{g}_i\otimes\vek{g}_j=S_{ij}\,\vek{g}^i\otimes\vek{g}^j.
\end{align*}
Predpostavljamo, da bralec pozna pravila za višanje in nižanje indeksov s pomočjo metričnih koeficientov
ter pravila za transformacijo komponent tenzorjev pri prehodu iz ene baze v drugo.

\subsection{Odvajanje}

Posvetimo se sedaj odvajanju tenzorskih polj. Pri matematični analizi se spoznamo z odvodi med normiranimi prostori,
običajno v poglavju o variacijskem računu.
Če je $F:D^{\mathrm{odp}}\subseteq W_1\rightarrow W_2$ preslikava med normiranima prostoroma $W_1$ in $W_2$,
potem je njen odvod $DF$ element prostora $W_2\otimes W_1$ (tenzorski produkt vektorskih prostorov).
Za dosledno definicijo odvoda tenzorske funkcije na Evklidskem prostoru, ki sam po sebi še ni vektorski prostor,
potrebujemo naslednji dogovor.
\begin{enumerate}
	\item Če je $x\in\E$ in $\vek{v}\in V$, potem je $x+\vek{v}=y\in\E$, tako da je $\vek{v}=\vec{xy}$.
	\item Za $x,y\in \E$ je njuna razlika $y-x=\vec{xy}\in V$.
\end{enumerate}
Evklidska metrika $d$ na $\E$ je tako
\begin{equation*}
	d(x,y)=\|y-x\|=\|\vec{xy}\|.
\end{equation*}
\begin{definicija}
	Naj bo $f:\mathcal{D}\times I\rightarrow W$ tenzorsko polje, kjer je $\mathcal{D}$ odprta
	podmnožica v metričnem prostoru ($\E,d)$. $f$ je \emph{odvedljiva} ali \emph{diferenciabilna} v točki $x\in\E$,
	če obstaja taka linearna preslikava $A:V\rightarrow W$, da za vsak $\vek{h}\in V$ velja
	\begin{equation*}
		f(x+\vek{h},t)=f(x,t)+A[\vek{h}]+o(\vek{h}),
	\end{equation*}
	kjer je $o(\vek{h})\in W$ količina, za katero je
	\[ \lim_{\| h\|_V\to 0}\frac{\|o(\vek{h})\|_W}{\|\vek{h}\|_V}=0. \]
	Če taka $A$ obstaja, ji rečemo \emph{gradient} oziroma \emph{(krepki) odvod} funkcije $f$ po spremenljivki $x$
	in ga označimo z $\grad f(x,t)=\nabla f(x,t)$.
\end{definicija}
Imamo: $\nabla f \in W\otimes V$. Odvod, kadar obstaja, je en sam. Gradiente računamo s pomočjo
\emph{šibkega} oziroma \emph{smernega odvoda}:
\begin{equation*}
	\delta f(x,t)[\vek{h}]=\lim_{s\to 0}\frac{1}{s}\Big(f(x+s\vek{h},t)-f(x,t)\Big)=
	\at{\frac{d}{ds}f(x+s\vek{h},t)}{s=0}.
\end{equation*}
Znano je, da če obstaja krepki odvod (včasih imenovan tudi \emph{Fréchetov odvod}),
potem obstaja tudi šibki odvod in sta enaka: $\nabla f(x,t)[\vek{h}]=\delta f(x,t)[\vek{h}]$ za vsak $\vek{h}\in V$.
Za tovrstne odvode veljajo enaki izreki, kot veljajo za preslikave med normiranimi prostori, ki jih
poznamo iz matematične analize: izrek o posrednem odvajanju, izrek o odvajanju produkta\footnote{
Izrek o odvajanju produkta velja za katerikoli produkt, ki je bilinearna preslikava. Med njimi so
npr. skalarni in vektorski produkt vektorjev, produkt tenzorja s skalarjem, tenzorski produkt,
delovanje tenzorja na vektorju itd.},
izrek o totalnem odvodu itd.

Na podoben način definiramo tudi odvod tenzorske funkcije $f$ po času. Tega označimo z $\partial f/\partial t$
in je element prostora $W\otimes\R\cong W$. Totalni odvod funkcije $f$ vključuje tako gradient kot tudi odvod po času.

\subsection{Gradient in divergenca}

Oglejmo si gradiente nekaterih posebnih tenzorskih polj. Za začetek naj bo $f$ skalarno polje. Potem je
$\nabla f \in \R\otimes V \cong V$. Če je
$\psi^k(x)=x^k$ $k$-ta koordinata točke $x\in\E$, potem s posrednim odvajanjem v točki $x$ dobimo
\begin{equation*}
	\nabla f=\frac{\partial f}{\partial x^k}\,\nabla\psi^k=\frac{\partial f}{\partial x^k}\,\vek{g}^k.
\end{equation*}

Vektorji iz naravne baze so (časovno neodvisna) vektorska polja, njihovi gradienti v točki
$x$ pripadajo prostoru $V\otimes V$, torej
so tenzorji drugega reda in jih lahko zapišemo v komponentni obliki glede na naravno bazo:
\begin{align*}
	\nabla \vek{g}_i(x)&=\cs{i}{j}{k}(x)\,\vek{g}_j(x)\otimes\vek{g}^k(x)\\
	\nabla \vek{g}^i(x)&=\topbot{\Gamma}{i}{jk}(x)\,\vek{g}^j(x)\otimes\vek{g}^k(x).
\end{align*}
Koeficienti $\cs{i}{j}{k}$ in $\topbot{\Gamma}{i}{jk}$ se imenujejo \emph{Christoffelovi simboli} in
jih ne smemo zamenjati s koeficienti tenzorjev tretjega reda. Navedimo nekaj lastnosti Christoffelovih
simbolov, katerih izpeljavo lahko najdemo v \cite[str. 275--277]{liu} ali \cite[str. 59]{haupt}:
\begin{equation*}
	\cs{j}{i}{k}=-\topbot{\Gamma}{i}{jk},\quad \topbot{\Gamma}{i}{jk}=\topbot{\Gamma}{i}{kj},\quad
	\cs{j}{i}{k}=\cs{k}{i}{j},
\end{equation*}
\begin{equation*}
	\cs{i}{j}{k}=\frac{1}{2}\,g^{jl}\left(\frac{\partial g_{li}}{\partial x^k}+
	\frac{\partial g_{lk}}{\partial x^i}-\frac{\partial g_{ik}}{\partial x^l}\right).
\end{equation*}

Naj bo sedaj $\vek{f}$ vektorsko polje. Njegov gradient v točki $x$ je
\begin{align*}
	\nabla \vek{f}&=\nabla(f^i\,\vek{g}_i)=\vek{g}_i\otimes\nabla f^i + f^i\,\nabla\vek{g}_i=\\
	&=\vek{g}_i\otimes\frac{\partial f^i}{\partial x^k}\,\vek{g}^k+f^i\,\cs{i}{j}{k}\,\vek{g}_j\otimes\vek{g}^k=\\
	&=\left(\frac{\partial f^j}{\partial x^k}+f^i\,\cs{i}{j}{k}\right)\vek{g}_j\otimes\vek{g}^k
\end{align*}
in je tenzor iz prostora $V\otimes V$. V komponentni obliki ga zapišemo kot
\begin{equation*}
	\nabla\vek{f}= \topbot{f}{j}{,k}\:\vek{g}_j\otimes\vek{g}^k,\quad
	\topbot{f}{j}{,k}=\frac{\partial f^j}{\partial x^k}+f^i\,\cs{i}{j}{k}.
\end{equation*}
Vejica v izrazu $\topbot{f}{j}{,k}$ označuje operacijo, imenovano \emph{kovariantni odvod}.

Medtem ko gradient viša red tenzorskega polja, ga divergenca niža.
\begin{definicija}
	\emph{Divergenca vektorskega polja} $\vek{f}$ v točki $x\in\E$ je
	\begin{equation*}
		\div \vek{f}=\topbot{f}{k}{,k}=\frac{\partial f^k}{\partial x^k}+f^i\,\cs{i}{k}{k}\;\in\R.
	\end{equation*}
\end{definicija}


\section{Materialni in prostorski opis}

Materialno telo spremljajo različne fizikalne količine, ki jih predstavimo s skalarji,
vektorji ali pa tenzorji. Naj $W$ označuje neki normirani prostor
skalarjev, vektorjev ali pa tenzorjev, $I\subseteq\R$ pa naj bo časovni interval.
Fizikalno količino iz prostora $W$, ki pripada materialni točki $\P$ ob času $t$, določa funkcija
\begin{align*}
	f \colon \B\times I &\rightarrow W \\
	(\P,t) &\mapsto w = f(\P,t).
\end{align*}
Z uporabo oznak in dogovorov iz razdelka \ref{sec-ep} podajmo naslednjo definicijo.
\begin{definicija}
	\emph{Prostorski} ali \emph{Eulerjev opis} je izražava fizikalne količine $w=f(\P,t)$ s
	tenzorskim poljem
	\begin{align*}
		&\bar{f}\colon\body_t\times I \rightarrow W \\
		&\bar{f}(x,t)=f\left(\chi_t^{\phantom{t}-1}(x),t\right)=w.
	\end{align*}
	\emph{Materialni} ali \emph{Lagrangejev opis} je izražava fizikalne količine $w=f(\P,t)$ s
	tenzorskim poljem
	\begin{align*}
		&\hat{f}\colon\body_{\r}\times I \rightarrow W \\
		&\hat{f}(X,t)=f\left(\r^{-1}(X),t\right)=w.
	\end{align*}
\end{definicija}
\begin{opomba}
	Preko koordinatnega sistema $\psi$ dobi tenzorsko polje $\bar{f}$ enakovredno izražavo
	$\bar{f}:\chi_t[\B]\times I\rightarrow W$ v pripadajočih prostorskih koordinatah,
	tenzorsko polje $\hat{f}$ pa enakovredno izražavo $\hat{f}:\r[\B]\times I\rightarrow W$
	v materialnih koordinatah.
\end{opomba}
S pomočjo preslikav (\ref{eq-chire}) in (\ref{eq-chirinve}) dobimo prehod med prostorskim in
materialnim opisom:
\begin{subequations}\begin{align}
	\hat{f}(X,t)&=\bar{f}\left(\chi_{\r}(X,t),t\right), \\
	\bar{f}(x,t)&=\hat{f}\left(\chi_{\r}^{\phantom{\r}-1}(x,t),t\right). \label{eq-transf-pro-mat}
\end{align}\end{subequations}
\begin{definicija}
	Če je $w(t) = f(\P,t)$ časovno odvisna fizikalna količina, definirana za materialne točke, potem njen
	časovni odvod
	\begin{equation*}
		\dot{w}(t)=\dot{f}(\P,t)=\frac{d}{dt}f(\P,t)
	\end{equation*}
	imenujemo \emph{materialni odvod}.
\end{definicija}
V materialnem opisu določeni točki $X\in\E$ pripada ena in ista materialna točka ne glede na čas,
zato je materialni odvod kar parcialni odvod po času,
\begin{equation}\label{eq-odv-mat}
	\dot{w}(t)=\frac{\partial \hat{f}}{\partial t}(X,t).
\end{equation}
\begin{primer}
	\emph{Hitrost} $\vek{v}$ delca $\P$ ob času $t$ je vektorska količina,
	definirana kot materialni odvod pozicijske točke $X=\chi_{\r}(\P,t)\in\E$. Materialni
	opis hitrosti je tako
	\begin{equation}\label{eq-velocity}
		\hat{\vek{v}}(X,t)=\frac{\partial \chi_{\r}}{\partial t}(X,t).
	\end{equation}
	Iz enačbe (\ref{eq-transf-pro-mat}) dobimo še prostorski opis hitrosti
	\begin{equation}\label{eq-velocity-pro}
		\bar{\vek{v}}(x,t)=\hat{\vek{v}}\left(\chi_{\r}^{\phantom{\r}-1}(x,t),t\right).
	\end{equation}
\end{primer}
V prostorskem opisu pa določeni točki $x\in\E$ pripadajo različne materialne točke, odvisno od časa.
Zato materialni odvod, ki je totalni odvod po času, v prostorskem opisu dobimo s posrednim odvajanjem
glede na spremenljivko $x=\chi_{\r}(X,t)$. Z upoštevanjem (\ref{eq-velocity}) in (\ref{eq-velocity-pro}) dobimo
\begin{equation}\label{eq-odv-pro}
	\dot{w}(t)=\frac{\partial \bar{f}}{\partial t}(x,t)+\nabla\bar{f}(x,t)\,\bar{\vek{v}}(x,t).
\end{equation}
\begin{primer}
	Materialni odvod hitrosti imenujemo \emph{pospešek} in ga označimo z $\vek{a}$.
	Iz enačb (\ref{eq-odv-mat}) in (\ref{eq-odv-pro}) dobimo materialni in prostorski opis pospeška:
	\begin{align*}
		\hat{\vek{a}}(X,t)&=\frac{\partial \hat{\vek{v}}}{\partial t}(X,t)=
		\frac{\partial^2 \chi_{\r}}{\partial t^2}(X,t) \\
		\bar{\vek{v}}(x,t)&=\frac{\partial \bar{\vek{v}}}{\partial t}(x,t)+\nabla\bar{\vek{v}}(x,t)\,\bar{\vek{v}}(x,t).
	\end{align*}
\end{primer}