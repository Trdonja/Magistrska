\chapter{Kinematika kontinuuma}


Materialno telo je določeno z izbrano množico točk $\B\subseteq\E$. Tej določitvi rečemo
\emph{referenčna konfiguracija} materialnega telesa in služi zgolj označbi telesa; ni nujno,
da se telo dejansko kadarkoli nahaja v tem položaju. Oznaka $\B$ bo odslej vedno prestavljala
referenčno konfiguracijo telesa.


\section{Gibanje}


\begin{definicija}
	\emph{Gibanje} materialnega telesa v časovnem
	intervalu $I=(t_0,t_1)$ je zvezno, časovno odvisno vektorsko polje
	\begin{equation}\label{e:chi}
		\chi\colon\B\times I\to\E,\qquad \chi\colon (\vek{X},t)\mapsto\vek{x},
	\end{equation}
	za katerega je za vsak $t\in I$ preslikava
	\[ \chi_t\colon\B\to\B_t, \qquad \chi_t\colon\vek{X}\mapsto\chi(\vek{X},t)=\vek{x} \]
	bijektivna. Slika $\B_t$ se imenuje \emph{trenutna konfiguracija} materialnega
	telesa \emph{ob času} $t$.
\end{definicija}
Gibanje (\ref{e:chi}) ni bijektivna preslikava, zato nima inverza. Kljub temu na smiselen
način definiramo \emph{inverzno gibanje}
\[ \chi^{-1}(\vek{x},t)\equiv\bottop{\chi}{t}{-1}(\vek{x}), \]
za katerega zahtevamo, da je zvezno na domeni $\{\B_t\times\{t\}\,|\,t\in I\}$.
Dodatno bomo predpostavili, da sta gibanje in inverzno gibanje razreda $C^1$, razen če ne bo navedeno drugače.

%V predstavitvi gibanja (\ref{e:chi}) se krajevni vektor $\vek{X}$ imenuje \emph{materialni vektor},
%točka $X$, ki jo $\vek{X}$ predstavlja, se imenuje \emph{materialna točka} in njene koordinate
%$(X^1,X^2,X^3)$ se imenujejo \emph{materialne koordinate}. Krajevni vektor $\vek{x}$ se imenuje
%\emph{prostorski vektor}, točka $x$, ki jo predstavlja, se imenuje \emph{prostorska točka},
%njene koordinate $(x^1,x^2,x^3)$ pa se imenujejo \emph{prostorske koordinate}.

\emph{Hitrost} $\vek{v}$ in \emph{pospešek} $\vek{a}$ gibanja $\chi$ v časovnem intervalu $I$ sta vektorski polji
\begin{align}
	\vek{v}\colon\B\times I\to V \qquad & \vek{v} = \frac{\partial}{\partial t}\chi(\vek{X},t), \label{e:v} \\
	\vek{a}\colon\B\times I\to V \qquad & \vek{a} = \frac{\partial^2}{\partial t^2}\chi(\vek{X},t). \label{e:a}
\end{align}
Pri tem mora biti $\chi$ vsaj dvakrat odvedljivo po času. \emph{Deformacijski gradient} gibanja $\chi$ je
tenzorsko polje
\begin{equation} \label{e:F} F=\Grad\chi(\vek{X},t). \end{equation}
Za determinanto $J=\det F$ se predpostavi, da je pozitivna.


\section{Materialni in prostorski opis}


Materialno telo spremljajo različne fizikalne lastnosti, katerih vrednosti predstavimo z elementi nekega
normiranega prostora $W$. Vrednosti fizikalnih količin
se med gibanjem v časovnem intervalu $I\subseteq\R$ spreminjajo, opišemo pa jih lahko na
dva različna načina.
\begin{definicija}
	\emph{Materialni} ali \emph{Lagrangejev opis} je predstavitev fizikalne količine $f$, ki spremlja materialno telo
	med gibanjem $\chi$, s tenzorskim poljem
	\[ \hat{f}\colon\B\times I \to W, \qquad \hat{f}\colon (\vek{X},t)\mapsto w. \]
	\emph{Prostorski} ali \emph{Eulerjev opis} je pri vsakem $t\in I$ predstavitev taiste fizikalne količine s tenzorskim poljem
	\[ \bar{f}(\cdot,t)\colon\B_t\to W, \qquad \bar{f}\colon (\vek{x},t)\mapsto w. \]
\end{definicija}
Med materialnim in prostorskim opisom veljata zvezi
\begin{equation}\label{e:mpo-zveza}
	\hat{f}(\vek{X},t) = \bar{f}\big(\chi(\vek{X},t),t\big),\qquad \bar{f}(\vek{x},t) = \hat{f}\big(\chi^{-1}(\vek{x},t),t\big).
\end{equation}
Kasneje bomo strešico ali črtico v oznaki za polje izpustili, če bo jasno iz konteksta, kateri opis uporabljamo.
Veliki $\vek{X}$ v argumentu polja bo vedno ponazarjal materialni opis, mali $\vek{x}$ pa prostorski opis.
Do dvomov lahko pride, kadar ne pišemo argumentov polja, še posebej, če so vključeni odvodi. Temu dvomu se izognemo tako,
da uporabljamo različno notacijo za odvode. V materialnem opisu pišemo oznaki za gradient in divergenco z
veliko začetnico
\[ \Grad f\equiv\grad \hat{f}(\vek{X},t),\qquad \Div f \equiv \div \hat{f}(\vek{X},t), \]
v prostorskem opisu pa z malo
\[ \grad f\equiv\grad \bar{f}(\vek{x},t),\qquad \div f \equiv \div \bar{f}(\vek{x},t). \]
Če je $\phi$ skalarno, $\vek{u}$ pa vektorsko polje, potem je zveza med gradientoma
\begin{equation}\label{e:gz}
	\Grad\phi=F^{\,T}\grad\phi,\qquad \Grad\vek{u}=(\grad\vek{u})F.
\end{equation}
Res, če je $\vek{w}$ poljubno vektorsko polje, s posrednim odvajanjem dobimo
\begin{align*}
	&\Grad\phi\cdot\vek{w}=\grad\phi\cdot(\Grad\chi)\vek{w}=
	\grad\phi\cdot F\vek{w}=F^{\,T}\grad\phi\cdot\vek{w}, \\
	&(\Grad\vek{u})\vek{w}=(\grad\vek{u})(\Grad\chi)\vek{w}
	=(\grad\vek{u})F\vek{w}.
\end{align*}

\begin{definicija}
	Časovni odvod
	\[ \dot{f}=\frac{d}{dt}f \]
	fizikalne količine $f$ imenujemo \emph{materialni odvod}.
\end{definicija}
Materialno odvod meri naglost spreminjanja fizikalne količine v dani materialni točki tekom
časa. Od tod tudi tako poimenovanje. V prostorskem opisu fizikalne količine $f$ je materialni
odvod kar parcialni odvod po času
\[ \dot{\hat{f}}=\frac{\partial}{\partial t}\hat{f}(\vek{X},t). \]
Ker v prostorskem opisu določeno točko prostora tekom časa zavzemajo različne
materialne točke, dobimo materialni odvod v prostorskem opisu s totalnim odvodom po času,
\begin{equation} \label{e:matodv}
	\dot{\bar{f}} = \frac{\partial}{\partial t}\bar{f}(\vek{x},t) +
	\grad \bar{f}(\vek{x},t)\,[\bar{\vek{v}}(\vek{x},t)].
\end{equation}
Pri izpeljavi te enakosti se uporabi še pravilo za posredno odvajanje,
definicijo hitrosti (\ref{e:v}) ter zvezo (\ref{e:mpo-zveza}).
\begin{primer}
	Hitrost in pospešek sta prvi in drugi materialni odvod krajevnega vektorja,
	\[ \vek{v}=\dot{\vek{x}},\qquad \vek{a}=\ddot{\vek{x}}. \]
	Pospešek je materialni odvod hitrosti in se v prostorskem opisu izraža kot
	\[ \vek{a}=\dot{\vek{v}}=\frac{\partial}{\partial t}\vek{v}+L\vek{v}, \]
	kjer je $L=\grad\vek{v}$ t.~i.~\emph{hitrostni gradient}. Iz $\Grad\vek{v}=\Grad{\dot{\vek{x}}}=\dot{F}$ in
	zveze (\ref{e:gz}) dobimo
	\begin{equation} \label{e:L} L=\grad\vek{v}=\dot{F}F^{-1}. \end{equation}
\end{primer}


\section{Površinski in prostorninski element}


Gladka krivulja znotraj telesa je podana kot slika gladke preslikave
\[\vek{C}\colon \Theta\to\Br, \quad \vek{C}\colon\alpha\mapsto\vek{C}(\alpha),\]
kjer je $\Theta$ odprti interval v $\R$.
V trenutni konfiguraciji telesa se ta krivulja nahaja na lokaciji, ki je določena s sliko pripadajoče preslikave
\[ \vek{c}\colon \Theta\to\B_t,\quad \vek{c}\colon\alpha\mapsto\chi\big(\vek{C}(\alpha),t\big). \]
S posrednim odvajanjem dobimo zvezo
\begin{equation} \label{e:3101}
	\vek{c}'(\alpha) = \big( \Grad\chi\big(\vek{C}(\alpha),t\big)\big)\vek{C}'(\alpha).
\end{equation}
\begin{definicija}
	Naj bosta $\vek{C}$ in $\vek{c}$ definirana kot v prejšnjem odstavku.
	Pri danem $\vek{X}=\vek{C}(\alpha_0)$ in $\vek{x}=\vek{c}(\alpha_0)$, $\alpha_0\in\Theta$, imenujemo
	infinitezimalna tangentna vektorja
	\[ d\vek{X}=\vek{C}'(\alpha_0)d\alpha \quad \mathrm{in} \quad d\vek{x}=\vek{c}'(\alpha_0)d\alpha \]
	\emph{materialni dolžinski element} v referenčni oziroma trenutni konfiguraciji.
\end{definicija}

\begin{definicija}
	Če sta $d\vek{X}_1$, $d\vek{X}_2$ in $d\vek{x}_1$, $d\vek{x}_2$ materialna dolžinska elementa,
	potem infinitezimalna vektorja
	\[ d\vek{S}= d\vek{X}_1\times d\vek{X}_2\quad\mathrm{in}\quad d\vek{s}= d\vek{x}_1\times d\vek{x}_2 \]
	imenujemo \emph{materialni površinski element} v referenčni oz. trenutni konfiguraciji.
	Za materialne dolžinske elemente $d\vek{X}_1$, $d\vek{X}_2$, $d\vek{X}_3$
	in $d\vek{x}_1$, $d\vek{x}_2$, $d\vek{x}_3$ se infinitezimalni števili
	\[ dV= d\vek{X}_1\times d\vek{X}_2\cdot d\vek{X}_3\quad\mathrm{in}\quad dv= d\vek{x}_1\times d\vek{x}_2\cdot d\vek{x}_3 \]
	imenujeta \emph{materialni prostorninski element} v referenčni oziroma trenutni konfiguraciji.
\end{definicija}
\begin{trditev}
	Za pare materialnih dolžinskih, površinskih in volumskih elementov $(d\vek{X},d\vek{x})$, $(d\vek{S},d\vek{s})$ in $(dV,dv)$ velja
	\begin{equation}\label{e:dxdX}
		d\vek{x}=\ten{F}d\vek{X},\quad d\vek{s}=J\ten{F}^{-T}d\vek{S}\quad\mathrm{in}\quad
		dv=JdV,
	\end{equation}
	kjer je $\ten{F}$ deformacijski gradient (\ref{e:F}) z determinanto $J>0$.
\end{trditev}
\proof
	Prva enakost sledi neposredno iz (\ref{e:3101}). Tretja enakost sledi iz lastnosti determinante,
	\begin{align*}
		dv&=d\vek{x}_1\times d\vek{x}_2\cdot d\vek{x}_3=
		\ten{F}d\vek{X}_1\times \ten{F}d\vek{X}_2\cdot \ten{F}d\vek{X}_3=\\
		&=Jd\vek{X}_1\times d\vek{X}_2\cdot d\vek{X}_3 = JdV.
	\end{align*}
	Za poljuben vektor $\vek{u}$ je
	\begin{align*}
		d\vek{s}\cdot\vek{u}&=\ten{F}d\vek{X}_1\times\ten{F}d\vek{X}_2\cdot\vek{u}=
		\ten{F}d\vek{X}_1\times\ten{F}d\vek{X}_2\cdot\ten{F}(\ten{F}^{-1}\vek{u})=\\
		&=J d\vek{X}_1\times d\vek{X}_2 \cdot \ten{F}^{-1}\vek{u} =J d\vek{S}\cdot\ten{F}^{-1}\vek{u}=\\
		&=J\ten{F}^{-T}d\vek{S}\cdot\vek{u},
	\end{align*}
	iz česar sledi druga enakost. 
\endproof

Površinski element se običajno piše kot
\[ d\vek{S}=\vek{N}dS \quad\mathrm{oziroma}\quad d\vek{s}=\vek{n}ds, \]
kjer sta
\[
	\vek{N}=\frac{d\vek{X}_1\times d\vek{X}_2}{\|d\vek{X}_1\times d\vek{X}_2\|} \quad\mathrm{in}\quad
	\vek{n}=\frac{d\vek{x}_1\times d\vek{x}_2}{\|d\vek{x}_1\times d\vek{x}_2\|}
\]
\emph{enotski normali} ter
\[ dS=\|d\vek{X}_1\times d\vek{X}_2\| \quad\mathrm{in}\quad ds=\|d\vek{x}_1\times d\vek{x}_2\| \]
\emph{ploščinska elementa}.


\section{Singularna ploskev}


Naj bo $\Theta$ odprta podmnožica v $\R^2$ in $\zeta\colon\Theta\times I\to\E$ preslikava
razreda $C^2$ na $\Theta\times I$, za katero zahtevamo, da je pri vsakem $t\in I$ parcialni
odvod $D_{\theta}\zeta(\theta,t)\colon\R^2\to\V$ linearna preslikava ranga 2
v vsaki točki $\theta=(\theta^1,\theta^2)\in\Theta$, ter da je $\zeta(\cdot,t)\colon\Theta\to\E$
injektivna. Pri danem $t\in I$ je $S(t)=\zeta(\Theta, t)$ ploskev
v $\E$ in $\zeta(\cdot, t)$ je njena regularna parametrizacija. Odslej bomo uporabljali
oznako $S(t)$ za $S(t)\cap B_t$ ter $I_c$ za podinterval v $I$, v katerem je $S(t)\cap B_t$ neprazen.

Dodatno naj velja, da pri vsakem $t\in I_c$ ploskev $S(t)$ razdeljuje materialno telo $B_t$
na disjunktna, med seboj nepovezana dela $B_t^+$ in $B_t^-$. Velja
\[ B_t^+\cup B_t^- = B_t\setminus S(t)\quad\mathrm{in}\quad \overline{B_t^+}\cap \overline{B_t^-}=S(t). \]
Vpeljimo še oznake
\[ \Omega^+=\{(x, t)\: ; \: x\in B_t^+,\ t\in I_c\},\quad\Omega^-=\{(x, t)\: ; \: x\in B_t^-,\ t\in I_c\} \]
\begin{equation*} \textrm{in}\quad\Sigma = \{(x, t)\: ; \: x\in S(t),\ t\in I_c\}. \end{equation*}

\begin{definicija}
	Pravimo, da je gibajoča se ploskev $S(t)$ \emph{singularna} glede na tenzorsko polje
	$f\colon\Omega\to\W$, če je $f$ zvezno na $\Omega^+$ in $\Omega^-$, a ni zvezno na $\Sigma$.
\end{definicija}
\begin{definicija}
	Naj bo $S(t)$ singularna glede na polje $f$. Naj bo $f|_{\Omega^+}$
	skrčitev $f$ na $\Omega^+$ in $f|_{\Omega^-}$ skrčitev $f$ na $\Omega^-$. Označimo z
	$f^+$ zvezno razširitev $f|_{\Omega^+}$ na $\overline{\Omega^+}=\Omega^+\cup\Sigma$ in z
	$f^-$ zvezno razširitev $f|_{\Omega^-}$ na $\overline{\Omega^-}=\Omega^-\cup\Sigma$. \emph{Skok} polja $f$
	je funkcija
	\[ \llbracket f\rrbracket\colon\Sigma\to\W,\qquad \llbracket f\rrbracket=f^+-f^-. \]
\end{definicija}
Kot opombo velja omeniti, da je tako definiran $\llbracket f\rrbracket$ zvezen na $\Sigma$,
če sta normi od $f^+$ in $f^-$ končni na $\Sigma$.
Singularne ploskve predstavljajo širjenje vala skozi prostor v določenem časovnem intervalu.

Označimo parcialne odvode parametrizacije $\zeta$ z
\[
	\zeta_{,\alpha}(\theta^1,\theta^2,t)=\frac{\partial\zeta}{\partial\theta^{\alpha}}(\theta^1,\theta^2,t),\ \alpha=1,2,\qquad
	\zeta_{,t}(\theta^1,\theta^2,t)=\frac{\partial\zeta}{\partial t}(\theta^1,\theta^2,t).
\]
Označimo z $\vek{n}(x,t)$ enotsko normalo na ploskev $S(t)$ v točki $x$, za katero predpostavimo,
da je usmerjena v območje $B_t^+$. Glede na parametrizacijo $\zeta$ je v točki $x=\zeta(\theta^1,\theta^2,t)$
enotska normala
\[
	\vek{n}(x,t)=
	\frac{\zeta_{,1}\times\zeta_{,2}}{\displaystyle\left\| \zeta_{,1}\times\zeta_{,2} \right\|}(\theta^1,\theta^2,t)
\]
in predpostaviti smemo, da je usmerjena v območje $B_t^+$. Če temu ni tako, lahko to
dosežemo z reparametrizacijo. Hitrost točke $(\theta^1,\theta^2)\in\Theta$, ki
ob času $t\in I$ zavzema prostorsko točko $x=\zeta(\theta^1,\theta^2,t)$, je
\[
	\vek{w}(x,t)=\zeta_{,t}(\theta^1,\theta^2,t).
\] 
V splošnem je $\vek{w}(x,t)\neq\vek{v}(x,t)$. Čeprav v obeh primerih $x$ predstavlja isto točko
v prostoru, jo ob času $t$ hkrati zavzemata ploskovna in materialna točka.
$\vek{w}(x,t)$ je hitrost ploskovne točke, $\vek{v}(x,t)$ pa hitrost materialne točke.

Parametrizaciji $\zeta$ pripada parametrizacija $\omega\colon\Theta\times I\to\E_{\r}$, definirana kot
\[ \omega(\theta,t)=\chi^{-1}\big(\zeta(\theta,t),t\big). \]
S $S_{\r}(t)$ označimo sliko $\omega(\Theta, t)$ oz.~kar $\omega(\Theta, t)\cap B$, kar
je ploskev, ki se s časom širi po referenčni konfiguraciji materialnega telesa, in
v časovnem intervalu $I_c$ razdeljuje materialno telo $B$ na disjunktna, med seboj nepovezana
dela. Vpeljimo še oznake
\[ \Omega_{\r}^+=\chi^{-1}(\Omega^+),\quad\Omega_{\r}^-=\chi^{-1}(\Omega^-),\quad\Sigma_{\r}=\chi^{-1}(\Sigma). \]
Za gibanje $\chi$ bomo predpostavili, da je zvezno na $B\times I_c$ ter da je razreda $C^2$ na $\Omega_{\r}^+$
ter $\Omega_{\r}^+$. \textcolor[rgb]{1,0,0}{Zunaj intervala $I_c$ je $\chi$ še vedno razreda $C^2$.}

Če je $f\colon B\to\W$ tenzorsko polje, zvezno na $\Omega_{\r}^+$ in $\Omega_{\r}^-$, ni pa zvezno na
$\Sigma_{\r}$, potem zvezno razširitev skrčitve $f|_{\Omega_{\r}^+}$ na $\overline{\Omega_{\r}^+}=\Omega_{\r}^+\cup\Sigma_{\r}$
označimo z $f^+$. Podobno definiramo še $f^-$. Skok polja $f$ je funkcija
\[ \llbracket f\rrbracket\colon\Sigma_{\r}\to\W,\qquad \llbracket f\rrbracket=f^+-f^-. \]

Hitrost ploskovne točke $\theta\in\Theta$, ki v referenčni konfiguraciji zavzema materialno točko
$X=\omega(\theta,t)$, je
\[
	\vek{w}_{\r}(X,t)=\frac{d}{dt}\omega(\theta, t)=\frac{d}{dt}\chi^{-1}\big(\zeta(\theta,t),t\big).
\]
Če uporabimo verižno pravilo za odvajanje in upoštevamo ? ter ?, dobimo
\begin{align}
	\vek{w}_{\r}&=(\nabla_{x}\chi^{-1})^{\pm}\zeta_{,t}+\left(\frac{\partial}{\partial t}\chi^{-1}\right)^{\pm} \nonumber \\
	&=(\ten{F}^{-1})^+(\vek{w}-\vek{v}^+) \label{e:w} \\
	&=(\ten{F}^{-1})^-(\vek{w}-\vek{v}^-). \nonumber
\end{align}
Čeprav je inverzno gibanje $\chi^{-1}$ zvezno v točkah hiperploskve $\Sigma$, njegov gradient in
parcialni odvod po času nista nujno. Zato smo dobili dve enačbi, iz katerih, če ju med seboj odštejemo, dobimo pogoj
\[ \llbracket \ten{F}^{-1}(\vek{w}-\vek{v}) \rrbracket = \vek{0}. \]


\section{Transportni izrek}


V tem razdelku bomo podali transportni izrek, ki je v bistvu posplošitev izreka o odvajanju
integrala s parametrom, ki ga poznamo iz matematične analize. S transportnim izrekom
bomo dobili enačbo za časovni odvod integrala po območju trenutne konfiguracije materialnega telesa.
Še prej pa potrebujemo formulo za materialni odvod determinante deformacijskega gradienta.

Najprej poiščimo odvod za determinanto $\det\colon\L(V)\to\R$, kjer naj bo tokrat $V$
realni $n$-razsežni vektorski prostor. Naj bo $\omega\colon V^n\to\R$ netrivialna
alternirajoča $n$-linearna forma. Spomnimo, za determinanto in sled velja
\begin{align*}
	&\omega(A\vek{u}_1,\dots,A\vek{u}_n)=(\det A)\,\omega(\vek{u}_1,\dots,\vek{u}_n), \\
	&\sum_{j=1}^n\omega(\vek{u}_1,\dots,A\vek{u}_j,\dots,\vek{u}_n)=(\tr A)\,\omega(\vek{u}_1,\dots,\vek{u}_n).
\end{align*}
Po definiciji krepkega odvoda na Banachovih prostorih velja za odvod determinante
\[ (\partial_{A}\det A)[S] = \det(A+S) - \det A + o(S). \]
%kjer je $o(S)$ količina, za katero velja \[ \lim_{\|S\|\to 0}\frac{|o(S)|}{\|S\|}=0. \]
Torej imamo
\begin{multline*}
	(\partial_{A}\det A)[S]\,\omega(\vek{u}_1,\dots,\vek{u}_n)=\\
	\omega\big((A+S)\vek{u}_1,\dots,(A+S)\vek{u}_n\big) -
	\omega(A\vek{u}_1,\dots,A\vek{u}_n) + o(S).
\end{multline*}
Nadalje uporabimo linearnost forme $\omega$, člene, v katerih nastopa $S$ vsaj dvakrat,
pa vključimo v izraz $o(S)$, in tako iz desne strani zadnje enakosti dobimo
\begin{align*}
	&= \sum_{j=1}^n\omega(A\vek{u}_1,\dots,S\vek{u}_j,\dots,A\vek{u}_n) + o(S) = \\
	&= \sum_{j=1}^n\omega(A\vek{u}_1,\dots,AA^{-1}S\vek{u}_j,\dots,A\vek{u}_n) + o(S) = \\
	&= (\det A)\sum_{j=1}^n\omega(\vek{u}_1,\dots,A^{-1}S\vek{u}_j,\dots,\vek{u}_n) + o(S) = \\
	&= (\det A\big)\tr(A^{-1}S)\,\omega(\vek{u}_1,\dots,\vek{u}_n) + o(S),
\end{align*}
iz česar sledi
\[
	(\partial_{A}\det A)[S] = (\det A)\tr(A^{-1}S) = (\det A)A^{-T}\cdot S=(\det A)A^{-T}[S],
\]
Pravilo za odvod determinante je torej
\[ \partial_{A}\det A=(\det A)A^{-T}. \]
Z uporabo verižnega pravila za odvajanje, pravkar izpeljane enakosti za odvod determinante
ter enačbe (\ref{e:L}) dobimo pravilo za materialni odvod determinante deformacijskega gradienta:
\begin{align}
	\dot{J}&=(\det\ten{F})\dot{}=(\partial_{\ten{F}}\det\ten{F})\big[\dot{\ten{F}}\big]=J\ten{F}^{-T}\cdot\dot{\ten{F}}=
	\nonumber \\ &= J\tr(\ten{F}^{-1}\dot{\ten{F}}) = J\tr(\grad\vek{v})=J\div\vek{v}. \label{e:dotJ}
\end{align}

\begin{izrek}[Transportni izrek] \label{i:transport}
	Naj bo $\mathcal{P}\subseteq\B$ del materialnega telesa v referenčni konfiguraciji in naj
	$\mathcal{P}_t=\chi(\mathcal{P},t)\subseteq\B$ označuje njegovo trenutno konfiguracijo ob času $t$.
	Naj bosta $\hat{f}\colon\mathcal{P}\to W$ in $\bar{f}\colon\mathcal{P}_t\to W$
	materialni oziroma prostorski opis neke fizikalne količine, ki sta razreda $C^1$. Potem velja
	\begin{equation*}
		\frac{d}{dt}\int_{\mathcal{P}_t}\bar{f}\,dv =
		\int_{\mathcal{P}_t}(\dot{\bar{f}}+\bar{f}\div\vek{v})\,dv.
	\end{equation*}
	Nadalje, če je območje $\mathcal{P}_t$ omejeno ter je njegov rob sestavljen iz končnega števila
	gladkih ploskev in če sta $\hat{f}$ in $\bar{f}$ skalarno ali pa vektorsko polje
	razreda $C^1$ na zaprtju $\overline{\mathcal{P}}$ oz.~$\overline{\mathcal{P}_t}$, potem velja
	\begin{equation*}
		\frac{d}{dt}\int_{\mathcal{P}_t}\bar{f}\,dv =
		\int_{\mathcal{P}_t}\frac{\partial\bar{f}}{\partial t}\,dv +
		\int_{\partial\mathcal{P}_t}(\vek{v}\cdot\vek{n})\bar{f}\,ds.
	\end{equation*}
\end{izrek}
\proof
	Z uporabo (\ref{e:dxdX})${}_3$ in (\ref{e:dotJ}) pridemo do
	\begin{align*}
		\frac{d}{dt}\int_{\mathcal{P}_t}\bar{f}\,dv &= \frac{d}{dt}\int_{\mathcal{P}}\hat{f}J\,dV =
		\int_{\mathcal{P}}\frac{d}{dt}(\hat{f}J)\,dV = \int_{\mathcal{P}}(\dot{\hat{f}}J+\hat{f}\dot{J})\,dV =\\
		&=\int_{\mathcal{P}}(\dot{\hat{f}}+\hat{f}\div\vek{v})J\,dV = \int_{\mathcal{P}_t}(\dot{\bar{f}}+\bar{f}\div\vek{v})\,dv.
	\end{align*}
	S čimer je dokazan prvi del izreka.
	Pri tem smo na drugem koraku smeli zamenjati vrstni red odvajanja in integriranja, ker se referenčna konfiguracija
	s časom ne spreminja. Formalno to utemeljimo tako, da odvod zapišemo kot limito diferenčnega kvocienta,
	dejstvo, da se v obeh integralih v diferenčnem kvocientu integrira po enakem območju, izkoristimo za uporabo
	linearnosti integrala in tako celoten diferenčni kvocient spravimo pod en sam integral, nazadnje pa še zamenjamo
	limito in integral, kar tudi smemo po Lebesguevem izreku o dominantni konvergenci, saj je $\dot{\hat{f}}$
	po predpostavki zvezen \textcolor[rgb]{1,0,0}{na kompaktnem območju}, torej omejen.
	
	Če sedaj na desni strani prve enačbe iz izreka razpišemo $\dot{\bar{f}}$ v skladu z enačbo (\ref{e:matodv}),
	nato pa upoštevamo tretjo oz.~četrto enačbo iz trditve \ref{t:divprop} ter nato še drugo oz.~tretjo enakost
	iz divergenčnega izreka \ref{i:divtheo}, odvisno glede na to, ali je $\bar{f}$ skalarno ali
	vektorsko polje, dobimo enačbo iz drugega dela izreka.
\endproof

\begin{izrek}[Transportni izrek]
	Naj bo $S(t)$ singularna ploskev glede na polje $f\colon\Omega\to\W$. Velja
	\begin{equation*}
		\frac{d}{dt}\int_B \hat{f}\,dV = \int_B \frac{\partial \hat{f}}{\partial t}\,dV -
		\int_{S_{\r}(t)} \llbracket\hat{f}\rrbracket(\vek{w}_{\r}\cdot\vek{N})\,dS\quad\textrm{in}
	\end{equation*}
	\[
		\frac{d}{dt}\int_{B_t} f\,dV = \int_{B_t} \frac{\partial f}{\partial t}\,dV +
		\int_{\partial B_t} f(\vek{v}\cdot\vek{n})\,dS - \int_{S(t)} \llbracket f\rrbracket(\vek{w}\cdot\vek{n})\,dS.
	\]
\end{izrek}
\proof
	Izrek bomo dokazali za primer, ko je gibanje singularne ploskve $S(t)$ nepovratno, tj.~parametrizacija
	$\omega$ je injektivna preslikava. Izrek velja tudi sicer, le dokaz je bolj zapleten.
	Najprej dokažimo prvo enakost. Opazimo, da velja
	\[
		\frac{d}{dt}\int_B \hat{f}(X,t)\,dV = \frac{d}{dt}\left(
		\int_{B^-(t)} \hat{f}(X,t)\,dV + \int_{B^+(t)} \hat{f}(X,t)\,dV \right).
	\]
	Posvetimo se le odvajanju prvega integrala, odvajanja drugega integrala poteka na enak način.
	Po definiciji odvoda je
	\begin{gather*}
		\frac{d}{dt}\int_{B^-(t)} \hat{f}(X,t)\,dV =\\ =\lim_{h\to 0} \frac{1}{h}\left(
		\int_{B^-(t+h)} \hat{f}(X,t+h)\,dV - \int_{B^-(t)} \hat{f}(X,t)\,dV \right) \\
		=\lim_{h\to 0} \frac{1}{h}\left( \int_{B^-(t+h)} \hat{f}(X,t+h)\,dV - \int_{B^-(t)} \hat{f}(X,t+h)\,dV \right) + \\
		+ \lim_{h\to 0} \frac{1}{h}\left( \int_{B^-(t)} \hat{f}(X,t+h)\,dV - \int_{B^-(t)} \hat{f}(X,t)\,dV \right) \\
		= \lim_{h\to 0} \frac{1}{h}\int_{B^-(t+h)\setminus B^-(t)} \hat{f}(X,t+h)\,dV +
		\int_{B^-(t)} \frac{\partial\hat{f}}{\partial t}(X,t)\,dV.
	\end{gather*}
	Območje $B^-(t)\setminus B^-(t+h)$ je po začetni predpostavki o nepovratnosti singularne ploskve prazno,
	zato smo integral po tem območju izpustili. Limito bomo računali po komponentah. Polje $\hat{f}$
	lahko namreč zapišemo v komponentni obliki glede na neko fiksno bazo $\{b_k\}_{k=1}^m$ prostora $\W$ kot
	\[ \hat{f}(X,t)=\sum_{k=1}^m \hat{f}_k(X,t)b_k. \]
	Zaradi začetne predpostavke lahko
	območje $B^-(t+h)\setminus B^-(t)$ regularno parametriziramo kar s skrčeno bijektivno parametrizacijo
	\[
		\omega(\theta,\tau)\colon\Theta_{\tau}\times [t,t+h]\to B^-(t+h)\setminus B^-(t),\quad
		\Theta_{\tau}=\{\theta\in \Theta\: ;\: \omega(\theta,\tau)\in B\},
	\]
	zato je
	\begin{multline*}
		F_k(t)=\lim_{h\to 0} \frac{1}{h}\int_{B^-(t+h)\setminus B^-(t)} \hat{f}_k(X,t+h)\,dV=\\
		=\lim_{h\to 0} \frac{1}{h}\int_{t}^{t+h}\iint_{\Theta_{\tau}}\hat{f}_k\big(\omega(\theta,\tau),t+h\big)
		\left|\frac{\partial\omega}{\partial\theta^1}\times\frac{\partial\omega}{\partial\theta^2}\cdot\frac{\partial\omega}{\partial\tau}\right|
		(\theta,\tau)\,d\theta^1 d\theta^2 d\tau.
	\end{multline*}
	Po Lagrangevem izreku o srednji vrednosti obstaja $\xi\in [t,t+h]$, da je
	\[
		F_k(t)=\lim_{h\to 0} \frac{1}{h}h\iint_{\Theta_{\xi}}\hat{f}_k\big(\omega(\theta,\xi),t+h\big)
		\left|\frac{\partial\omega}{\partial\theta^1}\times\frac{\partial\omega}{\partial\theta^2}\cdot\frac{\partial\omega}{\partial\tau}\right|
		(\theta,\xi)\,d\theta^1 d\theta^2.
	\]
	Ko gre $h$ proti $0$, gre $\xi$ proti $t$ in
	\[ \hat{f}_k\big(\omega(\theta,\xi),t+h\big)\to\hat{f}^-_k\big(\omega(\theta,t),t\big). \]
	Upoštevamo še
	\[
		\left|\frac{\partial\omega}{\partial\theta^1}\times\frac{\partial\omega}{\partial\theta^2}\cdot
		\frac{\partial\omega}{\partial\tau}\right|\,d\theta^1 d\theta^2=
		\vek{N}\cdot\vek{w}_{\r}
		\left\|\frac{\partial\omega}{\partial\theta^1}\times\frac{\partial\omega}{\partial\theta^2}\right\|\,d\theta^1 d\theta^2=
		\vek{N}\cdot\vek{w}_{\r}\,dS,
	\]
	pri čemer je skalarni produkt $\vek{w}_{\r}\cdot\vek{N}$ pozitiven, zato smo absolutno vrednost izspustili.
	Hitrost $\vek{w}_{\r}$ in normala $\vek{N}$ namreč zaradi začetne predpostavke vedno kažeta v isti polprostor v $\E$,
	določen s tangentno ravnino, katere normala je $\vek{N}$. Tako dobimo
	\[ F_k(t)=\int_{S_{\r}(t)}\hat{f}_k^-(\vek{w}_{\r}\cdot\vek{N})\,dS \]
	in če vsak $F_k(t)$ pomnožimo z $b_k$, nato pa tvorimo vsoto po vseh $k$, dobimo
	\[
		\lim_{h\to 0} \frac{1}{h}\int_{B^-(t+h)\setminus B^-(t)} \hat{f}(X,t+h)\,dV =
		\int_{S_{\r}(t)}\hat{f}^-(\vek{w}_{\r}\cdot\vek{N})\,dS.
	\]
	Dokazali smo
	\begin{equation} \label{e:vmrez}
		\frac{d}{dt}\int_{B^-(t)} \hat{f}\,dV =
		\int_{S_{\r}(t)}\hat{f}^-(\vek{w}_{\r}\cdot\vek{N})\,dS +
		\int_{B^-(t)} \frac{\partial\hat{f}}{\partial t}\,dV.
	\end{equation}
	Na popolnoma enak način dobimo še
	\begin{equation*}
		\frac{d}{dt}\int_{B^+(t)} \hat{f}\,dV =
		-\int_{S_{\r}(t)}\hat{f}^+(\vek{w}_{\r}\cdot\vek{N})\,dS +
		\int_{B^+(t)} \frac{\partial\hat{f}}{\partial t}\,dV.
	\end{equation*}
	Ploskev $S_{\r}(t)$ je pri tem orientirana nasprotno, kot v prejšnjem primeru,
	zato je njena normala $-\vek{N}$ in posledično je ploskovni integral v tem primeru
	nasprotno predznačen. Če obe dobljeni enačbi seštejemo, dobimo prvo enačbo
	iz izreka.
	
	Drugo enakost bomo dokazali za primer, ko je $f$ skalarno ali pa vektorsko polje.
	Tudi tu integral najprej razcepimo na vsoto dveh integralov,
	enega po območju $B_t^-$ in drugega po območju $B_t^+$. Zopet se posvetimo le odvajanju
	prvega integrala, saj pri računanju odvoda drugega integrala uporabimo povsem enake metode.
	Z uporabo (\ref{e:dxdX})${}_3$ in malo prej izpeljane enakosti (\ref{e:vmrez}) dobimo
	\begin{align}
		\frac{d}{dt}\int_{B_t^-}f\,dV&=\frac{d}{dt}\int_{B^-(t)}\hat{f}J\,dV \nonumber \\
		&=\int_{B^-(t)}\frac{\partial(\hat{f}J)}{\partial t}\,dV + \int_{S_{\r}(t)}\hat{f}^-J^-(\vek{w}_{\r}\cdot\vek{N})\,dS. \label{e:vr2}
	\end{align}
	Integrand v prvem integralu lahko s pomočjo enačbe (\ref{e:dotJ}) za odvod determinante $J$ razpišemo kot
	\[ \frac{\partial(\hat{f}J)}{\partial t}=\left(\frac{\partial\hat{f}}{\partial t}+\hat{f}\widehat{\div\vek{v}}\right)J, \]
	integral pa potem s pomočjo (\ref{e:dxdX}) in (\ref{e:matodv}) transformiramo nazaj v
	\[
		\int_{B^-(t)}\frac{\partial(\hat{f}J)}{\partial t}\,dV =
		\int_{B_t^-}\left(\frac{\partial f}{\partial t}+(\nabla_{x}f)[\vek{v}]+f\div\vek{v}\right)\,dV.
	\]
	Sedaj uporabimo tretjo oz.~četrto enačbo iz trditve \ref{t:divprop} ter nato še drugo oz.~tretjo enakost
	iz divergenčnega izreka \ref{i:divtheo}, odvisno od tega, ali je $f$ skalarno ali
	vektorsko polje, ter dobimo
	\begin{equation} \label{e:vr3}
		\int_{B^-(t)}\frac{\partial(\hat{f}J)}{\partial t}\,dV =
		\int_{B_t^-}\frac{\partial f}{\partial t}\,dV + \int_{\partial B_t^-}f^-(\vek{v}^-\cdot\vek{n})\,dS.
	\end{equation}
	Prevedimo še preostali integral:
	\begin{multline} \label{e:vr4}
		\int_{S_{\r}(t)}\hat{f}^-J^-(\vek{w}_{\r}\cdot\vek{N})\,dS =
		\int_{S_{\r}(t)}\hat{f}^-\big((\ten{F}^{-1})^-\ten{F}^-\vek{w}_{\r}\cdot J^-\vek{N}\big)\,dS= \\
		=\int_{S_{\r}(t)}\hat{f}^-\big(\ten{F}^-\vek{w}_{\r}\cdot J^-(\ten{F}^{-T})^-\vek{N}\big)\,dS
		=\int_{S(t)}f^-\big((\vek{w}-\vek{v}^-)\cdot\vek{n}\big)\,dS.
	\end{multline}
	Tu smo na zadnjem koraku uporabili (\ref{e:dxdX})${}_2$ in $(\ref{e:w})$. V enačbi (\ref{e:vr3})
	lahko ploskovni integral po območju $\partial B_t^-$ zapišemo kot vsoto dveh ploskovnih integralov,
	enega po ploskvi $\partial B_t^-\setminus S(t)$ in drugega po ploskvi $S(t)$.
	Iz (\ref{e:vr2}), (\ref{e:vr3}) in (\ref{e:vr4}) sedaj dobimo
	\[
		\frac{d}{dt}\int_{B_t^-}f\,dV=
		\int_{B_t^-}\frac{\partial f}{\partial t}\,dV + \int_{\partial B_t^-\setminus S(t)}f(\vek{v}\cdot\vek{n})\,dS +
		\int_{S(t)}f^-(\vek{w}\cdot\vek{n})\,dS.
	\]
	Na enak način pridemo do
	\[
		\frac{d}{dt}\int_{B_t^+}f\,dV=
		\int_{B_t^+}\frac{\partial f}{\partial t}\,dV + \int_{\partial B_t^+\setminus S(t)}f(\vek{v}\cdot\vek{n})\,dS -
		\int_{S(t)}f^+(\vek{w}\cdot\vek{n})\,dS.
	\]
	Pri tem je ploskev $S(t)$ orientirana nasprotno, kot v prejšnjem primeru, zato pride ploskovni integral po
	ploskvi $S(t)$ negativno predznačen. Če sedaj seštejemo dobljeni enakosti, dobimo drugo enakost iz izreka.
\endproof


\section{Zakon o ohranitvi mase}


Če je $\mathcal{P}\subseteq\B$ poljuben del materialnega telesa, ki v trenutni konfiguraciji pri času $t$
zavzema območje $\mathcal{P}_t=\chi(\mathcal{P},t)\subseteq\B_t$, potem je \emph{masa} tega déla telesa
ob času $t$ podana z integralom
\[ m(\mathcal{P},t)=\int_{\mathcal{P}_t}\rho\,dv, \]
kjer je
\[ \rho(\cdot,t)\colon\B_t\to(0,\infty) \]
pozitivno integrabilno skalarno polje, imenovano
\emph{masna gostota trenutne konfiguracije}. V klasični mehaniki se prepostavi naslednji zakon.
\begin{aksiom}[Zakon o ohranitvi mase]
	Masa katerega koli déla telesa $\mathcal{P}\subseteq\B$ se z gibanjem telesa ne spreminja, t.j.
	\[ \frac{d}{dt}m(\mathcal{P},t)=\frac{d}{dt}\int_{\mathcal{P}_t}\rho\,dv=0. \]
\end{aksiom}
V skladu z zakonom o ohranitvi mase pripada materialnemu telesu \emph{masna gostota referenčne konfiguracije}
\[ \rho_{\r}\colon\B\to(0,\infty), \]
tako da velja
\[ m(\mathcal{P},t)\equiv m(\mathcal{P})=\int_{\mathcal{P}}\rho_{\r}\,dV=\int_{\mathcal{P}_t}\rho\,dv. \]
Ker to velja za kateri koli del telesa $\mathcal{P}\subseteq\B$, lahko iz lastnosti (\ref{e:dxdX})${}_3$ in trditve \ref{t:oiz}
za masni gostoti $\rho_{\r}$ in $\rho$, če sta zvezni, dobimo zvezo
\[ \rho_{\r}(\vek{X})=J(\vek{X},t)\hat{\rho}(\vek{X},t). \]

\begin{izrek}[Lokalna oblika zakona o ohranitvi mase]
	\textcolor[rgb]{1,0,0}{Predpostavke o $\rho$.} Potem velja
	\begin{equation} \label{e:lozom} \dot{\bar{\rho}}+\rho\div\vek{v}=0. \end{equation}
\end{izrek}
\proof
	Sledi neposredno iz globalne oblike zakona o ohranitvi mase, če uporabimo transportni izrek \ref{i:transport}
	in nato trditev \ref{t:oiz}.
\endproof

\begin{izrek}
	Če je $f$ fizikalna količina materialnega telesa, podana na masno enoto, in je $\bar{f}\colon\B_t\to W$ njen
	prostorski opis, potem za vsak $\mathcal{P}\subseteq\B$ velja
	\[ \frac{d}{dt}\int_{\chi_t(\mathcal{P})}\bar{f}\rho\,dv=\int_{\chi_t(\mathcal{P})}\dot{\bar{f}}\rho\,dv. \]
\end{izrek}
\proof
	Iz transportnega izreka \ref{i:transport} dobimo
	\begin{align*}
		\frac{d}{dt}\int_{\chi_t(\mathcal{P})}\bar{f}\rho\,dv
		&= \int_{\chi_t(\mathcal{P})}\big((\bar{f}\rho)\dot{}+\bar{f}\rho\div\vek{v}\big)\,dv= \\
		&= \int_{\chi_t(\mathcal{P})}(\dot{\bar{f}}\rho+\bar{f}\dot{\rho}+\bar{f}\rho\div\vek{v})\,dv=\\
		&= \int_{\chi_t(\mathcal{P})}\dot{\bar{f}}\rho\,dv+
		\int_{\chi_t(\mathcal{P})}\bar{f}(\dot{\rho}+\rho\div\vek{v})\,dv,
	\end{align*}
	od koder sledi enakost iz izreka, če upoštevamo (\ref{e:lozom}).
\endproof
