\chapter{Hamiltonov princip in variacijske metode}


\section{Dopustno gibanje in variacija}


V tem poglavju bomo s pomočjo t.~i.~\emph{Hamiltonovega principa} poiskali potrebne pogoje,
ki jim mora zadoščati pravo gibanje telesa. Še naprej naj $B$ označuje referenčno konfiguracijo
telesa (zaprto regularno območje v $\E_R$) in $I=[t_1,t_2]\subset\R$ zaprt časovni interval.

Recimo, da pravo gibanje iščemo med vsemi gibanji
$\chi\colon B\times I\to\E$ razreda $C^2$, ki zadoščajo \emph{začetnim pogojem}
\[ \chi(X,t_1)=\chi_{t_1}(X)\quad\textrm{in}\quad \chi(X,t_2)=\chi_{t_2}(X), \]
kjer sta $\chi_{t_1},\chi_{t_2}\colon B\to\E$ znani konfiguraciji razreda $C^2$, ki
določata položaj telesa ob začetnem in končnem času $t_1$ oz.~$t_2$.
Poleg začetnih pogojev so vnaprej predpisani še robni pogoji. \emph{Geometrijski robni pogoj} je predpis
\begin{equation*} \label{e:grp}
	\chi(X,t)=\chi_g(X,t)\quad\mathrm{za}\quad (X,t)\in \partial_g B\times I,
\end{equation*}
kjer je $\partial_g B\subseteq\partial B$ del robu območja $B$.
Geometrijski robni pogoj torej v naprej določa položaj robnih točk telesa.
Na preostalem delu $\partial_d B=\partial B \setminus \partial_g B$
je potem predpisan \emph{dinamični robni pogoj}, ki ga bomo predstavili v naslednjem razdelku.
Dopuščamo možnost, da je $\partial_g B$ ali $\partial_d B$ prazna množica, torej,
da imamo le geometrijske ali pa le dinamične robne pogoje. Vendar pa mora vedno veljati
$\partial_g B\cup\partial_d B=\partial B$ in $\partial_g B\cap\partial_d B=\emptyset$.

Množica $\mathcal{X}$ vseh preslikav $\chi\colon B\times I\to\E$
razreda $C^2$ tvori vektorski prostor.
$\mathcal{X}$ postane Banachov prostor, če na njem smiselno definiramo normo. Možna izbira
predpisa za normo je npr.
\[
	\|\cdot\|\colon\mathcal{X}\to[0,\infty),\quad
	\|\chi\|=\sup\big\{\|\chi(X,t)\|\: ;\ (X,t)\in B\times I\big\}.
\]
Pri tem je z normo, ki se pojavi znotraj množice, mišljena Evklidska norma krajevnega vektorja točke $x=\chi(X,t)$.
Dejansko lahko supremum nadomestimo z maksimumom, ker je vsak $\chi\in\mathcal{X}$
zvezna preslikava na kompaktni domeni $B\times I$.

\begin{definicija}
	\begin{enumerate}
		\item Preslikava $\chi\in\mathcal{X}$, ki zadošča predpisanim začetnim in geometrijskim robnim pogojem,
			se imenuje \emph{dopustno gibanje}. Množico vseh dopustnih gibanj bomo označili z $\mathcal{A}$.
		\item Vektorsko polje $\vek{\eta}\colon B\times I\to\V$, ki je razreda $C^2$ in
			ustreza pogojem
			\[ \vek{\eta}(X,t_1)=\vek{0}\quad \textrm{in} \quad\vek{\eta}(X,t_2)=\vek{0}\quad \textrm{za vse}\ X\in B\quad\textrm{ter} \]
			\[ \vek{\eta}(X,t)=\vek{0}\quad \textrm{za vse}\ (X,t)\in \partial_g B\times I, \]
			se imenuje \emph{variacija gibanja}. Množico vseh variacij gibanja bomo ozna\-čili s $\mathcal{T}$.
		\item Za dopustno gibanje $\chi$, variacijo gibanja $\vek{\eta}$ in realno konstanto $\varepsilon$ se gibanje
			$\chi+\varepsilon\vek{\eta}$ imenuje \emph{bližnje gibanje} gibanja $\chi$.
	\end{enumerate}
\end{definicija}

Takoj je potrebno opozoriti, da kljub poimenovanju dopustno gibanje in bližnje gibanje nista nujno gibanji,
kot smo ju definirali v poglavju \ref{chp:kinkon}, ker nismo postavili pogoja o pozitivnosti jacobijana.
Ni se težko prepričati, da je $\mathcal{T}$ vektorski podprostor prostora $\mathcal{X}$, če na elemente iz $\mathcal{X}$
gledamo kot na vektorska polja s kodomeno $\V$ namesto $\E$.
Prav tako zlahka ugotovimo, da za poljubni dopustni gibanji $\chi_1$ in $\chi_2$
velja $\chi_2-\chi_1\in\mathcal{T}$. $\mathcal{A}$ je torej afini podprostor prostora $\mathcal{X}$, saj je
\[ \mathcal{A}=\big\{ \chi_0+\vek{\eta}\: ;\ \vek{\eta}\in\mathcal{T} \big\}=\chi_0+\mathcal{T}, \]
kjer je $\chi_0$ poljubno dopustno gibanje.

\begin{definicija}
	Naj bo $U$ množica, ki je dobljena kot presek neke odprte množice v $\mathcal{X}$ in afinega
	podprostora $\mathcal{A}$. Naj bo $\mathcal{Y}$ Banachov prostor in $F\colon U\to\mathcal{Y}$
	odvedljiva preslikava. Smerni odvod
	\[
		\delta F\colon U\to\L(\mathcal{T},\mathcal{Y}),\qquad
		\delta F(\chi)(\vek{\eta})=\at{\frac{d}{d\varepsilon}F(\chi+\varepsilon\vek{\eta})}{\varepsilon=0}
	\]
	se imenuje tudi \emph{variacija} preslikave $F$.
\end{definicija}

\begin{notacija}
	Naj bo preslikava $F$ definirana kot v prejšnji definiciji. Če velja $F(\chi)=Y$,
	potem oznaka $Y^*$ pomeni $Y^*=F(\chi+\varepsilon\vek{\eta})$ za neka $\vek{\eta}\in\mathcal{T}$ in $\varepsilon\in\R$,
	oznaka $\delta Y$ pa pomeni $\delta Y=\delta F(\chi)(\vek{\eta})$. V tej notaciji velja
	\[ \delta Y = \at{\Big(\frac{dY^*}{d\varepsilon}\Big)}{\varepsilon=0}, \]
	$\delta Y$ pa imenujemo tudi \emph{variacija} količine $Y$.
\end{notacija}

\begin{primeri}
	Poglejmo si nekaj posebnih primerov za preslikavo $F$ iz zadnje definicije.
	\begin{enumerate}
		\item
			Če $F$ dopustnemu gibanju $\chi$ priredi vektorsko polje hitrosti $\vek{v}$,
			$F(\chi)=d\chi/dt=\dot{\chi}=\vek{v}$, potem je
			\[ \vek{v}^*=\frac{d}{dt}(\chi+\varepsilon\vek{\eta})=\vek{v}+\varepsilon\dot{\vek{\eta}}, \]
			variacija polja hitrosti pa je
			\[ \delta\vek{v}=\dot{\vek{\eta}}. \]
		\item
			Če $F$ dopustnemu gibanju $\chi$ priredi polje deformacijskega gradienta $\ten{F}$,
			$F(\chi)=\Grad{\chi}=\ten{F}$, potem je
			\[ \ten{F}^*=\Grad(\chi+\varepsilon\vek{\eta})=\ten{F}+\varepsilon\Grad\vek{\eta}, \]
			variacija pa je
			\begin{equation} \label{e:deltaF}
				\delta\ten{F}=\Grad\vek{\eta}.
			\end{equation}
		\item
			Če je $J$ polje jacobijana dopustnega gibanja $\chi$, potem je
			\[ J^*=\det(\Grad(\chi+\varepsilon\vek{\eta}))=\det(\ten{F}^*). \]
			Pri računanju variacije polja $J$ se uporabi podobne prijeme,
			kot smo jih uporabili pri računanju enakosti (\ref{e:dotJ}). Dobimo
			\begin{align*}
				\delta J&=D\det(\ten{F})(\delta\ten{F})=J\ten{F}^{-T}\cdot\Grad\vek{\eta}\\
				&=J\tr(\ten{F}^{-1}\Grad\vek{\eta})=J\tr(\grad\vek{\eta})=J\div\vek{\eta}.
			\end{align*}
			Na predzadnjem koraku smo uporabili enakost (\ref{e:gz}).
		\item
			Za masno gostoto $\rho_R$ referenčne konfiguracije se predpostavi, da je znana. Če je
			$\rho$ polje masne gostote trenutne konfiguracije glede na gibanje $\chi$,
			dobimo iz relacije (\ref{e:rojror})
			\[ \rho^*=\frac{\rho_R}{J^*}. \]
			Izračunajmo še variacijo:
			\begin{equation} \label{e:deltarho}
				\delta\rho=\rho_R\delta\left(\frac{1}{J}\right)=
				\rho_R\frac{-1}{J^2}(\delta J)=-\frac{\rho_R}{J}\frac{J\div\vek{\eta}}{J}=-\rho\div\vek{\eta}.
			\end{equation}
	\end{enumerate}
\end{primeri}


\section{Hamiltonov princip}


\begin{rdece}Ta razdelek sem napisal že pred časom in ga sedaj nisem pregledoval
in popravljal, ker sem želel čimprej oddati v vpogled. Zato so nekatere oznake še
neskladne, npr.~$dA$ je namesto $dS$ pa še kaj.

Tukaj mi je na misel prišla ideja,
da standardna tenzorska polja predstavim kot funkcionale. Npr.~vektorsko polje hitrosti
$\vek{v}$ bi zapisal kot preslikavo
\[ \vek{v}\colon\mathcal{X}\to C^2(B\times I,\V),\qquad \vek{v}(\chi)(X,t)=\frac{d\chi}{dt}(X,t). \]
Podobno bi potem npr.~kinetično energijo zapisal kot $T(\chi)(t)$ itd. Ampak mi je zmanjkalo časa.

Vseeno prosim, preglejte tudi ta razdelek, da boste povedali, če je v redu zastavljeno.
\end{rdece}

\emph{Kinetična energija} telesa glede na dopustno gibanje $\chi$ je definirana kot
\begin{equation*}
	T = \int\limits_{B}\frac{1}{2}\vek{v}\cdot\vek{v}\rho_R\, dV =
	\int\limits_{B_t}\frac{1}{2}\vek{v}\cdot\vek{v}\rho\, dv
\end{equation*}
in je funkcija časa $t\in I$. Kinetična energija bližnjega gibanja $\chi+\varepsilon\vek{\eta}$
je potemtakem
\begin{equation*}
	T^* = \int\limits_{B}\frac{1}{2}\vek{v}^*\cdot\vek{v}^*\rho_R\, dV =
	\int\limits_{\bottop{B}{t}{*}}\frac{1}{2}\vek{v}^*\cdot\vek{v}^*\rho^*\, dv.
\end{equation*}
Variacijo kinetične energije je lažje izračunati iz prvega izraza, saj lahko odvajanje prenesemo pod
integralski znak, ker se integracijsko območje ne spreminja. Skalarni produkt odvajamo
po pravilu za odvod produkta in dobimo
\[
	\delta T =
	\int\limits_{B}\delta\vek{v}\cdot\vek{v}\rho_R\, dV=
	\int\limits_{B}\dot{\vek{\eta}}\cdot\vek{v}\rho_R\, dV.
\]
Zanimal nas bo še integral kinetične energije od časa $t_1$ do časa $t_2$,
\[ \mathcal{I}=\int_{t_1}^{t_2} T\,dt=\int_{t_1}^{t_2} \int_{B}\frac{1}{2}\vek{v}\cdot\vek{v}\rho_R\, dV \,dt, \]
katerega variacija je
\begin{equation}\label{e:varInf}
	\delta \mathcal{I}=\int_{t_1}^{t_2}\int_{B}\dot{\vek{\eta}}\cdot\vek{v}\rho_R\, dV\, dt.
\end{equation}
Zapišemo jo lahko še nekoliko drugače, v obliki, ki bo v nadaljevanju za nas bolj pomembna.
V (\ref{e:varInf}) smemo po Fubinijevem izreku zamenjati vrstni red integriranja.
Notranji integral, ki je sedaj integral po času, integriramo per-partes:
\[
	\int_{t_1}^{t_2}\rho_R\vek{v}\cdot\dot{\vek{\eta}}\, dt =
	\left.\Big(\rho_R\vek{v}\cdot\vek{\eta}\Big)\right|_{t_1}^{t_2}-
	\int_{t_1}^{t_2}\rho_R\vek{a}\cdot\vek{\eta}\,dt.
\]
Prvi izraz na desni strani je enak 0, saj je $\vek{\eta}(\cdot,t_1)=\vek{\eta}(\cdot,t_2)=\vek{0}$.
Rezultat vstavimo nazaj v (\ref{e:varInf}) in dobimo
\begin{equation}\label{e:varkin}
	\delta \mathcal{I} =
	-\int_{t_1}^{t_2}\int_{B}\rho_R\vek{a}\cdot\vek{\eta}\, dV\, dt=
	-\int_{t_1}^{t_2}\int_{B_t}\rho\vek{a}\cdot\vek{\eta}\, dv\, dt.
\end{equation}
Pri tem druga enakost velja zaradi REF.

Rezultanto vseh zunanjih sil, ki delujejo na telo med dopustnim gibanjem $\chi$, lahko zapišemo v obliki
\begin{equation}\label{e:zunsil}
	\int\limits_{B_t}\vek{b}\rho\, dv + \int\limits_{\partial B_t}\vek{t}\, da =
	\int\limits_{B}\vek{b}\rho_R\, dV + \int\limits_{\partial B}\vek{t}_R\, dA.
\end{equation}
Vektorsko polje $\vek{b}$ je \emph{gostota prostorninske sile}, njegove vrednosti imajo fizikalno enoto
pospeška, torej \textit{meter na kvadratno sekundo}, $ms^{-2}$. Primer takega vektorskega polja
je npr. gravitacijski pospešek. Predpostavili bomo, da je $\vek{b}$ konzervativno vektorsko polje,
kar pomeni, da obstaja $C^1$ skalarno polje $\phi\colon\E\times I\to\R$, da je
\begin{equation*} \label{e:gps}
	\vek{b}(x,t)=\nabla_{x}\phi(x,t).
\end{equation*}
Vektorsko polje
\begin{equation*} \label{e:csv}
	\vek{t}=\vek{t}(x,t),\quad (x,t)\in \textcolor[rgb]{1,0,0}{\big\{\partial B_t\times\{t\}\: ;\ t\in I\big\}}
\end{equation*}
se imenuje \emph{Cauchyjev napetostni vektor}, polje
\begin{equation*} \label{e:pksv}
	\vek{t}_R=\vek{t}_R(X,t),\quad (X,t)\in \partial B\times I
\end{equation*}
pa \emph{Piola-Kirchoffov napetostni vektor}. Oba
imata fizikalno enoto tlaka, torej \textit{Newton na kvadratni meter}, $Nm^{-2}$.
Primer napetostnega vektorja je npr. atmosferski tlak.
Velja zveza
\begin{equation}\label{e:ttr}
	\vek{t}\, ds=\vek{t}_R\, dS.
\end{equation}
Tudi za napetosni vektor bomo predpostavili, da je konzervativno vektorsko polje v smislu,
da obstaja v času zvezno odvedljiv, v prostoru pa odsekoma zvezno odvedljiv\footnote{Namesto gradienta imamo v tem primeru
smerni odvod, definiran le v smereh iz tangentne ravnine na ploskev $\partial B_t$; vseeno bomo uporabljali oznako za gradient.}
skalarni potencial $\psi\colon\E\to\R$, da velja
\[ \vek{t}_R(x,t)=\nabla_{x}\psi(x,t)\quad\textrm{za}\quad (x,t)\in \textcolor[rgb]{1,0,0}{\big\{\partial B_t\times\{t\}\: ;\ t\in I\big\}}, \]
kjer je $\vek{t}_R(x,t)=\vek{t}_R(\chi(X,t),t)$ prostorski opis polja $\vek{t}_R$.

Za polje $\vek{b}$ bomo predpostavili, da je znano. Predpostavili bomo tudi, da poznamo predpis za
polje $\vek{t}_R$ na območju $(X,t)\in \partial_d B\times I$ in ta predpis
se imenuje dinamični robni pogoj.

Izraz
\begin{equation*}
	W = \int\limits_{B}\phi\rho_R\, dV + \int\limits_{\partial_d B}\psi\, dA,
\end{equation*}
predstavlja del \emph{potencialne energije} telesa in je prav tako funkcija časa.
Preostali del potencialne energije tvori \emph{notranja energija}, ki jo bomo označili z $V$.
Predpis za notranjo energijo je odvisen od materiala. Mi si bomo v tem délu ogledali primer
idealnih tekočin in elastičnih trdnin. 

Skalarni polji $\phi$ in $\psi$ sta na prostoru $\E\times I$ definirani neodvisno od gibanja.
Seveda to ne velja za njun materialni opis, od gibanja $\chi$ sta odvisni preko zvez
\[ \hat{\phi}(X,t)=\phi\big(\chi(X,t),t\big)\quad\textrm{in}\quad\hat{\psi}(X,t)=\psi\big(\chi(X,t),t\big) \]
za $(X,t)\in B\times I$.
Če imamo namesto gibanja $\chi$ njegovo bližnje gibanje $\chi+\varepsilon\vek{\eta}$, potem je
\[
	\hat{\phi}^*(X,t)=\phi\big(\chi(X,t)+\varepsilon\vek{\eta}(X,t),t\big)\quad\textrm{in}
	\quad\hat{\psi}^*(X,t)=\psi\big(\chi(X,t)+\varepsilon\vek{\eta}(X,t),t\big).
\]
Če ta dva izraza odvajamo po $\varepsilon$, nato pa postavimo $\varepsilon=0$, dobimo variaciji
\[
	\delta\phi=\grad\phi\cdot\vek{\eta}=\vek{b}\cdot\vek{\eta},\quad
	\delta\psi=\grad\psi\cdot\vek{\eta}=\vek{t}_R\cdot\vek{\eta}.
\]
Variacijo izraza $W$ imamo sedaj na dlani:
\begin{equation} \label{e:varvide}
	\delta W = \int\limits_{B}\rho_R\vek{b}\cdot\vek{\eta}\, dV + \int\limits_{\partial_d B}\vek{t}_R\cdot\vek{\eta}\, dA=
	\int\limits_{B_t}\rho\vek{b}\cdot\vek{\eta}\, dv + \int\limits_{\partial_d B_t}\vek{t}\cdot\vek{\eta}\, da.
\end{equation}
Tu smo z $\partial_d B_t$ označili $\chi_t(\partial_d B)$ in uporabili zvezo (\ref{e:ttr}).
Variacija $\delta W$ je sicer splošno znana
pod imenom \emph{virtualno delo} zunanjih sil (\ref{e:zunsil}). Kot bomo videli v nadaljevanju,
ni potrebno, da poznamo skalarna potenciala $\phi$ in $\psi$, ker nas bo zanimalo le virtualno delo.
Pomembno je le, da predpostavimo njun obstoj.

\begin{definicija}
	Naj bo $U\subseteq\mathcal{A}$ množica, dobljena kot presek afinega podprostora $\mathcal{A}$ in
	neke odprte množice v prostoru $\mathcal{X}$.
	Funkcional $F\colon U\to\R$ ima lokalni minimum pri dopustnem gibanju $\chi_{0}\in U$, če obstaja $\varepsilon >0$,
	da za vsak $\chi\in U$, za katerega je $\|\chi-\chi_{0}\|<\varepsilon$, velja $F(\chi_{0})<F(\chi)$.
\end{definicija}

\begin{definicija}
	Funkcional, ki gibanju $\chi\in U$ s kinetično energijo $T$ in potencialno
	energijo $W+V$ priredi realno število
	\[
		H=\int_{t_1}^{t_2}(T+W+V)\,dt,
	\]
	se imenuje \emph{Hamiltonov funkcional}.
\end{definicija}

\begin{aksiom}[Hamiltonov princip]
	Za pravo gibanje telesa velja, da je njegov jacobijan pozitiven na
	celotnem definicijskem območju, Hamiltonov funkcional pa pri pravem gibanju
	zavzame lokalni minimum.
\end{aksiom}

Hamiltonov princip zajema načelo minimalnega odpora. Telo se bo gibalo tako, da
bo za pot porabilo čim manj energije. Omenimo še, da zaradi zveznosti operatorja,
ki dopustnemu gibanju priredi njegov jacobijan, obstaja okolica pravega gibanja,
kjer je jacobijan vsakega gibanja iz te okolice še prav tako pozitiven.

\begin{trditev}
	Naj bo $U$ odprta podmnožica Banachovega prostora $\mathcal{X}$.
	Če funkcional $F\colon U\to\R$ zavzame lokalni minimum pri dopustnem gibanju $\chi$, potem velja
	\[ \delta F(\chi)(\vek{\eta})=0\quad\textrm{za vsak}\ \vek{\eta}\in\mathcal{T}. \]
\end{trditev}

\proof
	Recimo, da $F$ zavzame lokalni minimum pri $\chi\in U$ in naj bo $\vek{\eta}\in\mathcal{T}$ poljubna.
	Definirajmo funkcijo $\varphi\colon\R\to\R$ s predpisom
	\[ \varphi(\varepsilon)=F(\chi+\varepsilon\vek{\eta}). \]
	$\varphi$ ima lokalni minimum pri $\varepsilon=0$, zato velja $\varphi'(0)=0.$
	Po drugi strani pa je
	\[
		\varphi'(0)=\at{\frac{d}{d\varepsilon}F(\chi+\varepsilon\vek{\eta})}{\varepsilon=0}=
		\delta F(\chi)(\vek{\eta}).
	\]
\endproof

\begin{posledica} \label{p:varham0}
	Potreben pogoj, ki mu mora pravo gibanje zadoščati, je pozitivnost jacobijana
	na celotnem definicijskem območju ter
	\[ \delta H = \int_{t_1}^{t_2}(\delta T+\delta W+\delta U)\,dt = 0. \]
\end{posledica}

Če poznamo izraz za notranjo energijo in njeno variacijo, potem iz posledice
dobimo integralsko enačbo. Izraz za variacijo kinetične energije in virtualno delo 
smo namreč že izpeljali. Iz integralske enačbe se da dobiti sistem lokalnih diferencialnih enačb
za tenzorska polja. Med rešitvami tega sistema je tudi sámo gibanje telesa.
Da pa dobimo lokalne enačbe, potrebujemo nekaj izrekov, ki so podani v naslednjem razdelku.


\section{Osnovne leme variacijskega računa}


Skozi ta razdelek naj oznaka $\W$ stoji za
končnorazsežen vektorski prostor nad poljem realnih števil, opremljen s skalarnim produktom.
Naj bo $U$ podmnožica nekega metričnega prostora.
\emph{Nosilec} preslikave $\vek{w}\colon U\to\W$ je množica
\[ \mathrm{supp}\,\vek{w}=\overline{\{ x\in U\,;\ \vek{w}(x)\neq\vek{0} \}}. \]
Črta nad množico pomeni zaprtje množice. Če je $\mathrm{supp}\,\vek{w}$ kompaktna množi\-ca, potem rečemo,
da je $\vek{w}$ \emph{preslikava s kompaktnim nosilcem}. V nadaljevanju bomo pokazali, da obstajajo gladka
tenzorska polja s kompaktnim nosilcem, saj jih bomo potrebovali v izrekih, ki jih bomo navedli v tem razdelku.

Prepričajmo se, da je funkcija
\[
	g\colon\R\to\R,\quad g(x)=\left\{\begin{array}{ccl}
	\exp(-1/x)&;&x>0 \\ 0&;&x\leq 0 \end{array}\right.
\]
gladka. Očitno je to res za $x\neq 0$. Izračun limite v točki $0$ pokaže, da
je $g$ zvezna v 0. Za $x>0$ so odvodi funkcije $g$
\[
	g'(x)=\exp\left(-\frac{1}{x}\right)\frac{1}{x^2},\qquad
	g^{(k)}(x)=\exp\left(-\frac{1}{x}\right)p_k\left(\frac{1}{x}\right),
\]
kjer so $p_k$ neki polinomi, $k\in\mathbb{N}$. Če naredimo limito teh odvodov, ko
gre $x$ z desne proti 0, dobimo $g^{(k)}(0)=0$ za vse $k=1,2,\dots$. Leva limita
odvodov funkcije $g$ je pa očitno vedno 0. Obe limiti sta enaki, torej so vsi odvodi
funkcije $g$ zvezni v točki 0.

Oglejmo si sedaj funkcijo
\[
	\beta\colon\R\to\R,\quad \beta(x)=\left\{\begin{array}{ccl}
	\exp\left(-\dfrac{1}{1-x^2}\right)&;&|x|<1 \\ 0&;&|x|\geq 1 \end{array}\right. .
\]
Zapišemo jo lahko kot kompozitum funkcije $g$ ter funkcije $x\mapsto 1-x^2$. Obe funkciji sta
gladki povsod in tak je zato je tudi njun kompozitum $\beta$. Nosilec funkcije $\beta$ je
$[-1,1]$, v notranjosti nosilca pa je vrednost funkcije $\beta$ pozitivna.

Definirajmo skalarno polje $\gamma\colon\E_R\to\R$ s predpisom
\[
	\gamma(X)=\left\{\begin{array}{ccl}
	\exp\left(-\dfrac{1}{1-\|X\|^2}\right)&;&\|X\|<1 \\ 0&;&\|X\|\geq 1 \end{array}\right.
\]
Zapišemo ga lahko kot kompozitum $X\mapsto\|X\|\mapsto \beta(\|X\|)$, zato je polje
$\gamma$ gladko s kompaktnim nosilcem
\[ \textrm{supp}\,\gamma=\overline{\big\{X\in\E_R\,;\ \|X\|<1 \big\}}=\overline{K}(0,1). \]

Naj bo $\{\vek{b}_j\}_{j=1}^{n}$ baza prostora $\W$. Za neki $1\leq k\leq n$, neki $\alpha>0$ in
za neka $X_0\in\E_R$ ter $t_0\in\R$ definirajmo tenzorsko polje
\begin{equation} \label{e:funiw}
	\vek{w}\colon\E_R\times\R\to\W,\qquad\vek{w}(X,t)=\beta\left(\frac{t-t_0}{\alpha}\right)
	\gamma\left(\frac{X-X_0}{\alpha}\right)\vek{b}_k.
\end{equation}
Hitro se lahko prepričamo, da je
\begin{equation} \label{e:supaw}
	\mathrm{supp}\,\vek{w}=\overline{K}(X_0,\alpha)\times[t_0-\alpha,t_0+\alpha],
\end{equation}
kjer je $\overline{K}(X_0,\alpha)=\{X\in\E_R\,;\ \|X-X_0\|\leq\alpha\}$ zaprta krogla
s središčem v $X_0$ in radijem $\alpha$. Tako definirano polje $\vek{w}$ je
tudi gladko, saj je produkt gladkih polj. V notranjosti nosilca je njegova 
edina neničelna (t.j.~$k$-ta) komponenta glede na bazo $\{\vek{b}_j\}_{j=1}^{n}$ pozitivna.

Omenimo še to pomembno dejstvo, da za vsako zvezno preslikavo $\vek{w}$ s kompaktnim
nosilcem velja, da je $\vek{w}={0}$ na robu nosilca. Komplement nosilca je namreč odprta množica, na kateri
je $\vek{w}={0}$, in ker je $\vek{w}$ zvezna, je po zvezni razširitvi $\vek{w}={0}$ tudi na robu komplementa,
ki je enak robu nosilca.

\begin{lema}\label{l:1}
	Naj bo $\vek{f}\colon B\times [t_1,t_2]\to\W$ zvezno polje. Če velja
	\begin{equation}\label{e:lem1}
		\int_{t_1}^{t_2}\int_B\vek{f}\cdot\vek{w}\,dV\,dt = 0
	\end{equation}
	za vsako gladko polje $\vek{w}\colon B\times[t_1,t_2]\to\W$, za katerega velja
	\[
		\vek{w}(\cdot,t_1)=\vek{w}(\cdot,t_2)=\vek{0}\quad\textrm{in}
		\quad\vek{w}=\vek{0}\ \textrm{na}\ \partial B\times[t_1,t_2],
	\]
	potem je $\vek{f}=\vek{0}$ na $B\times [t_1,t_2]$.
\end{lema}

\proof
	Glede na fiksno bazo $\{\vek{b}_j\}_{j=1}^n$ prostora $\W$ lahko $\vek{f}$
	zapišemo v komponentni obliki kot $\vek{f}=f_j\vek{b}_j$.
	Recimo, da obstaja točka $(X_0,t_0)$ v notranjosti območja $B\times[t_1,t_2]$, da je $f_{k}(X_0,t_0)\neq 0$
	za neki $1\leq k\leq n$. Potem zaradi zveznosti polja $\vek{f}$
	obstaja okolica $U$ točke $(X_0,t_0)$, na kateri je $f_k$
	vseskozi različen od 0 in enakega predznaka, ter
	je $U$ še v celoti vsebovana v notranjosti območja $B\times[t_1,t_2]$.
	
	Za dovolj majhen $\alpha>0$ lahko definiramo polje
	$\vek{w}$ kot v (\ref{e:funiw}), če je $f_{k}(X_0,t_0) > 0$, oz.~kot $-\vek{w}$ v (\ref{e:funiw}),
	če je $f_{k}(X_0,t_0) < 0$, tako da je množica (\ref{e:supaw}) v celoti vsebovana v okolici $U$.
	Skrčitev polja $\vek{w}$ na domeno $B\times[t_0,t_1]$ tako ustreza vsem pogojem iz leme.
	Skalarni produkt $\vek{f}\cdot\vek{w}$ je pozitiven na notranjosti množice
	$\mathrm{supp}\,\vek{w}$ in je 0 drugje, zato je
	\[
		\int_{t_1}^{t_2}\int_B \vek{f}\cdot \vek{w}\,dV\,dt=
		\int_{t_0-\alpha}^{t_0+\alpha}\int_{\overline{K}(X_0,\alpha)} \vek{f}\cdot \vek{w}\,dV\,dt> 0,
	\]
	kar nasprotuje enačbi (\ref{e:lem1}). Torej mora veljati $f_k=0$ za vsak $k$,
	tj.~$\vek{f}=\vek{0}$ povsod v notranjosti območja $B\times [t_1,t_2]$,
	zaradi zveznosti pa tudi na zaprtju.
\endproof

\begin{lema}\label{l:2}
	Naj rob $\partial B$ sestoji iz komplementarnih ploskev $\partial_g B$ in
	$\partial_d B$. Naj bo $V$ množica vseh tistih regularnih točk ploskve $\partial_d B$,
	ki niso na robu ploskve $\partial_d B$ in naj bo 
	polje $\vek{f}\colon \partial_d B\times[t_1,t_2]\to \W$ zvezno na podobmočju $V\times[t_1,t_2]$.
	Če velja
	\begin{equation}\label{e:lem2}
		\int_{t_1}^{t_2}\int_{\partial_d B} \vek{f}\cdot \vek{w}\,dS\,dt = 0
	\end{equation}
	za vsako gladko polje $\vek{w}\colon B\times[t_1,t_2]\to\W$, za katero je
	\[
		\vek{w}(\cdot,t_1)=\vek{w}(\cdot,t_2)=\vek{0}\quad\textrm{in}
		\quad\vek{w}=\vek{0}\ \textrm{na}\ \partial_g B\times[t_1,t_2],
	\]
	potem je $\vek{f}=\vek{0}$ na $V\times [t_1,t_2]$.
\end{lema}

\proof
	Podobno, kot v prejšnjem dokazu, zapišemo $\vek{f}=f_j\vek{b}_j$ in predpostavimo
	$f_{k}(X_0,t_0)\neq 0$ za neke $1\leq k\leq n$, $t_0\in (t_1,t_2)$ in $X_0\in V$.
	
	Ker točka $X_0$ ni na robu in je regularna, obstaja odprta okolica $U\subset\E_R$ točke
	$X_0$, ki ne seka ploskve $\partial_g B$ in so vse točke iz množice $U\cap\partial_d B$
	regularne ter niso na robu.
	Za dovolj majhen $\alpha>0$ je $\overline{K}(X_0,\alpha)$ vsebovana v $U$ in
	hkrati je $f_k$ zaradi zveznosti $\vek{f}$ ves čas različna od 0 in enakega predznaka
	na območju $(K(X_0,\alpha)\cap\partial_dB)\times(t_1,t_2)$.
	
	Definirajmo polje $\vek{w}$ kot v
	(\ref{e:funiw}), če je $f_{k}(X_0,t_0) > 0$, oz.~kot $-\vek{w}$ v (\ref{e:funiw}),
	če je $f_{k}(X_0,t_0) < 0$. Skrčitev tako definiranega polja $\vek{w}$ na območje
	$B\times[t_1,t_2]$ ustreza predpostavkam iz leme.
	Skalarni produkt $\vek{f}\cdot\vek{w}$ je pozitiven na
	$(K(X_0,\alpha)\cap\partial_dB)\times(t_1,t_2)$ in je 0 drugje na $V\times[t_1,t_2]$,
	zato je\footnote{Ploščina množice $\partial_dB\setminus V$ je 0.}
	\[
		\int_{t_1}^{t_2}\int_{\partial_d B} \vek{f}\cdot \vek{w}\,dS\,dt=
		\int_{t_0-\alpha}^{t_0+\alpha}\int_{K(X_0,\alpha)\cap\partial_dB} \vek{f}\cdot \vek{w}\,dS\,dt> 0,
	\]
	kar nasprotuje enačbi (\ref{e:lem2}). Torej mora veljati $\vek{f}=\vek{0}$ na
	$V\times [t_1,t_2]$.
\endproof

\begin{lema} \label{lema3}
	Naj bo $\chi\colon B\times[t_1,t_2]\to\E$ gibanje razreda $C^1$ s pozitivnim jacobijanom $J$ in
	$\vek{f}\colon B\times[t_1,t_2]\to\W$ zvezno tenzorsko polje. Če velja
	\begin{equation} \label{e:lem3}
		\int_{t_1}^{t_2}\int_{\chi_t(B)}\vek{f}\cdot \vek{w} \,dv\,dt=0
	\end{equation}
	za vsako gladko polje $\vek{w}\colon B\times[t_1,t_2]\to\W$, za katerega velja
	\[
		\vek{w}(\cdot,t_1)=\vek{w}(\cdot,t_2)=\vek{0}\quad\textrm{in}
		\quad\vek{w}=\vek{0}\ \textrm{na}\ \partial B\times[t_1,t_2],
	\]
	potem je $\vek{f}=\vek{0}$ na $B\times[t_1,t_2]$ oz.~na $\Omega$.
\end{lema}
V enačbi (\ref{e:lem3}) nastopata polji $\vek{f}$ in $\vek{w}$ v prostorskem opisu.

\proof
	Enačba (\ref{e:lem3}) je po izreku \ref{i:prointrel} ekvivalentna enačbi
	\[ \int_{t_1}^{t_2}\int_{B}J\vek{f}\cdot \vek{w} \,dV\,dt=0. \]
	
	Ker je polje $J\vek{f}$ zvezno na $B\times[t_1,t_2]$, je
	po lemi (\ref{l:1}) $J\vek{f}=\vek{0}$
	na $B\times[t_1,t_2]$. Ker je $J>0$, je $\vek{f}=\vek{0}$ na $B\times[t_1,t_2]$,
	v prostorskem opisu pa je potem tudi $\vek{f}=\vek{0}$ na $\Omega$.
\endproof

\begin{lema} \label{lema4}
	Naj rob $\partial B$ sestoji iz komplementarnih ploskev $\partial_g B$ in $\partial_d B$.
	Naj bo $\chi\colon B\times[t_1,t_2]\to\E$ gibanje razreda $C^1$ s pozitivnim jacobijanom $J$ in
	$f\colon\partial_d B\times[t_1,t_2]\to\R$ skalarno polje, zvezno na podobmočju $V\times[t_1,t_2]$,
	ki je definirano enako, kot v lemi \ref{l:2}.
	Če velja
	\begin{equation}\label{e:lem4}
		\int_{t_1}^{t_2}\int_{\chi_t(\partial_d B)}f\vek{n}\cdot\vek{w}\,ds\,dt=0
	\end{equation}
	za vsako gladko vektorsko polje $\vek{w}\colon B\times[t_1,t_2]\to\V$, za katero je
	\[
		\vek{w}(\cdot,t_1)=\vek{w}(\cdot,t_2)=\vek{0}\quad\textrm{in}
		\quad\vek{w}=\vek{0}\ \textrm{na}\ \partial_g B\times[t_1,t_2],
	\]
	potem je $f=0$ na $V\times[t_1,t_2]$ oz.~na $\{(x,t)\,;\ x\in\chi_t(V),\ t\in[t_1,t_2] \}$.
\end{lema}

\proof
	Enačba (\ref{e:lem4}) je po izreku \ref{i:suittra} ekvivalentna enačbi
	\[ \int_{t_1}^{t_2}\int_{\partial_d B}fJ\ten{F}^{-T}\vek{N}\cdot\vek{w}\,dS\,dt=0. \]
	Polje $fJ\ten{F}^{-T}\vek{N}$ je zvezno na $V\times[t_1,t_2]$,
	zato je po lemi (\ref{l:2}) $fJ\ten{F}^{-T}\vek{N}=\vek{0}$ na $V\times[t_1,t_2]$.
	Ker so točke iz $V$ regularne, je $\vek{N}\neq \vek{0}$, hkrati je po predpostavki $J>0$,
	zato je tudi $\ten{F}^{-T}$ nesingularno. Od tod sledi, da je $f=0$ na $V\times[t_1,t_2]$,
	v prostorskem opisu pa $f=0$ na $\{(x,t)\,;\ x\in\chi_t(V),\ t\in[t_1,t_2] \}$.
\endproof


\section{Primera}


\subsection{Elastični fluid}


Idealni fluid je model tekočine, v katerem zanemarimo učinke viskoznosti. Če je tekočina
stisljiva, potem je njena notranja energija
\[ U=\int_{B}\rho_R e(\rho)\,dV=\int_{\chi_t(B)}\rho e(\rho)\,dv, \]
kjer je $e\colon[0,\infty)\to\R$ funkcija razreda $C^2$, imenovana
\emph{gostota notranje energije} in ima fizikalno enoto $m^2s^{-2}$.
Predpostavili bomo, da ima $e$ znano funkcijsko obliko.

Izračunajmo variacijo notranje energije:
\begin{align*} 
	\delta U &=\at{\frac{dU^*}{d\varepsilon}}{\varepsilon=0}=\at{\frac{d}{d\varepsilon}
	\int_{B}\rho_R e(\rho^*)\,dV }{\varepsilon=0} =
	\at{\int_{B}\rho_R e'(\rho^*)\frac{d\rho^*}{d\varepsilon}\,dV }{\varepsilon=0} \\
	&= -\int_{B}\rho_R e'(\rho)\rho\div\vek{\eta}\,dV
	= -\int_{\chi_t(B)}\rho^2 e'(\rho)\div\vek{\eta}\,dv
\end{align*}
Pri tem smo smeli zamenjati vrstni red odvajanja in integriranja, ker se referenčna
konfiguracija $B$ ne spreminja, nato pa smo uporabili verižno pravilo, enačbo
(\ref{e:deltarho}) in posledico \ref{p:roji}.
Če sedaj na dobljenem rezultatu uporabimo relacijo 3 iz trditve \ref{t:divprop} in
nato divergenčni izrek \ref{i:divtheo}$_3$, dobimo
\begin{equation} \label{e:varpot}
	\delta U=\int_{\chi_t(B)}\grad\big(\rho^2 e'(\rho)\big)\cdot\vek{\eta}\,dv
	-\int_{\chi_t(\partial B)}\rho^2 e'(\rho)\vek{n}\cdot\vek{\eta}\,ds.
\end{equation}

Geometrijskih robnih pogojev ni, zato imamo zgolj dinamične robne pogoje.
Naj bo $V\subseteq\partial B$ množica regularnih točk ploskve $\partial B$ in
predpostavimo, da imamo podan \emph{zunanji tlak}, t.j.~zvezno skalarno polje
$p\colon \E\times[t_1,t_2]\to\R$, da je $\vek{t}=-p\vek{n}$ na
$\{(x,t)\,;\ x\in\chi_t(V),\ t\in[t_1,t_2]\}$.

Če sedaj upoštevamo rezultate (\ref{e:varkin}), (\ref{e:varvide}) in (\ref{e:varpot}),
mora po posledici \ref{p:varham0} pravo gibanje zadoščati enačbi
\begin{multline} \label{e:elflupog}
	\int_{t_1}^{t_2}\bigg( \int_{\chi_t(B)}\Big(-\rho\vek{a}-
	\grad\big(\rho^2 e'(\rho)\big)+\rho\vek{b}\Big)\cdot\vek{\eta}\,dv+ \\
	+\int_{\chi_t(\partial B)}\big(\rho^2 e'(\rho)-p\big)\vek{n}\cdot\vek{\eta}\,ds\bigg)\,dt=0
\end{multline}
za vsako variacijo gibanja $\vek{\eta}\in\mathcal{T}$. Med drugim mora to veljati za
vsako variacijo gibanja iz množice
\[
	\big\{ \vek{\eta}\in C^{\infty}(B\times[t_1,t_2],\V)\,;\ \vek{\eta}(\cdot,t_1)=
	\vek{\eta}(\cdot,t_2)=\vek{0},\ \vek{\eta}=\vek{0}\ \textrm{na}\ \partial B\times[t_1,t_2]
	\big\}\subset\mathcal{T};
\]
pri teh variacijah je vrednost drugega integrala enaka 0, na preostali enačbi
pa lahko potem uporabimo lemo \ref{lema3} in dobimo
\begin{equation*}
	\rho\vek{a}=\grad\big(\rho^2 e'(\rho)\big)+\rho\vek{b}\quad
	\textrm{na}\ \Omega.
\end{equation*}
Če ta rezultat sedaj upoštevamo v (\ref{e:elflupog}), dobimo, da mora
za vsako variacijo gibanja $\vek{\eta}$ veljati
\[
	\int_{t_1}^{t_2}\int_{\chi_t(\partial B)}\big(\rho^2 e'(\rho)-p\big)\vek{n}\cdot\vek{\eta}\,ds\,dt=0,
\]
med drugim tudi za vsako variacijo iz množice
\[
	\big\{ \vek{\eta}\in C^{\infty}(B\times[t_1,t_2],\V)\,;\ \vek{\eta}(\cdot,t_1)=
	\vek{\eta}(\cdot,t_2)=\vek{0} \big\}\subset\mathcal{T},
\]
zato lahko uporabimo lemo \ref{lema4} (v tem primeru je $\partial_gB=\emptyset$) in dobimo še pogoj
\[
	\rho^2 e'(\rho)=p \quad\textrm{na}\ \{ (x,t)\,;\ x\in\chi_t(V),\ t\in[t_1,t_2] \}.
\]


\subsection{Elastično trdno telo}


\emph{Elastično trdno telo} je model telesa iz trdnega materiala, v katerem se predpostavi,
da je notranja energija telesa podana kot
\[ U=\int_{B}\rho_R e(\ten{F})\,dV=\int_{\chi_t(B)}\rho e(\ten{F})\,dv, \]
kjer je $e\colon\L(\V)\to\R$ \emph{gostota notranje energije}, vrednost katere je odvisna od
deformacijskega gradienta $\ten{F}$. Za funkcijo $e$ bomo predpostavili, da je znana in da je
razreda $C^1$ na svoji domeni.

Pri danem $\ten{F}$ je $De(\ten{F})$ linearen funkcional
na vektorskem prostoru $\L(\V)$ s svojim skalarnim produktom, zato po Riezsovem izreku
o reprezentaciji linearnih funkcionalov pripada funkcionalu $De(\ten{F})$ natanko
določen tenzor $\tilde{\ten{S}}\in\L(\V)$, da za vsak $\ten{A}\in\L(\V)$ velja
\[ De(\ten{F})(\ten{A})=\tilde{\ten{S}}\cdot\ten{A}=\tr(\tilde{\ten{S}}^{T}\ten{A}). \]
Zato bomo v bodoče kar enačili odvod $De(\ten{F})$ s pripadajočim tenzorjem $\tilde{\ten{S}}$.
Definirajmo tenzorsko polje
\begin{equation} \label{e:piola}
	\ten{S}\colon B\times I\to\L(\V),\qquad
	\ten{S}(X,t)=\rho_R(X)De(\ten{F}(X,t)).%\quad\textrm{oz.~krajše}\quad
	%\ten{S}=\rho_RDe(\ten{F}).
\end{equation}
$\ten{S}(X,t)$ se imenuje \emph{prvi Piola-Kirchoffov napetostni tenzor}.

Izračunajmo variacijo notranje energije:
\begin{align*} 
	\delta U &=\at{\frac{dU^*}{d\varepsilon}}{\varepsilon=0}=\at{\frac{d}{d\varepsilon}
	\int_{B}\rho_R e(\ten{F}^*)\,dV }{\varepsilon=0} =
	\at{\int_{B}\rho_R De(\ten{F}^*)\cdot\frac{d\ten{F}^*}{d\varepsilon}\,dV }{\varepsilon=0} \\
	&= \int_{B}\rho_R De(\ten{F})\cdot\Grad\vek{\eta}\,dV
	=\int_{B}\ten{S}\cdot\Grad\vek{\eta}\,dV.
	%= -\int_{\chi_t(B)}\rho^2 e'(\rho)\div\vek{\eta}\,dv
\end{align*}
Pri tem smo uporabili enakost (\ref{e:deltaF}). Če sedaj upoštevamo zvezi (\ref{e:gz}) in
(\ref{e:skatra}) ter izrek (\ref{i:prointrel}), dobimo
\begin{align*} 
	\delta U = \int_{B}J\frac{1}{J}\ten{S}\ten{F}^{T}\cdot\grad\vek{\eta}\,dV =
	\int_{\chi_t(B)}\ten{T}\cdot\grad\vek{\eta}\,dv,
\end{align*}
kjer se vrednost tenzorskega polja
\begin{equation} \label{e:napetostni}
	\ten{T}=\frac{1}{J}\ten{S}\ten{F}^{T}=\rho D\sigma(\ten{F})\ten{F}^{T}
\end{equation}
imenuje \emph{Cauchyjev napetostni tenzor}. Obe dobljeni enačbi za $\delta U$ lahko
s pomočjo izrekov \ref{e:divSu} in \ref{i:divtheo} zapišemo v obliki
\begin{align}
	\delta U &= \int_{\partial_dB}\ten{S}\vek{N}\cdot\vek{\eta}\,dS-
	\int_B\Div\ten{S}\cdot\vek{\eta}\,dV \label{e:vapoesr} \\
	&= \int_{\chi_t(\partial_dB)}\ten{T}\vek{n}\cdot\vek{\eta}\,dS-
	\int_{\chi_t(B)}\div\ten{T}\cdot\vek{\eta}\,dV. \label{e:vapoesp}
\end{align}
Pri tem smo upoštevali, da sta ploskovna integrala po ploskvah $\partial_gB$ in
$\chi_t(\partial_gB)$ enaka 0, saj je tam zaradi geometrijskega robnega pogoja $\vek{\eta}=\vek{0}$.

Po posledici \ref{p:varham0} mora pravo gibanje zadoščati enačbama
\begin{multline} \label{e:globoref}
	\int_{t_1}^{t_2}\bigg( \int_{B}(-\rho_R\vek{a}+
	\Div\ten{S}+\rho_R\vek{b})\cdot\vek{\eta}\,dV+ \\
	+\int_{\partial_d B}(\vek{t}_R-\ten{S}\vek{N})\cdot\vek{\eta}\,dS\bigg)\,dt=0
\end{multline}
in
\begin{multline} \label{e:globoprost}
	\int_{t_1}^{t_2}\bigg( \int_{\chi_t(B)}(-\rho\vek{a}+
	\div\ten{T}+\rho\vek{b})\cdot\vek{\eta}\,dv+ \\
	+\int_{\chi_t(\partial_d B)}(\vek{t}-\ten{T}\vek{n})\cdot\vek{\eta}\,ds\bigg)\,dt=0
\end{multline}
za vsa dopustna gibanja $\vek{\eta}\in\mathcal{T}$. Pri tem smo zopet upoštevali
že prej izpeljana rezultata (\ref{e:varkin}) in (\ref{e:varvide}) ter
variaciji notranje energije (\ref{e:vapoesr}) oz.~(\ref{e:vapoesp}).
Od tu dalje postopamo na enak način, kot smo pri primeru za elastični fluid.
Iz enačbe (\ref{e:globoref}) dobimo z uporabo leme \ref{l:1}
\begin{equation*}
	\rho_R\vek{a}=\Div\ten{S}+\rho_R\vek{b}\quad\textrm{na}\ B\times[t_1,t_2],
\end{equation*}
iz enačbe (\ref{e:globoprost}) pa z uporabo leme \ref{lema3} dobimo
\begin{equation*}
	\rho\vek{a}=\div\ten{T}+\rho\vek{b}\quad\textrm{na}\ \Omega.
\end{equation*}
Če dobljena rezultata upoštevamo v (\ref{e:globoref}) oz.~v (\ref{e:globoprost}) in
nato uporabimo lemo \ref{l:2} oz.~lemo \ref{lema4}, dobimo še
\begin{equation*}
	\ten{S}\vek{N}=\vek{t}_R\quad\textrm{na}\ \partial_dB\times[t_1,t_2]
\end{equation*}
oziroma
\begin{equation*}
	\ten{T}\vek{n}=\vek{t}\quad\textrm{na}\ \{(x,t)\,;\ x\in\chi_t(\partial_dB),\ t\in[t_1,t_2] \}.
\end{equation*}