\chapter{Hamiltonov princip in variacijske metode} \label{pog:ham}


\section{Dopustno gibanje in variacija}


V tem poglavju bomo s pomočjo t.~i.~\emph{Hamiltonovega principa} poiskali potrebne pogoje,
ki jim mora zadoščati pravo gibanje telesa. Še naprej naj $\B$ označuje referenčno konfiguracijo
telesa (zaprto regularno območje v $\E_R$) in $I=[t_1,t_2]\subset\R$ zaprt časovni interval.

Recimo, da pravo gibanje iščemo med vsemi gibanji
$\chi\colon \B\times I\to\E$ razreda $C^2$, ki zadoščajo v naprej predpisanim začetnim
in robnim pogojem.
\begin{enumerate}
	\item \emph{Začetni pogoj} je predpis
		\begin{equation} \label{e:zacetnipogoj}
			\chi(X,t_1)=\chi_{t_1}(X)\quad\textrm{in}\quad \chi(X,t_2)=\chi_{t_2}(X),\quad\forall\;X\in\B,
		\end{equation}
		kjer sta $\chi_{t_1},\chi_{t_2}\colon \B\to\E$ znani konfiguraciji razreda $C^2$, ki
		določata položaj telesa ob začetnem in končnem času $t_1$ oz.~$t_2$.
	\item \emph{Robni pogoji} so lahko treh vrst. \emph{Kinematični robni pogoj} je predpis
		\begin{equation} \label{e:grp}
			\chi(X,t)=q(X,t)\quad\mathrm{za}\quad (X,t)\in \partial\B\times I,
		\end{equation}
		kjer je $q\colon\partial\B\times I\to\E$ poznana.
		Kinematični robni pogoj torej v naprej določa položaj robnih točk telesa.
		\emph{Robni pogoj napetosti} je predpis za napetostni vektor, ki deluje na rob telesa.
		Ta pogoj bomo predstavili v naslednjem razdelku. Če je kinematični robni pogoj (\ref{e:grp})
		podan le na $\partial_1\B\times I$, kjer je $\partial_1\B\subset\partial\B$, na
		$\partial_2\B\times I$ pa je podan robni pogoj napetosti, kjer je $\partial_2\B=\partial\B\setminus\partial_1\B$,
		potem tak robni pogoj imenujemo \emph{mešani robni pogoj}.
\end{enumerate}

Množica $\mathcal{X}$ vseh preslikav $\chi\colon \B\times I\to\E$
razreda $C^2$ tvori vektorski prostor.
$\mathcal{X}$ postane Banachov prostor, če na njem smiselno definiramo normo. Možna izbira
predpisa za normo je npr.
\[
	\|\cdot\|\colon\mathcal{X}\to[0,\infty),\quad
	\|\chi\|=\sup\big\{\|\vek{\chi}(X,t)\|\: ;\ (X,t)\in \B\times I\big\}.
\]

Pri tem je v skladu z dogovorom \ref{d:dogovor} $\vek{\chi}=\iota_o\circ\chi$, norma, ki se pojavi
znotraj množice, pa je Evklidska norma na prostoru $\V$.
Dejansko lahko supremum nadomestimo z maksimumom, ker je vsak $\chi\in\mathcal{X}$
zvezna preslikava na kompaktni domeni $B\times I$.

\begin{definicija}
	\begin{enumerate}
		\item Preslikava $\chi\in\mathcal{X}$, ki zadošča predpisanim začetnim pogojem in kinematičnim robnim pogojem,
			se imenuje \emph{dopustno gibanje}. Množico vseh dopustnih gibanj bomo označili z $\mathcal{A}$.
		\item Vektorsko polje $\vek{\eta}\colon \B\times I\to\V$, ki je razreda $C^2$ in
			ustreza pogojem
			\[
				\vek{\eta}(X,t_1)=\vek{0}\quad \textrm{in} \quad
				\vek{\eta}(X,t_2)=\vek{0}\quad \textrm{za vse}\ X\in \B\quad\textrm{ter}
			\]
			\[
				\vek{\eta}(X,t)=\vek{0}\quad \textrm{za vse}\ (X,t)\in \partial_1 \B\times I,
			\]
			se imenuje \emph{variacija gibanja}. Množico vseh variacij gibanja bomo ozna\-čili s $\mathcal{T}$.
		\item Za dopustno gibanje $\chi$, variacijo gibanja $\vek{\eta}$ in realno število $\varepsilon$ se gibanje\footnote{
			V skladu z dogovorom na strani \pageref{seto} je vsota točke in vektorja točka, zato preslikava
			$\chi+\varepsilon\vek{\eta}$ pripada prostoru $\mathcal{X}$.}
			$\chi+\varepsilon\vek{\eta}$ imenuje \emph{bližnje gibanje} gibanja $\chi$.
	\end{enumerate}
\end{definicija}

Takoj je potrebno opozoriti, da kljub poimenovanju dopustno gibanje in bližnje gibanje nista nujno gibanji,
kot smo ju definirali v poglavju \ref{chp:kinkon}, ker tukaj nismo nič zahtevali, da morajo biti
trenutne konfiguracije dopustnega ali bližnjega gibanja bijektivne.
Ta pogoj je avtomatsko izpolnjen, če je pripadajoča determinanta deformacijskega
gradienta vseskozi različna od 0. V bistvu je to potreben in tudi zadosten pogoj, da so trenutne konfiguracije
dopustnega ali bližnjega gibanja res bijektivne za vsak $t\in I$.\footnote{
To velja za odvedljive preslikave, kar pa dopustna in bližnja gibanja po definiciji so.}
Ko bomo iskali pravo gibanje, bo neničelnost jacobijana eden od potrebnih pogojev.
V bistvu bomo predpostavili, da so konfiguracija $\chi_{t_1}$ iz začetnega pogoja (\ref{e:zacetnipogoj})
ter opazovališči za $\E_R$ in $\E$ izbrani tako, da je $J(\cdot,t_1)>0$. Zato bo za pravo gibanje
potreben dodaten pogoj, da je jacobijan pozitiven na celotnem definicijskem območju.

Ni se težko prepričati, da je $\mathcal{T}$ vektorski podprostor prostora $\mathcal{X}$, če na elemente iz $\mathcal{X}$
gledamo kot na vektorska polja glede na opazovališče $\iota_o$ za $\E$.
Prav tako zlahka ugotovimo, da za poljubni dopustni gibanji $\chi_1$ in $\chi_2$
velja $\chi_2-\chi_1\in\mathcal{T}$. $\mathcal{A}$ je torej afini podprostor prostora $\mathcal{X}$, saj je
\[ \mathcal{A}=\big\{ \chi_0+\vek{\eta}\: ;\ \vek{\eta}\in\mathcal{T} \big\}=\chi_0+\mathcal{T}, \]
kjer je $\chi_0$ poljubno dopustno gibanje.

\begin{definicija}
	Naj bo $U$ množica, ki je dobljena kot presek neke odprte množice v $\mathcal{X}$ in afinega
	podprostora $\mathcal{A}$. Naj bo $\mathcal{Y}$ Banachov prostor in $F\colon U\to\mathcal{Y}$
	odvedljiva preslikava. Smerni odvod
	\[
		\delta F\colon U\to\L(\mathcal{T},\mathcal{Y}),\qquad
		\delta F(\chi)(\vek{\eta})=\at{\frac{d}{d\varepsilon}F(\chi+\varepsilon\vek{\eta})}{\varepsilon=0}
	\]
	se imenuje tudi \emph{variacija} preslikave $F$.
\end{definicija}

\begin{notacija}
	Naj bo preslikava $F$ definirana kot v prejšnji definiciji. Če $F$ gibanju $\chi$
	priredi neko njegovo fizikalno količino $Y$, $F(\chi)=Y$,
	potem oznaka $Y^*$ pomeni pripadajočo fizikalno količino bližnjega gibanja,
	$Y^*=F(\chi+\varepsilon\vek{\eta})$ za neka $\vek{\eta}\in\mathcal{T}$ in $\varepsilon\in\R$,
	oznaka $\delta Y$ pa pomeni $\delta Y=\delta F(\chi)(\vek{\eta})$. V tej notaciji velja
	\[ \delta Y = \at{\Big(\frac{dY^*}{d\varepsilon}\Big)}{\varepsilon=0}, \]
	$\delta Y$ pa imenujemo tudi \emph{variacija} količine $Y$.
\end{notacija}

\begin{primeri}
	Poglejmo si nekaj posebnih primerov za preslikavo $F$ iz zadnje definicije.
	\begin{enumerate}
		\item
			Če $F$ dopustnemu gibanju $\chi$ priredi vektorsko polje hitrosti $\vek{v}$,
			$F(\chi)=d\vek{\chi}/dt=\dot{\vek{\chi}}=\vek{v}$, potem je
			\[ \vek{v}^*=\frac{d}{dt}(\vek{\chi}+\varepsilon\vek{\eta})=\vek{v}+\varepsilon\dot{\vek{\eta}}, \]
			variacija polja hitrosti pa je
			\[ \delta\vek{v}=\dot{\vek{\eta}}. \]
		\item
			Če $F$ dopustnemu gibanju $\chi$ priredi polje deformacijskega gradienta $\ten{F}$,
			$F(\chi)=\Grad{\vek{\chi}}=\ten{F}$, potem je
			\[ \ten{F}^*=\Grad(\vek{\chi}+\varepsilon\vek{\eta})=\ten{F}+\varepsilon\Grad\vek{\eta}, \]
			variacija pa je
			\begin{equation} \label{e:deltaF}
				\delta\ten{F}=\Grad\vek{\eta}.
			\end{equation}
		\item
			Če $F$ dopustnemu gibanju $\chi$ priredi polje jacobijana
			$J=\det(\Grad\vek{\chi})$, potem je
			\[ J^*=\det(\Grad(\vek{\chi}+\varepsilon\vek{\eta}))=\det(\ten{F}^*). \]
			Pri računanju variacije polja $J$ se uporabi podobne prijeme,
			kot smo jih uporabili pri računanju enakosti (\ref{e:dotJ}). Dobimo
			\begin{align*}
				\delta J&=D\det(\ten{F})(\delta\ten{F})=\langle J\ten{F}^{-T},\Grad\vek{\eta}\rangle \\
				&=J\tr(\ten{F}^{-1}\Grad\vek{\eta})=J\tr(\grad\vek{\eta})=J\div\vek{\eta}.
			\end{align*}
			Na predzadnjem koraku smo uporabili enakost (\ref{e:gz}).
		\item
			Za masno gostoto $\rho_R$ referenčne konfiguracije se predpostavi, da je znana. Če je
			$\rho$ polje masne gostote trenutne konfiguracije glede na gibanje $\chi$,
			dobimo iz relacije (\ref{e:rojror})
			\[ \rho^*=\frac{\rho_R}{J^*}. \]
			Izračunajmo še variacijo:
			\begin{equation} \label{e:deltarho}
				\delta\rho=\rho_R\delta\left(\frac{1}{J}\right)=
				\rho_R\frac{-1}{J^2}(\delta J)=-\frac{\rho_R}{J}\frac{J\div\vek{\eta}}{J}=-\rho\div\vek{\eta}.
			\end{equation}
	\end{enumerate}
\end{primeri}


\section{Hamiltonov princip}


Preden podamo Hamiltonov princip, ki nam bo povedal, kako poiskati pravo gibanje telesa,
moramo definirati še nekaj fizikalnih količin, ki nastopajo v Hamiltonovem funkcionalu.


\subsection{Kinetična energija}


\emph{Kinetična energija} telesa glede na dopustno gibanje $\chi$ je definirana kot
\begin{equation*}
	T = \int_{\B}\frac{1}{2}\langle\vek{v},\vek{v}\rangle\rho_R\, dV =
	\int_{\B_t}\frac{1}{2}\langle\vek{v},\vek{v}\rangle\rho\, dv
\end{equation*}
in je funkcija časa $t\in I$. Kinetična energija bližnjega gibanja $\chi+\varepsilon\vek{\eta}$
je potemtakem
\begin{equation*}
	T^* = \int_{\B}\frac{1}{2}\langle\vek{v}^*,\vek{v}^*\rangle\rho_R\, dV =
	\int_{\bottop{\B}{t}{*}}\frac{1}{2}\langle\vek{v}^*,\vek{v}^*\rangle\rho^*\, dv.
\end{equation*}
Pri tem je $\bottop{\B}{t}{*}=\chi(\B,t)+\varepsilon\vek{\eta}(\B,t)$.
Variacijo kinetične energije je lažje izračunati iz prvega izraza, saj lahko odvajanje prenesemo pod
integralski znak, ker se integracijsko območje z $\varepsilon$ ne spreminja. Skalarni produkt pod integralom
odvajamo po pravilu za odvod produkta in na koncu dobimo
\[
	\delta T =
	\int_{\B}\langle\delta\vek{v},\vek{v}\rangle\rho_R\, dV=
	\int_{\B}\langle\dot{\vek{\eta}},\vek{v}\rangle\rho_R\, dV.
\]
Zanimal nas bo še integral kinetične energije od časa $t_1$ do časa $t_2$,
\[
	\mathcal{I}=\int_{t_1}^{t_2} T\,dt=\int_{t_1}^{t_2}
	\int_{\B}\frac{1}{2}\langle\vek{v},\vek{v}\rangle\rho_R\, dV \,dt,
\]
katerega variacija je
\begin{equation}\label{e:varInf}
	\delta \mathcal{I}=\int_{t_1}^{t_2}\int_{\B}\langle\dot{\vek{\eta}},\vek{v}\rangle\rho_R\, dV\, dt.
\end{equation}
Zapišemo jo lahko še nekoliko drugače, v obliki, ki bo v nadaljevanju za nas bolj pomembna.
V (\ref{e:varInf}) smemo po Fubinijevem izreku zamenjati vrstni red integriranja.
Notranji integral, ki je sedaj integral po času, integriramo per-partes:
\[
	\int_{t_1}^{t_2}\langle\rho_R\vek{v},\dot{\vek{\eta}}\rangle\, dt =
	\langle\rho_R\vek{v},\vek{\eta}\rangle\Big|_{t_1}^{t_2}-
	\int_{t_1}^{t_2}\langle\rho_R\vek{a},\vek{\eta}\rangle\,dt.
\]
Prvi izraz na desni strani je enak 0, saj je $\vek{\eta}(\cdot,t_1)=\vek{\eta}(\cdot,t_2)=\vek{0}$.
Rezultat vstavimo nazaj v (\ref{e:varInf}) in dobimo
\begin{equation}\label{e:varkin}
	\delta \mathcal{I} =
	-\int_{t_1}^{t_2}\int_{\B}\langle\rho_R\vek{a},\vek{\eta}\rangle\, dV\, dt=
	-\int_{t_1}^{t_2}\int_{\B_t}\langle\rho\vek{a},\vek{\eta}\rangle\, dv\, dt.
\end{equation}
Pri tem druga enakost velja zaradi posledice \ref{p:roji}.


\subsection{Potencialna energija}


Zunanje sile so prostorninske in površinske sile, ki delujejo na materialno telo med gibanjem $\chi$
in so posledica interakcije telesa z okoljem. Rezultanto $\vek{f}(t)$ vseh zunanjih sil, ki delujejo
na materialno telo ob času $t\in I$, lahko zapišemo kot vsoto
\[
	\vek{f}(t)=\vek{f}_b(t)+\vek{f}_s(t),
\]
kjer je
\[
	\vek{f}_b(t)=\int_{B_t}\vek{b}\rho\, dv=\int_{\B}\vek{b}\rho_R\, dV
\]
\emph{telesna sila}, ki deluje znotraj telesa, ter
\begin{equation}\label{e:zunsil}
	\vek{f}_s(t)=\int_{\partial \B_t}\vek{t}\, da=\int_{\partial \B}\vek{t}_R\, dA
\end{equation}
\emph{stična sila}, ki deluje na rob telesa.
Vektorsko polje $\vek{b}$ je \emph{gostota prostorninske sile}, njegove vrednosti imajo fizikalno enoto
pospeška, torej \textit{meter na kvadratno sekundo}, $ms^{-2}$. Primer takega vektorskega polja
je npr. gravitacijski pospešek. Predpostavili bomo, da je $\vek{b}$ konzervativno vektorsko polje,
kar pomeni, da obstaja $C^1$ skalarno polje $\phi\colon\E\times I\to\R$, da je
\begin{equation*} \label{e:gps}
	\vek{b}(x,t)=-\grad\phi(x,t).
\end{equation*}
Vektorsko polje
\begin{equation*} \label{e:csv}
	\vek{t}=\vek{t}(x,t),\quad (x,t)\in \{(x,t)\:;\ x=\chi_t(\partial \B),\ t\in I\}
\end{equation*}
se imenuje \emph{Cauchyjev napetostni vektor}, polje
\begin{equation*} \label{e:pksv}
	\vek{t}_R=\vek{t}_R(X,t),\quad (X,t)\in \partial \B\times I
\end{equation*}
pa \emph{Piola-Kirchhoffov napetostni vektor}. Oba
imata fizikalno enoto tlaka, torej \textit{Newton na kvadratni meter}, $Nm^{-2}$.
Primer napetostnega vektorja je npr. atmosferski tlak.
Iz enačbe (\ref{e:zunsil}) vidimo, da mora veljati zveza
\begin{equation}\label{e:ttr}
	\vek{t}\, da=\vek{t}_R\, dA.
\end{equation}
Tudi za napetostni vektor bomo predpostavili, da je konzervativno vektorsko polje, v smislu,
da obstaja v času zvezno odvedljiv, v prostoru pa odsekoma zvezno odvedljiv\footnote{Namesto gradienta imamo v tem primeru
smerni odvod, definiran le v smereh iz tangentne ravnine na ploskev $\partial B_t$; vseeno bomo uporabljali oznako za gradient.}
skalarni potencial $\psi\colon\E\to\R$, da velja
\[ \vek{t}_R(x,t)=-\grad\psi(x,t)\quad\textrm{za}\quad (x,t)\in \{(x,t)\:;\ x=\chi_t(\partial \B),\ t\in I\}, \]
kjer je $\vek{t}_R(x,t)=\vek{t}_R(\chi(X,t),t)$ prostorski opis polja $\vek{t}_R$.

Za polje $\vek{b}$ bomo predpostavili, da je znano.
Če imamo podan predpis za polje $\vek{t}_R$ na območju $\partial\B\times I$, potem je ta predpis
robni pogoj napetosti. Pri mešanih robnih pogojih poznamo predpis za polje $\vek{t}_R$ na
območju $\partial\B_2\times I$, na območju $\partial\B_1\times I$ pa je podan predpis $\chi=q$.
Na kinematične robne pogoje in robne pogoje napetosti lahko gledamo kot na poseben
primer mešanih robnih pogojev; v prvem primeru je $\partial_2\B=\emptyset$, v drugem pa
$\partial_1\B=\emptyset$.

Izraz
\begin{equation*}
	W = \int_{\B}\phi\rho_R\, dV + \int_{\partial_2 \B}\psi\, dA
\end{equation*}
predstavlja del \emph{potencialne energije} telesa, ki izvira iz potencialov zunanjih sil.
Preostali del potencialne energije tvori \emph{notranja} ali \emph{deformacijska energija},
ki jo bomo označili z $U$, in je odvisna od materiala. Celotna potencialna energija je torej $W+U$.
V razdelku \ref{sec:primeri} si bomo ogledali primera
elastičnih tekočin in hiperelastičnih trdnih teles in takrat bomo tudi podali izraz
za notranjo energijo za ta dva primera.


Skalarni polji $\phi$ in $\psi$ sta na prostoru $\E\times I$ definirani neodvisno od gibanja.
Seveda to ne velja za njun materialni opis, od gibanja $\chi$ sta odvisni preko zvez
\[ \hat{\phi}(X,t)=\phi(\chi(X,t),t)\quad\textrm{in}\quad\hat{\psi}(X,t)=\psi(\chi(X,t),t) \]
za $(X,t)\in \B\times I$.
Če imamo namesto gibanja $\chi$ njegovo bližnje gibanje $\chi+\varepsilon\vek{\eta}$, potem je
\[
	\hat{\phi}^*(X,t)=\phi\big(\chi(X,t)+\varepsilon\vek{\eta}(X,t),t\big)\quad\textrm{in}
	\quad\hat{\psi}^*(X,t)=\psi\big(\chi(X,t)+\varepsilon\vek{\eta}(X,t),t\big).
\]
Če ta dva izraza odvajamo po $\varepsilon$, nato pa postavimo $\varepsilon=0$, dobimo variaciji
\[
	\delta\phi=\langle\grad\phi,\vek{\eta}\rangle=-\langle\vek{b},\vek{\eta}\rangle,\qquad
	\delta\psi=\langle\grad\psi,\vek{\eta}\rangle=-\langle\vek{t}_R,\vek{\eta}\rangle.
\]
Variacijo izraza $W$ imamo sedaj na dlani:
\begin{equation} \label{e:varvide}
	\delta W = -\int_{\B}\langle\rho_R\vek{b},\vek{\eta}\rangle\, dV
	- \int_{\partial_2 \B}\langle\vek{t}_R,\vek{\eta}\rangle\, dA=
	-\int_{\B_t}\langle\rho\vek{b},\vek{\eta}\rangle\, dv
	- \int_{\partial_2 \B_t}\langle\vek{t},\vek{\eta}\rangle\, da.
\end{equation}
Tu smo z $\partial_2 \B_t$ označili $\chi_t(\partial_2 \B)$ in uporabili zvezo (\ref{e:ttr}).
Variacija $\delta W$ je sicer splošno znana
pod imenom \emph{virtualno delo} zunanjih sil. Kot bomo videli v nadaljevanju,
ni potrebno, da poznamo skalarna potenciala $\phi$ in $\psi$, ker nas bo zanimalo le virtualno delo.
Pomembno je le, da predpostavimo njun obstoj.


\subsection{Hamiltonov funkcional in Hamiltonov princip}


\begin{definicija}
	Funkcional, ki gibanju $\chi\in \mathcal{D}$ s kinetično energijo $T$ in potencialno
	energijo $U+W$ priredi realno število
	\[
		H=\int_{t_1}^{t_2}(T-U-W)\,dt,
	\]
	se imenuje \emph{Hamiltonov funkcional}.
\end{definicija}

\begin{definicija}
	Naj bo $\mathcal{D}\subseteq\mathcal{A}$ množica, dobljena kot presek afinega podprostora $\mathcal{A}$ in
	neke odprte množice v prostoru $\mathcal{X}$.
	Funkcional $F\colon \mathcal{D}\to\R$ ima lokalni minimum pri dopustnem gibanju
	$\chi_{0}\in \mathcal{D}$, če obstaja $\varepsilon >0$,
	da za vsak $\chi\in \mathcal{D}$, za katerega je $\|\chi-\chi_{0}\|<\varepsilon$, velja $F(\chi_{0})<F(\chi)$.
\end{definicija}

\begin{hampri}
	Za pravo gibanje telesa velja, da je njegov jacobijan pozitiven na
	celotnem definicijskem območju, Hamiltonov funkcional pa pri pravem gibanju
	zavzame lokalni minimum.
\end{hampri}

Hamiltonov princip zajema načelo minimalnega odpora. Telo se bo gibalo tako, da
bo na poti razlika med kinetično in potencialno energijo čim manjša. Omenimo še, da zaradi zveznosti operatorja,
ki dopustnemu gibanju priredi njegov jacobijan, obstaja okolica pravega gibanja,
kjer je jacobijan vsakega gibanja iz te okolice še prav tako pozitiven.

\begin{trditev}
	Naj bo $\mathcal{D}\subseteq\mathcal{A}$ množica, dobljena kot presek afinega podprostora $\mathcal{A}$ in
	neke odprte množice v prostoru $\mathcal{X}$.
	Če funkcional $F\colon \mathcal{D}\to\R$ zavzame lokalni minimum pri dopustnem gibanju $\chi\in\mathcal{D}$, potem velja
	\[ \delta F(\chi)(\vek{\eta})=0\quad\textrm{za vsak}\ \vek{\eta}\in\mathcal{T}. \]
\end{trditev}

\proof
	Recimo, da $F$ zavzame lokalni minimum pri $\chi\in\mathcal{D}$ in naj bo $\vek{\eta}\in\mathcal{T}$ poljubna.
	Definirajmo funkcijo $\varphi\colon\R\to\R$ s predpisom
	\[ \varphi(\varepsilon)=F(\chi+\varepsilon\vek{\eta}). \]
	$\varphi$ ima lokalni minimum pri $\varepsilon=0$, zato velja $\varphi'(0)=0.$
	Po drugi strani pa je
	\[
		\varphi'(0)=\at{\frac{d}{d\varepsilon}F(\chi+\varepsilon\vek{\eta})}{\varepsilon=0}=
		\delta F(\chi)(\vek{\eta}).
	\]
\endproof

\begin{posledica} \label{p:varham0}
	Potreben pogoj, ki mu mora pravo gibanje zadoščati, je pozitivnost jacobijana
	na celotnem definicijskem območju ter
	\[ \delta H = \int_{t_1}^{t_2}(\delta T-\delta U-\delta W)\,dt = 0. \]
\end{posledica}

Če poznamo izraz za notranjo energijo in njeno variacijo, potem iz posledice
dobimo integralsko enačbo. Izraz za variacijo kinetične energije in virtualno delo 
smo namreč že izpeljali. Iz integralske enačbe se da dobiti sistem lokalnih diferencialnih enačb
za tenzorska polja. Med rešitvami tega sistema je tudi sámo gibanje telesa.
Da pa dobimo lokalne enačbe, potrebujemo nekaj izrekov, ki so podani v naslednjem razdelku.


\section{Osnovne leme variacijskega računa}


Skozi ta razdelek naj oznaka $\W$ predstavlja
končnorazsežen vektorski prostor nad poljem realnih števil, opremljen s skalarnim produktom.
Naj bo $U$ podmnožica nekega metričnega prostora.
\emph{Nosilec} preslikave $\vek{w}\colon U\to\W$ je množica
\[ \mathrm{supp}\,\vek{w}=\overline{\{ x\in U\,;\ \vek{w}(x)\neq\vek{0} \}}. \]
Črta nad množico pomeni zaprtje množice. Če je $\mathrm{supp}\,\vek{w}$ kompaktna množi\-ca, potem rečemo,
da je $\vek{w}$ \emph{preslikava s kompaktnim nosilcem}. V nadaljevanju bomo pokazali, da obstajajo gladka
tenzorska polja s kompaktnim nosilcem, saj jih bomo potrebovali v izrekih, ki jih bomo navedli v tem razdelku.

Prepričajmo se, da je funkcija
\[
	g\colon\R\to\R,\quad g(x)=\left\{\begin{array}{ccl}
	\exp(-1/x)&;&x>0 \\ 0&;&x\leq 0 \end{array}\right.
\]
gladka. Očitno je to res za $x\neq 0$. Izračun limite v točki $0$ pokaže, da
je $g$ zvezna v 0. Za $x>0$ so odvodi funkcije $g$
\[
	g'(x)=\exp\left(-\frac{1}{x}\right)\frac{1}{x^2},\qquad
	g^{(k)}(x)=\exp\left(-\frac{1}{x}\right)p_k\left(\frac{1}{x}\right),
\]
kjer so $p_k$ polinomi, $k\in\mathbb{N}$. Če naredimo limito teh odvodov, ko
gre $x$ z desne proti 0, dobimo $g^{(k)}(0)=0$ za vse $k=1,2,\dots$. Leva limita
odvodov funkcije $g$ je pa očitno vedno 0. Obe limiti sta enaki, torej so vsi odvodi
funkcije $g$ zvezni v točki 0.

Oglejmo si sedaj funkcijo
\[
	\beta\colon\R\to\R,\quad \beta(x)=\left\{\begin{array}{ccl}
	\exp\left(-\dfrac{1}{1-x^2}\right)&;&|x|<1 \\ 0&;&|x|\geq 1 \end{array}\right. .
\]
Zapišemo jo lahko kot kompozitum funkcije $g$ ter funkcije $x\mapsto 1-x^2$. Obe funkciji sta
gladki povsod in tak je zato je tudi njun kompozitum $\beta$. Nosilec funkcije $\beta$ je
$[-1,1]$, v notranjosti nosilca pa je vrednost funkcije $\beta$ pozitivna.

Definirajmo skalarno polje $\gamma\colon\E_R\to\R$ s predpisom
\[
	\gamma(X)=\left\{\begin{array}{ccl}
	\exp\left(-\dfrac{1}{1-\|\vek{X}\|^2}\right)&;&\|\vek{X}\|<1 \\ 0&;&\|\vek{X}\|\geq 1 \end{array}\right.
\]
Pri tem je $\vek{X}$ krajevni vektor točke $X$. $\gamma$
lahko zapišemo kot kompozitum $X\mapsto\|\vek{X}\|\mapsto \beta(\|\vek{X}\|)$, zato je polje
$\gamma$ gladko s kompaktnim nosilcem
\[ \textrm{supp}\,\gamma=\overline{\big\{X\in\E_R\,;\ \|\vek{X}\|<1 \big\}}=\overline{K}(O,1). \]
Tu je $\overline{K}(O,1)$ zaprta krogla z radijem 1 okoli izhodiščne točke $O\in\E_R$
opazovališča $\hat{\iota}_O$ za $\E_R$.

Naj bo $\{\vek{b}_j\}_{j=1}^{n}$ baza prostora $\W$. Za neke $1\leq k\leq n$, $\alpha>0$,
$X_0\in\E_R$ in $t_0\in\R$ definirajmo tenzorsko polje
\begin{equation} \label{e:funiw}
	\vek{w}\colon\E_R\times\R\to\W,\qquad\vek{w}(X,t)=\beta\left(\frac{t-t_0}{\alpha}\right)
	\gamma\left(\frac{X-X_0}{\alpha}\right)\vek{b}_k.
\end{equation}
Hitro se lahko prepričamo, da je
\begin{equation} \label{e:supaw}
	\mathrm{supp}\,\vek{w}=\overline{K}(X_0,\alpha)\times[t_0-\alpha,t_0+\alpha],
\end{equation}
kjer je $\overline{K}(X_0,\alpha)=\{X\in\E_R\,;\ \|X-X_0\|\leq\alpha\}$ zaprta krogla
s središčem v $X_0$ in radijem $\alpha$. Tako definirano polje $\vek{w}$ je
tudi gladko, saj je produkt gladkih polj. V notranjosti nosilca je njegova 
edina neničelna (tj.~$k$-ta) komponenta glede na bazo $\{\vek{b}_j\}_{j=1}^{n}$ pozitivna.

Omenimo še to pomembno dejstvo, da za vsako zvezno preslikavo $\vek{w}$ s kompaktnim
nosilcem velja, da je $\vek{w}=\vek{0}$ na robu nosilca. Komplement nosilca je namreč odprta množica, na kateri
je $\vek{w}=\vek{0}$, in ker je $\vek{w}$ zvezna, je po zvezni razširitvi $\vek{w}=\vek{0}$ tudi na robu komplementa,
ki je enak robu nosilca.

\begin{lema}\label{l:1}
	Naj bo $\vek{f}\colon\B\times [t_1,t_2]\to\W$ zvezno polje. Če velja
	\begin{equation}\label{e:lem1}
		\int_{t_1}^{t_2}\int_{\B}\langle\vek{f},\vek{w}\rangle\,dV\,dt = 0
	\end{equation}
	za vsako gladko polje $\vek{w}\colon\B\times[t_1,t_2]\to\W$, za katerega velja
	\[
		\vek{w}(\cdot,t_1)=\vek{w}(\cdot,t_2)=\vek{0}\quad\textrm{in}
		\quad\vek{w}=\vek{0}\ \textrm{na}\ \partial \B\times[t_1,t_2],
	\]
	potem je $\vek{f}=\vek{0}$ na $\B\times [t_1,t_2]$.
\end{lema}

\proof
	Glede na fiksno bazo $\{\vek{b}_j\}_{j=1}^n$ prostora $\W$ lahko $\vek{f}$
	zapišemo v komponentni obliki kot $\vek{f}=f_j\vek{b}_j$.
	Recimo, da obstaja točka $(X_0,t_0)$ v notranjosti območja $\B\times[t_1,t_2]$, da je $f_{k}(X_0,t_0)\neq 0$
	za neki $1\leq k\leq n$.
	Potem zaradi zveznosti polja $\vek{f}$ obstaja dovolj majhen $\alpha>0$, da je $f_k$ na okolici
	$U=K(X_0,\alpha)\times(t_0-\alpha,t_0+\alpha)$ točke $(X_0,t_0)$ različna
	od 0 in enakega predznaka, hkrati pa je $U$ še v celoti vsebovana v notranjosti območja $\B\times[t_1,t_2]$.
	
	Definirajmo polje
	$\vek{w}$ kot v (\ref{e:funiw}), če je $f_{k}(X_0,t_0) > 0$, oz.~kot $-\vek{w}$,
	če je $f_{k}(X_0,t_0) < 0$.
	Skrčitev polja $\vek{w}$ na domeno $\B\times[t_1,t_2]$ tako ustreza vsem pogojem iz leme.
	Skalarni produkt $\langle\vek{f},\vek{w}\rangle$ je pozitiven na množici $U$ (katere zaprtje
	je ravno nosilec od $\vek{w}$) in je 0 drugje, zato je
	\[
		\int_{t_1}^{t_2}\int_B \langle\vek{f},\vek{w}\rangle\,dV\,dt=
		\int_{t_0-\alpha}^{t_0+\alpha}\int_{\overline{K}(X_0,\alpha)} \langle\vek{f},\vek{w}\rangle\,dV\,dt> 0,
	\]
	kar nasprotuje enačbi (\ref{e:lem1}). Torej mora veljati $f_k=0$ za vsak $k$,
	tj.~$\vek{f}=\vek{0}$ povsod v notranjosti območja $\B\times [t_1,t_2]$,
	zaradi zveznosti pa tudi na zaprtju.
\endproof

\begin{lema}\label{l:2}
	Naj rob $\partial\B$ sestoji iz komplementarnih ploskev $\partial_1 \B$ in
	$\partial_2 \B$. Naj bo $\mathcal{R}$ množica vseh tistih regularnih točk ploskve $\partial_2\B$,
	ki niso na robu ploskve $\partial_2\B$ in naj bo 
	polje $\vek{f}\colon \partial_2\B\times[t_1,t_2]\to \W$ zvezno na podobmočju $\mathcal{R}\times[t_1,t_2]$.
	Če velja
	\begin{equation}\label{e:lem2}
		\int_{t_1}^{t_2}\int_{\partial_2\B} \langle\vek{f},\vek{w}\rangle\,dA\,dt = 0
	\end{equation}
	za vsako gladko polje $\vek{w}\colon\B\times[t_1,t_2]\to\W$, za katero je
	\[
		\vek{w}(\cdot,t_1)=\vek{w}(\cdot,t_2)=\vek{0}\quad\textrm{in}
		\quad\vek{w}=\vek{0}\ \textrm{na}\ \partial_1\B\times[t_1,t_2],
	\]
	potem je $\vek{f}=\vek{0}$ na $\mathcal{R}\times [t_1,t_2]$.
\end{lema}

\proof
	Podobno, kot v prejšnjem dokazu, zapišemo $\vek{f}=f_j\vek{b}_j$ in predpostavimo
	$f_{k}(X_0,t_0)\neq 0$ za neke $1\leq k\leq n$, $t_0\in (t_1,t_2)$ in $X_0\in \mathcal{R}$.
	
	Ker točka $X_0$ ni na robu ploskve $\partial_2 \B$ in je regularna,
	obstaja za dovolj majhen $\alpha>0$ okolica $K(X_0,\alpha)\times(t_0-\alpha,t_0+\alpha)$
	točke $(X_0,t_0)$, tako da je $K(X_0,\alpha)\cap\partial_2\B\subset\mathcal{R}$,
	$(t_0-\alpha,t_0+\alpha)\subset[t_1,t_2]$ in je
	$f_k$ zaradi zveznosti $\vek{f}$ različna od 0 in enakega predznaka
	na območju $(K(X_0,\alpha)\cap\partial_2\B)\times(t_0-\alpha,t_0+\alpha)$.
	
	Definirajmo polje $\vek{w}$ kot v
	(\ref{e:funiw}), če je $f_{k}(X_0,t_0) > 0$, oz.~kot $-\vek{w}$,
	če je $f_{k}(X_0,t_0) < 0$. Skrčitev tako definiranega polja $\vek{w}$ na območje
	$\B\times[t_1,t_2]$ ustreza predpostavkam iz leme.
	Skalarni produkt $\langle\vek{f},\vek{w}\rangle$ je pozitiven na
	$(K(X_0,\alpha)\cap\partial_2\B)\times(t_0-\alpha,t_0+\alpha)$ in je 0 drugje na $\mathcal{R}\times[t_1,t_2]$,
	zato je\footnote{Ploščina množice $\partial_2\B\setminus \mathcal{R}$ je 0.}
	\[
		\int_{t_1}^{t_2}\int_{\partial_2\B} \langle\vek{f},\vek{w}\rangle\,dA\,dt=
		\int_{t_0-\alpha}^{t_0+\alpha}\int_{K(X_0,\alpha)\cap\partial_2\B} \langle\vek{f},\vek{w}\rangle\,dA\,dt> 0,
	\]
	kar nasprotuje enačbi (\ref{e:lem2}). Torej mora veljati $\vek{f}=\vek{0}$ na
	$\mathcal{R}\times [t_1,t_2]$.
\endproof

\begin{lema} \label{lema3}
	Naj bo $\chi\colon \B\times[t_1,t_2]\to\E$ gibanje razreda $C^1$ s pozitivnim jacobijanom $J$ in
	$\vek{f}\colon \B\times[t_1,t_2]\to\W$ zvezno tenzorsko polje. Če velja
	\begin{equation} \label{e:lem3}
		\int_{t_1}^{t_2}\int_{\chi_t(\B)}\langle\vek{f},\vek{w}\rangle \,dv\,dt=0
	\end{equation}
	za vsako gladko polje $\vek{w}\colon \B\times[t_1,t_2]\to\W$, za katerega velja
	\[
		\vek{w}(\cdot,t_1)=\vek{w}(\cdot,t_2)=\vek{0}\quad\textrm{in}
		\quad\vek{w}=\vek{0}\ \textrm{na}\ \partial \B\times[t_1,t_2],
	\]
	potem je $\vek{f}=\vek{0}$ na $\B\times[t_1,t_2]$ oz.~na $\Omega$.
\end{lema}
V enačbi (\ref{e:lem3}) nastopata polji $\vek{f}$ in $\vek{w}$ v prostorskem opisu.

\proof
	Enačba (\ref{e:lem3}) je po izreku \ref{i:prointrel} ekvivalentna enačbi
	\[ \int_{t_1}^{t_2}\int_{\B}\langle J\vek{f},\vek{w}\rangle \,dV\,dt=0. \]
	
	Ker je polje $J\vek{f}$ zvezno na $\B\times[t_1,t_2]$, je
	po lemi (\ref{l:1}) $J\vek{f}=\vek{0}$
	na $\B\times[t_1,t_2]$. Ker je $J>0$, je $\vek{f}=\vek{0}$ na $\B\times[t_1,t_2]$,
	v prostorskem opisu pa je potem tudi $\vek{f}=\vek{0}$ na $\Omega$.
\endproof

\begin{lema} \label{lema4}
	Naj rob $\partial \B$ sestoji iz komplementarnih ploskev $\partial_1 \B$ in $\partial_2 \B$.
	Naj bo $\chi\colon \B\times[t_1,t_2]\to\E$ gibanje razreda $C^1$ s pozitivnim jacobijanom $J$ in
	$f\colon\partial_2 \B\times[t_1,t_2]\to\R$ skalarno polje, zvezno na podobmočju $\mathcal{R}\times[t_1,t_2]$.
	Pri tem je $\mathcal{R}$ definirana enako, kot v lemi \ref{l:2}. Če velja
	\begin{equation}\label{e:lem4}
		\int_{t_1}^{t_2}\int_{\chi_t(\partial_2 \B)}\langle f\vek{n},\vek{w}\rangle\,da\,dt=0
	\end{equation}
	za vsako gladko vektorsko polje $\vek{w}\colon \B\times[t_1,t_2]\to\V$, za katero je
	\[
		\vek{w}(\cdot,t_1)=\vek{w}(\cdot,t_2)=\vek{0}\quad\textrm{in}
		\quad\vek{w}=\vek{0}\ \textrm{na}\ \partial_1 \B\times[t_1,t_2],
	\]
	potem je $f=0$ na $\mathcal{R}\times[t_1,t_2]$ oz.~na $\{(x,t)\,;\ x\in\chi_t(\mathcal{R}),\ t\in[t_1,t_2] \}$.
\end{lema}

\proof
	Enačba (\ref{e:lem4}) je po izreku \ref{i:suittra} ekvivalentna enačbi
	\[ \int_{t_1}^{t_2}\int_{\partial_2 \B}\langle fJ\ten{F}^{-T}\vek{N},\vek{w}\rangle\,dA\,dt=0. \]
	Polje $fJ\ten{F}^{-T}\vek{N}$ je zvezno na $\mathcal{R}\times[t_1,t_2]$,
	zato je po lemi (\ref{l:2}) $fJ\ten{F}^{-T}\vek{N}=\vek{0}$ na $\mathcal{R}\times[t_1,t_2]$.
	Ker so točke iz $\mathcal{R}$ regularne, je $\vek{N}\neq \vek{0}$, hkrati je po predpostavki $J>0$,
	zato je tudi $\ten{F}^{-T}$ nesingularnen tenzor. Od tod sledi, da je $f=0$ na $\mathcal{R}\times[t_1,t_2]$,
	v prostorskem opisu pa $f=0$ na $\{(x,t)\,;\ x\in\chi_t(\mathcal{R}),\ t\in[t_1,t_2] \}$.
\endproof


\section{Primera} \label{sec:primeri}


V tem razdelku bomo podali dva primera, na katerih bomo prikazali uporabo Hamiltonovega principa
kot sredstvo za izpeljavo enačb gibanja.


\subsection{Elastični fluid}


Elastični fluid je model tekočine, v katerem zanemarimo učinke viskoznosti. Če je tekočina
stisljiva, potem je njena notranja ali deformacijska energija
\[ U=\int_{\B}\rho_R e(\rho)\,dV=\int_{\chi_t(\B)}\rho e(\rho)\,dv, \]
kjer je $e\colon[0,\infty)\to\R$ funkcija razreda $C^2$, imenovana
\emph{gostota notranje energije} in ima fizikalno enoto $m^2s^{-2}$.
Predpostavili bomo, da je funkcija $e$ poznana, recimo iz zakona o idealnih plinih.

Izračunajmo variacijo notranje energije:
\begin{align*} 
	\delta U &=\at{\frac{dU^*}{d\varepsilon}}{\varepsilon=0}=\at{\frac{d}{d\varepsilon}
	\int_{\B}\rho_R e(\rho^*)\,dV }{\varepsilon=0} =
	\at{\int_{\B}\rho_R e'(\rho^*)\frac{d\rho^*}{d\varepsilon}\,dV }{\varepsilon=0} \\
	&= -\int_{\B}\rho_R e'(\rho)\rho\div\vek{\eta}\,dV
	= -\int_{\chi_t(\B)}\rho^2 e'(\rho)\div\vek{\eta}\,dv.
\end{align*}
Pri tem smo smeli zamenjati vrstni red odvajanja in integriranja, ker se referenčna
konfiguracija $\B$ ne spreminja, nato pa smo uporabili verižno pravilo, enačbo
(\ref{e:deltarho}) in posledico \ref{p:roji}.
Če sedaj na dobljenem rezultatu uporabimo relacijo 3 iz trditve \ref{t:divprop} in
nato divergenčni izrek \ref{i:divtheo}$_3$, dobimo
\begin{equation} \label{e:varpot}
	\delta U=\int_{\chi_t(\B)}\big<\grad\big(\rho^2 e'(\rho)\big),\vek{\eta}\big>\,dv
	-\int_{\chi_t(\partial \B)}\big<\rho^2 e'(\rho)\vek{n},\vek{\eta}\big>\,da.
\end{equation}

Od robnih pogojev imamo v tem primeru le robni pogoj napetosti.
Naj bo $\mathcal{R}\subseteq\partial \B$ množica regularnih točk ploskve $\partial \B$ in
predpostavimo, da imamo podan \emph{zunanji tlak}, tj.~zvezno skalarno polje
$p\colon \E\times[t_1,t_2]\to\R$, da je $\vek{t}=-p\vek{n}$ na
$\{(x,t)\,;\ x\in\chi_t(\mathcal{R}),\ t\in[t_1,t_2]\}$.

Če sedaj upoštevamo rezultate (\ref{e:varkin}), (\ref{e:varvide}) in (\ref{e:varpot}),
mora po posledici \ref{p:varham0} pravo gibanje zadoščati enačbi
\begin{multline} \label{e:elflupog}
	\int_{t_1}^{t_2}\bigg( \int_{\chi_t(\B)}\Big<-\rho\vek{a}-
	\grad\big(\rho^2 e'(\rho)\big)+\rho\vek{b},\vek{\eta}\Big>\,dv+ \\
	+\int_{\chi_t(\partial \B)}\big<\big(\rho^2 e'(\rho)-p\big)\vek{n},\vek{\eta}\big>\,da\bigg)\,dt=0
\end{multline}
za vsako variacijo gibanja $\vek{\eta}\in\mathcal{T}$. Med drugim mora to veljati za
vsako variacijo gibanja iz množice
\[
	\big\{ \vek{\eta}\in C^{\infty}(\B\times[t_1,t_2],\V)\,;\ \vek{\eta}(\cdot,t_1)=
	\vek{\eta}(\cdot,t_2)=\vek{0},\ \vek{\eta}=\vek{0}\ \textrm{na}\ \partial \B\times[t_1,t_2]
	\big\}\subset\mathcal{T};
\]
pri teh variacijah je vrednost drugega integrala enaka 0, na preostali enačbi
pa lahko potem uporabimo lemo \ref{lema3} in dobimo
\begin{equation} \label{e:eflinmom}
	\rho\vek{a}=\grad\big(\rho^2 e'(\rho)\big)+\rho\vek{b}\quad
	\textrm{na}\ \Omega.
\end{equation}
Če ta rezultat sedaj upoštevamo v (\ref{e:elflupog}), vidimo, da mora
za vsako variacijo gibanja $\vek{\eta}$ veljati
\[
	\int_{t_1}^{t_2}\int_{\chi_t(\partial \B)}\big<\big(\rho^2 e'(\rho)-p\big)\vek{n},\vek{\eta}\big>\,da\,dt=0,
\]
med drugim tudi za vsako variacijo iz množice
\[
	\big\{ \vek{\eta}\in C^{\infty}(\B\times[t_1,t_2],\V)\,;\ \vek{\eta}(\cdot,t_1)=
	\vek{\eta}(\cdot,t_2)=\vek{0} \big\}\subset\mathcal{T},
\]
zato lahko uporabimo lemo \ref{lema4} (v tem primeru je $\partial_1\B=\emptyset$) in dobimo še pogoj
\begin{equation} \label{e:rpzef}
	\rho^2 e'(\rho)=p \quad\textrm{na}\ \{ (x,t)\,;\ x\in\chi_t(\mathcal{R}),\ t\in[t_1,t_2] \}.
\end{equation}

Enačba (\ref{e:eflinmom}) je \emph{enačba za ohranitev gibalne količine}.
Poleg enačb (\ref{e:eflinmom}) in (\ref{e:rpzef}) sodi v sistem še enačba (\ref{e:lozom})
za ohranitev mase.

V krivuljnih koordinatah $(x^i)$ za prostor $\E$ se enačba (\ref{e:eflinmom}) glasi
\[
	\rho\Big(\frac{\partial v^i}{\partial t}+v^j \topbot{v}{i}{,\, j}\Big)
	=g^{ij}\frac{\partial (\rho^2e'(\rho))}{\partial x^j}+\rho b^i,
\]
pri čemer so $v^i$ in $b^i$ komponente vektorjev $\vek{v}=v^i\vek{g}_i$ in $\vek{b}=b^i\vek{g}_i$,
$g^{ij}$ pa so metrični koeficienti.
Pospešek $\vek{a}$ smo pri tem izrazili kot materialni odvod hitrosti (\ref{e:maohit}).


\subsection{Hiperelastično trdno telo}


\emph{Hiperelastično trdno telo} je model telesa iz trdnega materiala, v katerem se predpostavi,
da je notranja ali deformacijska energija telesa podana kot
\[ U=\int_{\B}\rho_R \sigma(\ten{F})\,dV=\int_{\chi_t(\B)}\rho\sigma(\ten{F})\,dv, \]
kjer je $\sigma\colon\L(\V)\to\R$ \emph{funkcija shranjene energije} ali \emph{gostota deformacijske energije},
vrednost katere je odvisna od
deformacijskega gradienta $\ten{F}$. Za funkcijo $\sigma$ bomo v tem razdelku predpostavili, da je znana in da je
razreda $C^1$ na svoji domeni. V dodatku A je za funkcijo $\sigma$ navedenih nekaj dodatnih lastnosti,
bolj podrobno pa je obdelana za primer izotropnega hiperelastičnega trdnega telesa.

Pri danem $\ten{F}$ je $D\sigma(\ten{F})$ linearen funkcional
na vektorskem prostoru $\L(\V)$ s svojim skalarnim produktom, zato po Riezsovem izreku
o reprezentaciji linearnih funkcionalov pripada funkcionalu $D\sigma(\ten{F})$ natanko
določen tenzor $\tilde{\ten{S}}\in\L(\V)$, da za vsak $\ten{A}\in\L(\V)$ velja
\[ D\sigma(\ten{F})(\ten{A})=\langle\tilde{\ten{S}},\ten{A}\rangle=\tr(\tilde{\ten{S}}^{T}\ten{A}). \]
Po drugi točki dogovora \ref{d:riesz}
bomo odvod $D\sigma(\ten{F})$ enačili s pripadajočim tenzorjem $\tilde{\ten{S}}$.
Definirajmo tenzorsko polje
\begin{equation} \label{e:piola}
	\ten{S}\colon \B\times I\to\L(\V),\qquad
	\ten{S}(X,t)=\rho_R(X)D\sigma(\ten{F}(X,t)).%\quad\textrm{oz.~krajše}\quad
	%\ten{S}=\rho_RDe(\ten{F}).
\end{equation}
$\ten{S}(X,t)$ se imenuje \emph{prvi Piola-Kirchhoffov napetostni tenzor}.

Izračunajmo variacijo notranje energije:
\begin{align*} 
	\delta U &=\at{\frac{dU^*}{d\varepsilon}}{\varepsilon=0}=\at{\frac{d}{d\varepsilon}
	\int_{\B}\rho_R \sigma(\ten{F}^*)\,dV }{\varepsilon=0} =
	\at{\int_{\B}\rho_R \left< D\sigma(\ten{F}^*),\frac{d\ten{F}^*}{d\varepsilon}\right>\,dV }{\varepsilon=0} \\
	&= \int_{\B}\rho_R \langle D\sigma(\ten{F}),\Grad\vek{\eta}\rangle\,dV
	=\int_{\B}\langle\ten{S},\Grad\vek{\eta}\rangle\,dV.
	%= -\int_{\chi_t(B)}\rho^2 e'(\rho)\div\vek{\eta}\,dv
\end{align*}
Pri tem smo uporabili enakost (\ref{e:deltaF}). Če sedaj upoštevamo zvezi (\ref{e:gz}) in
(\ref{e:skatra}) ter izrek (\ref{i:prointrel}), dobimo
\begin{align*} 
	\delta U = \int_{\B}J\left<\frac{1}{J}\ten{S}\ten{F}^{T},\grad\vek{\eta}\right>\,dV =
	\int_{\chi_t(\B)}\langle\ten{T},\grad\vek{\eta}\rangle\,dv,
\end{align*}
kjer se vrednost tenzorskega polja
\begin{equation} \label{e:napetostni}
	\ten{T}=\frac{1}{J}\ten{S}\ten{F}^{T}=\rho D\sigma(\ten{F})\ten{F}^{T}
\end{equation}
imenuje \emph{Cauchyjev napetostni tenzor}. Obe dobljeni enačbi za $\delta U$ lahko
s pomočjo izrekov \ref{e:divSu} in \ref{i:divtheo} zapišemo v obliki
\begin{align}
	\delta U &= \int_{\partial_2\B}\langle\ten{S}\vek{N},\vek{\eta}\rangle\,dA-
	\int_{\B}\langle\Div\ten{S},\vek{\eta}\rangle\,dV \label{e:vapoesr} \\
	&= \int_{\chi_t(\partial_2\B)}\langle\ten{T}\vek{n},\vek{\eta}\rangle\,da-
	\int_{\chi_t(\B)}\langle\div\ten{T},\vek{\eta}\rangle\,dv. \label{e:vapoesp}
\end{align}
Pri tem smo upoštevali, da sta ploskovna integrala po ploskvah $\partial_1\B$ in
$\chi_t(\partial_1\B)$ enaka 0, saj je tam zaradi kinematičnega robnega pogoja $\vek{\eta}=\vek{0}$.

Po posledici \ref{p:varham0} mora pravo gibanje zadoščati enačbama
\begin{multline} \label{e:globoref}
	\int_{t_1}^{t_2}\bigg( \int_{\B}\langle -\rho_R\vek{a}+
	\Div\ten{S}+\rho_R\vek{b},\vek{\eta}\rangle \,dV+ \\
	+\int_{\partial_2 \B}\langle\vek{t}_R-\ten{S}\vek{N},\vek{\eta}\rangle\,dA\bigg)\,dt=0
\end{multline}
in
\begin{multline} \label{e:globoprost}
	\int_{t_1}^{t_2}\bigg( \int_{\chi_t(\B)}\langle -\rho\vek{a}+
	\div\ten{T}+\rho\vek{b},\vek{\eta}\rangle\,dv+ \\
	+\int_{\chi_t(\partial_2 \B)}\langle\vek{t}-\ten{T}\vek{n},\vek{\eta}\rangle\,ds\bigg)\,dt=0
\end{multline}
za vsa dopustna gibanja $\vek{\eta}\in\mathcal{T}$. Pri tem smo zopet upoštevali
že prej izpeljana rezultata (\ref{e:varkin}) in (\ref{e:varvide}) ter
variaciji notranje energije (\ref{e:vapoesr}) oz.~(\ref{e:vapoesp}).
Od tu dalje postopamo na enak način, kot smo pri primeru za elastični fluid.
Iz enačbe (\ref{e:globoref}) dobimo z uporabo leme \ref{l:1}
\begin{equation} \label{e:heslmmat}
	\rho_R\vek{a}=\Div\ten{S}+\rho_R\vek{b}\quad\textrm{na}\ \B\times[t_1,t_2],
\end{equation}
iz enačbe (\ref{e:globoprost}) pa z uporabo leme \ref{lema3} dobimo
\begin{equation} \label{e:heslmpro}
	\rho\vek{a}=\div\ten{T}+\rho\vek{b}\quad\textrm{na}\ \Omega.
\end{equation}
Če dobljena rezultata upoštevamo v (\ref{e:globoref}) oz.~v (\ref{e:globoprost}) in
nato uporabimo lemo \ref{l:2} oz.~lemo \ref{lema4}, dobimo še
\begin{equation} \label{e:robokop}
	\ten{S}\vek{N}=\vek{t}_R\quad\textrm{na}\ \mathcal{R}\times[t_1,t_2]
\end{equation}
oziroma
\begin{equation} \label{e:robolom}
	\ten{T}\vek{n}=\vek{t}\quad\textrm{na}\ \{(x,t)\,;\ x\in\chi_t(\mathcal{R}),\ t\in[t_1,t_2] \}.
\end{equation}
Pri tem je $\mathcal{R}$ množica regularnih točk ploskve $\partial_2\B$.

Tudi tu sta (\ref{e:heslmmat}) in (\ref{e:heslmpro}) \emph{enačbi za ohranitev gibalne količine}.
Poleg enačb (\ref{e:heslmmat}) in (\ref{e:robokop}) oz.~(\ref{e:heslmpro}) in (\ref{e:robolom})
sodi v sistem še enačba ohranitve mase (\ref{e:lozom}).

V krivuljnih koordinatah $(x^i)$ za prostor $\E$ se enačbi (\ref{e:heslmpro}) in (\ref{e:robolom}) glasita
\begin{gather*}
	\rho\Big(\frac{\partial v^i}{\partial t}+v^j\topbot{v}{i}{,\, j}\Big)=\topbot{T}{ij}{,\, j}+\rho b^i,\\
	t^i=T^{ij}n_j=\topbot{T}{i}{j}n^j,
\end{gather*}
pri čemer so $\vek{v}=v^i\vek{g}_i$, $\vek{b}=b^i\vek{g}_i$, $\vek{t}=t^i\vek{g}_i$,
$\vek{n}=n^i\vek{g}_i=n_i\vek{g}^i$ in $\ten{T}=T^{ij}\vek{g}_i\otimes\vek{g}_j=\topbot{T}{i}{j}\vek{g}_i\otimes\vek{g}^j$.
Enačba (\ref{e:heslmmat}) se glasi
\begin{equation*}
	\rho_R\Big(\frac{\partial v^i}{\partial t}+v^j\topbot{v}{i}{,\, j}\Big)=\topbot{S}{iK}{,\, K}+\rho_R b^i,
\end{equation*}
pri čemer je $\ten{S}=S^{iK}\vek{g}_i\otimes\vek{G}_K$
in $\{\vek{G}_K\}$ je kovariantna baza za koordinatni sistem $(X^K)$ na $\B\subset\E_R$.