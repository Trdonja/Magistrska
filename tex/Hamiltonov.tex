\chapter{Hamiltonov princip in variacijske metode}


\section{Dopustno gibanje in variacija}


V tem poglavju bomo s pomočjo t.~i.~\emph{Hamiltonovega principa} poiskali potrebne pogoje,
ki jim mora zadoščati pravo gibanje telesa.
Recimo, da pravo gibanje iščemo med vsemi gibanji
$\chi\colon \overline{B}\times [t_1,t_2]\to\E$ razreda $C^2$, ki zadoščajo \emph{začetnim pogojem}
\[ \chi(X,t_1)=\chi_{t_1}(X)\quad\textrm{in}\quad \chi(X,t_2)=\chi_{t_2}(X), \]
kjer sta $\chi_{t_1},\chi_{t_2}\colon B\to\E$ znani konfiguraciji razreda $C^2$, ki
določata položaj telesa ob začetnem in končnem času $t_1$ oz.~$t_2$.
Poleg začetnih pogojev so vnaprej predpisani še robni pogoji. \emph{Geometrijski robni pogoj} je predpis
\begin{equation*} \label{e:grp}
	\chi(X,t)=\chi_g(X,t)\quad\mathrm{za}\quad (X,t)\in \partial_g B\times [t_1,t_2],
\end{equation*}
kjer je $\partial_g B\subseteq\partial B$ del robu območja $B$.
Geometrijski robni pogoj torej v naprej določa položaj robnih točk telesa.
Na preostalem delu $\partial_d B=\partial B \setminus \partial_g B$
je potem predpisan \emph{dinamični robni pogoj}, ki ga bomo predstavili v naslednjem razdelku.
Dopuščamo možnost, da je $\partial_g B$ ali $\partial_d B$ prazna množica, torej,
da imamo le geometrijske ali pa le dinamične robne pogoje. Vendar pa mora vedno veljati
$\partial_g B\cup\partial_d B=\partial B$ in $\partial_g B\cap\partial_d B=\emptyset$.

Množica $\mathcal{X}$ vseh preslikav $\chi\colon\overline{B}\times [t_1,t_2]\to\E$
razreda $C^2$ tvori vektorski prostor.
$\mathcal{X}$ postane Banachov prostor, če na njem smiselno definiramo normo. Možna izbira
predpisa za normo je npr.
\[
	\|\cdot\|\colon\mathcal{X}\to[0,\infty),\quad
	\|\chi\|=\max\big\{\|\chi(X,t)\|\: ;\ (X,t)\in B\times [t_1,t_2]\big\}.
\]
Pri tem je z normo, ki se pojavi znotraj množice, mišljena Evklidska norma krajevnega vektorja točke $\chi(X,t)$.

\begin{definicija}
	\begin{enumerate}
		\item Preslikava $\chi\in\mathcal{X}$, ki zadošča predpisanim začetnim in geometrijskim robnim pogojem,
			se imenuje \emph{dopustno gibanje}. Množico vseh dopustnih gibanj bomo označili z $\mathcal{A}$.
		\item Vektorsko polje $\vek{\eta}\colon\overline{B}\times [t_1,t_2]\to\V$, ki je razreda $C^2$ in
			ustreza pogojem
			\[ \vek{\eta}(X,t_1)=\vek{0}\quad \textrm{in} \quad\vek{\eta}(X,t_2)=\vek{0}\quad \textrm{za vse}\ X\in\overline{B}\quad\textrm{ter} \]
			\[ \vek{\eta}(X,t)=\vek{0}\quad \textrm{za vse}\ (X,t)\in \partial_g B\times [t_1,t_2], \]
			se imenuje \emph{variacija gibanja}. Množico vseh variacij gibanja bomo ozna\-čili s $\mathcal{T}$.
		\item Za dopustno gibanje $\chi$, variacijo gibanja $\vek{\eta}$ in realno konstanto $\varepsilon$ se gibanje
			$\chi+\varepsilon\vek{\eta}$ imenuje \emph{bližnje gibanje} gibanja $\chi$.
	\end{enumerate}
\end{definicija}

Takoj je potrebno opozoriti, da kljub poimenovanju dopustno gibanje in bližnje gibanje nista nujno gibanji,
kot smo ju definirali v poglavju \ref{chp:kinkon}, ker nismo postavili pogoja o pozitivnosti jacobijana.
Ni se težko prepričati, da je $\mathcal{T}$ vektorski podprostor prostora $\mathcal{X}$, če na elemente iz $\mathcal{X}$
gledamo kot na vektorska polja s kodomeno $\V$ namesto $\E$.
Prav tako zlahka ugotovimo, da za poljubni dopustni gibanji $\chi_1$ in $\chi_2$
velja $\chi_2-\chi_1\in\mathcal{T}$. $\mathcal{A}$ je torej afini podprostor prostora $\mathcal{X}$, saj je
\[ \mathcal{A}=\big\{ \chi_0+\vek{\eta}\: ;\ \vek{\eta}\in\mathcal{T} \big\}=\chi_0+\mathcal{T}, \]
kjer je $\chi_0$ poljubno dopustno gibanje.

\begin{definicija}
	Naj bo $U$ množica, ki je dobljena kot presek neke odprte množice v $\mathcal{X}$ in afinega
	podprostora $\mathcal{A}$. Naj bo $\mathcal{Y}$ Banachov prostor in $F\colon U\to\mathcal{Y}$
	odvedljiva preslikava. Smerni odvod
	\[
		\delta F\colon U\to\L(\mathcal{T},\mathcal{Y}),\qquad
		\delta F(\chi)(\vek{\eta})=\at{\frac{d}{d\varepsilon}F(\chi+\varepsilon\vek{\eta})}{\varepsilon=0}
	\]
	se imenuje tudi \emph{variacija} operatorja $F$.
\end{definicija}

\begin{notacija}
	Naj bo preslikava $F$ definirana kot v prejšnji definiciji. Če velja $F(\chi)=Y$,
	potem oznaka $Y^*$ pomeni $Y^*=F(\chi+\varepsilon\vek{\eta})$ za neka $\vek{\eta}\in\mathcal{T}$ in $\varepsilon\in\R$,
	oznaka $\delta Y$ pa pomeni $\delta Y=\delta F(\chi)(\vek{\eta})$. V tej notaciji velja
	\[ \delta Y = \at{\Big(\frac{dY^*}{d\varepsilon}\Big)}{\varepsilon=0}, \]
	$\delta Y$ pa imenujemo tudi \emph{variacija} količine $Y$.
\end{notacija}

\begin{primeri}
	Poglejmo si nekaj posebnih primerov za preslikavo $F$ iz zadnje definicije.
	\begin{enumerate}
		\item
			Če $F$ dopustnemu gibanju $\chi$ priredi vektorsko polje hitrosti $\vek{v}$,
			$F(\chi)=d\chi/dt=\dot{\chi}=\vek{v}$, potem je
			\[ \vek{v}^*=\frac{d}{dt}(\chi+\varepsilon\vek{\eta})=\vek{v}+\varepsilon\dot{\vek{\eta}}, \]
			variacija polja hitrosti pa je
			\[ \delta\vek{v}=\dot{\vek{\eta}}. \]
		\item
			Če $F$ dopustnemu gibanju $\chi$ priredi polje deformacijskega gradienta $\ten{F}$,
			$F(\chi)=\Grad{\chi}=\ten{F}$, potem je
			\[ \ten{F}^*=\Grad(\chi+\varepsilon\vek{\eta})=\ten{F}+\varepsilon\Grad\vek{\eta}, \]
			variacija pa je
			\[ \delta\ten{F}=\Grad\vek{\eta}. \]
		\item
			Če je $J$ polje jacobijana dopustnega gibanja $\chi$, potem je
			\[ J^*=\det(\Grad(\chi+\varepsilon\vek{\eta}))=\det(\ten{F}^*). \]
			Pri računanju variacije polja $J$ se uporabi podobne prijeme,
			kot smo jih uporabili pri računanju enakosti (\ref{e:dotJ}). Dobimo
			\begin{align*}
				\delta J&=D\det(\ten{F})(\delta\ten{F})=J\ten{F}^{-T}\cdot\Grad\vek{\eta}\\
				&=J\tr(\ten{F}^{-1}\Grad\vek{\eta})=J\tr(\grad\vek{\eta})=J\div\vek{\eta}.
			\end{align*}
			Na predzadnjem koraku smo uporabili enakost (\ref{e:gz}).
		\item
			Za masno gostoto $\rho_R$ referenčne konfiguracije se predpostavi, da je znana. Če je
			$\rho$ polje masne gostote trenutne konfiguracije glede na gibanje $\chi$,
			dobimo iz relacije (\ref{e:rojror})
			\[ \rho^*=\frac{\rho_R}{J^*}. \]
			Izračunajmo še variacijo:
			\[
				\delta\rho=\rho_R\delta\left(\frac{1}{J}\right)=
				\rho_R\frac{-1}{J^2}(\delta J)=-\frac{\rho_R}{J}\frac{J\div\vek{\eta}}{J}=-\rho\div\vek{\eta}.
			\]
	\end{enumerate}
\end{primeri}


\section{Hamiltonov princip}


\emph{Kinetična energija} telesa glede na dopustno gibanje $\chi$ je definirana kot
\begin{equation*}
	T = \int\limits_{B}\frac{1}{2}\vek{v}\cdot\vek{v}\rho_R\, dV =
	\int\limits_{B_t}\frac{1}{2}\vek{v}\cdot\vek{v}\rho\, dv
\end{equation*}
in je funkcija časa $t\in[t_0,t_1]$. Kinetična energija bližnjega gibanja $\chi+\varepsilon\vek{\eta}$
je potemtakem
\begin{equation*}
	T^* = \int\limits_{B}\frac{1}{2}\vek{v}^*\cdot\vek{v}^*\rho_R\, dV =
	\int\limits_{\bottop{B}{t}{*}}\frac{1}{2}\vek{v}^*\cdot\vek{v}^*\rho^*\, dv.
\end{equation*}
Variacijo kinetične energije je lažje izračunati iz prvega izraza, saj lahko odvajanje prenesemo pod
integralski znak, ker se integracijsko območje ne spreminja. Skalarni produkt odvajamo
po pravilu za odvod produkta in dobimo
\[
	\delta T =
	\int\limits_{B}\delta\vek{v}\cdot\vek{v}\rho_R\, dV=
	\int\limits_{B}\dot{\vek{\eta}}\cdot\vek{v}\rho_R\, dV.
\]
Zanimal nas bo še integral kinetične energije od časa $t_0$ do časa $t_1$,
\[ I=\int_{t_0}^{t_1} T\,dt=\int_{t_0}^{t_1} \int_{B}\frac{1}{2}\vek{v}\cdot\vek{v}\rho_R\, dV \,dt, \]
katerega variacija je
\begin{equation}\label{e:varInf}
	\delta I=\int_{t_0}^{t_1}\int_{B}\dot{\vek{\eta}}\cdot\vek{v}\rho_R\, dV\, dt.
\end{equation}
Zapišemo jo lahko še nekoliko drugače, v obliki, ki bo v nadaljevanju za nas bolj pomembna.
V (\ref{e:varInf}) smemo po Fubinijevem izreku zamenjati vrstni red integriranja.
Notranji integral, ki je sedaj integral po času, integriramo per-partes:
\[
	\int_{t_0}^{t_1}\rho_R\vek{v}\cdot\dot{\vek{\eta}}\, dt =
	\left.\Big(\rho_R\vek{v}\cdot\vek{\eta}\Big)\right|_{t_0}^{t_1}-
	\int_{t_0}^{t_1}\rho_R\vek{a}\cdot\vek{\eta}\,dt.
\]
Prvi izraz na desni strani je enak 0, saj je $\vek{\eta}(\cdot,t_0)=\vek{\eta}(\cdot,t_1)=\vek{0}$.
Rezultat vstavimo nazaj v (\ref{e:varInf}) in dobimo
\begin{equation*}
	\delta I =
	-\int_{t_0}^{t_1}\int_{B}\rho_R\vek{a}\cdot\vek{\eta}\, dV\, dt=
	-\int_{t_0}^{t_1}\int_{B_t}\rho\vek{a}\cdot\vek{\eta}\, dv\, dt.
\end{equation*}
Pri tem druga enakost velja zaradi REF.

Rezultanto vseh zunanjih sil, ki delujejo na telo med dopustnim gibanjem $\chi$, lahko zapišemo v obliki
\begin{equation}\label{e:zunsil}
	\int\limits_{B_t}\vek{b}\rho\, dv + \int\limits_{\partial B_t}\vek{t}\, da =
	\int\limits_{B}\vek{b}\rho_R\, dV + \int\limits_{\partial B}\vek{t}_R\, dA.
\end{equation}
Vektorsko polje $\vek{b}$ je \emph{gostota prostorninske sile}, njegove vrednosti imajo fizikalno enoto
pospeška, torej \textit{meter na kvadratno sekundo}, $ms^{-2}$. Primer takega vektorskega polja
je npr. gravitacijski pospešek. Predpostavili bomo, da je $\vek{b}$ konzervativno vektorsko polje,
kar pomeni, da obstaja $C^1$ skalarno polje $\phi\colon\E\times [t_0,t_1]\to\R$, da je
\begin{equation*} \label{e:gps}
	\vek{b}(x,t)=\nabla_{x}\phi(x,t).
\end{equation*}
Vektorsko polje
\begin{equation*} \label{e:csv}
	\vek{t}=\vek{t}(x,t),\quad (x,t)\in \big\{\partial B_t\times\{t\}\: ;\ t\in [t_0,t_1]\big\}
\end{equation*}
se imenuje \emph{Cauchyjev napetostni vektor}, polje
\begin{equation*} \label{e:pksv}
	\vek{t}_R=\vek{t}_R(X,t),\quad (X,t)\in \partial B\times [t_0,t_1]
\end{equation*}
pa \emph{Piola-Kirchoffov napetostni vektor}. Oba
imata fizikalno enoto tlaka, torej \textit{Newton na kvadratni meter}, $Nm^{-2}$.
Primer napetostnega vektorja je npr. atmosferski tlak.
Velja zveza
\begin{equation}\label{e:ttr}
	\vek{t}\, da=\vek{t}_R\, dA.
\end{equation}
Tudi za napetosni vektor bomo predpostavili, da je konzervativno vektorsko polje v smislu,
da obstaja v času zvezno odvedljiv, v prostoru pa odsekoma zvezno odvedljiv\footnote{Namesto gradienta imamo v tem primeru
smerni odvod, definiran le v smereh iz tangentne ravnine na ploskev $\partial B_t$; vseeno bomo uporabljali oznako za gradient.}
skalarni potencial $\psi\colon\E\to\R$, da velja
\[ \vek{t}_R(x,t)=\nabla_{x}\psi(x,t)\quad\textrm{za}\quad (x,t)\in \big\{\partial B_t\times\{t\}\: ;\ t\in [t_0,t_1]\big\}, \]
kjer je $\vek{t}_R(x,t)=\vek{t}_R(\chi(X,t),t)$ prostorski opis polja $\vek{t}_R$.

Za polje $\vek{b}$ bomo predpostavili, da je znano. Predpostavili bomo tudi, da poznamo predpis za
polje $\vek{t}_R$ na območju $(X,t)\in \partial_d B\times [t_0,t_1]$ in ta predpis
se imenuje dinamični robni pogoj.

Izraz
\begin{equation*}
	W = \int\limits_{B}\phi\rho_R\, dV + \int\limits_{\partial_d B}\psi\, dA,
\end{equation*}
predstavlja del \emph{potencialne energije} telesa in je prav tako funkcija časa.
Preostali del potencialne energije tvori \emph{notranja energija}, ki jo bomo označili z $U$.
Predpis za notranjo energijo je odvisen od materiala. Mi si bomo v tem délu ogledali primer
idealnih tekočin in elastičnih trdnin. 

Skalarni polji $\phi$ in $\psi$ sta na prostoru $\E\times [t_0,t_1]$ definirani neodvisno od gibanja.
Seveda to ne velja za njun materialni opis, od gibanja $\chi$ sta odvisni preko zvez
\[ \hat{\phi}(X,t)=\phi\big(\chi(X,t),t\big)\quad\textrm{in}\quad\hat{\psi}(X,t)=\psi\big(\chi(X,t),t\big) \]
za $(X,t)\in B\times [t_0,t_1]$.
Če imamo namesto gibanja $\chi$ njegovo bližnje gibanje $\chi+\varepsilon\vek{\eta}$, potem je
\[
	\hat{\phi}^*(X,t)=\phi\big(\chi(X,t)+\varepsilon\vek{\eta}(X,t),t\big)\quad\textrm{in}
	\quad\hat{\psi}^*(X,t)=\psi\big(\chi(X,t)+\varepsilon\vek{\eta}(X,t),t\big).
\]
Če ta dva izraza odvajamo po $\varepsilon$, nato pa postavimo $\varepsilon=0$, dobimo variaciji
\[
	\delta\phi=\grad\phi\cdot\vek{\eta}=\vek{b}\cdot\vek{\eta},\quad
	\delta\psi=\grad\psi\cdot\vek{\eta}=\vek{t}_R\cdot\vek{\eta}.
\]
Variacijo izraza $W$ imamo sedaj na dlani:
\[
	\delta W = \int\limits_{B}\rho_R\vek{b}\cdot\vek{\eta}\, dV + \int\limits_{\partial_d B}\vek{t}_R\cdot\vek{\eta}\, dA=
	\int\limits_{B_t}\rho\vek{b}\cdot\vek{\eta}\, dv + \int\limits_{\partial_d B_t}\vek{t}\cdot\vek{\eta}\, da.
\]
Tu smo z $\partial_d B_t$ označili $\chi_t(\partial_d B)$ in uporabili zvezo (\ref{e:ttr}).
Variacija $\delta W$ je sicer splošno znana
pod imenom \emph{virtualno delo} zunanjih sil (\ref{e:zunsil}). Kot bomo videli v nadaljevanju,
ni potrebno, da poznamo skalarna potenciala $\phi$ in $\psi$, ker nas bo zanimalo le virtualno delo.
Pomembno je le, da predpostavimo njun obstoj.

\begin{definicija}
	Naj bo $D\subseteq\mathcal{A}$ množica, dobljena kot presek afinega podprostora $\mathcal{A}$ in
	neke odprte množice v prostoru $\mathcal{X}$.
	Funkcional $F\colon D\to\R$ ima lokalni minimum pri dopustnem gibanju $\chi^{0}\in D$, če obstaja $\varepsilon >0$,
	da za vsak $\chi\in D$, za katerega je $\|\chi-\chi^{0}\|<\varepsilon$, velja $F(\chi^{0})<F(\chi)$.
\end{definicija}

\begin{definicija}
	Funkcional, ki gibanju $\chi\in\mathcal{D}$ s kinetično energijo $T$ in potencialno
	energijo $W+U$ priredi realno število
	\[
		H=\int_{t_0}^{t_1}(T+W+U)\,dt,
	\]
	se imenuje \emph{Hamiltonov funkcional}.
\end{definicija}

\begin{aksiom}[Hamiltonov princip]
	Za pravo gibanje telesa velja, da je njegov jacobijan pozitiven na
	celotnem definicijskem območju, Hamiltonov funkcional pa pri pravem gibanju
	zavzame lokalni minimum.
\end{aksiom}

Hamiltonov princip zajema načelo minimalnega odpora. Telo se bo gibalo tako, da
bo za pot porabilo čim manj energije. Omenimo še, da zaradi zveznosti operatorja,
ki dopustnemu gibanju priredi njegov jacobijan, obstaja okolica pravega gibanja,
kjer je jacobijan vsakega gibanja iz te okolice še prav tako pozitiven.

\begin{trditev}
	Če funkcional $F\colon D\to\R$ zavzame lokalni minimum pri dopustnem gibanju $\chi$, potem velja
	\[ \big(\delta F(\chi)\big)[\vek{\eta}]=0\quad\textrm{za vsak}\ \vek{\eta}\in\mathcal{T}. \]
\end{trditev}

\proof
	Recimo, da $F$ zavzame lokalni minimum pri $\chi\in D$ in naj bo $\eta\in\mathcal{T}$ poljubna.
	Definirajmo funkcijo $\varphi\colon\R\to\R$ s predpisom
	\[ \varphi(\varepsilon)=F(\chi+\varepsilon\vek{\eta}). \]
	$\varphi$ ima lokalni minimum pri $\varepsilon=0$, zato velja $\varphi'(0)=0.$
	Po drugi strani pa je
	\[
		\varphi'(0)=\at{\frac{d}{d\varepsilon}F(\chi+\varepsilon\vek{\eta})}{\varepsilon=0}=
		\big(\delta F(\chi))[\vek{\eta}].
	\]
\endproof

\begin{posledica}
	Potreben pogoj, ki mu mora pravo gibanje zadoščati, je pozitivnost jacobijana
	na celotnem definicijskem območju ter
	\[ \delta H = \int_{t_0}^{t_1}(\delta T+\delta W+\delta U)\,dt = 0. \]
\end{posledica}

Če poznamo izraz za notranjo energijo in njeno variacijo, potem iz posledice
dobimo integralsko enačbo. Izraz za variacijo kinetične energije in virtualno delo 
smo namreč že izpeljali. Iz integralske enačbe se da dobiti sistem lokalnih diferencialnih enačb
za tenzorska polja. Med rešitvami tega sistema je tudi samo gibanje telesa.
Da pa dobimo lokalne enačbe, potrebujemo nekaj izrekov, ki so podani v naslednjem razdelku.


\section{Osnovne leme variacijskega računa}


Skozi ta razdelek naj oznaka $\W$ stoji za
končnorazsežen vektorski prostor nad poljem realnih števil, opremljen s skalarnim produktom.
\emph{Nosilec} tenzorskega polja $w\colon\E\times\R\to\W$ je množica
\[ \mathrm{supp}\,w=\overline{\{ (X,t)\in\E\times\R\; ;\ w(X,t)\neq 0 \}}. \]
Črta nad množico pomeni zaprtje množice. Če je nosilec od $w$ kompaktna množica, potem rečemo,
da je $w$ polje s kompaktnim nosilcem. V nadaljevanju bomo pokazali, da obstajajo gladka polja
s kompaktnim nosilcem, saj jih bomo potrebovali v izrekih, ki jih bomo navedli v tem razdelku.

Prepričajmo se, da je funkcija
\[
	g\colon\R\to\R,\quad g(x)=\left\{\begin{array}{ccl}
	\exp\left(-\frac{1}{x}\right)&;&x>0 \\ 0&;&x\leq 0 \end{array}\right.
\]
gladka. Očitno je to res za $x\neq 0$. Očitno je tudi, da je v točki $0$ zvezna. Za $x>0$
so odvodi funkcije $g$
\[
	g'(x)=\exp\left(-\frac{1}{x}\right)\frac{1}{x^2};\quad
	g^{(k)}(x)=\exp\left(-\frac{1}{x}\right)p_k\left(\frac{1}{x}\right),
\]
kjer so $p_k$ neki polinomi, $k=1,2,\dots$ Če naredimo limito teh odvodov, ko
gre $x$ z desne proti 0, dobimo $g^{(k)}(0)=0$ za vse $k=1,2,\dots$. Leva limita
odvodov funkcije $g$ je pa očitno vedno 0 in tako smo dokazali, da so vsi odvodi funkcije $g$ zvezni
v točki 0.

Oglejmo si sedaj funkcijo
\[
	h\colon\R\to\R,\quad h(x)=\left\{\begin{array}{ccl}
	\exp\left(-\frac{1}{1-x^2}\right)&;&|x|<1 \\ 0&;&|x|\geq 1 \end{array}\right. .
\]
Zapišemo jo lahko kot kompozitum funkcije $g$ ter funkcije $x\mapsto 1-x^2$. Obe funkciji sta
gladki povsod in tak je zato je tudi njun kompozitum $h$. Nosilec funkcije $h$ je
$[-1,1]$, v notranjosti nosilca pa je vrednost funkcije $h$ pozitivna.

Definirajmo skalarno polje $u\colon\E\times\R\to\R$ s predpisom
\[
	u(X,t)=\left\{\begin{array}{ccl}
	\exp\left(-\frac{1}{1-\|(X,t)\|^2}\right)&;&\|(X,t)\|<1 \\ 0&;&\|(X,t)\|\geq 1 \end{array}\right.
\]
Zapišemo ga lahko kot kompozitum $(X,t)\mapsto\|(X,t)\|\mapsto h(\|(X,t)\|)$, zato je
$u$ gladko s kompaktnim nosilcem
\[ \textrm{supp}\, u=\overline{\big\{(X,t)\in\E\times\R\; ;\ \|(X,t)\|<1 \big\}}. \]
Naj bo $(X_c,t_c)\in\E\times\R$, $\alpha > 0$ in označimo z
\begin{equation}\label{e:mnU}
	U_{\alpha}=\big\{(X,t)\in\E\times\R\; ;\ \|(X,t)-(X_c,t_c)\|<\alpha \big\}
\end{equation}
$\alpha$-okolico točke $(X_c,t_c)$. Skalarno polje
\begin{equation}\label{e:funu}
	u_{\alpha}(X,t)=u\left(\frac{(X,t)-(X_c,t_c)}{\alpha}\right)
\end{equation}
je gladko s kompaktnim nosilcem $\overline{U_{\alpha}}$.

Omenimo še to pomembno dejstvo, da za vsako gladko tenzorska polje $w\colon\E\times\R\to\W$ s kompaktnim
nosilcem velja, da je $w=0$ na robu nosilca. Komplement nosilca je namreč odprta množica, na kateri
je $w=0$, in ker je $w$ zvezno, je po zvezni razširitvi $w=0$ tudi na robu komplementa, ki je enak
robu nosilca.

\begin{lema}\label{l:1}
	Naj bo $f\colon\overline{B}\times [t_0,t_1]\to\W$ zvezno polje. Če velja
	\begin{equation}\label{e:lem1}
		\int_{t_0}^{t_1}\int_B f\cdot w\,dV\,dt = 0
	\end{equation}
	za vsako $C^{\infty}$ polje $w\colon\E_R\times\R\to\W$ s kompaktnim nosilcem
	\[ \mathrm{supp}\, w\subseteq \overline{B}\times[t_0,t_1], \]
	potem je $f=0$ na $\overline{B}\times [t_0,t_1]$.
\end{lema}

\proof
	Glede na ortonormirano bazo $\{b_j\}$ prostora $\W$ lahko $f$ zapišemo v komponentni obliki kot $f=f_jb_j$.
	Recimo, da obstaja točka $(X_c,t_c)$ v notranjosti območja $B\times(t_0,t_1)$, da je $f_{k}(X_c,t_c)\neq 0$
	za neki $k$. Potem zaradi zveznosti polja $f$ za neki dovolj majhen
	$\alpha > 0$ obstaja okolica $U_{\alpha}$ točke $(X_c,t_c)$, definirana kot v (\ref{e:mnU}), kjer je $f_k$
	vseskozi različen od 0 in enakega predznaka, ter
	je $U_{\alpha}$ še v celoti vsebovana v notranjosti odprtega območja $B\times(t_0,t_1)$.
	
	Naj bo polje
	$w$ definirano kot polje $u_{\alpha}b_k$, če je $f_{k}(X_c,t_c) > 0$, oz. kot $-u_{\alpha}b_k$,
	če je $f_{k}(X_c,t_c) < 0$, kjer je $u_{\alpha}$ definirano kot v (\ref{e:funu}). Tak $w$ ima
	kompaktni nosilec $\overline{U_{\alpha}}$, vsebovan v območju $\overline{B}\times[t_0,t_1]$.
	Skalarni produkt $f\cdot w$ je pozitiven na množici $U_{\alpha}$ in je 0 drugje, zato je
	\[ \int_{t_0}^{t_1}\int_B f\cdot w\,dV\,dt > 0, \]
	kar nasprotuje enačbi (\ref{e:lem1}). Torej mora veljati $f=0$ povsod na na $B\times (t_0,t_1)$,
	zaradi zveznosti pa tudi na zaprtju $\overline{B}\times [t_0,t_1]$.
\endproof

\begin{lema}\label{l:2}
	Naj rob $\partial B$ odprtega območja $B\subset\E_R$ sestoji iz komplementarnih regularnih ploskev
	$\partial_g B$ in $\partial_d B$. Naj bo polje $f\colon \partial_d B\times[t_0,t_1]\to \W$ odsekoma
	regularno in zvezno v času. Če velja
	\begin{equation}\label{e:lem2}
		\int_{t_0}^{t_1}\int_{\partial_d B} f\cdot w\,dS\,dt = 0
	\end{equation}
	za vsako $C^{\infty}$ polje $w\colon\E_R\times\R\to\W$ s kompaktnim nosilcem,
	ki je vsebovan v $\E_R\times[t_0,t_1]$ in ima prazen presek z množico $\partial_g B\times [t_0,t_1]$,
	potem je $f=0$ na $\partial_d B\times [t_0,t_1]$.
\end{lema}

\proof
	Naj bo $X_c\in\partial_d B$ regularna točka, ki ni na robu ploskve $\partial_d B$ in
	naj bo $t_c\in (t_0,t_1)$. Predpostavimo, da je $f_{k}(X_c,t_c)\neq 0$ za neki $k$. Potem
	za neki $\alpha>0$ obstaja okolica $U_{\alpha}$ točke $(X_c,t_c)$, definirana kot v (\ref{e:mnU}),
	da je $f$ na množici $U_{\alpha}\cap\partial_d B\times[t_0,t_1]$ vseskozi različno od 0 in enakega
	predznaka. Prav tako je za dovolj majhen $\alpha>0$ presek $\overline{U_{\alpha}}\cap\partial_g B\times[t_0,t_1]$
	prazen in $\overline{U_{\alpha}}$ je vsebovan v $\E_R\times[t_0,t_1]$.
	
	Definirajmo polje $w$ kot $u_{\alpha}b_k$, če je $f_{k}(X_c,t_c) > 0$, oz. kot $-u_{\alpha}b_k$,
	če je $f_{k}(X_c,t_c) < 0$, kjer je $u_{\alpha}$ definirano kot v (\ref{e:funu}). Tak $w$ ustreza pogojem iz leme.
	Skalarni produkt $f\cdot w$ je pozitiven na množici $U_{\alpha}\cap\partial_d B\times[t_0,t_1]$
	in je 0 drugje na $\partial_d B\times [t_0,t_1]$, zato je
	\[ \int_{t_0}^{t_1}\int_{\partial_d B} f\cdot w\,dS\,dt > 0, \]
	kar nasprotuje enačbi (\ref{e:lem2}). Torej mora veljati $f=0$ za vse regularne točke območja
	$\partial_d B\times [t_0,t_1]$.
\endproof

\begin{lema}
	Naj bo $\chi\colon\overline{B}\times[t_0,t_1]\to\E$ $C^2$ gibanje s pozitivnim jacobijanom $J$ in
	$f\colon\overline{B}\times[t_0,t_1]\to\W$ zvezno tenzorsko polje. Če velja
	\begin{equation}\label{e:lem3}
		\int_{t_0}^{t_1}\int_{B_t}\bar{f}\cdot \bar{w} \,dv\,dt=0
	\end{equation}
	za vsako $C^{\infty}$ polje $w\colon\E_R\times\R\to\W$ s kompaktnim nosilcem
	\[ \mathrm{supp}\, w\subseteq \overline{B}\times[t_0,t_1], \]
	potem je $\bar{f}=0$ na $\chi(\overline{B},[t_0,t_1])$.
\end{lema}

\proof
	Enačba (\ref{e:lem3}) je ekvivalentna enačbi
	\[ \int_{t_0}^{t_1}\int_{B}Jf\cdot w \,dV\,dt=0. \]
	Ker je $Jf$ zvezna na $\overline{B}\times[t_0,t_1]$, je po lemi (\ref{l:1}) $Jf=0$
	na $\overline{B}\times[t_0,t_1]$. Ker je $J>0$, je $f=0$ na $\overline{B}\times[t_0,t_1]$,
	torej je tudi $\bar{f}=0$ na $\chi(\overline{B},[t_0,t_1])$.
\endproof

\begin{lema}
	Naj rob $\partial B$ odprtega območja $B\subset\E_R$ sestoji iz komplementarnih regularnih ploskev
	$\partial_g B$ in $\partial_d B$.
	Naj bo $\chi\colon\overline{B}\times[t_0,t_1]\to\E$ $C^2$ gibanje s pozitivnim jacobijanom $J$ in
	$f\colon\partial_d B\times[t_0,t_1]\to\R$ odsekoma regularno in v času zvezno skalarno polje.
	Če velja
	\begin{equation}\label{e:lem4}
		\int_{t_0}^{t_1}\int_{\chi_t(\partial_d B)}\bar{f}\vek{n}\cdot\bar{\vek{w}}\,ds\,dt=0
	\end{equation}
	za vsako $C^{\infty}$ vektorsko polje $\vek{w}\colon\E_R\times\R\to\V$ s kompaktnim nosilcem,
	ki je vsebovan v $\E_R\times[t_0,t_1]$ in ima prazen presek z množico $\partial_g B\times [t_0,t_1]$,
	potem je $\bar{f}=0$ na $\chi(\partial_d B, [t_0,t_1])$.
\end{lema}

\proof
	Enačba (\ref{e:lem4}) je ekvivalentna enačbi
	\[ \int_{t_0}^{t_1}\int_{\partial_d B}fJ\ten{F}^{-T}\vek{N}\cdot\vek{w}\,dS\,dt=0. \]
	Polje $fJ\ten{F}^{-T}\vek{N}$ je odsekoma regularno in zvezno v času na $\partial_d B\times[t_0,t_1]$,
	zato je po lemi (\ref{l:2}) $fJ\ten{F}^{-T}\vek{N}=\vek{0}$ na $\partial_d B\times[t_0,t_1]$,
	iz REF pa nato sledi $\bar{f}=0$ na $\chi(\partial_d B, [t_0,t_1])$.
\endproof


\section{Primera}


\subsection{Elastični fluid}


Idealni fluid je model tekočine, v katerem zanemarimo učinke viskoznosti. Če je tekočina
stisljiva, potem je njena notranja energija
\[ U=\int_{B}\rho_R e(\rho)\,dV=\int_{B_t}\rho e(\rho)\,dv, \]
kjer je $e(\rho)$ skalarna funkcija, ki je od $(X,t)\in B\times[t_0,t_1]$ oz. od
$(x,t)\in\chi(B,[t_0,t_1])$ odvisna le preko funkcije masne gostote $\rho$. $e(\rho)$
se imenuje \emph{gostota notranje energije} in ima fizikalno enoto $m^2s^{-2}$.
Predpostavili bomo, da ima $e\colon R_+\to\R$ znano funkcijsko obliko in da je
vsaj dvakrat zvezno odvedljiva.

Variacija notranje energije je
\begin{align*} 
	\delta U &=\at{\frac{dU^*}{d\varepsilon}}{\varepsilon=0}=\at{\frac{d}{d\varepsilon}
	\int_{B}\rho_R e(\rho^*)\,dV }{\varepsilon=0} =
	\at{\int_{B}\rho_R e'(\rho^*)\frac{d\rho^*}{d\varepsilon}\,dV }{\varepsilon=0} = \\
	&= -\int_{B}\rho_R e'(\rho)\rho\div\vek{\eta}\,dV.
\end{align*}
Z upoštevanjem relacij REF, REF, nato pa še REF, dobimo
\begin{equation}\label{e:varUbasic}
	\delta U = -\int_{B}\rho^2e'(\rho)\Div\vek{\eta}\,dV = -\int_{B_t}\rho^2 e'(\rho)\div\vek{\eta}\,dv
\end{equation}
Če sedaj na vsakem od teh izrazov posebej uporabimo relacijo REF in nato divergenčni izrek REF,
lahko (\ref{e:varUbasic}) prepišemo v obliki
\begin{align*}
	\delta U &= \int_{B}\Grad\big(\rho^2 e'(\rho)\big)\cdot\vek{\eta}\,dV
	-\int_{\partial B}\rho^2 e'(\rho)\vek{N}\cdot\vek{\eta}\,dS \\
	&= \int_{B_t}\grad\big(\rho^2 e'(\rho)\big)\cdot\vek{\eta}\,dv
	-\int_{\partial B_t}\rho^2 e'(\rho)\vek{n}\cdot\vek{\eta}\,ds
\end{align*}

Predpostavimo, da imamo podan \emph{zunanji tlak}, t.j. skalarno polje $p\colon\partial B\times[t_0,t_1]$,
da je $\vek{t}=-p\vek{n}$. Potem je po (\ref{e:ttr}) in REF
\[ \vek{t}_R=-pJ\ten{F}^{-T}\vek{N}. \]
Geometrijskih robnih pogojev ni.

\begin{align*}
	\int_{t_0}^{t_1}\bigg( &\int_{B}\Big(-\rho\vek{a}-\grad\big(\rho^2 e'(\rho)\big)+\rho\vek{b}\Big)\cdot\vek{\eta}\,dv \\
	&+\int_{\partial B}\big(\rho^2 e'(\rho)-p\big)\vek{n}\cdot\vek{\eta}\,ds\bigg)\,dt=0
\end{align*}

\begin{align*}
	\int_{t_0}^{t_1}\bigg( &\int_{B}\Big(-\rho_R\vek{a}-\Grad\big(\rho^2 e'(\rho)\big)+\rho_R\vek{b}\Big)\cdot\vek{\eta}\,dV \\
	&+\int_{\partial B}\big(\rho^2 e'(\rho)\ten{I}-pJ\ten{F}^{-T}\big)\vek{N}\cdot\vek{\eta}\,dS\bigg)\,dt=0
\end{align*}