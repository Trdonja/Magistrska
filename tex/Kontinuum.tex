\chapter{Kinematika kontinuuma} \label{chp:kinkon}


\section{Konfiguracije in gibanje telesa}


Naj bosta $\E_R$ in $\E$ Evklidska točkovna prostora, ki imata sicer enake lastnosti,
vendar ju bomo kljub temu razlikovali. Prostor $\E_R$ bo služil določitvi materialnega
telesa: z množico točk izbranega regularnega območja $\B\subset\E_R$ je določeno \emph{materialno
telo} ali \emph{kontinuum}. Množici $\B$ bomo rekli \emph{referenčna} ali \emph{sklicna konfiguracija}
materialnega telesa. Privzeli bomo, da je $\B$ zaprta.
Prostor $\E$ pa bo služil opisu dejanskega položaja materialnega telesa v prostoru.

Točke prostora $\E_R$, njihove koordinate ter njihove krajevne vektorje bomo označevali
z velikimi simboli: $X$, $(X_1,X_2,X_3)$, $\vek{X}$, in jih bomo imenovali
\emph{materialne točke oz.~koordinate oz.~vektorji}.
Točke prostora $\E$, njihove koordinate ter njihove krajevne vektorje bomo,
kot doslej, označevali z malimi simboli: $x$, $(x_1,x_2,x_3)$, $\vek{x}$, in jih bomo imenovali
\emph{prostorske točke oz.~koordinate oz.~vektorji}.

\begin{definicija}
	\emph{Konfiguracija (materialnega) telesa} je zvezna in injektivna preslikava
	\[ \kappa\colon \B\to\E,\qquad \kappa\colon X\mapsto x=\kappa(X). \]
	Tudi sliki preslikave $\kappa$, t.~j.~$\kappa(\B)$, rečemo \emph{konfiguracija telesa}.
\end{definicija}
S konfiguracijo torej določamo položaj telesa v prostoru $\E$. Ker je $\B$ kompaktna
množica, je tudi inverz $\kappa^{-1}\colon\kappa(\B)\to\B$ konfiguracije $\kappa$ zvezna preslikava.

Naj bo $I=[t_1,t_2]\subset\R$ časovni interval.
\begin{definicija}
	\emph{Gibanje (materialnega) telesa} je zvezna preslikava
	\begin{equation}\label{e:chi}
		\chi\colon\B\times I\to\E,\qquad \chi\colon(X,t)\mapsto x
	\end{equation}
	z lastnostjo, da je za vsak $t\in I$ preslikava
	\[ \chi_t\colon\B\to\E,\qquad \chi_t(x):=\chi(X,t) \]
	konfiguracija. Preslikava $\chi_t$ in množica $\B_t:=\chi_t(\B)$ se imenujeta
	\emph{trenutna konfiguracija} telesa ob času $t$.
\end{definicija}

V skladu z dogovorom \ref{d:dogovor} bomo s krepkimi simboli $\vek{\chi}$, $\vek{\chi}_t$, \dots
označevali vektorska polja $\vek{\chi}=\iota_o\circ\chi$, $\vek{\chi}_t=\iota_o\circ\chi$, \dots,
kjer je $\iota_o$ opazovališče za $\E$.

Na gibanje lahko gledamo tudi kot na enoparametrično družino konfiguracij
$\{ \chi_t\,;\ t\in I \}$, kjer je preslikava $t\mapsto\chi_t$ (enakomerno) zvezna,
t.j.~za vsak $\varepsilon>0$ obstaja $\delta>0$, tako da za vsaka $t_1,t_2\in I$,
za katera je $|t_2-t_1|<\delta$, velja $\sup_{X\in\B}\|\vek{\chi}_{t_2}(X)-\vek{\chi}_{t_1}(X)\|<\varepsilon$.
Pri tem je $\|\cdot\|$ norma prostora $\V$.

Z oznako $\Omega$ bomo označili množico
\[ \Omega=\{(x,t)\;;\ x\in\chi_t(\B),\ t\in I\}\subset\E\times I. \]
Gibanje (\ref{e:chi}) ni injektivna preslikava\footnote{Zaradi zveznosti preslikave $\chi$
obstajata različna časa $t,t'\in I$, ki sta dovolj blizu skupaj, tako da imata množici $\chi(\B,t)$
in $\chi(\B,t')$ neprazen presek, torej obstaja točka $x$ iz tega preseka in točki $X_1,X_2\in\B$,
da je $\chi(X_1,t)=x=\chi(X_2,t')$, vendar pa $(X_1,t)\neq(X_2,t')$, ker je vsaj $t\neq t'$.},
zato nima inverza. Kljub temu na smiselen način definiramo \emph{inverzno gibanje}
\[
	\chi^{-1}\colon\Omega\to \B,\qquad
	\chi^{-1}\colon(x,t)\mapsto X:=\bottop{\chi}{t}{-1}(x).
\]
S $\chi^{-1}$ smo torej označili inverz preslikave $(X,t)\mapsto(x,t)=(\chi(X,t),t)$,
ki pa je injektivna.
Dodatno bomo do nadaljnjega predpostavili, da sta gibanje in iverzno gibanje razreda $C^2$.

Tenzorsko polje
\[
	\ten{F}\colon \B\times I\to\L(\V),\qquad \ten{F}(X,t)=\Grad\vek{\chi}(X,t)=\Grad\vek{\chi}_t(X)
\]
se imenuje \emph{deformacijski gradient} gibanja. Po izreku o inverzni funkciji velja
\[
	\ten{F}^{-1}(x,t)=\grad\vek{\chi}_t^{-1}(x)=\grad\vek{\chi}^{-1}(x,t).
\]

Absolutni vrednosti determinante deformacijskega gradienta rečemo \emph{jacobijan} in jo označimo z
\[
	J=|\det\ten{F}|.
\]
Ker so konfiguracije $\chi_t$
injektivne oz.~bijektivne na svojo sliko, mora biti $\det\ten{F}\neq 0$ povsod na $\B\times I$.
Zaradi zveznosti tenzorskega polja $\ten{F}$ mora biti potem skalarno polje $\det\ten{F}$,
ki je potem takem tudi zvezno, enakega predznaka povsod na $\B\times I$.
Običajno, ni pa nujno, je začetna konfiguracija ob začetnem času $t_1$ časovnega intervala $I$
taka, da za vse $X\in\B$ velja $\hat{\iota}_{O}(X)=\iota_o(\chi(X,t_1))$. Povedano drugače,
krajevni vektorji točk iz referenčne konfiguracije (glede na opazovališče $\hat{\iota}_{O}$ za $\E_R$)
so enaki krajevnim vektorjem pripadajočih točk v začetni konfiguraciji (glede na opazovališče $\iota_o$ za $\E$).
V tem primeru je $\ten{F}(\cdot,t_1)=\ten{1}$ in $\det\ten{F}(\cdot,t_1)=1$, zato je po prejšnjem
premisleku $\det\ten{F}>0$ na celotnem območju $\B\times I$ in absolutno vrednost v izrazu za $J$
lahko izpustimo. V tem in še naslednjem poglavju bomo privzeli, da je $J>0$.

\emph{Hitrost} $\vek{v}$ in \emph{pospešek} $\vek{a}$ gibanja $\chi$ sta vektorski polji
\begin{align}
	\vek{v}\colon \B\times I\to \V \qquad & \vek{v}(X,t) = \frac{\partial\vek{\chi}}{\partial t}(X,t), \label{e:v} \\
	\vek{a}\colon \B\times I\to \V \qquad & \vek{a}(X,t) = \frac{\partial^2\vek{\chi}}{\partial t^2}(X,t). \label{e:a}
\end{align}
Alternativna oznaka za $\vek{v}$ je tudi $\dot{\vek{x}}$, za $\vek{a}$ pa $\ddot{\vek{x}}$.


\section{Materialni in prostorski opis}


Vsakemu tenzorskemu polju $\vek{f}\colon \B\times I\to\W$ pripada glede na gibanje
(\ref{e:chi}) enakovreden predpis
\[
	\bar{\vek{f}}\colon\Omega\to\W,\qquad
	\bar{\vek{f}}(x,t):=\vek{f}(\chi^{-1}(x,t),t)=\vek{f}(X,t)
\]
Tenzorsko polje $\bar{\vek{f}}$ se imenuje \emph{prostorski opis} tenzorskega polja $\vek{f}$.

Na enak način pripada vsakemu tenzorskemu polju $\vek{f}\colon\Omega\to\W$
glede na gibanje (\ref{e:chi}) enakovreden predpis
\begin{equation} \label{e:matopi}
	\hat{\vek{f}}\colon \B\times I\to\W,\qquad
	\hat{\vek{f}}(X,t):=\vek{f}(\chi(X,t),t)=\vek{f}(x,t).
\end{equation}
$\hat{\vek{f}}$ imenujemo \emph{materialni opis} tenzorskega polja $\vek{f}$.

Pogosto bomo strešico ali črtico v oznakah za materialni ali prostorski opis opustili,
če to ne bo pustilo dvomov o tem, na kateri domeni je definirano polje.
Pri integraciji bo že iz integracijske domene razvidno, za kateri opis gre.
Če poleg polja pišemo še argumente, potem mali $x$ v argumentu nakazuje na prostorski,
veliki $X$ pa na materialni opis.
Pri gradientu in divergenci se dvomom izognemo z uporabo različnih notacij
za ta dva diferencialna operatorja.
V materialnem opisu pišemo oznaki za gradient in divergenco z veliko začetnico,
\[ \Grad\vek{f}:=\nabla\hat{\vek{f}},\qquad \Div\vek{f}:=\div\hat{\vek{f}}, \]
v prostorskem opisu pa z malo,
\[ \grad\vek{f}:=\nabla\bar{\vek{f}},\qquad \div\vek{f}:=\div\bar{\vek{f}}. \]

Če je $\phi$ skalarno, $\vek{u}$ pa vektorsko polje, potem je zveza med gradientoma
\begin{equation}\label{e:gz}
	\Grad\phi=\ten{F}^{T}\grad\phi,\qquad \Grad\vek{u}=(\grad\vek{u})\ten{F}.
\end{equation}
Res, če je $\vek{w}$ poljubno vektorsko polje, dobimo iz (\ref{e:matopi})
z uporabo verižnega pravila
\begin{gather*}
	\langle\Grad\phi,\vek{w}\rangle=\langle\grad\phi,(\Grad\chi)\vek{w}\rangle=
	\langle\grad\phi, \ten{F}\vek{w}\rangle=\langle\ten{F}^{T}\grad\phi,\vek{w}\rangle, \\
	(\Grad\vek{u})\vek{w}=(\grad\vek{u})(\Grad\chi)\vek{w}
	=(\grad\vek{u})\ten{F}\vek{w}.
\end{gather*}

\begin{definicija}
	Časovni odvod tenzorskega polja $\vek{f}\colon \B\times I\to\W$ označimo z
	$\dot{\vek{f}}$ ali $d\vek{f}/dt$ in ga imenujemo \emph{materialni časovni odvod};
	\[ \dot{\vek{f}}(X,t)=\frac{d\vek{f}}{dt}(X,t)=\frac{\partial\vek{f}}{\partial t}(X,t). \]
\end{definicija}
Iz (\ref{e:matopi}) dobimo z uporabo verižnega pravila še materialni časovni odvod za prostorski opis:
\begin{equation} \label{e:matodv}
	\dot{\vek{f}}=\frac{d\vek{f}}{dt}=
	\frac{\partial\vek{f}}{\partial t}+(\grad\vek{f})(\vek{v}),
\end{equation}
kjer je $\vek{v}$ hitrost gibanja (\ref{e:v}), $\partial\vek{f}/\partial t$ pa
časovni dvod polja $\vek{f}(x,t)$ (odvod preslikave $\vek{f}(\cdot,t)\colon I\to\E$).
Kadar poleg tenzorskega polja ne bomo pisali argumentov,
bo oznaka $\partial\vek{f}/\partial t$ vedno pomenila časovni odvod prostorskega opisa.

\begin{primer} %\label{e:L}
	Hitrost in pospešek sta prvi in drugi materialni časovni odvod gibanja $\vek{x}(X,t)=\vek{\chi}(X,t)$,
	$\vek{v}=\dot{\vek{x}}$, $\vek{a}=\ddot{\vek{x}}$.
	Pospešek je materialni časovni odvod hitrosti in se v prostorskem opisu izraža kot
	\[ \vek{a}=\dot{\vek{v}}=\frac{\partial\vek{v}}{\partial t}+(\grad\vek{v})\vek{v}. \]
	Tenzorsko polje $\ten{L}=\grad\vek{v}$ se imenuje \emph{hitrostni gradient}.
	Če je gibanje $\chi$ razreda $C^2$, potem je
	\[
		\frac{d}{dt}\Grad\vek{\chi}=\Grad\frac{d\vek{\chi}}{dt}\qquad\textrm{oziroma}\qquad
		\dot{\ten{F}}=\Grad\vek{v},
	\]
	iz česar z uporabo zveze (\ref{e:gz}) dobimo
	\begin{equation} \label{e:L}
		\ten{L}:=\grad\vek{v}=\dot{\ten{F}}\ten{F}^{-1}.
	\end{equation}
\end{primer}


\section{Površinski in prostorninski element}


Naj bo $\varepsilon$ pozitivno realno število in naj bo $C\colon(-\varepsilon,\varepsilon)\to\B$
preslikava razreda $C^1$. Slika preslikave $C$ je krivulja znotraj ali pa na robu
materialnega telesa $\B$. \emph{Dolžinski element} v točki $C(0)=X\in\B$ glede na parametrizacijo $C$
je infinitezimalni tangentni vektor
\[
	d\vek{X}=\frac{d}{d\alpha}\at{\vek{C}(\alpha)}{\alpha=0}d\alpha=\vek{C}'(0)\, d\alpha.
\]
Glede na trenutno konfiguracijo $\chi_t\colon\B\to\B_t$ ob času $t$ pripada preslikavi $C$ preslikava
\[
	c\colon(-\varepsilon,\varepsilon)\to\B_t,\qquad c(\alpha)=\chi_t(C(\alpha)).
\]
Dolžinski element v točki $c(0)=x\in\B_t$ glede na parametrizacijo $c$ je infinitezimalni
tangentni vektor
\begin{align*}
	d\vek{x}&=\frac{d}{d\alpha}\at{\vek{c}(\alpha)}{\alpha=0}d\alpha
	=\frac{d}{d\alpha}\at{\vek{\chi_t}(C(\alpha))}{\alpha=0}d\alpha
	=\ten{F}(X,t)\vek{C}'(0)\,d\alpha \\
	&=\ten{F}(X,t)\,d\vek{X}.
\end{align*}

Naj bosta $d\vek{X}_1$ in $d\vek{X}_2$ dolžinska elementa v isti točki $X\in\B$, vendar
glede na različni parametrizaciji $C_1(\alpha_1)$ in $C_2(\alpha_2)$. Pripadajoča dolžinska elementa
v točki $x\in\B_t$ iz trenutne konfiguracije naj bosta $d\vek{x}_1=\ten{F}\,d\vek{X}_1$
in $d\vek{x}_2=\ten{F}\,d\vek{X}_2$. Infinitezimalna vektorja
\[
	d\vek{S}=d\vek{X}_1\times d\vek{X}_2\quad\textrm{in}\quad d\vek{s}=d\vek{x}_1\times d\vek{x}_2
\]
se imenujeta \emph{(materialni) površinski element} v referenčni oz.~trenutni konfiguraciji.
Za poljuben vektor $\vek{w}\in\V$ velja
\begin{align*}
	\langle\vek{w},d\vek{s}\rangle &=
	\langle\vek{w},d\vek{x}_1\times d\vek{x}_2\rangle
	=\langle\vek{w},\ten{F}\,d\vek{X}_1\times\ten{F}\,d\vek{X}_2\rangle \\
	&= \langle\ten{F}(\ten{F}^{-1}\vek{w}),\ten{F}\,d\vek{X}_1\times\ten{F}\,d\vek{X}_2\rangle
	=(\det\ten{F})\langle\ten{F}^{-1}\vek{w},d\vek{X}_1\times d\vek{X}_2\rangle \\
	&=\langle\vek{w}, J\ten{F}^{-T}(d\vek{X}_1\times d\vek{X}_2)\rangle,
\end{align*}
od koder sledi enakost
\begin{equation} \label{e:kaligula}
	d\vek{s}=J\ten{F}^{-T}\,d\vek{S}.
\end{equation}
Pri tem smo upoštevali, da je mešani produkt $\langle\cdot,\cdot\times\cdot\rangle$
alternirajoča 3-linearna forma. Enakost (\ref{e:kaligula}) lahko zapišemo tudi kot
\begin{equation} \label{e:nsns}
	\vek{n}\,da=J\ten{F}^{-T}\vek{N}\,dA,
\end{equation}
pri čemer sta
\[
	\vek{N}=\frac{\vek{C}_1'(0)\times\vek{C}_2'(0)}{\|\vek{C}_1'(0)\times\vek{C}_2'(0)\|}
	\quad\textrm{in}\quad
	\vek{n}=\frac{\vek{c}_1'(0)\times\vek{c}_2'(0)}{\|\vek{c}_1'(0)\times\vek{c}_2'(0)\|}
\]
enotski normali in
\[
	dA=\|\vek{C}_1'(0)\times\vek{C}_2'(0)\|\,d\alpha_1\,d\alpha_2\quad\textrm{ter}
	\quad da=\|\vek{c}_1'(0)\times\vek{c}_2'(0)\|\,d\alpha_1\,d\alpha_2
\]
\emph{ploščinska elementa}.

Če je $r=r(u,v)$ regularna parametrizacija neke ploskve v $\B_t$, ki je lahko tudi
del robu $\partial B$, potem lahko dolžinska elementa $d\vek{x}_1\times d\vek{x}_2$,
ki nastopata v površinskem elementu $d\vek{s}=d\vek{x}_1\times d\vek{x}_2$, razumemo kot
infinitezimalna tangentna vektorja
\[
	d\vek{x}_1=\frac{\partial \vek{r}}{\partial u}\,du,\qquad
	d\vek{x}_2=\frac{\partial \vek{r}}{\partial v}\,dv.
\]
Podobno velja za parametrizacijo pripadajoče ploskve v $\B$.
Ker lahko integrale po ploskvah izrazimo s pomočjo parametrizacije ploskve, dobimo
s pomočjo enačbe (\ref{e:nsns}) naslednji izrek.

\begin{izrek} \label{i:suittra}
	Naj bo $\mathcal{S}$ ploskev v $\B$ in $\mathcal{S}_t=\chi_t(\mathcal{S})$
	pripadajoča ploskev v trenutni konfiguraciji $\B_t$.
	Če je $\vek{f}\colon \mathcal{S}\to\W$ integrabilno tenzorsko polje, potem velja
	\begin{align*}
		\int_{\mathcal{S}_t}\vek{f}[\vek{n}]\,da=\int_{\mathcal{S}}\vek{f}[J\ten{F}^{-T}\vek{N}]\,dA ,\\
		\int_{\mathcal{S}}\vek{f}[\vek{N}]\,dA=\int_{\mathcal{S}_t}\vek{f}[J^{-1}\ten{F}^{T}\vek{n}]\,da,
	\end{align*}
	kjer izraz $\vek{f}[\,\cdot\,]$ nadomestimo z ustreznim produktom
	polja $\vek{f}$ in vsebine $[\,\cdot\,]$.
\end{izrek}

Naj bodo sedaj, podobno kot prej, za $j=1,2,3$ $d\vek{X}_j$
dolžinski elementi v isti točki $X\in\B$, vendar
glede na različne parametrizacije $C_j(\alpha_j)$.
Pripadajoči dolžinski elementi v točki $x\in\B_t$ iz trenutne konfiguracije naj bodo $d\vek{x}_j=\ten{F}\,d\vek{X}_j$.
Infinitezimalna skalarja
\[
	dV=|\langle d\vek{X}_1,d\vek{X}_2\times d\vek{X}_3\rangle|\quad\textrm{in}\quad
	dv=|\langle d\vek{x}_1,d\vek{x}_2\times d\vek{x}_3\rangle|
\]
se imenujeta \emph{(materialni) prostorninski element} v referenčni oz.~trenutni konfiguraciji.
Imamo
\begin{equation} \label{e:projojo}
	dv=|\langle \ten{F}\,d\vek{X}_1,\ten{F}\,d\vek{X}_2\times \ten{F}\,d\vek{X}_3\rangle|
	=|\det\ten{F}||\langle d\vek{x}_1,d\vek{x}_2\times d\vek{x}_3\rangle|
	=J\,dV.
\end{equation}

Tudi integrale po prostornini lahko izrazimo s pomočjo parametrizacije (npr.~s
pomočjo koordinatnega sistema) in s pomočjo enakosti (\ref{e:projojo}) dobimo
naslednji izrek.

\begin{izrek} \label{i:prointrel}
	Naj bo $\mathcal{P}\subseteq\B$ regularno območje z neprazno notranjostjo,
	naj bo $\mathcal{P}_t=\chi_t(\mathcal{P})$ in
	naj bo $\vek{f}\colon \mathcal{P} \to\W$ integrabilno tenzorsko polje. Potem velja
	\begin{equation*}
		\int_{\mathcal{P}_t}\vek{f}\,dv=\int_{\mathcal{P}}\vek{f}J\,dV \qquad\textrm{in}\qquad
		\int_{\mathcal{P}}\vek{f}\,dV=\int_{\mathcal{P}_t}\vek{f}J^{-1}\,dv.
	\end{equation*}
\end{izrek}


\section{Transportni izrek}


V tem razdelku bomo podali transportni izrek, katerega vsebina je
enačba za časovni odvod integrala po območju trenutne konfiguracije materialnega telesa.
Še prej pa potrebujemo formulo za materialni časovni odvod determinante deformacijskega gradienta.

Naj bo $\W$ vektorski prostor nad obsegom $\R$ dimenzije $n\in\mathbb{N}$.
Najprej poiščimo odvod za determinanto $\det\colon\L(\W)\to\R$.
Naj bo $\omega\colon \W^n\to\R$ netrivialna alternirajoča $n$-linearna forma
in naj bo $\ten{A}\in\L(\W)$ linearna preslikava.
Spomnimo, determinanta in sled linearne preslikave $\ten{A}$ sta definirani kot\footnote{
Glej podrazdelek \ref{pdrten2}.}
\begin{gather*}
	\omega(\ten{A}\vek{u}_1,\dots,\ten{A}\vek{u}_n)=(\det \ten{A})\,\omega(\vek{u}_1,\dots,\vek{u}_n), \\
	\sum_{j=1}^n\omega(\vek{u}_1,\dots,\ten{A}\vek{u}_j,\dots,\vek{u}_n)=(\tr \ten{A})\,\omega(\vek{u}_1,\dots,\vek{u}_n)
\end{gather*}
in definicija je neodvisna od izbire netrivialne forme $\omega$.

Naj bo $\ten{S}\in\L(\W)$ še ena linearna preslikava in $\varepsilon>0$. Potem je
\begin{align*}
	\det&(\ten{A}+\varepsilon \ten{S})\,\omega(\vek{u}_1,\dots,\vek{u}_n)=
	\omega((\ten{A}+\varepsilon \ten{S})\vek{u}_1,\dots,(\ten{A}+\varepsilon \ten{S})\vek{u}_n)\\
	&=\omega(\ten{A}\vek{u}_1,\dots,\ten{A}\vek{u}_n)+
	\sum_{j=1}^n\varepsilon\,\omega(\ten{A}\vek{u}_1,\dots,\ten{S}\vek{u}_j,\dots,\ten{A}\vek{u}_n)+o(\varepsilon)\\
	&=(\det \ten{A})\,\omega(\vek{u}_1,\dots,\vek{u}_n)+\varepsilon
	\sum_{j=1}^n\omega(\ten{A}\vek{u}_1,\dots,\ten{A}\ten{A}^{-1}\ten{S}\vek{u}_j,\dots,\ten{A}\vek{u}_n)+o(\varepsilon)\\
	&=(\det \ten{A})\Big(\omega(\vek{u}_1,\dots,\vek{u}_n)+\varepsilon
	\sum_{j=1}^n\omega(\vek{u}_1,\dots,\ten{A}^{-1}\ten{S}\vek{u}_j,\dots,\vek{u}_n)\Big)+o(\varepsilon)\\
	&=(\det \ten{A})(1+\varepsilon\,\tr(\ten{A}^{-1}\ten{S}))\,\omega(\vek{u}_1,\dots,\vek{u}_n)+o(\varepsilon).
\end{align*}
Pri tem smo člene, kjer nastopajo potence števila $\varepsilon$, spravili
v izraz $o(\varepsilon)$. Od tu sedaj lahko izračunamo smerni odvod
\begin{multline*}
	\at{\frac{d}{d\varepsilon}\det(\ten{A}+\varepsilon \ten{S})}{\varepsilon=0}
	\,\omega(\vek{u}_1,\dots,\vek{u}_n)=
	\at{\frac{d}{d\varepsilon}\Big[\det(\ten{A}+\varepsilon \ten{S})
	\,\omega(\vek{u}_1,\dots,\vek{u}_n)\Big]}{\varepsilon=0}\\
	=(\det \ten{A})\tr(\ten{A}^{-1}\ten{S})\,\omega(\vek{u}_1,\dots,\vek{u}_n)=
	\big<(\det \ten{A})\ten{A}^{-T}, \ten{S}\big>\,\omega(\vek{u}_1,\dots,\vek{u}_n).
\end{multline*}
Pri tem smo na zdanjem koraku uporabili definicijo skalarnega produkta na prostoru $\L(\W)$.
Determinanta je odvedljiva preslikava, zato je krepki odvod enak pravkar izračunanemu
smernemu odvodu, torej imamo
\begin{equation} \label{e:odvodet}
	D\det(\ten{A})(\ten{S})=\langle(\det \ten{A})\ten{A}^{-T},\ten{S}\rangle
	\quad\textrm{oz.}\quad D\det(\ten{A})=(\det \ten{A})\ten{A}^{-T}
\end{equation}

Sedaj imamo pripravljeno vse, da izračunamo materialni časovni odvod jacobijana:
\begin{align}
	\dot{J}&=(\det\ten{F})\,\dot{}=D\det(\ten{F})(\dot{\ten{F}})=\langle J\ten{F}^{-T},\dot{\ten{F}}\rangle
	\nonumber \\ &= J\tr(\dot{\ten{F}}\ten{F}^{-1}) = J\tr(\grad\vek{v})=J\div\vek{v}. \label{e:dotJ}
\end{align}
Pri tem smo uporabili verižno pravilo, malo prej izpeljano pravilo za odvod determinante
(kjer vzamemo $\W=\V$), komutativnost skalarnega produkta ter zvezo (\ref{e:L}).

\begin{izrek}[Transportni izrek] \label{i:transport}
	Naj bo $\mathcal{P}\subseteq \B$ regularno območje znotraj referenčne konfiguracije in naj
	$\mathcal{P}_t=\chi(\mathcal{P},t)\subseteq \B_t$ označuje njegovo trenutno konfiguracijo ob času $t$.
	Naj bo $\vek{f}\colon \mathcal{P}\times I\to W$ razreda $C^1$ na zaprtju $\overline{\mathcal{P}}\times I$.
	Potem velja
	\begin{equation*}
		\frac{d}{dt}\int_{\mathcal{P}_t}\vek{f}\,dv =
		\int_{\mathcal{P}_t}(\dot{\vek{f}}+\vek{f}\div\vek{v})\,dv=
		\int_{\mathcal{P}_t}\frac{\partial\vek{f}}{\partial t}\,dv +
		\int_{\partial \mathcal{P}_t}\vek{f}\langle\vek{v},\vek{n}\rangle\,da.
	\end{equation*}
\end{izrek}
Ne pozabimo, da pri tem $\partial\vek{f}/\partial t$ pomeni časovni (parcialni)
odvod prostorskega opisa polja $\vek{f}$.

\proof
	Z uporabo izreka \ref{i:prointrel} in enačbe (\ref{e:dotJ}) pridemo do
	\begin{align*}
		\frac{d}{dt}\int_{\mathcal{P}_t}\vek{f}\,dv &= \frac{d}{dt}\int_{\mathcal{P}}\vek{f}J\,dV =
		\int_{\mathcal{P}}\frac{d}{dt}(\vek{f}J)\,dV = \int_{\mathcal{P}}(\dot{\vek{f}}J+\vek{f}\dot{J})\,dV =\\
		&=\int_{\mathcal{P}}(\dot{\vek{f}}+\vek{f}\div\vek{v})J\,dV = \int_{\mathcal{P}_t}(\dot{\vek{f}}+\vek{f}\div\vek{v})\,dv,
	\end{align*}
	s čimer je dokazan prvi del enačbe iz izreka.
	Pri tem smo na drugem koraku smeli zamenjati vrstni red odvajanja in integriranja,
	ker se referenčna konfiguracija s časom ne spreminja.
	
	Preostalo enakost bomo dokazali za primer, ko je $\vek{f}$ skalarno ali vektorsko
	polje, za katero lahko uporabimo tretjo oz.~četrto enačbo iz trditve \ref{t:divprop}.
	Izrek sicer velja tudi za splošna tenzorska polja, le dokaz je bolj zapleten.
	
	Z uporabo enačbe (\ref{e:matodv}), trditve \ref{t:divprop}$_{3,4}$
	in divergenčnega izreka \ref{i:divtheo} pridemo do
	\begin{multline*}
		\int_{\mathcal{P}_t}(\dot{\vek{f}}+\vek{f}\div\vek{v})\,dv
		=\int_{\mathcal{P}_t}\Big(\frac{\partial\vek{f}}{\partial t}+
		(\grad\vek{f})(\vek{v})+\vek{f}\div\vek{v}\Big)\,dv \\
		=\int_{\mathcal{P}_t}\Big(\frac{\partial\vek{f}}{\partial t}+\div(\vek{f}[\vek{v}])\Big)\,dv
		=\int_{\mathcal{P}_t}\frac{\partial\vek{f}}{\partial t}\,dv+
		\int_{\partial \mathcal{P}_t}\vek{f}\langle\vek{v},\vek{n}\rangle\,da.
	\end{multline*}
	Pri tem oznaka $\vek{f}[\vek{v}]$ pomeni $f\vek{v}$, če je $\vek{f}=f$ skalarno polje,
	oz.~$\vek{f}\otimes\vek{v}$, če je $\vek{f}$ vektorsko polje.
\endproof


\section{Zakon o ohranitvi mase}


\emph{Masa} poljubne podmnožice $\mathcal{P}\subseteq \B$ je definirana kot vrednost integrala
\[ M(\mathcal{P})=\int_{\mathcal{P}}\rho_R\,dV, \]
kjer je $\rho_R\colon \B\to(0,\infty)$ integrabilno skalarno polje, imenovano
\emph{masna gostota referenčne konfiguracije}.
Za $\rho_R$ se običajno predpostavi še dodatne pogoje o zveznosti ali odvedljivosti.

\begin{comment}
Poljubni konfiguraciji materialnega telesa $\kappa\colon B\to\E$ pripada
integrabilno skalarno polje $\rho_{\kappa}\colon B\to(0,\infty)$, imenovano
\emph{masna gostota konfiguracije $\kappa$}, tako da za vsako množico $P\subseteq B$ velja
\[ M(P)=\int_{\kappa(P)}\rho_{\kappa}\,dv. \]
\end{comment}

V mehaniki kontinuuma se predpostavi naslednji zakon.
\begin{aksiom}[Zakon o ohranitvi mase]
	Gibanju $\chi(X,t)=\chi_t(X)$ pripada integrabilno, časovno odvisno skalarno polje
	$\rho\colon\Omega\to(0,\infty)$, imenovano \emph{masna gostota trenutne konfiguracije},
	tako da za vsako množico $\mathcal{P}\subseteq \B$ velja
	\[ M(\mathcal{P})=\int_{\chi_t(\mathcal{P})}\rho\,dv\qquad\textrm{za vsak}\ t\in I. \]
\end{aksiom}

Iz tega zakona z uporabo izreka \ref{i:prointrel} dobimo
\[ \int_{\mathcal{P}}(\rho_R-\rho J)\,dV=0 \qquad\textrm{za vsak}\ t\in I, \]
in ker to velja za vsako množico $\mathcal{P}\subseteq \B$, dobimo po trditvi \ref{t:oiz}
naslednjo relacijo za masni gostoti:
\begin{equation} \label{e:rojror}
	\rho_R(X)=\rho(X,t)J(X,t)\qquad\textrm{za vsak}\ X\in \B,\ t\in I.
\end{equation}
Seveda potem velja $\rho_R=\rho J$ tudi v prostorskem opisu na $\Omega$.

\begin{posledica} \label{p:roji}
	Za vsak $\mathcal{P}\subseteq \B$ in vsako
	tenzorsko polje $\vek{f}\colon \mathcal{P}\times I\to\W$, ki je integrabilno na $\mathcal{P}$
	pri vsakem $t\in I$, velja
	\[
		\int_{\mathcal{P}}\vek{f}\rho_R\,dV=\int_{\chi_t(\mathcal{P})}\vek{f}\rho\,dv.
	\]
\end{posledica}

\proof
	Uporabimo relacijo (\ref{e:rojror}) in izrek \ref{i:prointrel}.
\endproof

Še ena takojšnja posledica zakona o ohranitvi mase je ta, da se masa katerega koli
dela delesa med gibanjem ne spreminja, torej velja
\[ \frac{d}{dt}\int_{\chi_t(\mathcal{P})}\rho(x,t)\,dv = 0\quad\textrm{za vsak}\ t\in I. \]
Če je $\rho$ razreda $C^1$, potem z uporabo transportnega izreka \ref{i:transport}
pridem do
\[ \int_{\chi_t(\mathcal{P})}(\dot{\rho}+\rho\div\vek{v})\,dv=0, \]
in ker to velja za poljubno množico $\chi_t(\mathcal{P})\subseteq\chi_t(\B)$, je po trditvi \ref{t:oiz}
\begin{equation} \label{e:lozom}
	\dot{\rho}+\rho\div\vek{v}=0
\end{equation}
na celotnem območju $\chi_t(\B)$ in za vsak $t\in I$, torej na celotnem območju $\Omega$,
enako pa potem velja tudi v materialnem opisu na $\B\times I$.
Enačba (\ref{e:lozom}) je znana pod imenom \emph{lokalna oblika zakona o ohranitvi mase}.

\begin{izrek}
	Naj bo tenzorsko polje $\vek{f}\colon \B\times I\to W$ razreda $C^1$ in naj bo masna gostota
	$\rho\colon\Omega\to(0,\infty)$ prav tako razreda $C^1$.
	Potem za vsak $\mathcal{P}\subseteq\B$ velja
	\[
		\frac{d}{dt}\int_{\chi_t(\mathcal{P})}\vek{f}\rho\,dv=
		\int_{\chi_t(\mathcal{P})}\dot{\vek{f}}\rho\,dv.
	\]
\end{izrek}
\proof
	Iz transportnega izreka \ref{i:transport} dobimo
	\begin{align*}
		\frac{d}{dt}\int_{\chi_t(\mathcal{P})}\vek{f}\rho\,dv
		&= \int_{\chi_t(\mathcal{P})}\big((\vek{f}\rho)\,\dot{}+\vek{f}\rho\div\vek{v}\big)\,dv \\
		&= \int_{\chi_t(\mathcal{P})}(\dot{\vek{f}}\rho+\vek{f}\dot{\rho}+\vek{f}\rho\div\vek{v})\,dv\\
		&= \int_{\chi_t(\mathcal{P})}\dot{\vek{f}}\rho\,dv+
		\int_{\chi_t(\mathcal{P})}\vek{f}(\dot{\rho}+\rho\div\vek{v})\,dv,
	\end{align*}
	od koder sledi enakost iz izreka, če upoštevamo (\ref{e:lozom}).
\endproof
Druga možnost za dokaz izreka je uporaba posledice \ref{p:roji} in
izreka \ref{i:prointrel}.