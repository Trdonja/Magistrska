\chapter{Singularne ploskve}


V tem poglavju naj velja pravilo, da imajo grški indeksi ($\alpha$,$\beta$,$\gamma$,$\dots$) zalogo
vrednosti $\{1,2\}$, latinski indeksi ($\dots$,$i$,$j$,$k$,$\dots$) pa zalogo vrednosti $\{1,2,3\}$.


\section{Krivočrtne koordinate}


\subsection{Koordinatni sistem}


\begin{definicija}
	Naj bo $U\subseteq\E$ odprta množica. \emph{Koordinatni sistem} na $U$ je bijektivna in
	gladka preslikava $\psi\colon U\to V$, kjer je $V$ odprta
	množica v $\R^3$, inverz $\psi^{-1}$ pa je prav tako gladka preslikava.
\end{definicija}

Naj bo $x\in U$ in
\[ \psi\colon x\mapsto (x^1,x^2,x^3)=\psi(x). \]
$(x^1,x^2,x^3)$ so \emph{(krivočrtne) koordinate} točke $x$, funkcije
\[ \psi^i\colon U\to\R,\qquad \psi^i(x)=x^i,\qquad i=1,2,3 \]
pa se imenujejo \emph{$i$-ta koordinatna funkcija} koordinatnega sistema $\psi$.
Po dogovoru tudi $(x^i)$ rečemo koordinatni sistem na $U$.

Zopet naj za $x\in U$ velja $\psi(x)=(x^1,x^2,x^3)$. Slika preslikave
\begin{equation} \label{e:kokri}
	\gamma_i\colon(-\varepsilon,\varepsilon)\to U,\qquad
	\gamma_i(t)=\psi^{-1}\big((x^1,x^2,x^3)+te_i\big),\qquad i=1,2,3,
\end{equation}
kjer je $e_i$ $i$-ti vektor iz standardne baze za $\R^3$,
je \emph{$i$-ta koordinatna krivulja} v $U$, ki gre pri $t=0$ skozi točko $x$.
V točki $x$ lahko definiramo \emph{tangentne vektorje}
\begin{equation} \label{e:tanvek}
	\vek{g}_i(x)=\at{\frac{d}{dt}\gamma_i(t)}{t=0}=\dot{\gamma}_i(0)
	=\at{\frac{\partial\psi^{-1}}{\partial x^i}}{(x^1,x^2,x^3)}
	, \qquad i=1,2,3
\end{equation}
in \emph{gradientne vektorje}
\begin{equation} \label{e:gradvek}
	\vek{g}^i(x)=\nabla\psi^i(x),\qquad i=1,2,3.
\end{equation}

\begin{trditev}
	Množica $\{\vek{g}_i(x)\}_i$ tvori bazo za translacijski prostor $\V$.
\end{trditev}

\proof
	Naj bo $\vek{u}\in\V$ poljuben in definirajmo krivuljo skozi $x$ s predpisom
	\[ \gamma\colon(-\varepsilon,\varepsilon)\to U,\qquad \gamma(t)=x+t\vek{u}. \]
	Velja
	\[ \gamma(t)=\psi^{-1}\big(\psi^1(x+t\vek{u}),\psi^2(x+t\vek{u}),\psi^3(x+t\vek{u})\big), \]
	zato je
	\[
		\vek{u}=\at{\frac{d}{dt}\gamma(t)}{t=0}=
		\at{\frac{\partial\psi^{-1}}{\partial x^i}}{(x^1,x^2,x^3)}
		\at{\frac{d}{dt}\psi^i(x+t\vek{u})}{t=0}=
		\at{\frac{d}{dt}\psi^i(x+t\vek{u})}{t=0}\vek{g}_i(x).
	\]
	Z drugimi besedami, $\{\vek{g}_i(x)\}_{i}$ razpenja prostor $\V$.
\endproof

Tangentni in gradientni vektorji so definirani v vsaki točki množice $U$, torej
gre v bistvu za vektorska polja. V vsaki točki $x\in U$ se baza $\{\vek{g}_i(x)\}_i$
imenuje \emph{naravna baza} koordinatnega sistema $(x^i)$ za prostor $\V$.
Množica $\{\vek{g}^i(x)\}_i$ pa je prav tako baza za $\V$, ki je dualna bazi $\{\vek{g}_i(x)\}_i$.
Res, iz zveze
\[ x^i=\psi^i(\psi^{-1}(x^1,x^2,x^3)) \]
dobimo s posrednim odvajanjem
\begin{equation} \label{e:gigj}
	\topbot{\delta}{i}{j}=\frac{\partial x^i}{\partial x^j}=
	(\nabla\psi^i)\cdot\frac{\partial\psi^{-1}}{\partial x^j}=\vek{g}^i\cdot\vek{g}_j.
\end{equation}
Koeficienti
\[ g_{ij}=\vek{g}_i\cdot\vek{g}_j\qquad\textrm{in}\qquad g^{ij}=\vek{g}^i\cdot\vek{g}^j \]
so \emph{koeficienti metričnega tenzorja} in so seveda odvisni od točke $x\in U$. 
Iz zadnjih treh zvez hitro vidimo, da velja
\[ g^{ik}g_{kj}=\topbot{\delta}{i}{j},\qquad \vek{g}^i=g^{ik}\vek{g}_k,\qquad \vek{g}_i=g_{ik}\vek{g}^k. \]


\subsection{Koordinatna transformacija}


Naj bosta $\psi$ in $\bar{\psi}$ oz.~$(x^i)$ in $(\bar{x}^i)$ koordinatna sistema za
$U\subseteq\E$ z naravnima bazama $\{\vek{g}_i(x)\}_i$ oz.~$\{\bar{\vek{g}}_i(x)\}_i$.
Koordinatna transformacija je podana z
\[
	x^i=x^i(\bar{x}^1,\bar{x}^2,\bar{x}^3)\quad\Longleftrightarrow\quad
	\bar{x}^k=\bar{x}^k(x^1,x^2,x^3).
\]
Za vektorje iz naravne baze velja
\begin{equation} \label{e:gtog}
	\vek{g}^i=\frac{\partial x^i}{\partial\bar{x}^k}\bar{\vek{g}}^k,\qquad
	\vek{g}_i=\frac{\partial\bar{x}^k}{\partial x^i}\bar{\vek{g}}_k.
\end{equation}
Prvo enakost dobimo kot gradient funkcije $x^i$ iz koordinatne transformacije,
drugo enakost pa dobimo s parcialnim odvajanjem enakosti
\[ 
	\psi^{-1}(x^1,x^2,x^3)=\bar{\psi}^{-1}(\bar{x}^1(x^1,x^2,x^3),
	\bar{x}^2(x^1,x^2,x^3),\bar{x}^3(x^1,x^2,x^3)).
\]

Naj bo $(x^1,x^2,x^3)$ in $(\bar{x}^1,\bar{x}^2,\bar{x}^3)$ dvoje koordinat točke $x\in U\subseteq\E$.
Če je $f\colon U\to\R$ skalarno polje, potem bomo brez posebnih oznak pisali
\[ f(x)=f(x^1,x^2,x^3)=f(\bar{x}^1,\bar{x}^2,\bar{x}^3). \]
Če je $\vek{u}\colon U\to\V$ vektorsko polje, potem ga lahko zapišemo v komponentni
obliki glede na naravno bazo prostora $\V$:
\begin{align}
	\vek{u}&=u_i\,\vek{g}^i=u^i\,\vek{g}_i \nonumber \\
	&=\bar{u}_i\,\bar{\vek{g}}^i=\bar{u}^i\,\bar{\vek{g}}_i, \label{e:vekkomp}
\end{align}
Prav tako lahko tenzorsko polje $\ten{S}\colon U\to\L(\V)$ zapišemo v
komponentni obliki glede na bazo prostora $\L(\V)$:
\begin{align}
	\ten{S}&=S_{ij}\,\vek{g}^i\otimes\vek{g}^j=\topbot{S}{i}{j}\,\vek{g}_i\otimes\vek{g}^j \nonumber \\
	&=\bar{S}_{ij}\,\bar{\vek{g}}^i\otimes\bar{\vek{g}}^j=
	\topbot{\bar{S}}{i}{j}\,\bar{\vek{g}}_i\otimes\bar{\vek{g}}^j \label{e:tenkomp}
\end{align}
S pomočjo (\ref{e:gtog}) dobimo naslednje zveze med posameznimi komponentami:
\begin{gather*}
	\bar{u}_i=\frac{\partial x^k}{\partial\bar{x}^i}u_k, \qquad
	\bar{u}^i=\frac{\partial\bar{x}^i}{\partial x_k}u_k, \\
	\bar{S}_{ij}=\frac{\partial x^k}{\partial\bar{x}^i}\frac{\partial x^l}{\partial\bar{x}^j}S_{kl}, \qquad
	\topbot{\bar{S}}{i}{j}=\frac{\partial\bar{x}^i}{\partial x^k}\frac{\partial x^l}{\partial\bar{x}^j}\topbot{S}{k}{l}.
\end{gather*}
Posamezne komponente vektorskih in tenzoskih polj dobimo s pomočjo zveze (\ref{e:gigj})
iz (\ref{e:vekkomp}) in (\ref{e:tenkomp}):
\begin{gather}
	u_i=\vek{g}_i\cdot\vek{u},\qquad u^i=\vek{g}^i\cdot\vek{u}, \label{e:kompvek} \\
	S_{ij}=\vek{g}_i\cdot\ten{S}\vek{g}_j,\qquad\topbot{S}{i}{j}=\vek{g}^i\cdot\ten{S}\vek{g}_j.\label{e:kompten}
\end{gather}


\subsection{Gradient in kovariantni odvod}


Naj bo $(x^i)$ koordinatni sistem na $U\subseteq\E$ z naravnima bazama $\{\vek{g}_i\}_i$ in $\{\vek{g}^i\}_i$.
Gradiente tenzorskih polj lahko zapišemo v komponentni obliki glede na naravno bazo koordinatnega sistema.

Naj bo $\vek{f}\colon U\to\W$ zvezno odvedljivo tenzorsko polje. Če je $\gamma_i$ $i$-ta
koordinatna krivulja, ki gre pri $t=0$ skozi točko $x$, potem iz (\ref{e:tanvek}) dobimo
\begin{equation*}
	\at{\frac{d}{dt}\vek{f}\big(\gamma_i(t)\big)}{t=0}=
	(\nabla \vek{f}(x))[\dot{\gamma}_i(0)]=(\nabla \vek{f}(x))[\vek{g}_i(x)],
\end{equation*}
po drugi strani pa imamo po (\ref{e:kokri})
\begin{equation*}
	\at{\frac{d}{dt}\vek{f}\big(\gamma_i(t)\big)}{t=0}=
	\at{\frac{d}{dt}\vek{f}\big(\psi^{-1}((x^1,x^2,x^3)+te_i)\big)}{t=0}=
	\at{\frac{\partial(\vek{f}\circ\psi^{-1})}{\partial x^i}}{(x^1,x^2,x^3)}.
\end{equation*}
Po našem dogovoru pišemo namesto $\vek{f}\circ\psi^{-1}$ kar $\vek{f}$. Če izenačimo oba rezultata, dobimo
\begin{equation} \label{e:sotf}
	\frac{\partial\vek{f}}{\partial x^i}=(\nabla \vek{f}(x))[\vek{g}_i(x)].
\end{equation}
V primeru, ko je $f$ skalarno polje, je po (\ref{e:kompvek}) $(\nabla f)\cdot\vek{g}_i=(\nabla f)_i$
ravno kovariantna komponenta gradienta $\nabla f$, torej je
\begin{equation} \label{e:gradskal}
	\nabla f = \frac{\partial f}{\partial x^i}\,\vek{g}^i.
\end{equation}

Preden nadaljujemo z gradienti vektorskih polj, vpeljimo najprej standardne oznake
za gradiente vektorjev iz naravne baze:
\begin{equation} \label{e:gradg}
	\nabla\vek{g}_i=\ten{\Gamma}_i,\qquad\nabla\vek{g}^i=\ten{\Gamma}^i.
\end{equation}
Tu sta $\ten{\Gamma}_i,\ten{\Gamma}^i\colon U\to\L(\V)$ tenzorski polji drugega reda in ju
v komponentni obliki zapišemo kot
\begin{equation} \label{e:cs}
	\ten{\Gamma}_i=\cs{i}{j}{k}\,\vek{g}_j\otimes\vek{g}^k,\qquad\ten{\Gamma}^i=\ks{i}{jk}\,\vek{g}^j\otimes\vek{g}^k.
\end{equation}
Koeficienti $\cs{i}{j}{k}$ in $\ks{i}{jk}$ se imenujejo \emph{Christoffelovi simboli}
in ne gre za komponente kakega tenzorja tretjega reda. Če v enačbi (\ref{e:sotf})
za $\vek{f}$ vstavimo vektorsko polje $\vek{g}_i$ oz.~$\vek{g}^i$ in upoštevamo
(\ref{e:kompten}), (\ref{e:gradg}) in (\ref{e:cs}), dobimo
\begin{equation} \label{e:csexplicit}
	\cs{i}{j}{k}=\vek{g}^j\cdot\frac{\partial\vek{g}_i}{\partial x^k}
	=-\vek{g}_i\cdot\frac{\partial\vek{g}^j}{\partial x^k},\qquad
	\ks{i}{jk}=\vek{g}_j\cdot\frac{\partial\vek{g}_i}{\partial x^k}.
\end{equation}
Pri tem je drugi izraz za $\cs{i}{j}{k}$ dobljen iz prvega s parcialnim odvajanjem
enakosti $\vek{g}^i\cdot\vek{g}_j=\topbot{\delta}{i}{j}$ po spremenljivki $x^k$.

Če po pravilu (\ref{t:divprop})$_2$ izračunamo gradient izraza $(\vek{g}^i\cdot\vek{g}_j)$,
ki je enak 0, in upoštevamo, da velja\footnote{REF I-Shi-Liu str. 241}
\[(\vek{u}\otimes\vek{v})^{T}=\vek{v}\otimes\vek{u}\]
za poljubna vektorja $\vek{u}$ in $\vek{v}$ ter da je transponiranje linearna operacija,
potem dobimo zvezo
\begin{equation} \label{e:lcs1}
	\cs{j}{i}{k}=-\ks{i}{jk}.
\end{equation}
Nadalje, ker je $\ten{\Gamma}^i=\nabla(\nabla\psi^i)$ in je drugi gradient vedno
simetrični tenzor\footnote{REF I-Shi-Liu str. 271}, veljata še naslednji zvezi:
\[ \ks{i}{jk}=\ks{i}{kj},\qquad\cs{j}{i}{k}=\cs{k}{i}{j}. \]
Ker je zaradi teh zvez možno prehajati iz ene vrste simbolov v drugo vrsto,
so v uporabi zgolj simboli $\cs{i}{j}{k}$, imenovani Christoffelovi simboli druge vrste.

Naj bo sedaj $\vek{u}\colon U\to\V$ vektorsko polje, v komponentni obliki
zapisano kot
\[ \vek{u}=u^j\vek{g}_j=u_k\vek{g}^k. \]
Njegov gradient pripada prostoru $\L(\V)$, zato ga lahko zapišemo v komponentni obliki
\begin{equation} \label{e:gradukomp}
	\nabla\vek{u}=\topbot{u}{j}{,k}\,\vek{g}_j\otimes\vek{g}^k.
\end{equation}
Poiščimo izraz za komponente $\topbot{u}{j}{,k}$. Z upoštevanjem
(\ref{e:kompten}) in (\ref{e:sotf}) dobimo
\begin{align}
	\topbot{u}{j}{,k}&=\vek{g}^j\cdot\frac{\partial\vek{u}}{\partial x^k}=
	\vek{g}^j\cdot\frac{\partial(u^i\vek{g}_i)}{\partial x^k} \nonumber \\
	&=\vek{g}^j\cdot\Big(\frac{\partial u^i}{\partial x^k}\vek{g}_i+
	u^i\frac{\partial\vek{g}_i}{\partial x^k}\Big) \nonumber \\
	&=\frac{\partial u^j}{\partial x^k}+u^i\cs{i}{j}{k}. \label{e:kovod1}
\end{align}
Pri tem smo na zadnjem koraku upoštevali relacijo (\ref{e:csexplicit}). Dobljena enakost (\ref{e:kovod1})
za $\topbot{u}{j}{,k}$ je t.~i.~\emph{kovariantni odvod} komponentne $u^j$ po spremenljivki $x^k$.

Če zapišemo
\begin{equation} \label{e:nimena1}
	\nabla\vek{u}=u_{j,k}\,\vek{g}^j\otimes\vek{g}^k,
\end{equation}
in ponovimo prejšnji postopek, kjer dodatno uporabimo relacijo (\ref{e:lcs1}), dobimo
\begin{equation} \label{e:kov2}
	u_{j,k}=\frac{\partial u_j}{\partial x^k}-u_i\cs{j}{i}{k}.
\end{equation}
\begin{primer}
	Če v izrazu $\vek{u}=u_j\vek{g}^j$ vstavimo za $u_j=\partial f/\partial x^j$,
	kar je kovariantna komponenta od $\nabla f$ v izrazu (\ref{e:gradskal}), ter vstavimo v
	(\ref{e:kov2}), rezultat pa nato v (\ref{e:nimena1}), dobimo
	\begin{equation} \label{e:dvagrad}
		%\Big(\frac{\partial f}{\partial x^j}\Big)_{,k}
		\nabla(\nabla f)=\Big(\frac{\partial^2 f}{\partial x^j\partial x^k}-
		\frac{\partial f}{\partial x^i}\cs{j}{i}{k}\Big)\vek{g}^j\otimes\vek{g}^k.
	\end{equation}
\end{primer}

Naj bo $\ten{S}\colon U\to\L(\V)$. Potem je $\nabla\ten{S}\colon U\to\L(\V,\L(\V))$
in ga v komponentni obliki lahko razpišemo glede na bazo prostora $\L(\V,\L(\V))$ kot
\[ \nabla\ten{S}=\topbot{S}{ij}{,k}\vek{g}_i\otimes\vek{g}_j\otimes\vek{g}^k. \]
Iz (\ref{e:sotf}) dobimo
\[
	\frac{\partial\ten{S}}{\partial x^k}=(\nabla\ten{S})[\vek{g}_k]=
	(\topbot{S}{ij}{,l}\vek{g}_i\otimes\vek{g}_j\otimes\vek{g}^l)\vek{g}_k=
	\topbot{S}{ij}{,k}\vek{g}_i\otimes\vek{g}_j.
\]
Če na dobljeni enakosti uporabimo (\ref{e:kompten}), dobimo
\begin{align*}
	\topbot{S}{ij}{,k}&=\vek{g}^i\cdot\frac{\partial\ten{S}}{\partial x^k}\vek{g}^j=
	\vek{g}^i\cdot\frac{\partial(S^{lr}\vek{g}_l\otimes\vek{g}_r)}{\partial x^k}\vek{g}^j \\
	&=\vek{g}^i\cdot\Big(
	\frac{\partial S^{lr}}{\partial x^k}\vek{g}_l\otimes\vek{g}_r+
	S^{lr}\frac{\partial\vek{g}_l}{\partial x^k}\otimes\vek{g}_r+
	S^{lr}\vek{g}_l\otimes\frac{\partial\vek{g}_r}{\partial x^k}
	\Big)\vek{g}^j \\
	&=\frac{\partial S^{ij}}{\partial x^k}+S^{lj}\cs{l}{i}{k}+S^{ir}\cs{r}{j}{k}.
\end{align*}
Preostale komponente za $\nabla\ten{S}$ dobimo na enak način.


\subsection{Divergenca}


Spomnimo se, sled tenzorja $\ten{S}\in\L(\V)$ s komponentno obliko (\ref{e:tenkomp})
je\footnote{REF I-Shi-Liu str. 249} skalar
\[ \tr\,\ten{S}=\topbot{S}{i}{i}=g^{ij}S_{ij}. \]
Če je $\vek{u}$ vektorsko polje, potem iz (\ref{e:gradukomp}) in (\ref{e:nimena1}) dobimo
\begin{equation} \label{e:divu}
	\div\vek{u} = \tr(\nabla\vek{u}) = \topbot{u}{i}{,i} = g^{ij}u_{i,j}.
\end{equation}

Poiščimo še izraz za divergenco tenzorskega polja $\ten{S}\colon U\to\L(\V)$.
Iz (\ref{e:divu}) dobimo za poljubno vektorsko polje $\vek{u}$
\[
	\div(\ten{S}^{T}\vek{u})=\div(S^{ij}\vek{g}_j\otimes\vek{g}_i u_k\vek{g}^k)
	=\div(S^{ij}u_i\vek{g}_j)=(S^{ij}u_i)_{,\,j}.
\]
Bralec se lahko sam prepriča, da tudi za kovariantni odvod produkta velja
podobno pravilo, kot ga poznamo za običaji odvod, zato imamo
\[
	(S^{ij}u_i)_{,j}=\topbot{S}{ij}{,j}u_i+S^{ij}u_{i,j}=
	\topbot{S}{ij}{,j}\vek{g}_i\cdot\vek{u}+\tr(\ten{S}^{T}\nabla\vek{u})
\]
Dobili smo enakost
\begin{equation} \label{e:divStu}
	\div(\ten{S}^{T}\vek{u})=\topbot{S}{ij}{,j}\vek{g}_i\cdot\vek{u}+\tr(\ten{S}^{T}\nabla\vek{u}).
\end{equation}
Če je $\vek{u}$ konstantno vektorsko polje, potem je $\nabla\vek{u}=\ten{0}$ in v
enačbi (\ref{e:divStu}) s pomočjo definicije REF prepoznamo izraz za $\div\ten{S}$, ki je
\[
	\div\ten{S}=\topbot{S}{ij}{,j}\vek{g}_i.
\]
Če dobljeni izraz vstavimo nazaj v enačbo (\ref{e:divStu}) in zamenjamo $\ten{S}$ z
$\ten{S}^{T}$, dobimo naslednjo trditev.
\begin{trditev} \label{e:divSu}
	Za poljubno tenzorsko polje $\ten{S}\colon U\to\L(\V)$ in poljubno
	vektorsko polje $\vek{u}\colon U\to\V$ velja
	\[ \div(\ten{S}\vek{u})=\vek{u}\cdot\div\ten{S}^{T}+\tr(\ten{S}\nabla\vek{u}). \]
\end{trditev}


\section{Gibajoča se ploskev}


Naj bo $(x_i)$ kartezijev koordinatni sistem za $\E$, kot smo ga prestavili v REF, ter označimo
njegovo naravno bazo z $\{\vek{e}_i\}_i$. Ta baza je enaka v vsaki točki prostora in je sama
sebi dualna, $\vek{e}^i\cdot\vek{e}_j=\topbot{\delta}{i}{j}$, zato v kartezijevem
koordinatnem sistemu ni potrebno uporabljati dvonivojskega indeksiranja.

Naj bo $\Theta\subseteq\R^2$ odprta množica in $I=(t_1,t_2)$ odprti interval.
Oglejmo si preslikavo
\begin{gather*}
	\varphi\colon\Theta\times I\to\E,\qquad \varphi\colon(\theta^1,\theta^2,t)\mapsto \vek{x}=x_i\vek{e}_i,\\
	x_i=\varphi_i(\theta^1,\theta^2,t)
\end{gather*}
Zahtevamo, da je $\varphi$ razreda $C^2$ na svoji domeni, ter da je pri vsakem $t\in I$ 
odvod preslikave $\varphi(\cdot,t)\colon \Theta\to\E$ linearna preslikava iz $\R^2$ v $\V$ ranga 2 v vsaki
točki $(\theta^1,\theta^2)\in\Theta$. Slednji pogoj je ekvivalenten temu, da na celotni domeni
$\Theta\times I$ velja
\[ \frac{\partial\varphi}{\partial\theta^1}\times\frac{\partial\varphi}{\partial\theta^2}\neq \vek{0}. \]
Pri danem $t\in I$ je preslikava $\varphi(\cdot,t)$ regularna parametrizacija ploskve,
vložene v $\E$. Označili jo bomo s $\sigma(t)=\varphi(\Theta,t)$. Vse ploskve $\{\sigma(t)\}_{t\in I}$
skupaj predstavljajo s časom gibajočo se ploskev po prostoru $\E$.

Vpeljimo oznake
\[
	\vek{a}_{\alpha}=\frac{\partial\varphi}{\partial\theta^{\alpha}}
	=\frac{\partial\varphi_i}{\partial\theta^{\alpha}}\vek{e}_i,\quad
	a=\|\vek{a}_1\times\vek{a}_2\|,\quad
	\vek{n}=\frac{1}{a}\vek{a}_1\times\vek{a}_2.
\]
Vektorja $\vek{a}_1$ in $\vek{a}_2$ sta tangentna ali kovariantna bazna vektorja ploskve $\sigma(t)$
glede na parametrizacijo $\varphi(\cdot,t)$, vektor
$\vek{n}=n_j\vek{e}_j$ pa je normala na ploskev $\sigma(t)$. Velja
\[ \vek{a}_{\alpha}\cdot\vek{n}=0,\qquad \vek{n}\cdot\vek{n}=1,\qquad \vek{n}\cdot\frac{\partial\vek{n}}{\partial\theta^{\alpha}}=0. \]
Tretjo enakost dobimo s parcialnim odvajanjem druge enakosti. Koeficienti
\[
	a_{\alpha\beta}=\vek{a}_{\alpha}\cdot\vek{a}_{\beta}=
	\frac{\partial\varphi_i}{\partial\theta^{\alpha}}\frac{\partial\varphi_i}{\partial \theta^{\beta}}
\]
so koeficienti prve fundamentalne forme ploskve,
\[
	b_{\alpha\beta}=\frac{\partial^2\varphi}{\partial\theta^{\alpha}\partial\theta^{\beta}}
	\cdot\vek{n}=\frac{\partial\vek{a}_{\alpha}}{\partial\theta^{\beta}}\cdot\vek{n}
\]
pa so koeficienti druge fundamentalne forme.

Kartezijeve koordinate točke $\vek{x}=x_j\vek{e}_j\in\E$ so $(x_1,x_2,x_3)$. Naj bosta $\theta\in\Theta$
in $t\in I$ poljubna. Za neko dovolj majhno okolico $U\subseteq\Theta$ točke $\theta$ in neki dovolj
majhen $c>0$ definirajmo na množici $U\times(-c,c)$ koordinatno transformacijo
\[
	x_j(\theta^1,\theta^2,\theta^3)=\varphi_j(\theta^1,\theta^2,t)+\theta^3 n_j(\theta^1,\theta^2,t).
\]
Pripadajoči tangentni vektorji so
\[ \vek{g}_{\alpha}=\vek{a}_{\alpha}+\theta^3\frac{\partial\vek{n}}{\partial\theta^{\alpha}},\qquad \vek{g}_3=\vek{n}. \]
Pri $\theta^3=0$, torej na ploskvi, imamo
\begin{align*}
	\vek{g}_{\alpha}=\vek{a}_{\alpha},\qquad \vek{g}^{\alpha}=\vek{a}^{\alpha},\qquad \vek{g}_3=\vek{g}^3=\vek{n}, \\
	g_{\alpha\beta}=a_{\alpha\beta},\qquad g^{\alpha\beta}=a^{\alpha\beta},\qquad g_{j 3}=g^{j 3}=\delta_{j 3}, \\
	\vek{a}^{\alpha}=a^{\alpha\beta}\vek{a}_{\beta},\qquad \vek{a}_{\alpha}=a_{\alpha\beta}\vek{a}^{\beta},
	\qquad a_{\alpha\beta}a^{\beta\gamma}=\bottop{\delta}{\alpha}{\gamma}.
\end{align*}

\[\frac{\dtd g}{\dtd t}\]