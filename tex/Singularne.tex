\chapter{Singularne ploskve}


V tem poglavju naj velja pravilo, da imajo grški indeksi ($\alpha$,$\beta$,$\gamma$,$\dots$) zalogo
vrednosti $\{1,2\}$, latinski indeksi ($\dots$,$i$,$j$,$k$,$\dots$) pa zalogo vrednosti $\{1,2,3\}$.


\section{Gibajoča se ploskev}


Naj bo $(x_i)$ kartezijev koordinatni sistem za $\E$, kot smo ga prestavili v REF, ter označimo
njegovo naravno bazo z $\{\vek{e}_i\}_i$. Ta baza je enaka v vsaki točki prostora in je sama
sebi dualna, $\vek{e}^i\cdot\vek{e}_j=\topbot{\delta}{i}{j}$, zato v kartezijevem
koordinatnem sistemu ni potrebno uporabljati dvonivojskega indeksiranja.
Christoffelovi simboli za kartezijev koordinatni sistem so vsi enaki 0.

Naj bo $\Theta\subseteq\R^2$ odprta množica in $I=(t_1,t_2)$ odprti interval.
Oglejmo si preslikavo
\begin{equation} \label{e:pargip}
	\varphi\colon\Theta\times I\to\E,\qquad \varphi\colon(\theta^1,\theta^2,t)\mapsto x,
\end{equation}
ki jo lahko podamo s tremi koordinatnimi funkcijami
\[
	x_i=\varphi_i(\theta^1,\theta^2,t).
\]
Zahtevamo, da je $\varphi$ razreda $C^2$ na svoji domeni, ter da je pri vsakem $t\in I$ 
odvod preslikave $\varphi(\cdot,t)\colon \Theta\to\E$ linearna preslikava iz $\R^2$ v $\V$ ranga 2 v vsaki
točki $(\theta^1,\theta^2)\in\Theta$. Slednji pogoj je ekvivalenten temu, da na celotni domeni
$\Theta\times I$ velja
\[ \frac{\partial\varphi}{\partial\theta^1}\times\frac{\partial\varphi}{\partial\theta^2}\neq \vek{0}. \]
Pri danem $t\in I$ je preslikava $\varphi(\cdot,t)$ regularna parametrizacija ploskve,
vložene v $\E$. Označili jo bomo s $\sigma(t)=\varphi(\Theta,t)$. Vse ploskve $\{\sigma(t)\}_{t\in I}$
skupaj predstavljajo s časom gibajočo se ploskev po prostoru $\E$.

Vpeljimo oznake
\[
	\vek{a}_{\alpha}=\frac{\partial\varphi}{\partial\theta^{\alpha}}
	=\frac{\partial\varphi_i}{\partial\theta^{\alpha}}\vek{e}_i
	,\quad a=\|\vek{a}_1\times\vek{a}_2\|,\quad
	\vek{n}=\frac{1}{a}\vek{a}_1\times\vek{a}_2.
\]
Komponentna oblika za vektor $\vek{n}=n_i\vek{e}_i$ je
\[
	n_i=\frac{1}{a}\varepsilon_{ijk}\frac{\partial\varphi_j}{\partial\theta^1}
	\frac{\partial\varphi_k}{\partial\theta^2},
\]
kjer je $\varepsilon_{ijk}$
%\[
%	\varepsilon_{ijk}=\left\{\begin{array}{rl}
%		1;&(i,j,k)\ \textrm{je soda permutacija od}\ (1,2,3) \\
%		-1;&(i,j,k)\ \textrm{je liha permutacija od}\ (1,2,3) \\
%		0;&\textrm{sicer}
%	\end{array}\right.
%\]
t.~i.~\emph{permutacijski simbol}.
Vektorja $\vek{a}_1$ in $\vek{a}_2$ sta tangentna ali kovariantna bazna vektorja ploskve $\sigma(t)$
glede na parametrizacijo $\varphi(\cdot,t)$,
$\vek{n}$ pa je normala na ploskev $\sigma(t)$. Velja
\begin{equation} \label{e:skalnul}
	\vek{a}_{\alpha}\cdot\vek{n}=0,\qquad \vek{n}\cdot\vek{n}=1.
\end{equation}
S parcialnim odvajanjem teh enakosti dobimo
\[
	\frac{\partial\vek{a}_{\alpha}}{\partial\theta^{\beta}}\cdot\vek{n}=
	-\vek{a}_{\alpha}\cdot\frac{\partial\vek{n}}{\partial\theta^{\beta}},\qquad
	\vek{n}\cdot\frac{\partial\vek{n}}{\partial\theta^{\alpha}}=0.
\]
Koeficienti
\[
	a_{\alpha\beta}=\vek{a}_{\alpha}\cdot\vek{a}_{\beta}
	%=\frac{\partial\varphi_i}{\partial\theta^{\alpha}}\frac{\partial\varphi_i}{\partial \theta^{\beta}}
\]
so koeficienti prve fundamentalne forme ploskve,
\begin{equation} \label{e:bkoe}
	b_{\alpha\beta}=\frac{\partial^2\varphi}{\partial\theta^{\alpha}\partial\theta^{\beta}}
	\cdot\vek{n}=\frac{\partial\vek{a}_{\alpha}}{\partial\theta^{\beta}}\cdot\vek{n}
	=-\vek{a}_{\alpha}\cdot\frac{\partial\vek{n}}{\partial\theta^{\beta}}
\end{equation}
pa so koeficienti druge fundamentalne forme. Seveda velja $a_{\alpha\beta}=a_{\beta\alpha}$
in $b_{\alpha\beta}=b_{\beta\alpha}$.

Ploskev $\sigma(t)$ lahko ekvivalentno predstavimo tudi v 
implicitni obliki, kot množico točk $x\in\E$, ki rešijo enačbo
\begin{equation} \label{e:plimpl}
	f(x,t)=0\qquad\textrm{oz.}\qquad f(x_1,x_2,x_3,t)=0,
\end{equation}
kjer je $f$ skalarna funkcija razreda $C^2$ na $\E\times I$,
za katero velja, da je $\grad f$ različen od $\vek{0}$ na $\sigma$
(pogoj regularnosti). Normala na ploskev se potem izraža kot
\begin{equation} \label{e:ngrad}
	\vek{n}=\frac{\grad f}{\|\grad f\|}.
\end{equation}
S totalnim odvodom enačbe (\ref{e:plimpl}) po času dobimo na $\sigma$
\begin{equation} \label{e:fdot0}
	\frac{\partial f}{\partial t}+\vek{\nu}\cdot\grad f = 0,
\end{equation}
kjer je
\begin{equation} \label{e:nu}
	\vek{\nu}=\frac{\partial\varphi}{\partial t},\qquad
	\vek{\nu}=\nu_i\vek{e}_i,\qquad \nu_i=\frac{\partial\varphi_i}{\partial t}
\end{equation}
vektorsko polje hitrosti gibanja ploskve $\sigma(t)$. Skalarno polje
\begin{equation} \label{e:un}
	u_n=\vek{\nu}\cdot\vek{n}
\end{equation}
je hitrost ploskve v smeri normale oz.~\emph{normalna hitrost ploskve}.
Če (\ref{e:ngrad}) vstavimo v (\ref{e:un}) in upoštevamo (\ref{e:fdot0}), dobimo
\begin{equation} \label{e:nohit}
	u_n=-\frac{1}{\|\grad f\|}\frac{\partial f}{\partial t}.
\end{equation}


\section{Spremljajoči koordinatni sistem} \label{s:sks}


Naj bo $\Omega$ odprto in povezano območje v prostoru $\N=\E\times\R$ z lastnostjo, da so projekcije
\[ \B_t=\{x\in\E\;;\ (x,t)\in\Omega\} \]
neprazne, povezane in regularne za vsak $t\in I$. Nadalje naj velja, da je za vsak $t\in I$
presek $\sigma(t)\cap\B_t$ neprazen ter da ploskev $\sigma(t)$ razdeljuje območje
$\B_t$ na dve disjunktni odprti podmnožici, ki ju bomo označili z $\B_t^+$ in $\B_t^-$.
Podmnožica $\B_t^+$ naj bo tista, v katero je usmerjena normala $\vek{n}$.

Vpeljimo še oznako
\[
	\sigma=\{(x,t)\;;\ x\in\sigma(t)\cap \B_t,\ t\in I\}\subset\N,
\]
z oznako $\sigma(t)$ pa bomo od tu naprej zaznamovali presek $\sigma(t)\cap\B_t$.

Na neki okolici $U_t$ ploskve $\sigma(t)$ lahko definiramo koordinatni sistem $\psi_t$ oz.~$(\theta^i)$
preko njegovega inverza
\[
	\psi_t^{-1}\colon(\theta^1,\theta^2,\theta^3)\mapsto x=\varphi(\theta^1,\theta^2,t)+
	\theta^3\vek{n}(\theta^1,\theta^2,t)\in U_t,
\]
ali pa preko koordinatne transformacije
\[
	x_j(\theta^1,\theta^2,\theta^3)=\varphi_j(\theta^1,\theta^2,t)+\theta^3 n_j(\theta^1,\theta^2,t).
\]
Ta koordinatni sistem je odvisen od $t\in I$.
Z enačbo $\theta^3=0$ je določena ploskev $\sigma(t)$.
Tangentni vektorji koordinatnega sistema $(\theta^i)$ so
\[
	\vek{g}_{\alpha}=\vek{a}_{\alpha}+\theta^3\frac{\partial\vek{n}}{\partial\theta^{\alpha}},
	\qquad \vek{g}_3=\vek{g}^3=\vek{n}.
\]
Pri $\theta^3=0$, torej na ploskvi, imamo $\vek{g}_{\alpha}=\vek{a}_{\alpha}$ in
$g_{\alpha\beta}=a_{\alpha\beta}$. Pri $\theta^3=0$
bomo označili gradientne vektorje $\vek{g}^{\alpha}$ z $\vek{a}^{\alpha}$
in metrične koeficiente $g^{\alpha\beta}$ z $a^{\alpha\beta}$.

Ker pri $\theta^3=0$ velja $g_{\alpha 3}=0$, je
\[ \vek{a}_{\alpha}=a_{\alpha\beta}\vek{a}^{\beta}, \]
in če sedaj to enačbo skalarno pomnožimo z $\vek{a}^{\gamma}$, dobimo
\[ a_{\alpha\beta}a^{\beta\gamma}=\bottop{\delta}{\alpha}{\gamma}, \]
torej sta si matriki $[a_{\alpha\beta}]$ in $[a^{\alpha\beta}]$ inverzni.
Po Lagrangevi identiteti imamo
\[ a^2=\|\vek{a}_1\times\vek{a}_2\|^2=a_{11}a_{22}-(a_{12})^2=\det[a_{\alpha\beta}] \]
in inverz matrike
\begin{equation*}
	[a_{\alpha\beta}]=\left[\begin{array}{cc}
		a_{11}&a_{12}\\a_{12}&a_{22}
	\end{array}\right] \quad\textrm{je}\quad
	[a^{\alpha\beta}]=\frac{1}{a^2}\left[\begin{array}{rr}
		a_{22}&-a_{12}\\-a_{12}&a_{11}
	\end{array}\right].
\end{equation*}
Torej imamo eksplicitno izražavo za koeficiente $a^{\alpha\beta}$.  Iz enačbe
\[ \vek{n}=g^{3\alpha}\vek{a}_{\alpha}+g^{33}\vek{n} \]
zaradi linearne neodvisnosti vektorjev $\vek{a}_1,\vek{a}_2$ in $\vek{n}$ sledi
$g^{3i}=\delta^{3i}$, torej je
\begin{equation} \label{e:agd}
	\vek{a}^{\alpha}=a^{\alpha\beta}\vek{a}_{\beta}
\end{equation}
in gradientna vektorja $\vek{a}^{\alpha}$ lahko dobimo iz tangentnih vektorjev $\vek{a}_{\alpha}$.

Christoffelove simbole koordinatnega sistema $(\theta^i)$ dobimo iz
enačb (\ref{e:csexplicit}). Na ploskvi $\theta^3=0$ so
\begin{gather}
	\cs{\alpha}{\beta}{\gamma}=\vek{a}^{\beta}\cdot\frac{\partial\vek{a}_{\alpha}}{\partial\theta^{\gamma}},\quad
	\cs{\alpha}{3}{\beta}=\vek{a}_{\alpha}\cdot\frac{\partial\vek{n}}{\partial\theta^{\beta}}=b_{\alpha\beta},\nonumber\\
	\cs{3}{\alpha}{\beta}=\vek{a}^{\alpha}\cdot\frac{\partial\vek{n}}{\partial\theta^{\beta}}
	=a^{\alpha\varrho}\vek{a}_{\varrho}\cdot\frac{\partial\vek{n}}{\partial\theta^{\beta}}
	=-a^{\alpha\varrho}b_{\varrho\beta},\nonumber\\
	\cs{3}{3}{\alpha}=\vek{n}\cdot\frac{\partial\vek{n}}{\partial\theta^{\alpha}}=0,\quad
	\cs{3}{i}{3}=0. \label{e:csnasi}
\end{gather}

Iz (\ref{e:iten}) dobimo pri $\theta^3=0$ še naslednjo zvezo:
\[ \vek{a}^{\alpha}\otimes\vek{a}_{\alpha}=\ten{1}-\vek{n}\otimes\vek{n}. \]
Če jo razpišemo v komponentni obliki glede na bazo $\{\vek{e}_i\otimes\vek{e}_j\}$, dobimo
\[
	a^{\alpha\beta}\frac{\partial\varphi_i}{\partial\theta^{\beta}}\vek{e}_i
	\otimes\frac{\partial\varphi_j}{\partial\theta^{\alpha}}\vek{e}_j=
	\delta_{ij}\vek{e}_i\otimes\vek{e}_j-n_i\vek{e}_i\otimes n_j\vek{e}_j
\]
oziroma
\begin{equation} \label{e:mucenje}
	a^{\alpha\beta}\frac{\partial\varphi_i}{\partial\theta^{\beta}}
	\frac{\partial\varphi_j}{\partial\theta^{\alpha}}=\delta_{ij}-n_i n_j.
\end{equation}


\section{Kompatibilnostni pogoji}


Naj bo $g\colon\Omega\to\R$ skalarno polje s funkcijsko obliko
\[ g=g(x,t)=g(\kappa^{-1}(x_1,x_2,x_3),t)=:g(x_1,x_2,x_3,t), \]
kjer je $\kappa$ kartezijev koordinatni sistem na $\E$. Na območju hiperploskve $\sigma$
je lahko polje $g$ podano v parametrični obliki $g=\tilde{g}(\theta^1,\theta^2,t)$ kot
\begin{equation} \label{e:totog}
	\tilde{g}(\theta^1,\theta^2,t)=g(\varphi(\theta^1,\theta^2,t),t).
\end{equation}
V tem razdelku se bomo ukvarjali
z izpeljavo pogojev, ki jim mora zadoščati tako skalarno polje na območju $\sigma\subset\Omega$.
Za začetek naj bo $g$ razreda $C^1$ na $\Omega$.

Naj bo $t_0\in I$ poljuben in naj bo $U_{t_0}$ neka okolica ploskve $\sigma(t_0)$,
na kateri lahko definiramo poprej predstavljeni koordinatni sistem $(\theta^i)$ oz. $\psi_{t_0}$.
Od tu dalje se bomo ukvarjali z analizo odvodov polja $g$ na ploskvi $\sigma(t_0)$.
Dobljeni rezultati bodo potem seveda veljali za vse $t\in I$, torej na celotnem
območju $\sigma$.

Na $U_{t_0}$ lahko torej $g$ predstavimo še v funkcijski obliki
\[ g=g(\psi_{t_0}^{-1}(\theta^1,\theta^2,\theta^3),t_0)=:g(\theta^1,\theta^2,\theta^3,t_0). \]
Na $\sigma(t_0)$ velja
\begin{equation} \label{e:gingtilde}
	\tilde{g}(\theta^1,\theta^2,t_0)=g(\theta^1,\theta^2,\theta^3=0,t_0) \quad\textrm{in}\quad
	\frac{\partial \tilde{g}}{\partial\theta^{\alpha}}=\frac{\partial g}{\partial\theta^{\alpha}}.
\end{equation}

\emph{Normalni odvod} $\partial g/\partial n$ polja $g$ na območju $\sigma$
je smerni odvod polja $g$ v smeri normale, torej
\begin{equation} \label{e:nodovkat}
	\frac{\partial g}{\partial n}=(\grad g)\cdot\vek{n}=\frac{\partial g}{\partial x^i}n_i.
\end{equation}
Ker je $\vek{n}$ tretji tangetni vektor koordinatnega sistema $(\theta^i)$,
dobimo iz (\ref{e:sotf})
\begin{equation} \label{e:p3jpn}
	\frac{\partial g}{\partial n}=\frac{\partial g}{\partial \theta^3}.
\end{equation}

Gradient funkcije $g$ na ploskvi $\sigma(t_0)$ lahko iz enačbe
(\ref{e:gradskal}) zapišemo v komponentni obliki z obzirom na kartezijev koordinatni
sistem $(x^i)$ in na koordinatni sistem $(\theta^i)$. Z dodatnim upoštevanjem 
(\ref{e:agd}), (\ref{e:gingtilde}) in (\ref{e:p3jpn}) dobimo
\begin{equation} \label{e:ojej}
 \grad g = \frac{\partial g}{\partial x_j}\vek{e}_j
 =\frac{\partial \tilde{g}}{\partial\theta^{\alpha}}a^{\alpha\beta}\vek{a}_{\beta}+
 \frac{\partial g}{\partial n}\vek{n}.
\end{equation}
Če enačbo pomnožimo skalarno z $\vek{e}_i$, dobimo
\begin{equation} \label{e:kpspace}
 \frac{\partial g}{\partial x_i}=
 a^{\alpha\beta}\frac{\partial\varphi_i}{\partial\theta^{\beta}}
 \frac{\partial \tilde{g}}{\partial\theta^{\alpha}}+
 n_i\frac{\partial g}{\partial n}.
\end{equation}

Iz enačbe (\ref{e:totog}) dobimo
\[ \frac{d\tilde{g}}{dt}=\frac{\partial g}{\partial t}+(\grad g)\cdot\vek{\nu}, \]
iz (\ref{e:ojej}), (\ref{e:nu}) in (\ref{e:un}) pa nato
\begin{equation} \label{e:padofg}
	\frac{\partial g}{\partial t}=\frac{d\tilde{g}}{dt}-
	a^{\alpha\beta}\frac{\partial\varphi_i}{\partial\theta^{\beta}}
 \frac{\partial \tilde{g}}{\partial\theta^{\alpha}}\nu_i-
	u_n\frac{\partial g}{\partial n}.
\end{equation}
Za skalarne funkcije na $\Omega$ lahko na podobmočju $\sigma$ vpeljimo diferencialni operator
\[ \frac{\dtd}{\dtd t}=\frac{\partial}{\partial t}+u_n\frac{\partial}{\partial n}. \]
Iz (\ref{e:padofg}) vidimo, da je
\begin{equation} \label{e:dtdofg}
	\frac{\dtd g}{\dtd t}=\frac{d\tilde{g}}{d t}-
	a^{\alpha\beta}\frac{\partial\varphi_i}{\partial\theta^{\beta}}
	\frac{\partial \tilde{g}}{\partial\theta^{\alpha}}\nu_i
\end{equation}
in
\begin{equation} \label{e:kptime}
	\frac{\partial g}{\partial t}=\frac{\dtd g}{\dtd t}-u_n\frac{\partial g}{\partial n}.
\end{equation}

Če je $\tilde{g}$ znana funkcija na $\sigma$, potem je mogoče izračunati odvode
$\partial\tilde{g}/\partial\theta^{\alpha}$ in $\partial\tilde{g}/\partial t$,
iz (\ref{e:dtdofg}) pa lahko potem izračunamo tudi $\dtd g/\dtd t$.
Enačbi (\ref{e:kpspace}) in (\ref{e:kptime}) sta \emph{kompatibilnostna pogoja za prve parcialne odvode}
funkcije $g$ na $\sigma$.
Če poznamo $\tilde{g}$ in $\partial g/\partial n$ na $\sigma$, potem enačbi
(\ref{e:kpspace}) in (\ref{e:kptime}) določata prve parcialne odvode polja $g$ na $\sigma$.

Iz (\ref{e:sotf}), (\ref{e:csnasi}) in (\ref{e:bkoe}) imamo
\begin{equation*}
	\frac{\partial\vek{n}}{\partial\theta^{\alpha}}=
	(\cs{3}{\beta}{\gamma}\vek{a}_{\beta}\otimes\vek{a}^{\gamma})\vek{a}_{\alpha}=
	\cs{3}{\beta}{\alpha}\vek{a}_{\beta}
	=-a^{\beta\varrho}b_{\varrho\alpha}\vek{a}_{\beta}=
	-a^{\beta\varrho}\frac{\partial^2\varphi_i}{\partial\theta^{\varrho}\partial\theta^{\alpha}}
	n_i\vek{a}_{\beta},
\end{equation*}
zato je
\begin{align*}
	\frac{\partial u_n}{\partial\theta^{\alpha}}&=\frac{\partial(\vek{\nu}\cdot\vek{n})}{\partial\theta^{\alpha}}=
	\frac{\partial\vek{\nu}}{\partial\theta^{\alpha}}\cdot\vek{n}+
	\vek{\nu}\cdot\frac{\partial\vek{n}}{\partial\theta^{\alpha}}\\
	&=n_i\frac{d}{dt}\frac{\partial\varphi_i}{\partial\theta^{\alpha}}-
	a^{\beta\varrho}\frac{\partial}{\partial\theta^{\varrho}}\Big(\frac{\partial\varphi_i}{\partial\theta^{\alpha}}\Big)
	n_i\frac{\partial\varphi_j}{\partial\theta^{\beta}}\nu_j=
	n_i\frac{\dtd}{\dtd t}\Big(\frac{\partial\varphi_i}{\partial\theta^{\alpha}}\Big),
\end{align*}
kjer smo upoštevali (\ref{e:dtdofg}), (\ref{e:nu}) ter dejstvo $d\varphi_i/dt=\partial\varphi_i/\partial t$.
Hitro se lahko prepričamo, da za diferencialni operator $\dtd/\dtd t$ velja Leibnizevo pravilo
za odvod produkta, zato iz enačb (\ref{e:skalnul}), če jih zapišemo v komponentni obliki,
in iz pravkar izpeljanega rezultata dobimo
\[
	\frac{\partial\varphi_i}{\partial\theta^{\alpha}}\frac{\dtd n_i}{\dtd t}=
	-\frac{\partial u_n}{\partial\theta^{\alpha}}\quad\mathrm{in}\quad
	n_i\frac{\dtd n_i}{\dtd t}=0.
\]
Če prvo od teh enačb množimo z $a^{\alpha\beta}\partial\varphi_j/\partial\theta^{\beta}$
in seštejemo po indeksu $\alpha$, potem z upoštevanjem še druge enačbe in pa (\ref{e:mucenje})
pridemo do
\begin{equation} \label{e:dtdni}
	\frac{\dtd n_i}{\dtd t}=-a^{\alpha\beta}\frac{\partial\varphi_i}{\partial\theta^{\alpha}}
	\frac{\partial u_n}{\partial\theta^{\beta}}.
\end{equation}

Naj bo sedaj $g$ razreda $C^2$ na $\Omega$. Tudi drugi gradient funkcije $g$ na ploskvi $\sigma(t_0)$
lahko zapišemo v komponentni obliki z obzirom na kartezijev koordinatni
sistem $(x^i)$ in na koordinatni sistem $(\theta^i)$. Iz (\ref{e:dvagrad})
z upoštevanjem (\ref{e:gingtilde}) in Christoffelovih simbolov (\ref{e:csnasi}) dobimo
\begin{align}
	\grad(\grad g) =\;&\frac{\partial^2 g}{\partial x_k\partial x_l}\vek{e}_k\otimes\vek{e}_l\nonumber\\
	=\;&\Big( \frac{\partial^2\tilde{g}}{\partial\theta^{\alpha}\partial\theta^{\beta}}-
	\frac{\partial\tilde{g}}{\partial\theta^{\gamma}}\cs{\alpha}{\gamma}{\beta}-
	\frac{\partial g}{\partial n}b_{\alpha\beta}\Big)\vek{a}^{\alpha}\otimes\vek{a}^{\beta}+\nonumber\\
	&+\Big(\frac{\partial}{\partial\theta^{\beta}}\Big(\frac{\partial g}{\partial n}\Big)+
	a^{\gamma\varrho}b_{\varrho\beta}\frac{\partial\tilde{g}}{\partial\theta^{\gamma}}\Big)
	(\vek{n}\otimes\vek{a}^{\beta}+\vek{a}^{\beta}\otimes\vek{n})+\nonumber\\
	&+\frac{\partial^2 g}{\partial n^2}\vek{n}\otimes\vek{n}, \label{e:dgrf}
\end{align}
kjer je
\[
	\frac{\partial^2 g}{\partial n^2}=\frac{\partial^2 g}{\partial\theta^3\partial\theta^3}=
	\frac{\partial^2 g}{\partial x_i\partial x_j}n_in_j.
\]
Če enačbo (\ref{e:dgrf}) z desne pomnožimo z $\vek{e}_j$, nato pa še skalarno z $\vek{e}_i$, dobimo
\begin{align}
	\frac{\partial^2 g}{\partial x_i\partial x_j}=\;&
	a^{\alpha\varrho}a^{\beta\mu}\Big( \frac{\partial^2\tilde{g}}{\partial\theta^{\alpha}\partial\theta^{\beta}}-
	\frac{\partial\tilde{g}}{\partial\theta^{\gamma}}\cs{\alpha}{\gamma}{\beta}-
	\frac{\partial g}{\partial n}b_{\alpha\beta}\Big)
	\frac{\partial\varphi_i}{\partial\theta^{\varrho}}\frac{\partial\varphi_j}{\partial\theta^{\mu}}+ \nonumber \\
	&+a^{\beta\mu}\Big(\frac{\partial}{\partial\theta^{\beta}}\Big(\frac{\partial g}{\partial n}\Big)+
	a^{\gamma\varrho}b_{\varrho\beta}\frac{\partial\tilde{g}}{\partial\theta^{\gamma}}\Big)
	\Big(n_i\frac{\partial\varphi_j}{\partial\theta^{\mu}}+n_j\frac{\partial\varphi_i}{\partial\theta^{\mu}}\Big)+
	\frac{\partial^2 g}{\partial n^2}n_in_j.
	\label{e:kpzdpo1}
\end{align}

Če v enačbi (\ref{e:kpspace}) nadomestimo $\tilde{g}$ z $\partial g/\partial t$, dobimo
\begin{equation} \label{e:ncmd}
	\frac{\partial^2 g}{\partial x_i\partial t}=
	a^{\alpha\beta}\frac{\partial\varphi_i}{\partial\theta^{\beta}}
	\frac{\partial}{\partial\theta^{\alpha}}\Big(\frac{\partial g}{\partial t}\Big)+
	n_i\frac{\partial}{\partial n}\Big(\frac{\partial g}{\partial t}\Big).
\end{equation}
Iz (\ref{e:nodovkat}) dobimo po Leibnizevem pravilu
\[
	\frac{\dtd}{\dtd t}\Big(\frac{\partial g}{\partial n}\Big)=
	\frac{\dtd}{\dtd t}\Big(\frac{\partial g}{\partial x_i}\Big)n_i+
	\frac{\partial g}{\partial x_i}\frac{\dtd n_i}{\dtd t}
\]
in če sedaj upoštevamo (\ref{e:dtdni}) in (\ref{e:kptime}) skupaj z dejstvom, da normalni odvod
in parcialni odvod po času komutirata, dobimo
\begin{equation} \label{e:katastrophe}
	\frac{\dtd}{\dtd t}\Big(\frac{\partial g}{\partial n}\Big)=
	\frac{\partial}{\partial n}\Big(\frac{\partial g}{\partial t}\Big)+
	u_n\frac{\partial^2 g}{\partial n^2}-
	a^{\alpha\beta}\frac{\partial\tilde{g}}{\partial\theta^{\alpha}}\frac{\partial u_n}{\partial\theta^{\beta}}.
\end{equation}
Enačbo (\ref{e:ncmd}) lahko sedaj s pomočjo (\ref{e:katastrophe}) in (\ref{e:kptime})
zapišemo kot
\begin{align}
	\frac{\partial^2 g}{\partial x_i\partial t}=\;&
	a^{\alpha\beta}\frac{\partial\varphi_i}{\partial\theta^{\beta}}
	\frac{\partial}{\partial\theta^{\alpha}}
	\Big(\frac{\dtd g}{\dtd t}-u_n\frac{\partial g}{\partial n}\Big)+ \nonumber \\
	&+\Big(\frac{\dtd}{\dtd t}\Big(\frac{\partial g}{\partial n}\Big)+
	a^{\alpha\beta}\frac{\partial\tilde{g}}{\partial\theta^{\alpha}}
	\frac{\partial u_n}{\partial\theta^{\beta}}-u_n\frac{\partial^2 g}{\partial n^2}\Big)n_i.
	\label{e:kpzdpo2}
\end{align}

Če v enačbi (\ref{e:kptime}) nadomestimo $\tilde{g}$ z $\partial g/\partial t$
in upoštevamo še (\ref{e:katastrophe}), dobimo
\begin{align}
	\frac{\partial^2 g}{\partial t^2}=\;&
	\frac{\dtd}{\dtd t}\Big(\frac{\dtd g}{\dtd t}-u_n\frac{\partial g}{\partial n}\Big)- \nonumber \\
	&-u_n\Big(\frac{\dtd}{\dtd t}\Big(\frac{\partial g}{\partial n}\Big)+
	a^{\alpha\beta}\frac{\partial\tilde{g}}{\partial\theta^{\alpha}}
	\frac{\partial u_n}{\partial\theta^{\beta}}-u_n\frac{\partial^2 g}{\partial n^2}\Big).
	\label{e:kpzdpo3}
\end{align}

Enačbe (\ref{e:kpzdpo1}), (\ref{e:kpzdpo2}) in (\ref{e:kpzdpo3}) se imenujejo
\emph{kompatibilnostni pogoji za druge parcialne odvode} funkcije $g$ na $\sigma$.


\section{Singularna ploskev}


Spomnimo se oznak iz začetka razdelka (\ref{s:sks}).
Območje $\Omega$ je s hiperploskvijo $\sigma$ razdeljeno na odprti podobmočji
\[
	\Omega^+=\{(x,t)\;;\ x\in\B_t^+,\ t\in I\}\quad\textrm{in}\quad
	\Omega^-=\{(x,t)\;;\ x\in\B_t^-,\ t\in I\}.
\]
$\Omega$ lahko torej zapišemo kot disjunktno unijo
\[ \Omega=\Omega^-\cup\sigma\cup\Omega^+. \]
Hiperploskev $\sigma$ je del roba območja $\Omega^-$, kot tudi $\Omega^+$.

Naj bo $\vek{f}\colon\Omega\to\W$ tenzorsko polje.
Recimo, da je $\vek{f}$ zvezno na $\Omega^-$ in na $\Omega^+$, ni pa zvezno na $\sigma$.
Če obstajata zvezni razširitvi
\[ \vek{f}^-\colon\Omega^-\cup\sigma\to\W\quad\textrm{in}\quad \vek{f}^+\colon\Omega^+\cup\sigma\to\W \]
polja $\vek{f}$ iz območja $\Omega^-$ na območje $\Omega^-\cup\sigma$ ter iz območja $\Omega^+$ na
območje $\Omega^+\cup\sigma$, potem funkcijo
\[
	\jump{\vek{f}}\colon\sigma\to\W,\qquad \jump{\vek{f}}=\vek{f}^+-\vek{f}^-
\]
imenujemo \emph{skok} polja $\vek{f}$.
Zvezni razširitvi $\vek{f}^-$ in $\vek{f}^+$ ter posledično skok $\jump{\vek{f}}$ zagotovo obstajajo,
kadar je $\vek{f}$ po normi omejeno na $\Omega^-\cup\Omega^+$. Če zvezni razširitvi obstajata,
sta tudi enolični, saj gre za razširitev na del robu območja.

Če je pri nekem $t\in I$ $\jump{\vek{f}}\neq\vek{0}$ skoraj povsod na $\sigma(t)$,
potem se $\sigma(t)$ imenuje \emph{singularna ploskev} za $\vek{f}$. Če je
$\jump{\vek{f}}\neq\vek{0}$ skoraj povsod na $\sigma$, potem se
$\sigma$ imenuje \emph{singularna hiperploskev} za $\vek{f}$.

Če je $\vek{f}$ razreda $C^{r-1}$ na $\Omega$ za neki $r\geq 1$ in je hkrati tudi razreda $C^r$
na $\Omega^-$ ter na $\Omega^+$, potem, če obstaja $\jump{D^r\vek{f}}$ in je različen od nič
skoraj povsod na $\sigma(t)$ oz.~na $\sigma$, je $\sigma(t)$ oz.~$\sigma$ \emph{singularna
ploskev oz.~hiperploskev reda $r$} za $\vek{f}$.

Naj bo $g$ skalarno polje razreda $C^1$ na $\Omega^-$ in na $\Omega^+$ in naj
obstaja skok odvoda, $\jump{Dg}$, na $\sigma$.
Za $g^+$ in $g^-$ lahko na $\sigma$ zapišemo kompatibilnostna
pogoja (\ref{e:kpspace}) in (\ref{e:kptime}). Če potem upoštevamo še
\[
	\frac{\partial\jump{g}}{\partial\theta^{\alpha}}=
	\frac{\partial g^+}{\partial\theta^{\alpha}}-\frac{\partial g^-}{\partial\theta^{\alpha}}
	\quad\textrm{in}\quad
	\frac{\dtd\jump{g}}{\dtd t}=\frac{\dtd g^+}{\dtd t}-\frac{\dtd g^-}{\dtd t},
\]
kar sledi neposredno iz $\jump{g}=g^+-g^-$, dobimo
\begin{align}
	\jump{\frac{\partial g}{\partial x_i}}&=
	a^{\alpha\beta}\frac{\partial\varphi_i}{\partial\theta^{\beta}}
	\frac{\partial \jump{g}}{\partial\theta^{\alpha}}+
	n_i\jump{\frac{\partial g}{\partial n}}\nonumber\\
	\textrm{in}\quad\jump{\frac{\partial g}{\partial t}}&=
	\frac{\dtd \jump{g}}{\dtd t}-u_n\jump{\frac{\partial g}{\partial n}}.\label{e:jpd1}
\end{align}
Pri tem je seveda% za $x\in\sigma(t)$
%\begin{gather*}
%	\frac{\partial g^+}{\partial n}(x,t)=\lim_{\varepsilon\searrow 0}
%	\frac{g(x+\varepsilon\vek{n},t)}{\varepsilon},\qquad
%	\frac{\partial g^-}{\partial n}(x,t)=\lim_{\varepsilon\searrow 0}
%	\frac{g(x-\varepsilon\vek{n},t)}{\varepsilon}\\
%	\textrm{in}\quad\jump{\frac{\partial g}{\partial n}}=\frac{\partial g^+}{\partial n}-
%	\frac{\partial g^-}{\partial n}.
%\end{gather*}
\[
	\jump{\frac{\partial g}{\partial n}}=\frac{\partial g^+}{\partial n}-
	\frac{\partial g^-}{\partial n}=(\grad g)^+\vek{n}-(\grad g)^-\vek{n}.
\]
Če je $\sigma(t)$ singularna ploskev reda 1 za $g$, potem je $\jump{g}=0$ in enačbi
(\ref{e:jpd1}) se zreducirata na
\[
	\jump{\frac{\partial g}{\partial x_i}}=\jump{\frac{\partial g}{\partial n}}n_i\quad\textrm{in}
	\quad \jump{\frac{\partial g}{\partial t}}=-u_n\jump{\frac{\partial g}{\partial n}}.
\]

Naj bo sedaj $g$ skalarno polje razreda $C^2$ na $\Omega^-$ in na $\Omega^+$ in naj
obstaja $\jump{D^2g}$ na $\sigma$.
Iz kompatibilnostnih pogojev (\ref{e:kpzdpo1}), (\ref{e:kpzdpo2}) in (\ref{e:kpzdpo3})
za $g^+$ in $g^-$ na $\sigma$ sledi
\begin{align}
	\jump{\frac{\partial^2 g}{\partial x_i\partial x_j}}=\;&
	a^{\alpha\varrho}a^{\beta\mu}\Big( \frac{\partial^2\jump{g}}{\partial\theta^{\alpha}\partial\theta^{\beta}}-
	\frac{\partial\jump{g}}{\partial\theta^{\gamma}}\cs{\alpha}{\gamma}{\beta}-
	\jump{\frac{\partial g}{\partial n}}b_{\alpha\beta}\Big)
	\frac{\partial\varphi_i}{\partial\theta^{\varrho}}\frac{\partial\varphi_j}{\partial\theta^{\mu}}+ \nonumber \\
	&+a^{\beta\mu}\Big(\frac{\partial}{\partial\theta^{\beta}}\jump{\frac{\partial g}{\partial n}}+
	a^{\gamma\varrho}b_{\varrho\beta}\frac{\partial\jump{g}}{\partial\theta^{\gamma}}\Big)
	\Big(n_i\frac{\partial\varphi_j}{\partial\theta^{\mu}}+n_j\frac{\partial\varphi_i}{\partial\theta^{\mu}}\Big)+
	\nonumber\\&+\jump{\frac{\partial^2 g}{\partial n^2}}n_in_j,\nonumber\\
	\jump{\frac{\partial^2 g}{\partial x_i\partial t}}=\;&
	a^{\alpha\beta}\frac{\partial\varphi_i}{\partial\theta^{\beta}}
	\frac{\partial}{\partial\theta^{\alpha}}
	\Big(\frac{\dtd \jump{g}}{\dtd t}-u_n\jump{\frac{\partial g}{\partial n}}\Big)+ \nonumber \\
	&+\Big(\frac{\dtd}{\dtd t}\jump{\frac{\partial g}{\partial n}}+
	a^{\alpha\beta}\frac{\partial\jump{g}}{\partial\theta^{\alpha}}
	\frac{\partial u_n}{\partial\theta^{\beta}}-u_n\jump{\frac{\partial^2 g}{\partial n^2}}\Big)n_i,\nonumber\\
	\jump{\frac{\partial^2 g}{\partial t^2}}=\;&
	\frac{\dtd}{\dtd t}\Big(\frac{\dtd \jump{g}}{\dtd t}-u_n\jump{\frac{\partial g}{\partial n}}\Big)- \nonumber \\
	&-u_n\Big(\frac{\dtd}{\dtd t}\jump{\frac{\partial g}{\partial n}}+
	a^{\alpha\beta}\frac{\partial\jump{g}}{\partial\theta^{\alpha}}
	\frac{\partial u_n}{\partial\theta^{\beta}}-u_n\jump{\frac{\partial^2 g}{\partial n^2}}\Big).\label{e:jpd2}
\end{align}
Če je $\sigma(t)$ singularna ploskev reda 2 za $g$, potem sta $\jump{g}$ in
$\jump{\partial g/\partial n}$ enaka 0 in enačbe (\ref{e:jpd2}) se reducirajo na
\begin{gather*}
	\jump{\frac{\partial^2 g}{\partial x_i\partial x_j}}=
	\jump{\frac{\partial^2 g}{\partial n^2}}n_in_j,\qquad
	\jump{\frac{\partial^2 g}{\partial x_i\partial t}}=
	-u_n\jump{\frac{\partial^2 g}{\partial n^2}}n_i \\ \mathrm{in}\quad
	\jump{\frac{\partial^2 g}{\partial t^2}}=u_n^2\jump{\frac{\partial^2 g}{\partial n^2}}.
\end{gather*}

Enačbi (\ref{e:jpd1}) sta kompatibilnostna pogoja za nezveznost prvih parcialnih odvodov,
enačbe (\ref{e:jpd2}) pa so kompatibilnostni pogoji za nezveznost drugih parcialnih odvodov
polja $g$ na $\sigma$. Enačbi (\ref{e:jpd1})$_1$ in (\ref{e:jpd2})$_1$ se imenujeta
\emph{geometrijski kompatibilnostni pogoj}, enačbe (\ref{e:jpd1})$_2$, (\ref{e:jpd2})$_{2,3}$
pa \emph{kinematični kompatibilnostni pogoj}.


\section{Valovi}


Imejmo Evklidski prostor $\E_R$, ki služi referenčni konfiguraciji $B$ nekega telesa,
ter za množice $\B_t$ vzamemo kar $B$ pri vsakem $t\in I$. Telo naj se giblje po Evklidskem
prostoru $\E$. V tem prostoru za množice $\B_t$ vzamemo $B_t=\chi_t(B)$. Naj bo $\sigma(t)$
iz prejšnjih razdelkov gibajoča se ploskev v prostoru $\E$. Za domeno $\Omega$ imamo tako
v $\E_R$ kar $\Omega_R=B\times I$, v $\E$ pa
\[ \Omega=\{(x,t)\;;\ x\in\chi_t(B),\ t\in I\}. \]

Glede na gibanje $\chi$ telesa je
\[
	\varPhi\colon\Theta\times I\to\E_R,\qquad\varPhi=\chi^{-1}\circ\varphi,
\]
regularna parametrizacija gibajoče se ploskve $\Sigma(t)$ po prostoru $\E_R$
s komponentno obliko
\[ X_K=\varPhi_K(\theta^1,\theta^2,t), \]
kjer so $X_K$ kartezijeve koordinate točke $X\in\Sigma(t)$. Pripadajoča
hiperploskev je
\[ \Sigma = \{ (X,t)\;;\ X\in\Sigma(t),\ t\in I \}. \]
Metrične koeficiente ploskve $\Sigma(t)$ bomo označili
z $A_{\alpha\beta}$ oz.~$A^{\alpha\beta}$, koeficiente druge fundamentalne forme
pa z $B_{\alpha\beta}$. Christoffelove simbole spremljajočega koordinatnega
sistema za $\Sigma(t)$ bomo označili z $\tics{\alpha}{\beta}{\gamma}$.

V implicitni obliki je $\Sigma(t)$ predstavljena z enačbo
\[
	\hat{f}(X,t)=f\big(\chi(X,t),t\big)=0.
\]
Pri tem je seveda $\hat{f}$ materialni opis skalarnega polja $f$.
Enako, kot v (\ref{e:ngrad}) in (\ref{e:nohit}), sta
normala $\vek{N}=N_K\vek{e}_K$ in normalna hitrost $U_N$ ploskve $\Sigma(t)$
\begin{equation} \label{e:ninun}
	\vek{N}=\frac{\Grad f}{\|\Grad f\|}\quad\textrm{in}
	\quad U_N=-\frac{1}{\|\Grad f\|}\frac{\partial \hat{f}}{\partial t}
	=-\frac{\dot{f}}{\|\Grad f\|}.
\end{equation}
\emph{Lokalna hitrost širjenja oz.~propagiranja ploskve} je
\[ U=u_n-\vek{v}\cdot\vek{n}. \]
\begin{trditev}
	Če je $\chi$ razreda $C^1$ na $B\times I$ (torej obstajata $\ten{F}$ in $\vek{v}$
	in sta zvezna {\color[rgb]{1,0,0} Kaj pa, če nista?}), potem velja
	\begin{equation}
		\vek{N}=\frac{\ten{F}^{T}\vek{n}}{\|\ten{F}^{T}\vek{n}\|}\quad\textrm{in}
		\quad U_N=\frac{U}{\|\ten{F}^{T}\vek{n}\|}.
	\end{equation} 
\end{trditev}
\proof
	Če v (\ref{e:ninun})$_1$ števec in imenovalec delimo z $\|\grad f\|$ ter
	upoštevamo relacijo (\ref{e:gz}) za $f$ in enačbo (\ref{e:ngrad}) za $\vek{n}$,
	dokažemo enačbo za $\vek{N}$. Če zapišemo relacijo (\ref{e:matodv}) za $f$ in
	jo delimo z $\|\grad f\|$, dobimo z upoštevanjem (\ref{e:ngrad}) in (\ref{e:nohit})
	\[
		\frac{\dot{f}}{\|\grad f\|}=-u_n+\vek{n}\cdot\vek{v}=-U.
	\]
	Če v dobljeni enačbi upoštevamo
	\[
		\|\Grad f\|=\|\ten{F}^{T}\grad f\|=\|\ten{F}^{T}\vek{n}\|\|\grad f\|
	\]
	in (\ref{e:ninun})$_2$, dobimo še enačbo za $U_N$.
\endproof

Skalarno polje $g\colon\Omega\to\R$ ima na $\Omega_R$ pripadajoči materialni opis
\[ G(X,t)=g(\chi(x,t),t). \]
Kompatibilnostni pogoji (\ref{e:jpd1}) in (\ref{e:jpd2}) za polje $G(X,t)$ na $\Sigma$ so
\begin{align}
	\jump{\frac{\partial G}{\partial X_K}}=\;&
	A^{\alpha\beta}\frac{\partial\varPhi_K}{\partial\theta^{\beta}}
	\frac{\partial \jump{G}}{\partial\theta^{\alpha}}+
	N_K\jump{\frac{\partial G}{\partial N}},\nonumber\\
	\jump{\frac{\partial G}{\partial t}}=\;&
	\frac{\gdtd \jump{G}}{\dtd t}-U_N\jump{\frac{\partial G}{\partial N}},\nonumber\\
	\jump{\frac{\partial^2 G}{\partial X_K\partial X_L}}=\;&
	A^{\alpha\varrho}A^{\beta\mu}\Big( \frac{\partial^2\jump{G}}{\partial\theta^{\alpha}\partial\theta^{\beta}}-
	\frac{\partial\jump{G}}{\partial\theta^{\Gamma}}\tics{\alpha}{\Gamma}{\beta}-
	\jump{\frac{\partial G}{\partial N}}B_{\alpha\beta}\Big)
	\frac{\partial\varPhi_K}{\partial\theta^{\varrho}}\frac{\partial\varPhi_L}{\partial\theta^{\mu}}+ \nonumber \\
	&+A^{\beta\mu}\Big(\frac{\partial}{\partial\theta^{\beta}}\jump{\frac{\partial G}{\partial N}}+
	A^{\Gamma\varrho}B_{\varrho\beta}\frac{\partial\jump{G}}{\partial\theta^{\Gamma}}\Big)
	\Big(N_K\frac{\partial\varPhi_L}{\partial\theta^{\mu}}+N_L\frac{\partial\varPhi_K}{\partial\theta^{\mu}}\Big)
	\nonumber\\&+\jump{\frac{\partial^2 G}{\partial N^2}}N_K N_L,\nonumber\\
	\jump{\frac{\partial^2 G}{\partial X_K\partial t}}=\;&
	A^{\alpha\beta}\frac{\partial\varPhi_K}{\partial\theta^{\beta}}
	\frac{\partial}{\partial\theta^{\alpha}}
	\Big(\frac{\gdtd \jump{G}}{\dtd t}-U_N\jump{\frac{\partial G}{\partial N}}\Big)+ \nonumber \\
	&+\Big(\frac{\gdtd}{\dtd t}\jump{\frac{\partial G}{\partial N}}+
	A^{\alpha\beta}\frac{\partial\jump{G}}{\partial\theta^{\alpha}}
	\frac{\partial U_N}{\partial\theta^{\beta}}-U_N\jump{\frac{\partial^2 G}{\partial N^2}}\Big)N_K,\nonumber\\
	\jump{\frac{\partial^2 G}{\partial t^2}}=\;&
	\frac{\gdtd}{\dtd t}\Big(\frac{\gdtd \jump{G}}{\dtd t}-U_N\jump{\frac{\partial G}{\partial N}}\Big)- \nonumber \\
	&-U_N\Big(\frac{\gdtd}{\dtd t}\jump{\frac{\partial G}{\partial N}}+
	A^{\alpha\beta}\frac{\partial\jump{G}}{\partial\theta^{\alpha}}
	\frac{\partial U_N}{\partial\theta^{\beta}}-U_N\jump{\frac{\partial^2 G}{\partial N^2}}\Big).
\end{align}
Pri tem je
\[
	\frac{\gdtd}{\dtd t}=\frac{\partial}{\partial t}+U_N\frac{\partial}{\partial N}
\]
in
\[
	\frac{\partial G}{\partial N}=(\Grad G)\cdot\vek{N}=
	(\ten{F}^{T}\grad g)\cdot\frac{\ten{F}^{T}\vek{n}}{\|\ten{F}^{T}\vek{n}\|}=
	(\grad g)\cdot\frac{\ten{F}\ten{F}^{T}\vek{n}}{\|\ten{F}^{T}\vek{n}\|}.
\]

Singularna ploskev se imenuje \emph{val}, če je $U_N$ različna od nič.