\chapter{Uvod} \pagenumbering{arabic}


V mehaniki kontinuuma se enačbe gibanja fizikalnega telesa navadno izpelje
na klasičen način preko Newtonovih in Eulerjevih zakonov, ki povezujejo
gibanje telesa s silami in momenti, ki delujejo na telo. Alternativni pristop
za izpeljavo enačb gibanja telesa je Hamiltonov princip, ki je po svoji
naravi variacijski pristop. Pri Hamiltonovem principu med vsemi možnimi
gibanji poiščemo pravo gibanje kot minimum Hamiltonovega energijskega funkcionala.

Kdor se je že srečal s klasično mehaniko, verjetno pozna princip virtualnega dela,
ki je tudi variacijski princip, vendar je namenjen statiki. Hamiltonov
princip lahko smatramo kot posplošitev principa virtualnega dela za primer dinamike.

William Rowan Hamilton (1805-1865) je bil irski fizik in matematik, ki se je
v svojih zgodnjih raziskovalnih letih ukvarjal z optiko. Njegova teorija žarkov
je bila variacijska, temeljila je na predpostavki, da svetlobni žarek potuje
od ene točke prostora do druge po poti, za katero porabi najmanj časa. Med
svojim delom na področju optike je začel razmišljati tudi o razvoju podobne
teorije za dinamične sisteme delcev. Leta 1835 je nato tudi objavil članek, v katerem
je predstavil svoj rezultat, ki je danes poznan pod imenom Hamiltonov princip.

Hamiltonovo delo na področju dinamike je bilo prepoznano in cenjeno, toda
z vidika uporabnosti precej zapostavljeno vse do začetka dvajsetega stoletja,
ko so takratni fiziki spoznali, da Hamiltonova formulacija ponuja najbolj
naravno razširitev za takrat razvijajočo se teorijo kvantne mehanike.
Ravno Hamiltonovo formalno povezavo med optiko in klasično mehaniko se ima
za predhodnika kvantne mehanike.

V klasični mehaniki je bil Hamiltonov princip sprva uveden za sisteme delcev,
kjer se obravnava dinamiko končnega števila točkastih masnih delcev.
Ta princip je bil kmalu za tem razširjen za kontinuum, vendar se ga
z redkimi izjemami sprva ni uporabljalo, saj se je dalo dobiti enake rezultate
s pomočjo drugih, bolj direktnih metod. Do izraza je prišel s posplošitvijo
klasičnih teorij v mehaniki fluidov ter v mehaniki deformabilnih teles,
kjer se je izkazal kot zelo primeren pri matematični obravnavi materialov z mikrostrukturami
in v teoriji mešanic.

Skozi dvajseto stoletje se je nasploh povečalo zanimanje za variacijske metode.
Uporabljajo se recimo še
v numerični analizi pri metodi končnih elementov, ter pri študiji stabilnosti
rešitev za probleme iz mehanike fluidov in mehanike deformabilnih teles.

Namen tega déla je predstaviti Hamiltonov princip v mehaniki kontinuuma
kot sredstvo za iskanje enačb, ki opisujejo gibanje zveznga medija.
Delo je omejeno na klasično teorijo fluidov in deformabilnih teles,
zato je primerno za bralce, ki se s Hamiltonovim principom v mehaniki
kontinuuma šele spoznavajo. Od bralca se pričakuje osnovno poznavanje
funkcionalne analize, variacijskega računa, tenzorske analize in
dobro poznavanje matematične analize ter linearne algebre.
Potrebni pojmi in rezultati iz teh področij so povzeti v uvodnem
poglavju, kjer je tudi predstavljena notacija in poimenovanja.
Potrebni pojmi in rezultati iz mehanike kontinuuma bodo predstavljeni v tretjem
poglavju, zato predhodno znanje iz mehanike kontinuuma ni potrebno,
je pa dobrodošlo.

Jedro naloge je četrto poglavje. V njem bomo problem iskanja enačb gibanja
postavili v okvir variacijske naloge. Definirali bomo prostor dopustnih gibanj in
prostor dopustnih variacij, nato pa bomo pravo gibanje iskali kot stacionarno
točko Hamiltonovega funkcionala. Nato bomo podali nekaj pomožnih
izrekov variacijskega računa, s pomočjo katerih bomo lahko prišli do
diferencialnih enačb za gibanje. Uporaba Hamiltonovega principa
bo prikazana na primeru elastičnega fluida in hiperelastičnega trdnega telesa.
Dodatne konstitucijske enačbe za hiperelastično trdno telo bodo izpeljane v dodatku.
