\chapter{Matematični opis materialnega telesa}


\section{Evklidski prostor}


V klasični mehaniki opisujemo dogodke v \emph{Newtonovem prostor-času}, kar je produkt
trirazsežnega Evklidskega prostora ter prostora realnih števil $\R$. Evklidski prostor nam služi
za opis položaja in geometrije objektov, prostor $\R$ pa predstavlja časovno os.

\begin{definicija} \label{d:ep}
	Množica točk $\E$ je \emph{Evklidski točkovni prostor} in trirazsežni Evklidski vektorski prostor $V$ je
	\emph{translacijski prostor} za $\E$, če poljubnemu paru točk $p,q\in\E$ pripada vektor iz $V$,
	ki ga zapišemo kot $q-p$, tako da velja:
	\begin{enumerate}
		\item Za vsak $p\in\E$ je $p-p=\vek{0}$, ničelni vektor.
		\item Za vsak $p\in\E$ in vsak $\vek{v}\in V$ obstaja natanko ena točka $q\in\E$, da je
		$\vek{v}=q-p$. Pišemo: $q=p+\vek{v}$.
		\item Za vse $p,q,r\in\E$ velja $(q-p)+(r-q)=(r-p)$.
	\end{enumerate}
\end{definicija}

Razdalja med točkama $p,q\in\E$ je definirana kot $d(p,q)=\|q-p\|$, kjer $\|.\|$ označuje normo na
vektorskem prostoru $V$, porojeno iz standardnega skalarnega produkta. $(\E,d)$ je metrični prostor.

V prostoru $\E$ si izberimo točko, jo označimo z $o$ in jo imenujmo \emph{izhodišče}.
V skladu z aksiomi iz definicije pripada vsaki točki $p\in\E$ glede na izhodišče $o$ \emph{krajevni vektor}
$\vek{r}(p)=p-o\in V$. Naj bo $\{\vek{e}_1,\vek{e}_2,\vek{e}_3\}$ neka ortonormirana baza prostora $V$,
ki je desnosučna, tj.~$\vek{e}_3=(\vek{e}_1\times\vek{e}_2)$. Krajevni vektor $\vek{r}(p)$
ima enoličen razvoj po bazi, $\vek{r}(p)=y_k\vek{e}_k$, koeficientom $y_k$ tega razvoja pa rečemo
\emph{kartezijeve koordinate}. Te so odvisne od izbire izhodišča in ortonormirane baze.

Včasih si želimo prostor $\E$ opremiti s kakšnim drugim koordinatnim sistemom, ki ga podamo
s koordinatno transformacijo kartezijevih koordinat
\begin{equation}\label{e:kt}
	x^j = \hat{x}^j(y_1,y_2,y_3) \ \Leftrightarrow \ y_k=\hat{y}_k(x^1,x^2,x^3), \quad j,k=1,2,3.
\end{equation}
Za koordinatno transformacijo zahtevamo, da je bijektivna in gladka preslikava iz
$D^{\mathrm{odp}}\subseteq\R^3$ v $U^{\mathrm{odp}}\subseteq\R^3$ s prav tako gladkim inverzom.

Imamo torej bijektivno korespondenco med naslednjimi objekti:
\begin{itemize}
	\item točka $p\in\E$,
	\item krajevni vektor $\vek{r}(p)=y_k\vek{e}_k\in V$,
	\item koordinate $(x^1,x^2,x^3)\in\R^3$.
\end{itemize}
Zato bomo točke iz prostora $\E$ v bodoče identificirali z njihovimi krajevnimi vektorji ali pa z njihovimi koordinatami.

Newtonov prostor čas $\mathcal{N}=\E\times\R$ lahko naredimo za vektorski prostor s skalarnim produktom
\[ (\vek{x},t_1)\cdot(\vek{y},t_2) = \vek{x}\cdot\vek{y} + t_1 t_2. \]
Iz skalarnega produkta dobimo tudi normo in metriko.


\section{Tenzorska polja}


V tem razdelku bomo navedli nekaj bistvenih pojmov in rezultatov iz tenzorske analize. Predpostavlja
se, da bralec že pozna osnove tenzorske analize. Razdelek služi bolj predstavitvi oznak, ki jih
bomo uporabljali v nadaljevanju.

Funkciji $f\colon\mathcal{D}\subseteq\E\to W$, kjer je $W$ neki normirani vektorski prostor, rečemo
\emph{tenzorsko polje}. V posebnem primeru, ko je $W=\R$, rečemo funkciji $f$ \emph{skalarno},
v primeru $W=V$ pa \emph{vektorsko polje}. Tenzorskim poljem bomo včasih kot argument namesto točke raje
podali njen krajevni vektor ali pa njene koordinate, ne da bi pri tem spremenili oznako za tenzorsko polje.

\begin{definicija}
	Naj bo $f:\mathcal{D}^{\mathrm{odp}}\subseteq\E\to W$ tenzorsko polje. Funkcija $f$ je \emph{odvedljiva}
	v točki $x\in\mathcal{D}$, če obstaja taka linearna preslikava $\nabla f(x):V\to W$, da za vsak $\vek{h}\in V$ velja
	\begin{equation*}
		f(x+\vek{h})=f(x)+\nabla f(x)[\vek{h}]+o(\vek{h}),
	\end{equation*}
	kjer je $o(\vek{h})\in W$ količina, za katero je
	\[ \lim_{\| h\|\to 0}\frac{\|o(\vek{h})\|}{\|\vek{h}\|}=0. \]
	Če $\nabla f(x)$ obstaja, ji rečemo \emph{gradient} ali pa \emph{krepki} oz.~\emph{Fréchetov odvod} funkcije $f$ v točki $x$.
\end{definicija}
Krepki odvod, če obstaja, je tudi tenzorsko polje, vrednosti pa zavzema v prostoru $\L(V,W)$.
Drugi odvod oz. odvod odvoda, če obstaja, je tudi tenzorsko polje z vrednostmi v prostoru $\L\big(V,\L(V,W)\big)$,
in tako dalje.

Krepke odvode računamo s pomočjo \emph{šibkega} oz.~\emph{smernega} (včasih tudi \emph{Gâteauxovega}) \emph{odvoda}:
\begin{equation*}
	\delta f(x)[\vek{h}]=\lim_{s\to 0}\frac{1}{s}\Big(f(x+s\vek{h})-f(x)\Big)=
	\at{\frac{d}{ds}f(x+s\vek{h})}{s=0}.
\end{equation*}
Znano je, da če obstaja krepki odvod,
potem obstaja tudi šibki odvod in sta enaka: $Df(x)[\vek{h}]=\delta f(x)[\vek{h}]$ za vsak $\vek{h}\in V$.
Za krepke in šibke odvode veljajo enaki izreki, kot veljajo za preslikave med normiranimi prostori, ki jih
poznamo iz matematične analize: izrek o posrednem odvajanju, izrek o odvajanju produkta\footnote{
Izrek o odvajanju produkta velja za katerikoli produkt, ki je bilinearna preslikava. Med njimi so
npr. skalarni in vektorski produkt vektorjev, produkt tenzorja s skalarjem, tenzorski produkt,
delovanje tenzorja na vektorju itd.},
izrek o totalnem odvodu itd.

Gradient skalarnega polja $f$ zavzema vrednosti iz prostora $\L(V,\R)$, kar so linearni funkcionali.
S pojmom \emph{gradient} in oznako $\nabla f$ se v tem primeru označuje polje vektorjev, ki pripadajo polju
linearnih funkcionalov po Riezsovem izreku o reprezentaciji, tako da velja
\[ \nabla f[\vek{h}]=\nabla f \cdot \vek{h} \quad \forall\, \vek{h}\in V. \]

Medtem ko gradient viša red tenzorskega polja, ga divergenca niža.
\emph{Divergenca vektorskega polja} $\vek{u}$ je skalarno polje
\begin{equation*}
	\div\vek{u}=\tr(\nabla\vek{u})
\end{equation*}
\emph{Divergenca tenzorskega polja} $S\colon\mathcal{D}\to\L(V)$, je vektorsko polje $\div S$ z lastnostjo,
da za vsako konstantno vektorsko polje $\vek{v}$ velja
\[ \vek{v}\cdot\div S = \div(S^T\vek{v}). \]

Če je polje $f$ časovno odvisno, tj. $f$ je preslikava
\[ f\colon\mathcal{D}\times I\to W, \]
kjer je $\mathcal{D}\subseteq\E$ odprta množica, $I=(t_1,t_2)\subseteq\R$ (časovni) interval, $W$ pa neki normirani prostor,
potem je definiran še \emph{časovni odvod}
\[ \frac{\partial}{\partial t}f(x,t) = \lim_{s\to 0}\frac{1}{s}\Big(f(x,t+s)-f(x,t)\Big), \]
kar je zopet časovno odvisno tenzorsko polje z vrednostmi iz prostora $W$. $n$-ti
časovni odvod polja $f$ označimo z $\partial^n f/\partial t^n$.


\section{Gibanje materialnega telesa}


Materialno telo je določeno z izbrano množico točk $\B\subseteq\E$. Tej določitvi rečemo
\emph{referenčna konfiguracija} materialnega telesa in služi zgolj označbi telesa; ni nujno,
da se telo dejansko kadarkoli nahaja v tem položaju. Oznaka $\B$ bo odslej vedno prestavljala
referenčno konfiguracijo telesa.

\begin{definicija}
	\emph{Gibanje} materialnega telesa v časovnem
	intervalu $I=(t_0,t_1)$ je zvezno, časovno odvisno vektorsko polje
	\begin{equation}\label{e:chi}
		\chi\colon\B\times I\to\E,\qquad \chi\colon (\vek{X},t)\mapsto\vek{x},
	\end{equation}
	za katerega je za vsak $t\in I$ preslikava
	\[ \chi_t\colon\B\to\B_t, \qquad \chi_t\colon\vek{X}\mapsto\chi(\vek{X},t)=\vek{x} \]
	bijektivna. Slika $\B_t$ se imenuje \emph{trenutna konfiguracija} materialnega
	telesa \emph{ob času} $t$.
\end{definicija}
Gibanje (\ref{e:chi}) ni bijektivna preslikava, zato nima inverza. Kljub temu na smiselen
način definiramo \emph{inverzno gibanje}
\[ \chi^{-1}(\vek{x},t)=\bottop{\chi}{t}{-1}(\vek{x}), \]
za katerega tudi zahtevamo, da je zvezno.

V predstavitvi gibanja (\ref{e:chi}) se krajevni vektor $\vek{X}$ imenuje \emph{materialni vektor},
točka $X$, ki jo $\vek{X}$ predstavlja, se imenuje \emph{materialna točka} in njene koordinate
$(X^1,X^2,X^3)$ se imenujejo \emph{materialne koordinate}. Krajevni vektor $\vek{x}$ se imenuje
\emph{prostorski vektor}, točka $x$, ki jo predstavlja, se imenuje \emph{prostorska točka},
njene koordinate $(x^1,x^2,x^3)$ pa se imenujejo \emph{prostorske koordinate}.

\emph{Hitrost} $\vek{v}$ in \emph{pospešek} $\vek{a}$ gibanja $\chi$ v časovnem intervalu $I$ sta vektorski polji
\begin{align}
	\vek{v}\colon\B\times I\to V \qquad & \vek{v} = \frac{\partial}{\partial t}\chi(\vek{X},t), \label{e:v} \\
	\vek{a}\colon\B\times I\to V \qquad & \vek{a} = \frac{\partial^2}{\partial t^2}\chi(\vek{X},t). \label{e:a}
\end{align}
Pri tem mora biti $\chi$ vsaj dvakrat odvedljivo po času. \emph{Deformacijski gradient} gibanja $\chi$ je
tenzorsko polje
\[ F=\nabla\chi(\vek{X},t). \]
Za determinanto $j=\det F$ se predpostavi, da je pozitivna.


Materialno telo spremljajo različne fizikalne lastnosti, katerih vrednosti predstavimo z elementi nekega
normiranega prostora $W$. Vrednosti fizikalnih količin
se med gibanjem v časovnem intervalu $I\subseteq\R$ spreminjajo, opišemo pa jih lahko na
dva različna načina.
\begin{definicija}
	\emph{Materialni} ali \emph{Lagrangejev opis} je predstavitev fizikalne količine, ki spremlja materialno telo
	med gibanjem $\chi$, s tenzorskim poljem
	\[ f\colon\B\times I \to W, \qquad f\colon (\vek{X},t)\mapsto w. \]
	\emph{Prostorski} ali \emph{Eulerjev opis} je pri vsakem $t\in I$ predstavitev taiste fizikalne količine s tenzorskim poljem
	\[ \bar{f}(\cdot,t)\colon\B_t\to W, \qquad \bar{f}\colon (\vek{x},t)\mapsto w. \]
\end{definicija}
Med materialnim in prostorskim opisom veljata zvezi
\begin{equation}\label{e:mpo-zveza}
	f(\vek{X},t) = \bar{f}\big(\chi(\vek{X},t),t\big),\qquad \bar{f}(\vek{x},t) = f\big(\chi^{-1}(\vek{x},t),t\big).
\end{equation}
Kadar bomo uporabljali prostorski opis in bo to jasno iz konteksta, bomo črtico v oznaki za polje izpustili.
Do dvomov lahko pride, kadar ne pišemo argumentov polja, še posebej, če so vključeni odvodi. Temu dvomu se izognemo tako,
da uporabljamo različno notacijo za odvode. V materialnem opisu pišemo gradient in divergenco kot
\[ \nabla_{\vek{X}}f=\nabla f(\vek{X},t),\qquad \Div f = \div f(\vek{X},t), \]
v prostorskem opisu pa
\[ \nabla_{\vek{x}}f=\nabla \bar{f}(\vek{x},t),\qquad \div f = \div \bar{f}(\vek{x},t). \]
Če je $\phi$ skalarno, $\vek{u}$ pa vektorsko polje, potem je zveza med gradientoma
\[
	\nabla_{\!\!\vek{X}}\phi=F^{\,T}\,\nabla_{\!\!\vek{x}}\phi,\qquad \nabla_{\vek{X}}\vek{u}=\nabla_{\vek{x}}\vek{u}\,F.
\]
Res, če je $\vek{w}$ poljubno vektorsko polje, potem je
\begin{align*}
	&\nabla_{\vek{X}}\phi\cdot\vek{w}=\nabla_{\vek{x}}\phi\cdot\nabla_{\vek{X}}\chi\,[\vek{w}]=
	\nabla_{\vek{x}}\phi\cdot F\vek{w}=F^{\,T}\,\nabla_{\vek{x}}\phi\cdot\vek{w}, \\
	&\nabla_{\vek{X}}\vek{u}\,[\vek{w}]=\nabla_{\vek{x}}\vek{u}\,\big[\nabla_{\vek{X}}\chi\,[\vek{w}]\big]
	=\nabla_{\vek{x}}\vek{u}\,F\,[\vek{w}].
\end{align*}

\begin{definicija}
	Totalni odvod tenzorskega polja $f$ po času
	\[
		\dot{f}=\frac{d}{dt}f
	\]
	imenujemo \emph{materialni odvod}.
\end{definicija}
Poimenovanje naj nas ne zavede -- gre za časovni odvod polja $f$, ko to sledi póti materialne točke.
V materialnem opisu je
\[ \dot{f}=\frac{d}{dt}f(\vek{X},t)=\frac{\partial}{\partial t}f(\vek{X},t). \]
V prostorskem opisu določena točka prostora ob vsakem času v splošnem predstavlja
drugo materialno točko. Materialni odvod, ki je totalni odvod po času, je zato v prostorskem opisu
\[
	\dot{f}=\frac{d}{dt}\bar{f}(\vek{x},t) = \frac{\partial}{\partial t}\bar{f}(\vek{x},t) +
	\nabla\bar{f}(\vek{x},t)\,[\bar{\vek{v}}(\vek{x},t)].
\]
Pri izpeljavi te enakosti se zaporedoma uporabi pravilo za totalni odvod, pravilo za posredno odvajanje,
definicijo hitrosti (\ref{e:v}) ter zvezo (\ref{e:mpo-zveza}).
\begin{primer}
	Pospešek je materialni odvod hitrosti. V prostorskem opisu je
	\[ \bar{\vek{a}}=\frac{\partial}{\partial t}\bar{\vek{v}}+L\bar{\vek{v}}, \]
	kjer je $L=\nabla_{\vek{x}}\vek{v}$ t.~i.~\emph{hitrostni gradient}.
\end{primer}

