\section{Krivočrtne koordinate} \label{s:koordinate}


\subsection{Koordinatni sistem}


\begin{definicija}
	Naj bo $U\subseteq\E$ odprta množica. \emph{Koordinatni sistem} na $U$ je bijektivna
	preslikava $\psi\colon U\to V$ razreda $C^r$ (običajno je $r$ vsaj 2), kjer je $V$ odprta
	množica v $\R^3$, inverz $\psi^{-1}$ pa je prav tako razreda $C^r$.
\end{definicija}

Naj bo $x\in U$ in
\[ \psi\colon x\mapsto (x^1,x^2,x^3)=\psi(x). \]
$(x^1,x^2,x^3)$ so \emph{(krivočrtne) koordinate} točke $x$.
Za $i\in\{1,2,3\}$ se funkcija
\[ \psi^i\colon U\to\R,\qquad \psi^i(x)=x^i \]
imenuje \emph{$i$-ta koordinatna funkcija} koordinatnega sistema $\psi$.
Po dogovoru tudi oznaki $(x^i)$ rečemo koordinatni sistem na $U$.

Naj bodo $(x^1,x^2,x^3)$ koordinate točke $x\in U$. Z $\vek{j}_i$ označimo
$i$-ti vektor iz standardne baze za $\R^3$,
\[ \vek{j}_1=(1,0,0),\quad\vek{j}_2=(0,1,0),\quad\vek{j}_3=(0,0,1). \]
Za neki $\varepsilon>0$ je slika preslikave
\begin{equation} \label{e:kokri}
	\gamma_i\colon(-\varepsilon,\varepsilon)\to U,\qquad
	\gamma_i(t)=\psi^{-1}\big((x^1,x^2,x^3)+t\vek{j}_i\big),\qquad i=1,2,3,
\end{equation}
\emph{$i$-ta koordinatna krivulja} v $U$, ki gre pri $t=0$ skozi točko $x$.
V točki $x$ lahko definiramo \emph{tangentne vektorje}
\begin{equation} \label{e:tanvek}
	\vek{g}_i(x)=\at{\frac{d}{dt}\gamma_i(t)}{t=0}=\dot{\gamma}_i(0)
	=\at{\frac{\partial\psi^{-1}}{\partial x^i}}{(x^1,x^2,x^3)}
	, \qquad i=1,2,3
\end{equation}
in \emph{gradientne vektorje}
\begin{equation} \label{e:gradvek}
	\vek{g}^i(x)=\nabla\psi^i(x),\qquad i=1,2,3.
\end{equation}

\begin{trditev}
	Množica $\{\vek{g}_i(x)\}_i$ tvori bazo za translacijski prostor $\V$.
\end{trditev}

\proof
	Naj bo $\vek{u}\in\V$ poljuben in definirajmo krivuljo skozi $x$ s predpisom
	\[ \gamma\colon(-\varepsilon,\varepsilon)\to U,\qquad \gamma(t)=x+t\vek{u}. \]
	Velja
	\[
		\vek{u}=\at{\frac{d}{dt}\gamma(t)}{t=0}\quad\textrm{in}\quad
		\gamma(t)=\psi^{-1}\big(\psi^1(x+t\vek{u}),\psi^2(x+t\vek{u}),\psi^3(x+t\vek{u})\big),
	\]
	zato je
	\[
		\vek{u}=
		\at{\frac{\partial\psi^{-1}}{\partial x^i}}{(x^1,x^2,x^3)}
		\at{\frac{d}{dt}\psi^i(x+t\vek{u})}{t=0}=
		\at{\frac{d}{dt}\psi^i(x+t\vek{u})}{t=0}\vek{g}_i(x).
	\]
	Z drugimi besedami, $\{\vek{g}_i(x)\}_{i}$ razpenja prostor $\V$.
\endproof

Tangentni in gradientni vektorji so definirani v vsaki točki množice $U$, torej
gre v bistvu za vektorska polja. V vsaki točki $x\in U$ se baza $\{\vek{g}_i(x)\}_i$
imenuje \emph{naravna baza} koordinatnega sistema $(x^i)$ za prostor $\V$.
Množica $\{\vek{g}^i(x)\}_i$ pa je prav tako baza za $\V$, ki je dualna bazi $\{\vek{g}_i(x)\}_i$.
Res, iz zveze
\[ x^i=\psi^i(\psi^{-1}(x^1,x^2,x^3)) \]
dobimo po verižnem pravilu
\begin{equation} \label{e:gigj}
	\topbot{\delta}{i}{j}=\frac{\partial x^i}{\partial x^j}=
	(\nabla\psi^i)\cdot\frac{\partial\psi^{-1}}{\partial x^j}=\vek{g}^i\cdot\vek{g}_j.
\end{equation}
Skalarna polja
\[ g_{ij}=\vek{g}_i\cdot\vek{g}_j\qquad\textrm{in}\qquad g^{ij}=\vek{g}^i\cdot\vek{g}^j \]
imenujemo \emph{koeficienti metričnega tenzorja}. 
Iz zadnjih treh zvez hitro vidimo, da velja
\begin{equation} \label{e:relzag}
	g^{ik}g_{kj}=\topbot{\delta}{i}{j},\qquad \vek{g}^i=g^{ik}\vek{g}_k,\qquad \vek{g}_i=g_{ik}\vek{g}^k.
\end{equation}


\subsection{Koordinatna transformacija}


Naj bosta $\psi$ in $\bar{\psi}$ oz.~$(x^i)$ in $(\bar{x}^i)$ koordinatna sistema za
$U\subseteq\E$ z naravnima bazama $\{\vek{g}_i(x)\}_i$ oz.~$\{\bar{\vek{g}}_i(x)\}_i$.
Koordinatna transformacija je podana s preslikavami
\[
	x^i=x^i(\bar{x}^1,\bar{x}^2,\bar{x}^3)\quad\Longleftrightarrow\quad
	\bar{x}^k=\bar{x}^k(x^1,x^2,x^3).
\]
Za vektorje iz naravne baze velja
\begin{equation} \label{e:gtog}
	\vek{g}^i=\frac{\partial x^i}{\partial\bar{x}^k}\bar{\vek{g}}^k,\qquad
	\vek{g}_i=\frac{\partial\bar{x}^k}{\partial x^i}\bar{\vek{g}}_k.
\end{equation}
Prvo enakost dobimo kot gradient funkcije $x^i$ iz koordinatne transformacije,
drugo enakost pa dobimo s parcialnim odvajanjem enakosti
\[ 
	\psi^{-1}(x^1,x^2,x^3)=\bar{\psi}^{-1}(\bar{x}^1(x^1,x^2,x^3),
	\bar{x}^2(x^1,x^2,x^3),\bar{x}^3(x^1,x^2,x^3)).
\]

%Naj bo $(x^1,x^2,x^3)$ in $(\bar{x}^1,\bar{x}^2,\bar{x}^3)$ dvoje koordinat točke $x\in U\subseteq\E$.
%Če je $f\colon U\to\R$ skalarno polje, potem bomo brez posebnih oznak pisali
%\[ f(x)=f(x^1,x^2,x^3)=f(\bar{x}^1,\bar{x}^2,\bar{x}^3). \]
Če je $\vek{u}\colon U\to\V$ vektorsko polje, potem ga lahko zapišemo v komponentni
obliki glede na naravno bazo prostora $\V$:
\begin{align}
	\vek{u}&=u_i\,\vek{g}^i=u^i\,\vek{g}_i \nonumber \\
	&=\bar{u}_i\,\bar{\vek{g}}^i=\bar{u}^i\,\bar{\vek{g}}_i, \label{e:vekkomp}
\end{align}
Pri tem so komponente $u_i,u^i,\dots$ skalarna polja.
Prav tako lahko tenzorsko polje $\ten{S}\colon U\to\L(\V)$ zapišemo v
komponentni obliki glede na bazo prostora $\L(\V)$:
\begin{align}
	\ten{S}&=S_{ij}\,\vek{g}^i\otimes\vek{g}^j=\topbot{S}{i}{j}\,\vek{g}_i\otimes\vek{g}^j \nonumber \\
	&=\bar{S}_{ij}\,\bar{\vek{g}}^i\otimes\bar{\vek{g}}^j=
	\topbot{\bar{S}}{i}{j}\,\bar{\vek{g}}_i\otimes\bar{\vek{g}}^j \label{e:tenkomp}
\end{align}
Tudi tu so $S_{ij},\topbot{S}{i}{j},\dots$ skalarna polja.
Posamezne komponente vektorskih in tenzoskih polj dobimo s pomočjo zveze (\ref{e:gigj})
iz (\ref{e:vekkomp}) in (\ref{e:tenkomp}):
\begin{gather}
	u_i=\vek{g}_i\cdot\vek{u},\qquad u^i=\vek{g}^i\cdot\vek{u}, \label{e:kompvek} \\
	S_{ij}=\vek{g}_i\cdot\ten{S}\vek{g}_j,\qquad\topbot{S}{i}{j}=\vek{g}^i\cdot\ten{S}\vek{g}_j.\label{e:kompten}
\end{gather}
S pomočjo (\ref{e:gtog}) dobimo naslednje zveze med posameznimi komponentami:
\begin{gather*}
	\bar{u}_i=\frac{\partial x^k}{\partial\bar{x}^i}u_k, \qquad
	\bar{u}^i=\frac{\partial\bar{x}^i}{\partial x_k}u_k, \\
	\bar{S}_{ij}=\frac{\partial x^k}{\partial\bar{x}^i}\frac{\partial x^l}{\partial\bar{x}^j}S_{kl}, \qquad
	\topbot{\bar{S}}{i}{j}=\frac{\partial\bar{x}^i}{\partial x^k}\frac{\partial x^l}{\partial\bar{x}^j}\topbot{S}{k}{l}.
\end{gather*}

Če je $\vek{u}=u_k\vek{g}^k$ poljubno vektorsko polje, je
\begin{equation} \label{e:iten}
	(\vek{g}^i\otimes\vek{g}_i)\vek{u}=(\vek{g}^i\otimes\vek{g}_i)u_k\vek{g}^k=
	u_k\bottop{\delta}{i}{k}\vek{g}^i=u_k\vek{g}^k=\vek{u},
\end{equation}
torej je $\vek{g}^i\otimes\vek{g}_i=\ten{1}$, identična linearna preslikava, in to
v vsaki točki množice $U$. Seveda velja tudi $\vek{g}_i\otimes\vek{g}^i=\ten{1}$.


\subsection{Gradient in kovariantni odvod}


Do konca tega razdelka naj velja, da so vsa obravnavan tenzorska polja razreda
$C^{1}$ ali $C^{2}$, kar bo razvidno samo po sebi.

Naj bo $(x^i)$ koordinatni sistem na $U\subseteq\E$ z naravnima bazama $\{\vek{g}_i\}_i$ in $\{\vek{g}^i\}_i$.
Gradiente tenzorskih polj lahko zapišemo v komponentni obliki glede na naravno bazo.

Naj bo $\vek{f}\colon U\to\W$ tenzorsko polje. Če je $\gamma_i$ $i$-ta
koordinatna krivulja, ki gre pri $t=0$ skozi točko $x$, potem iz (\ref{e:tanvek}) dobimo
\begin{equation*}
	\at{\frac{d}{dt}\vek{f}(\gamma_i(t))}{t=0}=
	\nabla \vek{f}(x)(\dot{\gamma}_i(0))=\nabla \vek{f}(x)(\vek{g}_i(x)),
\end{equation*}
po drugi strani pa imamo po (\ref{e:kokri})
\begin{equation*}
	\at{\frac{d}{dt}\vek{f}(\gamma_i(t))}{t=0}=
	\at{\frac{d}{dt}\vek{f}\big(\psi^{-1}((x^1,x^2,x^3)+t\vek{j}_i)\big)}{t=0}=
	\at{\frac{\partial(\vek{f}\circ\psi^{-1})}{\partial x^i}}{(x^1,x^2,x^3)}.
\end{equation*}
Po našem dogovoru pišemo namesto $\vek{f}\circ\psi^{-1}$ kar $\vek{f}$. Če izenačimo oba rezultata, dobimo
\begin{equation} \label{e:sotf}
	\frac{\partial\vek{f}}{\partial x^i}(x^1,x^2,x^3)=\nabla \vek{f}(x)(\vek{g}_i(x)).
\end{equation}
V primeru, ko je $f$ skalarno polje, iz enačbe (\ref{e:kompvek}) in iz pravkar izpeljane enačbe
dobimo $(\nabla f)_i=(\nabla f)\cdot\vek{g}_i=\partial f/\partial x^i$, torej je
\begin{equation} \label{e:gradskal}
	\nabla f = \frac{\partial f}{\partial x^i}\,\vek{g}^i.
\end{equation}

Preden nadaljujemo z gradienti vektorskih polj, vpeljimo najprej standardne oznake
za gradiente vektorskih polj naravne baze:
\begin{equation} \label{e:cs} %former label: e:gradg
	\nabla\vek{g}_i=\ten{\Gamma}_i=\cs{i}{j}{k}\,\vek{g}_j\otimes\vek{g}^k,\qquad
	\nabla\vek{g}^i=\ten{\Gamma}^i=\ks{i}{jk}\,\vek{g}^j\otimes\vek{g}^k.
\end{equation}
Tu sta $\ten{\Gamma}_i,\ten{\Gamma}^i\colon U\to\L(\V)$ tenzorski polji drugega reda,
komponente $\cs{i}{j}{k}$ in $\ks{i}{jk}$ pa se imenujejo \emph{Christoffelovi simboli}
in ne gre za komponente kakega tenzorja tretjega reda. Če v enačbi (\ref{e:sotf})
za $\vek{f}$ vstavimo vektorsko polje $\vek{g}_i$ oz.~$\vek{g}^i$ in upoštevamo
(\ref{e:kompten}) in (\ref{e:cs}), dobimo
\begin{equation} \label{e:csexplicit}
	\cs{i}{j}{k}=\vek{g}^j\cdot\frac{\partial\vek{g}_i}{\partial x^k}
	=-\vek{g}_i\cdot\frac{\partial\vek{g}^j}{\partial x^k},\qquad
	\ks{i}{jk}=\vek{g}_j\cdot\frac{\partial\vek{g}_i}{\partial x^k}.
\end{equation}
Pri tem je drugi izraz za $\cs{i}{j}{k}$ dobljen iz prvega s parcialnim odvajanjem
enakosti $\vek{g}^i\cdot\vek{g}_j=\topbot{\delta}{i}{j}$ po spremenljivki $x^k$.

Če po pravilu (\ref{t:divprop})$_2$ izračunamo gradient izraza $(\vek{g}^i\cdot\vek{g}_j)$,
ki je $\vek{0}$, in upoštevamo, da velja\footnote{
$(\vek{u}\otimes\vek{v})\vek{a}\cdot\vek{b}=(\vek{v}\cdot\vek{a})(\vek{u}\cdot\vek{b})=
\vek{a}\cdot(\vek{v}\otimes\vek{u})\vek{b}$}
$(\vek{u}\otimes\vek{v})^{T}=\vek{v}\otimes\vek{u}$
za poljubna vektorja $\vek{u}$ in $\vek{v}$ ter da je transponiranje linearna operacija,
potem dobimo zvezo
\begin{equation} \label{e:lcs1}
	\cs{j}{i}{k}=-\ks{i}{jk}.
\end{equation}
Nadalje, ker je $\ten{\Gamma}^i=\nabla(\nabla\psi^i)$ in je drugi gradient vedno
simetrični tenzor\footnote{$(\nabla(\nabla\psi^i))^T=\nabla(\nabla\psi^i)$ \cite[str.~271]{liu}},
veljata še naslednji zvezi:
\[ \ks{i}{jk}=\ks{i}{kj},\qquad\cs{j}{i}{k}=\cs{k}{i}{j}. \]
Ker je zaradi teh zvez možno prehajati iz ene vrste simbolov v drugo vrsto,
so v uporabi zgolj simboli $\cs{i}{j}{k}$, imenovani Christoffelovi simboli druge vrste.

Naj bo sedaj $\vek{u}\colon U\to\V$ vektorsko polje, v komponentni obliki
zapisano kot
\[ \vek{u}=u^j\vek{g}_j=u_k\vek{g}^k. \]
Njegov gradient zavzema vrednosti v prostoru $\L(\V)$, zato ga lahko zapišemo v komponentni obliki
\begin{equation} \label{e:gradukomp}
	\nabla\vek{u}=\topbot{u}{j}{,k}\,\vek{g}_j\otimes\vek{g}^k.
\end{equation}
Poiščimo izraz za komponente $\topbot{u}{j}{,k}$. Z upoštevanjem
(\ref{e:kompten}) in (\ref{e:sotf}) dobimo
\begin{align}
	\topbot{u}{j}{,k}&=\vek{g}^j\cdot\frac{\partial\vek{u}}{\partial x^k}=
	\vek{g}^j\cdot\frac{\partial(u^i\vek{g}_i)}{\partial x^k} \nonumber \\
	&=\vek{g}^j\cdot\Big(\frac{\partial u^i}{\partial x^k}\vek{g}_i+
	u^i\frac{\partial\vek{g}_i}{\partial x^k}\Big) \nonumber \\
	&=\frac{\partial u^j}{\partial x^k}+u^i\cs{i}{j}{k}. \label{e:kovod1}
\end{align}
Pri tem smo na zadnjem koraku upoštevali relacijo (\ref{e:csexplicit}). Dobljena enakost (\ref{e:kovod1})
za $\topbot{u}{j}{,k}$ je t.~i.~\emph{kovariantni odvod} komponentne $u^j$ po spremenljivki $x^k$.

Če zapišemo
\begin{equation} \label{e:nimena1}
	\nabla\vek{u}=u_{j,k}\,\vek{g}^j\otimes\vek{g}^k,
\end{equation}
in ponovimo prejšnji postopek, kjer dodatno uporabimo relacijo (\ref{e:lcs1}), dobimo
\begin{equation} \label{e:kov2}
	u_{j,k}=\frac{\partial u_j}{\partial x^k}-u_i\cs{j}{i}{k}.
\end{equation}
\begin{primer}
	Če v izrazu $\vek{u}=u_j\vek{g}^j$ vstavimo za $u_j=\partial f/\partial x^j$,
	kar so komponente od $\nabla f$ v izrazu (\ref{e:gradskal}), ter vstavimo v
	(\ref{e:kov2}), rezultat pa nato v (\ref{e:nimena1}), dobimo
	\begin{equation} \label{e:dvagrad}
		%\Big(\frac{\partial f}{\partial x^j}\Big)_{,k}
		\nabla(\nabla f)=\Big(\frac{\partial^2 f}{\partial x^j\partial x^k}-
		\frac{\partial f}{\partial x^i}\cs{j}{i}{k}\Big)\vek{g}^j\otimes\vek{g}^k.
	\end{equation}
\end{primer}

Naj bo $\ten{S}\colon U\to\L(\V)$. Potem je $\nabla\ten{S}\colon U\to\L(\V,\L(\V))$
in ga v komponentni obliki lahko razpišemo glede na bazo prostora $\L(\V,\L(\V))$ kot
\[ \nabla\ten{S}=\topbot{S}{ij}{,k}\vek{g}_i\otimes\vek{g}_j\otimes\vek{g}^k. \]
Iz (\ref{e:sotf}) dobimo
\[
	\frac{\partial\ten{S}}{\partial x^k}=(\nabla\ten{S})(\vek{g}_k)=
	(\topbot{S}{ij}{,l}\vek{g}_i\otimes\vek{g}_j\otimes\vek{g}^l)\vek{g}_k=
	\topbot{S}{ij}{,k}\vek{g}_i\otimes\vek{g}_j.
\]
Če na dobljeni enakosti uporabimo (\ref{e:kompten}), dobimo
\begin{align*}
	\topbot{S}{ij}{,k}&=\vek{g}^i\cdot\frac{\partial\ten{S}}{\partial x^k}\vek{g}^j=
	\vek{g}^i\cdot\frac{\partial(S^{lr}\vek{g}_l\otimes\vek{g}_r)}{\partial x^k}\vek{g}^j \\
	&=\vek{g}^i\cdot\Big(
	\frac{\partial S^{lr}}{\partial x^k}\vek{g}_l\otimes\vek{g}_r+
	S^{lr}\frac{\partial\vek{g}_l}{\partial x^k}\otimes\vek{g}_r+
	S^{lr}\vek{g}_l\otimes\frac{\partial\vek{g}_r}{\partial x^k}
	\Big)\vek{g}^j \\
	&=\frac{\partial S^{ij}}{\partial x^k}+S^{lj}\cs{l}{i}{k}+S^{ir}\cs{r}{j}{k}.
\end{align*}
Preostale komponente za $\nabla\ten{S}$ dobimo na enak način.


\subsection{Divergenca}


Sled tenzorja $\ten{S}\in\L(\V)$ s komponentno obliko (\ref{e:tenkomp})
je\footnote{$\tr\ten{S}=\topbot{S}{i}{j}\tr(\vek{g}_i\otimes\vek{g}^j)=
\topbot{S}{i}{j}\vek{g}_i\cdot\vek{g}^j=\topbot{S}{i}{i}$ \cite[str.~249]{liu}}
\[ \tr\,\ten{S}=\topbot{S}{i}{i}=g^{ij}S_{ij}. \]
Če je $\vek{u}$ vektorsko polje, potem iz (\ref{e:gradukomp}) in (\ref{e:nimena1}) dobimo
\begin{equation} \label{e:divu}
	\div\vek{u} = \tr(\nabla\vek{u}) = \topbot{u}{i}{,i} = g^{ij}u_{i,j}.
\end{equation}

Poiščimo še izraz za divergenco tenzorskega polja $\ten{S}\colon U\to\L(\V)$.
Iz (\ref{e:divu}) dobimo za poljubno vektorsko polje $\vek{u}=u_k\vek{g}^k$
\[
	\div(\ten{S}^{T}\vek{u})=\div(S^{ij}\vek{g}_j\otimes\vek{g}_i u_k\vek{g}^k)
	=\div(S^{ij}u_i\vek{g}_j)=(S^{ij}u_i)_{,\,j}.
\]
Bralec se lahko sam prepriča, da tudi za kovariantni odvod produkta velja
podobno pravilo, kot ga poznamo za običaji odvod, zato imamo
\[
	(S^{ij}u_i)_{,\,j}=\topbot{S}{ij}{\,,j}u_i+S^{ij}u_{i,\,j}=
	\topbot{S}{ij}{,\,j}\vek{g}_i\cdot\vek{u}+\tr(\ten{S}^{T}\nabla\vek{u})
\]
Dobili smo enakost
\begin{equation} \label{e:divStu}
	\div(\ten{S}^{T}\vek{u})=\topbot{S}{ij}{,\,j}\vek{g}_i\cdot\vek{u}+\tr(\ten{S}^{T}\nabla\vek{u}).
\end{equation}
Če je $\vek{u}$ konstantno vektorsko polje, potem je $\nabla\vek{u}=\ten{0}$ in v
enačbi (\ref{e:divStu}) s pomočjo definicije \ref{def:div} prepoznamo izraz za $\div\ten{S}$, ki je
\[
	\div\ten{S}=\topbot{S}{ij}{,\,j}\vek{g}_i.
\]
Če dobljeni izraz vstavimo nazaj v enačbo (\ref{e:divStu}) in zamenjamo $\ten{S}$ z
$\ten{S}^{T}$, dobimo naslednjo trditev.
\begin{trditev} \label{e:divSu}
	Za poljubno tenzorsko polje $\ten{S}\in C^1(U,\L(\V))$ in poljubno
	vektorsko polje $\vek{u}\in C^1(U,\V)$ velja
	\[ \div(\ten{S}\vek{u})=\vek{u}\cdot\div\ten{S}^{T}+\tr(\ten{S}\nabla\vek{u}). \]
\end{trditev}



\section{Kinematika kontinuuma} \label{chp:kinkon}


\subsection{Konfiguracije in gibanje telesa}


Naj bosta $\E_R$ in $\E$ Evklidska točkovna prostora, ki imata sicer enake lastnosti,
vendar ju bomo kljub temu razlikovali. Prostor $\E_R$ bo služil določitvi materialnega
telesa: z množico točk izbranega regularnega območja $B\subset\E_R$ je določeno \emph{materialno
telo} ali \emph{kontinuum}. Množici $B$ bomo rekli \emph{referenčna} ali \emph{sklicna konfiguracija}
materialnega telesa. Privzeli bomo, da je $B$ zaprta.
Prostor $\E$ pa bo služil opisu dejanskega položaja materialnega telesa v prostoru.

Točke prostora $\E_R$, njihove koordinate ter njihove krajevne vektorje bomo označevali
z velikimi simboli: $X$, $(X_1,X_2,X_3)$, $\vek{X}$, in jih bomo imenovali
\emph{materialne točke oz.~koordinate oz.~vektorji}.
Točke prostora $\E$, njihove koordinate ter njihove krajevne vektorje bomo,
kot doslej, označevali z malimi simboli: $x$, $(x_1,x_2,x_3)$, $\vek{x}$, in jih bomo imenovali
\emph{prostorske točke oz.~koordinate oz.~vektorji}.

\begin{definicija}
	\emph{Konfiguracija (materialnega) telesa} je zvezna in injektivna preslikava
	\[ \kappa\colon B\to\E,\qquad \kappa\colon X\mapsto x=\kappa(X). \]
	Tudi sliki preslikave $\kappa$, tj.~$\kappa(B)$, rečemo \emph{konfiguracija telesa}.
\end{definicija}
S konfiguracijo torej določamo položaj telesa v prostoru $\E$. Ker je $B$ kompaktna
množica, je tudi inverz konfiguracije zvezna preslikava.

Naj bo $I=[t_1,t_2]\subset\R$ časovni interval, torej je $B\times I\subset\N$.
\begin{definicija}
	\emph{Gibanje (materialnega) telesa} je zvezna preslikava
	\begin{equation}\label{e:chi}
		\chi\colon B\times I\to\E,\qquad \chi\colon(X,t)\mapsto x=\chi(X,t)
	\end{equation}
	z lastnostjo, da je za vsak $t\in I$ preslikava
	\[ \chi_t\colon B\to\E, \qquad \chi_t(X):=\chi(X,t) \]
	konfiguracija. Preslikava $\chi_t$ in množica $ B_t:=\chi_t(B)$ se imenujeta
	\emph{trenutna konfiguracija} telesa \emph{ob času} $t$.
\end{definicija}

Z oznako $\Omega$ bomo, dokler ne bo povedano drugače, označili množico
\[ \Omega=\{(x,t)\;;\ x\in\chi_t(B),\ t\in I\}\subset\N. \]
Gibanje (\ref{e:chi}) ni injektivna preslikava, zato nima inverza. Kljub temu na smiselen
način definiramo \emph{inverzno gibanje}
\[
	\chi^{-1}\colon\Omega\to B,\qquad
	\chi^{-1}(x,t):=\bottop{\chi}{t}{-1}(x),
\]
ki je tudi zvezno. S $\chi^{-1}$ smo torej označili inverz preslikave
$(X,t)\mapsto(x,t)=(\chi(X,t),t)$. Dodatno bomo do nadaljnjega predpostavili, da sta gibanje
in inverzno gibanje razreda $C^2$.

Tenzorsko polje
\[
	\ten{F}\colon B\times I\to\L(\V),\qquad \ten{F}=\Grad\chi(X,t)=\Grad\chi_t(X)
\]
se imenuje \emph{deformacijski gradient} gibanja. Determinanti $J=\det\ten{F}$ 
rečemo \emph{jacobijan}. Ker so konfiguracije $\chi_t$
injektivne oz.~bijektivne na svojo sliko, mora biti $J\neq 0$ povsod na $B\times I$.
Zaradi zveznosti tenzorskega polja $\ten{F}$ mora biti potem skalarno polje $J$,
ki je potem takem tudi zvezno, vseskozi
enakega predznaka. Po dogovoru je $J>0$. Po izreku o inverzni funkciji velja
\[
	\ten{F}^{-1}(x,t)=\grad\chi_t^{-1}(x)=\grad\chi^{-1}(x,t).
\]

\emph{Hitrost} $\vek{v}$ in \emph{pospešek} $\vek{a}$ gibanja $\chi$ sta vektorski polji
\begin{align}
	\vek{v}\colon B\times I\to \V \qquad & \vek{v}(X,t) = \frac{\partial\chi}{\partial t}(X,t), \label{e:v} \\
	\vek{a}\colon B\times I\to \V \qquad & \vek{a}(X,t) = \frac{\partial^2\chi}{\partial t^2}(X,t). \label{e:a}
\end{align}
Alternativna oznaka za $\vek{v}$ je tudi $\dot{\vek{x}}$, za $\vek{a}$ pa $\ddot{\vek{x}}$.


\subsection{Materialni in prostorski opis}


Vsakemu tenzorskemu polju $\vek{f}\colon B\times I\to\W$ pripada glede na gibanje
(\ref{e:chi}) enakovreden predpis
\[
	\bar{\vek{f}}\colon\Omega\to\W,\qquad
	\bar{\vek{f}}(x,t):=\vek{f}(\chi^{-1}(x,t),t)=\vek{f}(X,t)
\]
Tenzorsko polje $\bar{\vek{f}}$ se imenuje \emph{prostorski opis} tenzorskega polja $\vek{f}$.

Na enak način pripada vsakemu tenzorskemu polju $\vek{f}\colon\Omega\to\W$
glede na gibanje (\ref{e:chi}) enakovreden predpis
\begin{equation} \label{e:matopi}
	\hat{\vek{f}}\colon B\times I\to\W,\qquad
	\hat{\vek{f}}(X,t):=\vek{f}(\chi(X,t),t)=\vek{f}(x,t).
\end{equation}
$\hat{\vek{f}}$ imenujemo \emph{materialni opis} tenzorskega polja $\vek{f}$.

Pogosto bomo strešico ali črtico v oznakah za materialni ali prostorski opis opustili,
če to ne bo pustilo dvomov o tem, na kateri domeni je definirano polje.
Pri integraciji bo že iz integracijske domene razvidno, za katero polje gre.
Če poleg polja pišemo še argumente, potem mali $x$ v argumentu nakazuje na prostorski,
veliki $X$ pa na materialni opis.
Pri gradientu in divergenci se dvomom izognemo z uporabo različnih notacij
za ta dva diferencialna operatorja.
V materialnem opisu pišemo oznaki za gradient in divergenco z veliko začetnico
\[ \Grad\vek{f}:=\nabla\hat{\vek{f}},\qquad \Div\vek{f}:=\div\hat{\vek{f}}, \]
v prostorskem opisu pa z malo
\[ \grad\vek{f}:=\nabla\bar{\vek{f}},\qquad \div\vek{f}:=\div\bar{\vek{f}}. \]

Če je $\phi$ skalarno, $\vek{u}$ pa vektorsko polje, potem je zveza med gradientoma
\begin{equation}\label{e:gz}
	\Grad\phi=\ten{F}^{T}\grad\phi,\qquad \Grad\vek{u}=(\grad\vek{u})\ten{F}.
\end{equation}
Res, če je $\vek{w}$ poljubno vektorsko polje, dobimo iz (\ref{e:matopi})
z uporabo verižnega pravila
\begin{gather*}
	\Grad\phi\cdot\vek{w}=\grad\phi\cdot(\Grad\chi)\vek{w}=
	\grad\phi\cdot \ten{F}\vek{w}=\ten{F}^{T}\grad\phi\cdot\vek{w}, \\
	(\Grad\vek{u})\vek{w}=(\grad\vek{u})(\Grad\chi)\vek{w}
	=(\grad\vek{u})\ten{F}\vek{w}.
\end{gather*}

\begin{definicija}
	Časovni odvod tenzorskega polja $\vek{f}\colon B\times I\to\W$ označimo z
	$\dot{\vek{f}}$ ali $d\vek{f}/dt$ in ga imenujemo \emph{materialni odvod};
	\[ \dot{\vek{f}}(X,t)=\frac{d\vek{f}}{dt}(X,t)=\frac{\partial\vek{f}}{\partial t}(X,t). \]
\end{definicija}
Iz (\ref{e:matopi}) dobimo z uporabo verižnega pravila še materialni odvod za prostorski opis:
\begin{equation} \label{e:matodv}
	\dot{\vek{f}}=\frac{d\vek{f}}{dt}=
	\frac{\partial\vek{f}}{\partial t}+(\grad\vek{f})(\vek{v}),
\end{equation}
kjer je $\vek{v}$ hitrost gibanja (\ref{e:v}), $\partial\vek{f}/\partial t$ pa
časovni dvod polja $\vek{f}(x,t)$. Kadar poleg tenzorskega polja ne bomo pisali argumentov,
bo oznaka $\partial\vek{f}/\partial t$ vedno pomenila časovni odvod prostorskega opisa.

\begin{primer} %\label{e:L}
	Hitrost in pospešek sta prvi in drugi materialni odvod gibanja $\vek{x}(X,t)=\chi(X,t)$,
	$\vek{v}=\dot{\vek{x}}$, $\vek{a}=\ddot{\vek{x}}$.
	Pospešek je materialni odvod hitrosti in se v prostorskem opisu izraža kot
	\[ \vek{a}=\dot{\vek{v}}=\frac{\partial\vek{v}}{\partial t}+(\grad\vek{v})\vek{v}. \]
	Tenzorsko polje $\ten{L}=\grad\vek{v}$ se imenuje \emph{hitrostni gradient}.
	Če je gibanje $\chi$ razreda $C^1$, potem je
	\[
		\frac{d}{dt}\Grad\chi=\Grad\frac{d\chi}{dt}\qquad\textrm{oziroma}\qquad
		\dot{\ten{F}}=\Grad\vek{v},
	\]
	iz česar z uporabo zveze (\ref{e:gz}) dobimo
	\begin{equation} \label{e:L}
		\ten{L}:=\grad\vek{v}=\dot{\ten{F}}\ten{F}^{-1}
	\end{equation}
\end{primer}


\subsection{Transformacija ploskovnih in volumskih integralov}


Naj bo $\Theta\subset\R^2$ odprta in $r\in C^1(\Theta,B)$ regularna parametrizacija
neke ploskve $S\subset B$, lahko je del robu $\partial B$. Normalni vektor
na ploskev $S$ glede na parametrizacijo $r(\vartheta_1,\vartheta_2)$ je
Vrednost vektorske funkcije
\[
	\vek{S}=\frac{\partial r}{\partial\vartheta_1}\times
	\frac{\partial r}{\partial\vartheta_2}.
\]
pri danem $(\vartheta_1,\vartheta_2)$ je normalni vektor na ploskev $S$
v točki $X=r(\vartheta_1,\vartheta_2)\in S$.
Preslikava $\chi_t\circ r$ je parametrizacija ploskve $s\subset B_t$, pripadajoča
vektorska funkcija normalnih vektorjev je
\[
	\vek{s}=\frac{\partial(\chi_t\circ r)}{\partial\vartheta_1}\times
	\frac{\partial(\chi_t\circ r)}{\partial\vartheta_2}=
	(\Grad\chi)\frac{\partial r}{\partial\vartheta_1}\times
	(\Grad\chi)\frac{\partial r}{\partial\vartheta_2}=
	\ten{F}\frac{\partial r}{\partial\vartheta_1}\times
	\ten{F}\frac{\partial r}{\partial\vartheta_2}.
\]
Če so $\vek{u}_1$, $\vek{u}_2$, $\vek{w}$ poljubni vektorji in $\ten{A}\in\L(V)$, potem je
\begin{align*}
	\vek{w}\cdot\ten{A}\vek{u}_1\times\ten{A}\vek{u}_2&=
	\ten{A}(\ten{A}^{-1}\vek{w})\cdot\ten{A}\vek{u}_1\times\ten{A}\vek{u}_2=
	(\det\ten{A})\ten{A}^{-1}\vek{w}\cdot\vek{u}_1\times\vek{u}_2 \\
	&=\vek{w}\cdot (\det\ten{A})\ten{A}^{-T}(\vek{u}_1\times\vek{u}_2),
\end{align*}
od koder sledi $\ten{A}\vek{u}_1\times\ten{A}\vek{u}_2=
(\det\ten{A})\ten{A}^{-T}(\vek{u}_1\times\vek{u}_2)$, zato je
\begin{equation} \label{e:nsns}
	\vek{s}=J\ten{F}^{-T}\vek{S}\qquad\textrm{oz.}\qquad
	\vek{n}\|\vek{s}\|=J\ten{F}^{-T}\vek{N}\|\vek{S}\|.
\end{equation}
kjer sta $\vek{N}=\|\vek{S}\|^{-1}\vek{S}$ in $\vek{n}=\|\vek{s}\|^{-1}\vek{s}$
%\[ \vek{N}=\frac{\vek{S}}{\|\vek{S}\|}\qquad\textrm{in}\qquad\vek{n}=\frac{\vek{s}}{\|\vek{s}\|} \]
vektorski funkciji enotskih normal.

\begin{izrek} \label{i:suittra}
	Naj bo $\vek{f}\colon S\times I\to\W$ zvezno tenzorsko polje. Potem velja
	\begin{align*}
		\int_s\vek{f}[\vek{n}]\,ds=\int_S\vek{f}[J\ten{F}^{-T}\vek{N}]\,dS ,\\
		\int_S\vek{f}[\vek{N}]\,dS=\int_s\vek{f}[J^{-1}\ten{F}^{T}\vek{n}]\,ds,
	\end{align*}
	kjer izraz $\vek{f}[\,\cdot\,]$ nadomestimo z ustreznim produktom
	polja $\vek{f}$ in vsebine $[\,\cdot\,]$.
\end{izrek}

\proof
	\begin{align*}
		\int_s\vek{f}(x,t)[\vek{n}]\,ds&=
		\int_{\Theta}\vek{f}(\chi_t(r(\vartheta_1,\vartheta_2)),t)
		[\vek{n}]\|\vek{s}\|\,d\vartheta_1d\vartheta_2\\
		&=\int_{\Theta}\vek{f}(r(\vartheta_1,\vartheta_2),t)
		[J\ten{F}^{-T}\vek{N}]\|\vek{S}\|\,d\vartheta_1d\vartheta_2\\
		&=\int_S\vek{f}(X,t)[J\ten{F}^{-T}\vek{N}]\,dS.
	\end{align*}
	Pri tem smo uporabili enakost (\ref{e:nsns}) in dejstvo, da je
	\[ \bar{\vek{f}}(\chi_t(r(\vartheta_1,\vartheta_2)),t)=\hat{\vek{f}}(r(\vartheta_1,\vartheta_2),t). \]
	Drugo enakost dokažemo na podoben način.
\endproof

Naj bo sedaj $\Theta\subset\R^3$ odprta in $r\in C^1(\Theta,B)$ regularna parametrizacija
nekega območja $P\subset B$, lahko je notranjost od $B$. Prostorninski element je
skalarna funkcija
\[
	V=\Big|\frac{\partial r}{\partial\vartheta_1}\cdot
	\frac{\partial r}{\partial\vartheta_2}\times\frac{\partial r}{\partial\vartheta_3}\Big|
\]
Preslikava $\chi_t\circ r$ je v tem primeru parametrizacija območja $P_t=\chi_t(P)\subset B_t$.
Pripadajoči prostorninski element je
\begin{align*}
	v&=\Big|\frac{\partial(\chi_t\circ r)}{\partial\vartheta_1}\cdot
	\frac{\partial(\chi_t\circ r)}{\partial\vartheta_2}\times\frac{\partial(\chi_t\circ r)}{\partial\vartheta_3}\Big|
	=\Big|\ten{F}\frac{\partial r}{\partial\vartheta_1}\cdot
	\ten{F}\frac{\partial r}{\partial\vartheta_2}\times\ten{F}\frac{\partial r}{\partial\vartheta_3}\Big| \\
	&=\det\ten{F}\Big|\frac{\partial r}{\partial\vartheta_1}\cdot
	\frac{\partial r}{\partial\vartheta_2}\times\frac{\partial r}{\partial\vartheta_3}\Big|.
\end{align*}
Pri tem smo upoštevali, da je $J=\det\ten{F}>0$.
Relacija med prostorninskima elementoma $v$ in $V$ je torej $v=JV$.
%\begin{equation*}
%	v=JV.
%\end{equation*}

Na podoben način, kot smo dokazali izrek \ref{i:suittra}, se s pomočjo pravkar dobljene
enakosti dokaže še naslednji izrek.
\begin{izrek} \label{i:prointrel}
	Naj bo $\vek{f}\colon P\times I\to\W$ zvezno tenzorsko polje. Potem velja
	\begin{equation*}
		\int_{P_t}\vek{f}\,dv=\int_P\vek{f}J\,dV \qquad\textrm{in}\qquad
		\int_P\vek{f}\,dV=\int_{P_t}\vek{f}J^{-1}\,dv.
	\end{equation*}
\end{izrek}


\subsection{Transportni izrek}


V tem razdelku bomo podali transportni izrek, katerega vsebina je
enačba za časovni odvod integrala po območju trenutne konfiguracije materialnega telesa.
Še prej pa potrebujemo formulo za materialni odvod determinante deformacijskega gradienta.

Naj bo $\W$ vektorski prostor nad obsegom $\R$ dimenzije $n\in\mathbb{N}$.
Najprej poiščimo odvod za determinanto $\det\colon\L(\W)\to\R$.
Naj bo $\omega\colon \W^n\to\R$ netrivialna alternirajoča $n$-linearna forma\footnote{
Alternirajoča $n$-linearna forma je definirana npr.~v \cite[str.~246]{liu}.}
in naj bo $A\in\L(\W)$ linearna preslikava.
Spomnimo, determinanta in sled linearne preslikave $A$ sta definirani kot
\begin{gather*}
	\omega(A\vek{u}_1,\dots,A\vek{u}_n)=(\det A)\,\omega(\vek{u}_1,\dots,\vek{u}_n), \\
	\sum_{j=1}^n\omega(\vek{u}_1,\dots,A\vek{u}_j,\dots,\vek{u}_n)=(\tr A)\,\omega(\vek{u}_1,\dots,\vek{u}_n)
\end{gather*}
in definicija je neodvisna od izbire netrivialne forme $\omega$.

Naj bo $S\in\L(\W)$ še ena linearna preslikava in $\varepsilon>0$. Potem je
\begin{align*}
	\det&(A+\varepsilon S)\,\omega(\vek{u}_1,\dots,\vek{u}_n)=
	\omega((A+\varepsilon S)\vek{u}_1,\dots,(A+\varepsilon S)\vek{u}_n)\\
	&=\omega(A\vek{u}_1,\dots,A\vek{u}_n)+
	\sum_{j=1}^n\varepsilon\,\omega(A\vek{u}_1,\dots,S\vek{u}_j,\dots,A\vek{u}_n)+o(\varepsilon)\\
	&=(\det A)\,\omega(\vek{u}_1,\dots,\vek{u}_n)+\varepsilon
	\sum_{j=1}^n\omega(A\vek{u}_1,\dots,AA^{-1}S\vek{u}_j,\dots,A\vek{u}_n)+o(\varepsilon)\\
	&=(\det A)\Big(\omega(\vek{u}_1,\dots,\vek{u}_n)+\varepsilon
	\sum_{j=1}^n\omega(\vek{u}_1,\dots,A^{-1}S\vek{u}_j,\dots,\vek{u}_n)\Big)+o(\varepsilon)\\
	&=(\det A)(1+\varepsilon\,\tr(A^{-1}S))\,\omega(\vek{u}_1,\dots,\vek{u}_n)+o(\varepsilon).
\end{align*}
Pri tem smo člene, kjer nastopajo potence števila $\varepsilon$, spravili
v izraz $o(\varepsilon)$. Od tu sedaj lahko izračunamo smerni odvod
\begin{multline*}
	\at{\frac{d}{d\varepsilon}\det(A+\varepsilon S)}{\varepsilon=0}
	\,\omega(\vek{u}_1,\dots,\vek{u}_n)=
	\at{\frac{d}{d\varepsilon}\Big[\det(A+\varepsilon S)
	\,\omega(\vek{u}_1,\dots,\vek{u}_n)\Big]}{\varepsilon=0}\\
	=(\det A)\tr(A^{-1}S)\,\omega(\vek{u}_1,\dots,\vek{u}_n)=
	\big[(\det A)A^{-T}\cdot S\big]\,\omega(\vek{u}_1,\dots,\vek{u}_n).
\end{multline*}
Pri tem smo na zdanjem koraku uporabili definicijo skalarnega produkta na prostoru $\L(\W)$.
Determinanta je odvedljiva preslikava, zato je krepki odvod enak pravkar izračunanemu
smernemu odvodu, torej imamo
\[
	D\det(A)(S)=(\det A)A^{-T}\cdot S\qquad\textrm{oz.}\qquad D\det(A)=(\det A)A^{-T}
\]
Druga enačba sledi iz prve, ker je skalarni produkt linearna preslikava.

Sedaj imamo pripravljeno vse, da izračunamo materialni odvod jacobijana:
\begin{align}
	\dot{J}&=(\det\ten{F})\,\dot{}=D\det(\ten{F})(\dot{\ten{F}})=J\ten{F}^{-T}\cdot\dot{\ten{F}}
	\nonumber \\ &= J\tr(\dot{\ten{F}}\ten{F}^{-1}) = J\tr(\grad\vek{v})=J\div\vek{v}. \label{e:dotJ}
\end{align}
Pri tem smo uporabili verižno pravilo, malo prej izpeljano pravilo za odvod determinante
(kjer vzamemo $\W=\V$), komutativnost skalarnega produkta ter zvezo (\ref{e:L}).

\begin{izrek}[Transportni izrek] \label{i:transport}
	Naj bo $P\subseteq B$ regularno območje znotraj referenčne konfiguracije in naj
	$P_t=\chi(P,t)\subseteq B_t$ označuje njegovo trenutno konfiguracijo ob času $t$.
	Naj bo $\vek{f}\colon P\times I\to W$ razreda $C^1$ na zaprtju $\overline{P}\times I$.
	Potem velja
	\begin{equation*}
		\frac{d}{dt}\int_{P_t}\vek{f}\,dv =
		\int_{P_t}(\dot{\vek{f}}+\vek{f}\div\vek{v})\,dv=
		\int_{P_t}\frac{\partial\vek{f}}{\partial t}\,dv +
		\int_{\partial P_t}(\vek{v}\cdot\vek{n})\vek{f}\,ds.
	\end{equation*}
\end{izrek}
Ne pozabimo, da pri tem $\partial\vek{f}/\partial t$ pomeni časovni (parcialni)
odvod prostorskega opisa polja $\vek{f}$.

\proof
	Z uporabo izreka \ref{i:prointrel} in enačbe (\ref{e:dotJ}) pridemo do
	\begin{align*}
		\frac{d}{dt}\int_{P_t}\vek{f}\,dv &= \frac{d}{dt}\int_{P}\vek{f}J\,dV =
		\int_{P}\frac{d}{dt}(\vek{f}J)\,dV = \int_{P}(\dot{\vek{f}}J+\vek{f}\dot{J})\,dV =\\
		&=\int_{P}(\dot{\vek{f}}+\vek{f}\div\vek{v})J\,dV = \int_{P_t}(\dot{\vek{f}}+\vek{f}\div\vek{v})\,dv,
	\end{align*}
	s čimer je dokazan prvi del enačbe iz ireka.
	Pri tem smo na drugem koraku smeli zamenjati vrstni red odvajanja in integriranja,
	ker se referenčna konfiguracija s časom ne spreminja.
	
	Preostalo enakost bomo dokazali za primer, ko je $\vek{f}$ skalarno ali vektorsko
	polje, za katero lahko uporabimo tretjo oz.~četrto enačbo iz trditve \ref{t:divprop}.
	Izrek sicer velja tudi za splošna tenzorska polja, le dokaz je bolj zapleten.
	
	V nadaljevanju z uporabo enačbe (\ref{e:matodv}), trditve \ref{t:divprop}$_{3,4}$
	in divergenčnega izreka \ref{i:divtheo} pridemo še do
	\begin{multline*}
		\int_{P_t}(\dot{\vek{f}}+\vek{f}\div\vek{v})\,dv
		=\int_{P_t}\Big(\frac{\partial\vek{f}}{\partial t}+
		(\grad\vek{f})(\vek{v})+\vek{f}\div\vek{v}\Big)\,dv \\
		=\int_{P_t}\Big(\frac{\partial\vek{f}}{\partial t}+\div(\vek{f}[\vek{v}])\Big)\,dv
		=\int_{P_t}\frac{\partial\vek{f}}{\partial t}\,dv+
		\int_{\partial P_t}(\vek{v}\cdot\vek{n})\vek{f}\,ds.
	\end{multline*}
	Pri tem oznaka $\vek{f}[\vek{v}]$ pomeni $f\vek{v}$, če je $\vek{f}=f$ skalarno polje,
	oz.~$\vek{f}\otimes\vek{v}$, če je $\vek{f}$ vektorsko polje.
\endproof


\subsection{Zakon o ohranitvi mase}


\emph{Masa} poljubne podmnožice $P\subseteq B$ je definirana kot vrednost integrala
\[ M(P)=\int_P\rho_R\,dV, \]
kjer je $\rho_R\colon B\to(0,\infty)$ integrabilno skalarno polje, imenovano
\emph{masna gostota referenčne konfiguracije}.
Za $\rho_R$ se običajno predpostavi še dodatne pogoje o zveznosti ali odvedljivosti.

\begin{comment}
Poljubni konfiguraciji materialnega telesa $\kappa\colon B\to\E$ pripada
integrabilno skalarno polje $\rho_{\kappa}\colon B\to(0,\infty)$, imenovano
\emph{masna gostota konfiguracije $\kappa$}, tako da za vsako množico $P\subseteq B$ velja
\[ M(P)=\int_{\kappa(P)}\rho_{\kappa}\,dv. \]
\end{comment}

V mehaniki kontinuuma se predpostavi naslednji zakon.
\begin{aksiom}[Zakon o ohranitvi mase]
	Gibanju $\chi(X,t)=\chi_t(X)$ pripada integrabilno, časovno odvisno skalarno polje
	$\rho\colon\Omega\to(0,\infty)$, imenovano \emph{masna gostota trenutne konfiguracije},
	tako da za vsako množico $P\subseteq B$ velja
	\[ M(P)=\int_{\chi_t(P)}\rho\,dv\qquad\textrm{za vsak}\ t\in I. \]
\end{aksiom}

Iz tega zakona z uporabo izreka \ref{i:prointrel} dobimo
\[ \int_{P}(\rho_R-\rho J)\,dV=0 \qquad\textrm{za vsak}\ t\in I, \]
in ker to velja za vsako množico $P\subseteq B$, dobimo po trditvi \ref{t:oiz}
naslednjo relacijo za masni gostoti:
\begin{equation} \label{e:rojror}
	\rho_R(X)=\rho(X,t)J(X,t)\qquad\textrm{za vsak}\ X\in B,\ t\in I.
\end{equation}
Seveda potem velja $\rho_R=\rho J$ tudi v prostorskem opisu na $\Omega$.

\begin{posledica} \label{p:roji}
	Za integrabilno tenzorsko polje $\vek{f}\colon B\to\W$ velja
	\[
		\int_{P}\vek{f}\rho_R\,dV=\int_{\chi_t(P)}\vek{f}\rho\,dv.
	\]
\end{posledica}

Še ena takojšnja posledica zakona o ohranitvi mase je ta, da se masa katerega koli
dela delesa med gibanjem ne spreminja, torej velja
\[ \frac{d}{dt}\int_{\chi_t(P)}\rho(x,t)\,dv = 0\quad\textrm{za vsak}\ t\in I. \]
Če je $\rho$ razreda $C^1$, potem z uporabo transportnega izreka \ref{i:transport}
pridem do
\[ \int_{\chi_t(P)}(\dot{\rho}+\rho\div\vek{v})\,dv=0, \]
in ker to velja za poljubno množico $\chi_t(P)\subseteq\chi_t(B)$, je po trditvi \ref{t:oiz}
\begin{equation} \label{e:lozom}
	\dot{\rho}+\rho\div\vek{v}=0
\end{equation}
na celotnem območju $\chi_t(B)$ in za vsak $t\in I$, torej na celotnem območju $\Omega$,
enako pa potem velja tudi v materialnem opisu na $B\times I$.
Enačba (\ref{e:lozom}) je znana pod imenom \emph{lokalna oblika zakona o ohranitvi mase}.

\begin{izrek}
	Naj bo tenzorsko polje $\vek{f}\colon B\times I\to W$ razreda $C^1$ in naj bo masna gostota
	$\rho\colon\Omega\to(0,\infty)$ prav tako razreda $C^1$.
	Potem za vsak $P\subseteq\B$ velja
	\[
		\frac{d}{dt}\int_{\chi_t(P)}\vek{f}\rho\,dv=
		\int_{\chi_t(P)}\dot{\vek{f}}\rho\,dv.
	\]
\end{izrek}
\proof
	Iz transportnega izreka \ref{i:transport} dobimo
	\begin{align*}
		\frac{d}{dt}\int_{\chi_t(P)}\vek{f}\rho\,dv
		&= \int_{\chi_t(P)}\big((\vek{f}\rho)\,\dot{}+\vek{f}\rho\div\vek{v}\big)\,dv \\
		&= \int_{\chi_t(P)}(\dot{\vek{f}}\rho+\vek{f}\dot{\rho}+\vek{f}\rho\div\vek{v})\,dv\\
		&= \int_{\chi_t(P)}\dot{\vek{f}}\rho\,dv+
		\int_{\chi_t(P)}\vek{f}(\dot{\rho}+\rho\div\vek{v})\,dv,
	\end{align*}
	od koder sledi enakost iz izreka, če upoštevamo (\ref{e:lozom}).
\endproof
Druga možnost za dokaz izreka je uporaba posledice \ref{p:roji} in
izreka \ref{i:prointrel}.