\section{Krivočrtne koordinate} \label{s:koordinate}


\subsection{Koordinatni sistem}


\begin{definicija}
	Naj bo $U\subseteq\E$ odprta množica. \emph{Koordinatni sistem} na $U$ je bijektivna
	preslikava $\psi\colon U\to V$ razreda $C^r$ (običajno je $r$ vsaj 2), kjer je $V$ odprta
	množica v $\R^3$, inverz $\psi^{-1}$ pa je prav tako razreda $C^r$.
\end{definicija}

Naj bo $x\in U$ in
\[ \psi\colon x\mapsto (x^1,x^2,x^3)=\psi(x). \]
$(x^1,x^2,x^3)$ so \emph{(krivočrtne) koordinate} točke $x$.
Za $i\in\{1,2,3\}$ se funkcija
\[ \psi^i\colon U\to\R,\qquad \psi^i(x)=x^i \]
imenuje \emph{$i$-ta koordinatna funkcija} koordinatnega sistema $\psi$.
Po dogovoru tudi oznaki $(x^i)$ rečemo koordinatni sistem na $U$.

Naj bodo $(x^1,x^2,x^3)$ koordinate točke $x\in U$. Z $\vek{j}_i$ označimo
$i$-ti vektor iz standardne baze za $\R^3$,
\[ \vek{j}_1=(1,0,0),\quad\vek{j}_2=(0,1,0),\quad\vek{j}_3=(0,0,1). \]
Za neki $\varepsilon>0$ je slika preslikave
\begin{equation} \label{e:kokri}
	\gamma_i\colon(-\varepsilon,\varepsilon)\to U,\qquad
	\gamma_i(t)=\psi^{-1}\big((x^1,x^2,x^3)+t\vek{j}_i\big),\qquad i=1,2,3,
\end{equation}
\emph{$i$-ta koordinatna krivulja} v $U$, ki gre pri $t=0$ skozi točko $x$.
V točki $x$ lahko definiramo \emph{tangentne vektorje}
\begin{equation} \label{e:tanvek}
	\vek{g}_i(x)=\at{\frac{d}{dt}\gamma_i(t)}{t=0}=\dot{\gamma}_i(0)
	=\at{\frac{\partial\psi^{-1}}{\partial x^i}}{(x^1,x^2,x^3)}
	, \qquad i=1,2,3
\end{equation}
in \emph{gradientne vektorje}
\begin{equation} \label{e:gradvek}
	\vek{g}^i(x)=\nabla\psi^i(x),\qquad i=1,2,3.
\end{equation}

\begin{trditev}
	Množica $\{\vek{g}_i(x)\}_i$ tvori bazo za translacijski prostor $\V$.
\end{trditev}

\proof
	Naj bo $\vek{u}\in\V$ poljuben in definirajmo krivuljo skozi $x$ s predpisom
	\[ \gamma\colon(-\varepsilon,\varepsilon)\to U,\qquad \gamma(t)=x+t\vek{u}. \]
	Velja
	\[
		\vek{u}=\at{\frac{d}{dt}\gamma(t)}{t=0}\quad\textrm{in}\quad
		\gamma(t)=\psi^{-1}\big(\psi^1(x+t\vek{u}),\psi^2(x+t\vek{u}),\psi^3(x+t\vek{u})\big),
	\]
	zato je
	\[
		\vek{u}=
		\at{\frac{\partial\psi^{-1}}{\partial x^i}}{(x^1,x^2,x^3)}
		\at{\frac{d}{dt}\psi^i(x+t\vek{u})}{t=0}=
		\at{\frac{d}{dt}\psi^i(x+t\vek{u})}{t=0}\vek{g}_i(x).
	\]
	Z drugimi besedami, $\{\vek{g}_i(x)\}_{i}$ razpenja prostor $\V$.
\endproof

Tangentni in gradientni vektorji so definirani v vsaki točki množice $U$, torej
gre v bistvu za vektorska polja. V vsaki točki $x\in U$ se baza $\{\vek{g}_i(x)\}_i$
imenuje \emph{naravna baza} koordinatnega sistema $(x^i)$ za prostor $\V$.
Množica $\{\vek{g}^i(x)\}_i$ pa je prav tako baza za $\V$, ki je dualna bazi $\{\vek{g}_i(x)\}_i$.
Res, iz zveze
\[ x^i=\psi^i(\psi^{-1}(x^1,x^2,x^3)) \]
dobimo po verižnem pravilu
\begin{equation} \label{e:gigj}
	\topbot{\delta}{i}{j}=\frac{\partial x^i}{\partial x^j}=
	(\nabla\psi^i)\cdot\frac{\partial\psi^{-1}}{\partial x^j}=\vek{g}^i\cdot\vek{g}_j.
\end{equation}
Skalarna polja
\[ g_{ij}=\vek{g}_i\cdot\vek{g}_j\qquad\textrm{in}\qquad g^{ij}=\vek{g}^i\cdot\vek{g}^j \]
imenujemo \emph{koeficienti metričnega tenzorja}. 
Iz zadnjih treh zvez hitro vidimo, da velja
\begin{equation} \label{e:relzag}
	g^{ik}g_{kj}=\topbot{\delta}{i}{j},\qquad \vek{g}^i=g^{ik}\vek{g}_k,\qquad \vek{g}_i=g_{ik}\vek{g}^k.
\end{equation}


\subsection{Koordinatna transformacija}


Naj bosta $\psi$ in $\bar{\psi}$ oz.~$(x^i)$ in $(\bar{x}^i)$ koordinatna sistema za
$U\subseteq\E$ z naravnima bazama $\{\vek{g}_i(x)\}_i$ oz.~$\{\bar{\vek{g}}_i(x)\}_i$.
Koordinatna transformacija je podana s preslikavami
\[
	x^i=x^i(\bar{x}^1,\bar{x}^2,\bar{x}^3)\quad\Longleftrightarrow\quad
	\bar{x}^k=\bar{x}^k(x^1,x^2,x^3).
\]
Za vektorje iz naravne baze velja
\begin{equation} \label{e:gtog}
	\vek{g}^i=\frac{\partial x^i}{\partial\bar{x}^k}\bar{\vek{g}}^k,\qquad
	\vek{g}_i=\frac{\partial\bar{x}^k}{\partial x^i}\bar{\vek{g}}_k.
\end{equation}
Prvo enakost dobimo kot gradient funkcije $x^i$ iz koordinatne transformacije,
drugo enakost pa dobimo s parcialnim odvajanjem enakosti
\[ 
	\psi^{-1}(x^1,x^2,x^3)=\bar{\psi}^{-1}(\bar{x}^1(x^1,x^2,x^3),
	\bar{x}^2(x^1,x^2,x^3),\bar{x}^3(x^1,x^2,x^3)).
\]

%Naj bo $(x^1,x^2,x^3)$ in $(\bar{x}^1,\bar{x}^2,\bar{x}^3)$ dvoje koordinat točke $x\in U\subseteq\E$.
%Če je $f\colon U\to\R$ skalarno polje, potem bomo brez posebnih oznak pisali
%\[ f(x)=f(x^1,x^2,x^3)=f(\bar{x}^1,\bar{x}^2,\bar{x}^3). \]
Če je $\vek{u}\colon U\to\V$ vektorsko polje, potem ga lahko zapišemo v komponentni
obliki glede na naravno bazo prostora $\V$:
\begin{align}
	\vek{u}&=u_i\,\vek{g}^i=u^i\,\vek{g}_i \nonumber \\
	&=\bar{u}_i\,\bar{\vek{g}}^i=\bar{u}^i\,\bar{\vek{g}}_i, \label{e:vekkomp}
\end{align}
Pri tem so komponente $u_i,u^i,\dots$ skalarna polja.
Prav tako lahko tenzorsko polje $\ten{S}\colon U\to\L(\V)$ zapišemo v
komponentni obliki glede na bazo prostora $\L(\V)$:
\begin{align}
	\ten{S}&=S_{ij}\,\vek{g}^i\otimes\vek{g}^j=\topbot{S}{i}{j}\,\vek{g}_i\otimes\vek{g}^j \nonumber \\
	&=\bar{S}_{ij}\,\bar{\vek{g}}^i\otimes\bar{\vek{g}}^j=
	\topbot{\bar{S}}{i}{j}\,\bar{\vek{g}}_i\otimes\bar{\vek{g}}^j \label{e:tenkomp}
\end{align}
Tudi tu so $S_{ij},\topbot{S}{i}{j},\dots$ skalarna polja.
Posamezne komponente vektorskih in tenzoskih polj dobimo s pomočjo zveze (\ref{e:gigj})
iz (\ref{e:vekkomp}) in (\ref{e:tenkomp}):
\begin{gather}
	u_i=\vek{g}_i\cdot\vek{u},\qquad u^i=\vek{g}^i\cdot\vek{u}, \label{e:kompvek} \\
	S_{ij}=\vek{g}_i\cdot\ten{S}\vek{g}_j,\qquad\topbot{S}{i}{j}=\vek{g}^i\cdot\ten{S}\vek{g}_j.\label{e:kompten}
\end{gather}
S pomočjo (\ref{e:gtog}) dobimo naslednje zveze med posameznimi komponentami:
\begin{gather*}
	\bar{u}_i=\frac{\partial x^k}{\partial\bar{x}^i}u_k, \qquad
	\bar{u}^i=\frac{\partial\bar{x}^i}{\partial x_k}u_k, \\
	\bar{S}_{ij}=\frac{\partial x^k}{\partial\bar{x}^i}\frac{\partial x^l}{\partial\bar{x}^j}S_{kl}, \qquad
	\topbot{\bar{S}}{i}{j}=\frac{\partial\bar{x}^i}{\partial x^k}\frac{\partial x^l}{\partial\bar{x}^j}\topbot{S}{k}{l}.
\end{gather*}

Če je $\vek{u}=u_k\vek{g}^k$ poljubno vektorsko polje, je
\begin{equation} \label{e:iten}
	(\vek{g}^i\otimes\vek{g}_i)\vek{u}=(\vek{g}^i\otimes\vek{g}_i)u_k\vek{g}^k=
	u_k\bottop{\delta}{i}{k}\vek{g}^i=u_k\vek{g}^k=\vek{u},
\end{equation}
torej je $\vek{g}^i\otimes\vek{g}_i=\ten{1}$, identična linearna preslikava, in to
v vsaki točki množice $U$. Seveda velja tudi $\vek{g}_i\otimes\vek{g}^i=\ten{1}$.


\subsection{Gradient in kovariantni odvod}


Do konca tega razdelka naj velja, da so vsa obravnavan tenzorska polja razreda
$C^{1}$ ali $C^{2}$, kar bo razvidno samo po sebi.

Naj bo $(x^i)$ koordinatni sistem na $U\subseteq\E$ z naravnima bazama $\{\vek{g}_i\}_i$ in $\{\vek{g}^i\}_i$.
Gradiente tenzorskih polj lahko zapišemo v komponentni obliki glede na naravno bazo.

Naj bo $\vek{f}\colon U\to\W$ tenzorsko polje. Če je $\gamma_i$ $i$-ta
koordinatna krivulja, ki gre pri $t=0$ skozi točko $x$, potem iz (\ref{e:tanvek}) dobimo
\begin{equation*}
	\at{\frac{d}{dt}\vek{f}(\gamma_i(t))}{t=0}=
	\nabla \vek{f}(x)(\dot{\gamma}_i(0))=\nabla \vek{f}(x)(\vek{g}_i(x)),
\end{equation*}
po drugi strani pa imamo po (\ref{e:kokri})
\begin{equation*}
	\at{\frac{d}{dt}\vek{f}(\gamma_i(t))}{t=0}=
	\at{\frac{d}{dt}\vek{f}\big(\psi^{-1}((x^1,x^2,x^3)+t\vek{j}_i)\big)}{t=0}=
	\at{\frac{\partial(\vek{f}\circ\psi^{-1})}{\partial x^i}}{(x^1,x^2,x^3)}.
\end{equation*}
Po našem dogovoru pišemo namesto $\vek{f}\circ\psi^{-1}$ kar $\vek{f}$. Če izenačimo oba rezultata, dobimo
\begin{equation} \label{e:sotf}
	\frac{\partial\vek{f}}{\partial x^i}(x^1,x^2,x^3)=\nabla \vek{f}(x)(\vek{g}_i(x)).
\end{equation}
V primeru, ko je $f$ skalarno polje, iz enačbe (\ref{e:kompvek}) in iz pravkar izpeljane enačbe
dobimo $(\nabla f)_i=(\nabla f)\cdot\vek{g}_i=\partial f/\partial x^i$, torej je
\begin{equation} \label{e:gradskal}
	\nabla f = \frac{\partial f}{\partial x^i}\,\vek{g}^i.
\end{equation}

Preden nadaljujemo z gradienti vektorskih polj, vpeljimo najprej standardne oznake
za gradiente vektorskih polj naravne baze:
\begin{equation} \label{e:cs} %former label: e:gradg
	\nabla\vek{g}_i=\ten{\Gamma}_i=\cs{i}{j}{k}\,\vek{g}_j\otimes\vek{g}^k,\qquad
	\nabla\vek{g}^i=\ten{\Gamma}^i=\ks{i}{jk}\,\vek{g}^j\otimes\vek{g}^k.
\end{equation}
Tu sta $\ten{\Gamma}_i,\ten{\Gamma}^i\colon U\to\L(\V)$ tenzorski polji drugega reda,
komponente $\cs{i}{j}{k}$ in $\ks{i}{jk}$ pa se imenujejo \emph{Christoffelovi simboli}
in ne gre za komponente kakega tenzorja tretjega reda. Če v enačbi (\ref{e:sotf})
za $\vek{f}$ vstavimo vektorsko polje $\vek{g}_i$ oz.~$\vek{g}^i$ in upoštevamo
(\ref{e:kompten}) in (\ref{e:cs}), dobimo
\begin{equation} \label{e:csexplicit}
	\cs{i}{j}{k}=\vek{g}^j\cdot\frac{\partial\vek{g}_i}{\partial x^k}
	=-\vek{g}_i\cdot\frac{\partial\vek{g}^j}{\partial x^k},\qquad
	\ks{i}{jk}=\vek{g}_j\cdot\frac{\partial\vek{g}_i}{\partial x^k}.
\end{equation}
Pri tem je drugi izraz za $\cs{i}{j}{k}$ dobljen iz prvega s parcialnim odvajanjem
enakosti $\vek{g}^i\cdot\vek{g}_j=\topbot{\delta}{i}{j}$ po spremenljivki $x^k$.

Če po pravilu (\ref{t:divprop})$_2$ izračunamo gradient izraza $(\vek{g}^i\cdot\vek{g}_j)$,
ki je $\vek{0}$, in upoštevamo, da velja\footnote{
$(\vek{u}\otimes\vek{v})\vek{a}\cdot\vek{b}=(\vek{v}\cdot\vek{a})(\vek{u}\cdot\vek{b})=
\vek{a}\cdot(\vek{v}\otimes\vek{u})\vek{b}$}
$(\vek{u}\otimes\vek{v})^{T}=\vek{v}\otimes\vek{u}$
za poljubna vektorja $\vek{u}$ in $\vek{v}$ ter da je transponiranje linearna operacija,
potem dobimo zvezo
\begin{equation} \label{e:lcs1}
	\cs{j}{i}{k}=-\ks{i}{jk}.
\end{equation}
Nadalje, ker je $\ten{\Gamma}^i=\nabla(\nabla\psi^i)$ in je drugi gradient vedno
simetrični tenzor\footnote{$(\nabla(\nabla\psi^i))^T=\nabla(\nabla\psi^i)$ \cite[str.~271]{liu}},
veljata še naslednji zvezi:
\[ \ks{i}{jk}=\ks{i}{kj},\qquad\cs{j}{i}{k}=\cs{k}{i}{j}. \]
Ker je zaradi teh zvez možno prehajati iz ene vrste simbolov v drugo vrsto,
so v uporabi zgolj simboli $\cs{i}{j}{k}$, imenovani Christoffelovi simboli druge vrste.

Naj bo sedaj $\vek{u}\colon U\to\V$ vektorsko polje, v komponentni obliki
zapisano kot
\[ \vek{u}=u^j\vek{g}_j=u_k\vek{g}^k. \]
Njegov gradient zavzema vrednosti v prostoru $\L(\V)$, zato ga lahko zapišemo v komponentni obliki
\begin{equation} \label{e:gradukomp}
	\nabla\vek{u}=\topbot{u}{j}{,k}\,\vek{g}_j\otimes\vek{g}^k.
\end{equation}
Poiščimo izraz za komponente $\topbot{u}{j}{,k}$. Z upoštevanjem
(\ref{e:kompten}) in (\ref{e:sotf}) dobimo
\begin{align}
	\topbot{u}{j}{,k}&=\vek{g}^j\cdot\frac{\partial\vek{u}}{\partial x^k}=
	\vek{g}^j\cdot\frac{\partial(u^i\vek{g}_i)}{\partial x^k} \nonumber \\
	&=\vek{g}^j\cdot\Big(\frac{\partial u^i}{\partial x^k}\vek{g}_i+
	u^i\frac{\partial\vek{g}_i}{\partial x^k}\Big) \nonumber \\
	&=\frac{\partial u^j}{\partial x^k}+u^i\cs{i}{j}{k}. \label{e:kovod1}
\end{align}
Pri tem smo na zadnjem koraku upoštevali relacijo (\ref{e:csexplicit}). Dobljena enakost (\ref{e:kovod1})
za $\topbot{u}{j}{,k}$ je t.~i.~\emph{kovariantni odvod} komponentne $u^j$ po spremenljivki $x^k$.

Če zapišemo
\begin{equation} \label{e:nimena1}
	\nabla\vek{u}=u_{j,k}\,\vek{g}^j\otimes\vek{g}^k,
\end{equation}
in ponovimo prejšnji postopek, kjer dodatno uporabimo relacijo (\ref{e:lcs1}), dobimo
\begin{equation} \label{e:kov2}
	u_{j,k}=\frac{\partial u_j}{\partial x^k}-u_i\cs{j}{i}{k}.
\end{equation}
\begin{primer}
	Če v izrazu $\vek{u}=u_j\vek{g}^j$ vstavimo za $u_j=\partial f/\partial x^j$,
	kar so komponente od $\nabla f$ v izrazu (\ref{e:gradskal}), ter vstavimo v
	(\ref{e:kov2}), rezultat pa nato v (\ref{e:nimena1}), dobimo
	\begin{equation} \label{e:dvagrad}
		%\Big(\frac{\partial f}{\partial x^j}\Big)_{,k}
		\nabla(\nabla f)=\Big(\frac{\partial^2 f}{\partial x^j\partial x^k}-
		\frac{\partial f}{\partial x^i}\cs{j}{i}{k}\Big)\vek{g}^j\otimes\vek{g}^k.
	\end{equation}
\end{primer}

Naj bo $\ten{S}\colon U\to\L(\V)$. Potem je $\nabla\ten{S}\colon U\to\L(\V,\L(\V))$
in ga v komponentni obliki lahko razpišemo glede na bazo prostora $\L(\V,\L(\V))$ kot
\[ \nabla\ten{S}=\topbot{S}{ij}{,k}\vek{g}_i\otimes\vek{g}_j\otimes\vek{g}^k. \]
Iz (\ref{e:sotf}) dobimo
\[
	\frac{\partial\ten{S}}{\partial x^k}=(\nabla\ten{S})(\vek{g}_k)=
	(\topbot{S}{ij}{,l}\vek{g}_i\otimes\vek{g}_j\otimes\vek{g}^l)\vek{g}_k=
	\topbot{S}{ij}{,k}\vek{g}_i\otimes\vek{g}_j.
\]
Če na dobljeni enakosti uporabimo (\ref{e:kompten}), dobimo
\begin{align*}
	\topbot{S}{ij}{,k}&=\vek{g}^i\cdot\frac{\partial\ten{S}}{\partial x^k}\vek{g}^j=
	\vek{g}^i\cdot\frac{\partial(S^{lr}\vek{g}_l\otimes\vek{g}_r)}{\partial x^k}\vek{g}^j \\
	&=\vek{g}^i\cdot\Big(
	\frac{\partial S^{lr}}{\partial x^k}\vek{g}_l\otimes\vek{g}_r+
	S^{lr}\frac{\partial\vek{g}_l}{\partial x^k}\otimes\vek{g}_r+
	S^{lr}\vek{g}_l\otimes\frac{\partial\vek{g}_r}{\partial x^k}
	\Big)\vek{g}^j \\
	&=\frac{\partial S^{ij}}{\partial x^k}+S^{lj}\cs{l}{i}{k}+S^{ir}\cs{r}{j}{k}.
\end{align*}
Preostale komponente za $\nabla\ten{S}$ dobimo na enak način.


\subsection{Divergenca}


Sled tenzorja $\ten{S}\in\L(\V)$ s komponentno obliko (\ref{e:tenkomp})
je\footnote{$\tr\ten{S}=\topbot{S}{i}{j}\tr(\vek{g}_i\otimes\vek{g}^j)=
\topbot{S}{i}{j}\vek{g}_i\cdot\vek{g}^j=\topbot{S}{i}{i}$ \cite[str.~249]{liu}}
\[ \tr\,\ten{S}=\topbot{S}{i}{i}=g^{ij}S_{ij}. \]
Če je $\vek{u}$ vektorsko polje, potem iz (\ref{e:gradukomp}) in (\ref{e:nimena1}) dobimo
\begin{equation} \label{e:divu}
	\div\vek{u} = \tr(\nabla\vek{u}) = \topbot{u}{i}{,i} = g^{ij}u_{i,j}.
\end{equation}

Poiščimo še izraz za divergenco tenzorskega polja $\ten{S}\colon U\to\L(\V)$.
Iz (\ref{e:divu}) dobimo za poljubno vektorsko polje $\vek{u}=u_k\vek{g}^k$
\[
	\div(\ten{S}^{T}\vek{u})=\div(S^{ij}\vek{g}_j\otimes\vek{g}_i u_k\vek{g}^k)
	=\div(S^{ij}u_i\vek{g}_j)=(S^{ij}u_i)_{,\,j}.
\]
Bralec se lahko sam prepriča, da tudi za kovariantni odvod produkta velja
podobno pravilo, kot ga poznamo za običaji odvod, zato imamo
\[
	(S^{ij}u_i)_{,\,j}=\topbot{S}{ij}{\,,j}u_i+S^{ij}u_{i,\,j}=
	\topbot{S}{ij}{,\,j}\vek{g}_i\cdot\vek{u}+\tr(\ten{S}^{T}\nabla\vek{u})
\]
Dobili smo enakost
\begin{equation} \label{e:divStu}
	\div(\ten{S}^{T}\vek{u})=\topbot{S}{ij}{,\,j}\vek{g}_i\cdot\vek{u}+\tr(\ten{S}^{T}\nabla\vek{u}).
\end{equation}
Če je $\vek{u}$ konstantno vektorsko polje, potem je $\nabla\vek{u}=\ten{0}$ in v
enačbi (\ref{e:divStu}) s pomočjo definicije \ref{def:div} prepoznamo izraz za $\div\ten{S}$, ki je
\[
	\div\ten{S}=\topbot{S}{ij}{,\,j}\vek{g}_i.
\]
Če dobljeni izraz vstavimo nazaj v enačbo (\ref{e:divStu}) in zamenjamo $\ten{S}$ z
$\ten{S}^{T}$, dobimo naslednjo trditev.
\begin{trditev} \label{e:divSu}
	Za poljubno tenzorsko polje $\ten{S}\in C^1(U,\L(\V))$ in poljubno
	vektorsko polje $\vek{u}\in C^1(U,\V)$ velja
	\[ \div(\ten{S}\vek{u})=\vek{u}\cdot\div\ten{S}^{T}+\tr(\ten{S}\nabla\vek{u}). \]
\end{trditev}



\section{Kinematika kontinuuma}


\subsection{Konfiguracije in gibanje telesa}


Naj bosta $\E_R$ in $\E$ Evklidska točkovna prostora, ki imata sicer enake lastnosti,
vendar ju bomo kljub temu razlikovali. Prostor $\E_R$ bo služil določitvi materialnega
telesa: z množico točk izbranega regularnega območja $B\subset\E_R$ je določeno \emph{materialno
telo} ali \emph{kontinuum}. Množici $B$ bomo rekli \emph{referenčna} ali \emph{sklicna konfiguracija}
materialnega telesa. Privzeli bomo, da je $B$ zaprta.
Prostor $\E$ pa bo služil opisu dejanskega položaja materialnega telesa v prostoru.

Točke prostora $\E_R$, njihove koordinate ter njihove krajevne vektorje bomo označevali
z velikimi simboli: $X$, $(X_1,X_2,X_3)$, $\vek{X}$, in jih bomo imenovali
\emph{materialne točke oz.~koordinate oz.~vektorji}.
Točke prostora $\E$, njihove koordinate ter njihove krajevne vektorje bomo,
kot doslej, označevali z malimi simboli: $x$, $(x_1,x_2,x_3)$, $\vek{x}$, in jih bomo imenovali
\emph{prostorske točke oz.~koordinate oz.~vektorji}.

\begin{definicija}
	\emph{Konfiguracija (materialnega) telesa} je zvezna in injektivna preslikava
	\[ \kappa\colon B\to\E,\qquad \kappa\colon X\mapsto x=\kappa(X). \]
	Tudi sliki preslikave $\kappa$, tj.~$\kappa(B)$, rečemo \emph{konfiguracija telesa}.
\end{definicija}
S konfiguracijo torej določamo položaj telesa v prostoru $\E$. Ker je $B$ kompaktna
množica, je tudi inverz konfiguracije zvezna preslikava.

Naj bo $I=[t_1,t_2]\subset\R$ časovni interval, torej je $B\times I\subset\N$.
\begin{definicija}
	\emph{Gibanje (materialnega) telesa} je zvezna preslikava
	\begin{equation}\label{e:chi}
		\chi\colon B\times I\to\E,\qquad \chi\colon(X,t)\mapsto x=\chi(X,t)
	\end{equation}
	z lastnostjo, da je za vsak $t\in I$ preslikava
	\[ \chi_t\colon B\to\E, \qquad \chi_t(X):=\chi(X,t) \]
	konfiguracija. Preslikava $\chi_t$ in množica $ B_t:=\chi_t(B)$ se imenujeta
	\emph{trenutna konfiguracija} telesa \emph{ob času} $t$.
\end{definicija}

Z oznako $\Omega$ bomo, dokler ne bo povedano drugače, označili množico
\[ \Omega=\{(x,t)\;;\ x\in\chi_t(B),\ t\in I\}\subset\N. \]
Gibanje (\ref{e:chi}) ni injektivna preslikava, zato nima inverza. Kljub temu na smiselen
način definiramo \emph{inverzno gibanje}
\[
	\chi^{-1}\colon\Omega\to B,\qquad
	\chi^{-1}(x,t):=\bottop{\chi}{t}{-1}(x),
\]
ki je tudi zvezno. S $\chi^{-1}$ smo torej označili inverz preslikave
$(X,t)\mapsto(x,t)=(\chi(X,t),t)$. Dodatno bomo do nadaljnjega predpostavili, da sta gibanje
in inverzno gibanje razreda $C^2$.

Tenzorsko polje
\[
	\ten{F}\colon B\times I\to\L(\V),\qquad \ten{F}=\Grad\chi(X,t)=\Grad\chi_t(X)
\]
se imenuje \emph{deformacijski gradient} gibanja. Determinanti $J=\det\ten{F}$ 
rečemo \emph{Jacobian}. Ker so konfiguracije $\chi_t$
injektivne oz.~bijektivne na svojo sliko, mora biti $J\neq 0$ povsod na $B\times I$.
Zaradi zveznosti tenzorskega polja $\ten{F}$ mora biti potem skalarno polje $J$,
ki je potem takem tudi zvezno, vseskozi
enakega predznaka. Po dogovoru je $J>0$. Po izreku o inverzni funkciji velja
\[
	\ten{F}^{-1}(x,t)=\grad\chi_t^{-1}(x)=\grad\chi^{-1}(x,t).
\]

\emph{Hitrost} $\vek{v}$ in \emph{pospešek} $\vek{a}$ gibanja $\chi$ sta vektorski polji
\begin{align}
	\vek{v}\colon B\times I\to \V \qquad & \vek{v}(X,t) = \frac{\partial\chi}{\partial t}(X,t), \label{e:v} \\
	\vek{a}\colon B\times I\to \V \qquad & \vek{a}(X,t) = \frac{\partial^2\chi}{\partial t^2}(X,t). \label{e:a}
\end{align}
Alternativna oznaka za $\vek{v}$ je tudi $\dot{\vek{x}}$, za $\vek{a}$ pa $\ddot{\vek{x}}$.


\subsection{Materialni in prostorski opis}


Vsakemu tenzorskemu polju $\vek{f}\colon B\times I\to\W$ pripada glede na gibanje
(\ref{e:chi}) enakovreden predpis
\[
	\bar{\vek{f}}\colon\Omega\to\W,\qquad
	\bar{\vek{f}}(x,t):=\vek{f}(\chi^{-1}(x,t),t)=\vek{f}(X,t)
\]
Tenzorsko polje $\bar{\vek{f}}$ se imenuje \emph{prostorski opis} tenzorskega polja $\vek{f}$.

Na enak način pripada vsakemu tenzorskemu polju $\vek{f}\colon\Omega\to\W$
glede na gibanje (\ref{e:chi}) enakovreden predpis
\begin{equation} \label{e:matopi}
	\hat{\vek{f}}\colon B\times I\to\W,\qquad
	\hat{\vek{f}}(X,t):=\vek{f}(\chi(X,t),t)=\vek{f}(x,t).
\end{equation}
$\hat{\vek{f}}$ imenujemo \emph{materialni opis} tenzorskega polja $\vek{f}$.

Pogosto bomo strešico ali črtico v oznakah za materialni ali prostorski opis opustili,
če to ne bo pustilo dvomov o tem, na kateri domeni je definirano polje.
Pri integraciji bo že iz integracijske domene razvidno, za katero polje gre.
Če poleg polja pišemo še argumente, potem mali $x$ v argumentu nakazuje na prostorski,
veliki $X$ pa na materialni opis.
Pri gradientu in divergenci se dvomom izognemo z uporabo različnih notacij
za ta dva diferencialna operatorja.
V materialnem opisu pišemo oznaki za gradient in divergenco z veliko začetnico
\[ \Grad\vek{f}:=\nabla\hat{\vek{f}},\qquad \Div\vek{f}:=\div\hat{\vek{f}}, \]
v prostorskem opisu pa z malo
\[ \grad\vek{f}:=\nabla\bar{\vek{f}},\qquad \div\vek{f}:=\div\bar{\vek{f}}. \]

Če je $\phi$ skalarno, $\vek{u}$ pa vektorsko polje, potem je zveza med gradientoma
\begin{equation}\label{e:gz}
	\Grad\phi=\ten{F}^{T}\grad\phi,\qquad \Grad\vek{u}=(\grad\vek{u})\ten{F}.
\end{equation}
Res, če je $\vek{w}$ poljubno vektorsko polje, dobimo iz (\ref{e:matopi})
z uporabo verižnega pravila
\begin{gather*}
	\Grad\phi\cdot\vek{w}=\grad\phi\cdot(\Grad\chi)\vek{w}=
	\grad\phi\cdot \ten{F}\vek{w}=\ten{F}^{T}\grad\phi\cdot\vek{w}, \\
	(\Grad\vek{u})\vek{w}=(\grad\vek{u})(\Grad\chi)\vek{w}
	=(\grad\vek{u})\ten{F}\vek{w}.
\end{gather*}

\begin{definicija}
	Časovni odvod tenzorskega polja $\vek{f}\colon B\times I\to\W$ označimo z
	$\dot{\vek{f}}$ ali $d\vek{f}/dt$ in ga imenujemo \emph{materialni odvod};
	\[ \dot{\vek{f}}(X,t)=\frac{d\vek{f}}{dt}(X,t)=\frac{\partial\vek{f}}{\partial t}(X,t). \]
\end{definicija}
Iz (\ref{e:matopi}) dobimo z uporabo verižnega pravila še materialni odvod za prostorski opis:
\[
	\dot{\vek{f}}=\frac{d\vek{f}}{dt}=
	\frac{\partial\vek{f}}{\partial t}+(\grad\vek{f})(\vek{v}),
\]
kjer je $\vek{v}$ hitrost gibanja (\ref{e:v}), $\partial\vek{f}/\partial t$ pa
časovni dvod polja $\vek{f}(x,t)$. Kadar poleg tenzorskega polja ne bomo pisali argumentov,
bo oznaka $\partial\vek{f}/\partial t$ vedno pomenila časovni odvod prostorskega opisa.

\begin{primer} %\label{e:L}
	Hitrost in pospešek sta prvi in drugi materialni odvod gibanja $\vek{x}(X,t)=\chi(X,t)$,
	$\vek{v}=\dot{\vek{x}}$, $\vek{a}=\ddot{\vek{x}}$.
	Pospešek je materialni odvod hitrosti in se v prostorskem opisu izraža kot
	\[ \vek{a}=\dot{\vek{v}}=\frac{\partial\vek{v}}{\partial t}+(\grad\vek{v})\vek{v}. \]
	Tenzorsko polje $\ten{L}=\grad\vek{v}$ se imenuje \emph{hitrostni gradient}.
	Če je gibanje $\chi$ razreda $C^1$, potem je
	\[
		\frac{d}{dt}\Grad\chi=\Grad\frac{d\chi}{dt}\qquad\textrm{oziroma}\qquad
		\dot{\ten{F}}=\Grad\vek{v},
	\]
	iz česar z uporabo zveze (\ref{e:gz}) dobimo $\ten{L}=\dot{\ten{F}}\ten{F}^{-1}$.
\end{primer}


\subsection{Površinski in prostorninski element}


Naj bo $\varepsilon>0$ in $\gamma\colon(-\varepsilon,\varepsilon)\to B$ preslikava
razreda $C^1$. Slika preslikave $\gamma$ je krivulja v referenčni konfiguraciji telesa,
slika preslikave $\chi_t\circ\gamma$ pa je krivulja v trenutni konfiguraciji.
V točki $X=\gamma(0)$ oz.~$x=\chi_t(\gamma(0))$ se tangentni vektor
\[
	d\vek{X}:=\gamma\,'(0)\,d\alpha\qquad\textrm{oz.}\qquad d\vek{x}=(\chi_t\circ\gamma)'(0)\,d\alpha
\]
imenuje \emph{dolžinski element v referenčni} oz.~\emph{trenutni konfiguraciji}. Pri tem
je $d\alpha$ majhno (infinitezimalno) realno število. Po verižnem pravilu je
\[ d\vek{x}=\Grad\chi_t(\gamma(0))(\gamma\,'(0)\,d\alpha)=\ten{F}(X,t)d\vek{X}. \]

Naj bosta $d\vek{X}_1$ in $d\vek{X}_2$ dolžinska elementa v točki $X\in B$ in
naj bosta $d\vek{x}_1=\ten{F}\vek{X}_1$ in $d\vek{x}_2=\ten{F}\vek{X}_2$ pripadajoča
dolžinska elementa v točki $x=\chi_t(X)$. Vektor
\[
	d\vek{S}=d\vek{X}_1\times d\vek{X}_2\qquad\textrm{oz.}
	d\vek{s}=d\vek{x}_1\times d\vek{x}_2
\]
se imenuje \emph{ploščinski element v referenčni} oz.~\emph{trenutni konfiguraciji}.
Velja

Naj bo $\Theta\subset\R^2$ odprta in $r\in C^1(\Theta,B)$ regularna parametrizacija
neke ploskve $S\subset B$, lahko je del robu $\partial B$. Normalni vektor
na ploskev $S$ glede na parametrizacijo $r(\vartheta_1,\vartheta_2)$ je
Vrednost vektorske funkcije
\[
	\vek{S}=\frac{\partial r}{\partial\vartheta_1}\times
	\frac{\partial r}{\partial\vartheta_2}.
\]
pri danem $(\vartheta_1,\vartheta_2)$ je normalni vektor na ploskev $S$
v točki $X=r(\vartheta_1,\vartheta_2)\in S$.
Preslikava $\chi_t\circ r$ je parametrizacija ploskve $s\subset B_t$, pripadajoča
vektorska funkcija normalnih vektorjev je
\[
	\vek{s}=\frac{\partial(\chi_t\circ r)}{\partial\vartheta_1}\times
	\frac{\partial(\chi_t\circ r)}{\partial\vartheta_2}=
	(\Grad\chi)\frac{\partial r}{\partial\vartheta_1}\times
	(\Grad\chi)\frac{\partial r}{\partial\vartheta_2}=
	\ten{F}\frac{\partial r}{\partial\vartheta_1}\times
	\ten{F}\frac{\partial r}{\partial\vartheta_2}.
\]
Če so $\vek{u}_1$, $\vek{u}_2$, $\vek{w}$ poljubni vektorji in $\ten{A}\in\L(V)$, potem je
\begin{align*}
	\vek{w}\cdot\ten{A}\vek{u}_1\times\ten{A}\vek{u}_2&=
	\ten{A}(\ten{A}^{-1}\vek{w})\cdot\ten{A}\vek{u}_1\times\ten{A}\vek{u}_2=
	(\det\ten{A})\ten{A}^{-1}\vek{w}\cdot\vek{u}_1\times\vek{u}_2 \\
	&=\vek{w}\cdot (\det\ten{A})\ten{A}^{-T}(\vek{u}_1\times\vek{u}_2),
\end{align*}
od koder sledi $\ten{A}\vek{u}_1\times\ten{A}\vek{u}_2=
(\det\ten{A})\ten{A}^{-T}(\vek{u}_1\times\vek{u}_2)$, zato je
\begin{equation} \label{e:nsns}
	\vek{s}=J\ten{F}^{-T}\vek{S}\qquad\textrm{oz.}\qquad
	\vek{n}\|\vek{s}\|=J\ten{F}^{-T}\vek{N}\|\vek{S}\|.
\end{equation}
kjer sta $\vek{N}=\|\vek{S}\|^{-1}\vek{S}$ in $\vek{n}=\|\vek{s}\|^{-1}\vek{s}$
%\[ \vek{N}=\frac{\vek{S}}{\|\vek{S}\|}\qquad\textrm{in}\qquad\vek{n}=\frac{\vek{s}}{\|\vek{s}\|} \]
vektorski funkciji enotskih normal.

\begin{izrek}
	Naj bo $\vek{f}\colon S\times I\to\W$ zvezno tenzorsko polje. Potem velja
	\begin{align*}
		\int_s\vek{f}[\vek{n}]\,ds=\int_S\vek{f}[J\ten{F}^{-T}\vek{N}]\,dS ,\\
		\int_S\vek{f}[\vek{N}]\,dS=\int_s\vek{f}[J^{-1}\ten{F}^{T}\vek{n}]\,ds,
	\end{align*}
	kjer izraz $\vek{f}[\,\cdot\,]$ nadomestimo z ustreznim produktom
	polja $\vek{f}$ in vsebine $[\,\cdot\,]$.
\end{izrek}

\proof
	\begin{align*}
		\int_s\vek{f}(x,t)[\vek{n}]\,ds&=
		\int_{\Theta}\vek{f}(\chi_t(r(\vartheta_1,\vartheta_2)),t)
		[\vek{n}]\|\vek{s}\|\,d\vartheta_1d\vartheta_2\\
		&=\int_{\Theta}\vek{f}(r(\vartheta_1,\vartheta_2),t)
		[J\ten{F}^{-T}\vek{N}]\|\vek{S}\|\,d\vartheta_1d\vartheta_2\\
		&=\int_S\vek{f}(X,t)[J\ten{F}^{-T}\vek{N}]\,dS.
	\end{align*}
	Pri tem smo uporabili enakost (\ref{e:nsns}) in dejstvo, da je
	\[ \bar{\vek{f}}(\chi_t(r(\vartheta_1,\vartheta_2)),t)=\hat{\vek{f}}(r(\vartheta_1,\vartheta_2),t). \]
	Drugo enakost dokažemo na podoben način.
\endproof

Ploščinska elementa, ki nastopata v ploskovnih integralih, sta
\[
	dS=\|\vek{S}\|\,d\vartheta_1d\vartheta_2\qquad\textrm{in}\qquad
	ds=\|\vek{s}\|\,d\vartheta_1d\vartheta_2.
\]
Posledica izpeljane enakosti (\ref{e:nsns}) je, da za poljubno zvezno tenzorsko polje
$\vek{f}\colon S\to\W$ velja
\[
	\int_s\vek{f}[\vek{n}]\,ds=\int_S\vek{f}[J\ten{F}^{-T}\vek{N}]\,dS\quad\textrm{in}\quad
	\int_S\vek{f}[\vek{N}]\,dS=\int_s\vek{f}[J^{-1}\ten{F}^{T}\vek{n}]\,ds.
\]

Gladka krivulja znotraj telesa je podana kot slika gladke preslikave
\[\vek{C}\colon \Theta\to\Br, \quad \vek{C}\colon\alpha\mapsto\vek{C}(\alpha),\]
kjer je $\Theta$ odprti interval v $\R$.
V trenutni konfiguraciji telesa se ta krivulja nahaja na lokaciji, ki je določena s sliko pripadajoče preslikave
\[ \vek{c}\colon \Theta\to\B_t,\quad \vek{c}\colon\alpha\mapsto\chi\big(\vek{C}(\alpha),t\big). \]
S posrednim odvajanjem dobimo zvezo
\begin{equation} \label{e:3101}
	\vek{c}'(\alpha) = \big( \Grad\chi\big(\vek{C}(\alpha),t\big)\big)\vek{C}'(\alpha).
\end{equation}
\begin{definicija}
	Naj bosta $\vek{C}$ in $\vek{c}$ definirana kot v prejšnjem odstavku.
	Pri danem $\vek{X}=\vek{C}(\alpha_0)$ in $\vek{x}=\vek{c}(\alpha_0)$, $\alpha_0\in\Theta$, imenujemo
	infinitezimalna tangentna vektorja
	\[ d\vek{X}=\vek{C}'(\alpha_0)d\alpha \quad \mathrm{in} \quad d\vek{x}=\vek{c}'(\alpha_0)d\alpha \]
	\emph{materialni dolžinski element} v referenčni oziroma trenutni konfiguraciji.
\end{definicija}

\begin{definicija}
	Če sta $d\vek{X}_1$, $d\vek{X}_2$ dolžinska elementa v referenčni
	in $d\vek{x}_1$, $d\vek{x}_2$ materialna dolžinska elementa,
	potem infinitezimalna vektorja
	\[ d\vek{S}= d\vek{X}_1\times d\vek{X}_2\quad\mathrm{in}\quad d\vek{s}= d\vek{x}_1\times d\vek{x}_2 \]
	imenujemo \emph{materialni površinski element} v referenčni oz. trenutni konfiguraciji.
	Za materialne dolžinske elemente $d\vek{X}_1$, $d\vek{X}_2$, $d\vek{X}_3$
	in $d\vek{x}_1$, $d\vek{x}_2$, $d\vek{x}_3$ se infinitezimalni števili
	\[ dV= d\vek{X}_1\times d\vek{X}_2\cdot d\vek{X}_3\quad\mathrm{in}\quad dv= d\vek{x}_1\times d\vek{x}_2\cdot d\vek{x}_3 \]
	imenujeta \emph{materialni prostorninski element} v referenčni oziroma trenutni konfiguraciji.
\end{definicija}
\begin{trditev}
	Za pare materialnih dolžinskih, površinskih in volumskih elementov $(d\vek{X},d\vek{x})$, $(d\vek{S},d\vek{s})$ in $(dV,dv)$ velja
	\begin{equation}\label{e:dxdX}
		d\vek{x}=\ten{F}d\vek{X},\quad d\vek{s}=J\ten{F}^{-T}d\vek{S}\quad\mathrm{in}\quad
		dv=JdV,
	\end{equation}
	kjer je $\ten{F}$ deformacijski gradient (\ref{e:F}) z determinanto $J>0$.
\end{trditev}
\proof
	Prva enakost sledi neposredno iz (\ref{e:3101}). Tretja enakost sledi iz lastnosti determinante,
	\begin{align*}
		dv&=d\vek{x}_1\times d\vek{x}_2\cdot d\vek{x}_3=
		\ten{F}d\vek{X}_1\times \ten{F}d\vek{X}_2\cdot \ten{F}d\vek{X}_3=\\
		&=Jd\vek{X}_1\times d\vek{X}_2\cdot d\vek{X}_3 = JdV.
	\end{align*}
	Za poljuben vektor $\vek{u}$ je
	\begin{align*}
		d\vek{s}\cdot\vek{u}&=\ten{F}d\vek{X}_1\times\ten{F}d\vek{X}_2\cdot\vek{u}=
		\ten{F}d\vek{X}_1\times\ten{F}d\vek{X}_2\cdot\ten{F}(\ten{F}^{-1}\vek{u})=\\
		&=J d\vek{X}_1\times d\vek{X}_2 \cdot \ten{F}^{-1}\vek{u} =J d\vek{S}\cdot\ten{F}^{-1}\vek{u}=\\
		&=J\ten{F}^{-T}d\vek{S}\cdot\vek{u},
	\end{align*}
	iz česar sledi druga enakost. 
\endproof

Površinski element se običajno piše kot
\[ d\vek{S}=\vek{N}dS \quad\mathrm{oziroma}\quad d\vek{s}=\vek{n}ds, \]
kjer sta
\[
	\vek{N}=\frac{d\vek{X}_1\times d\vek{X}_2}{\|d\vek{X}_1\times d\vek{X}_2\|} \quad\mathrm{in}\quad
	\vek{n}=\frac{d\vek{x}_1\times d\vek{x}_2}{\|d\vek{x}_1\times d\vek{x}_2\|}
\]
\emph{enotski normali} ter
\[ dS=\|d\vek{X}_1\times d\vek{X}_2\| \quad\mathrm{in}\quad ds=\|d\vek{x}_1\times d\vek{x}_2\| \]
\emph{ploščinska elementa}.


\subsection{Singularna ploskev}


Naj bo $\Theta$ odprta podmnožica v $\R^2$ in $\zeta\colon\Theta\times I\to\E$ preslikava
razreda $C^2$ na $\Theta\times I$, za katero zahtevamo, da je pri vsakem $t\in I$ parcialni
odvod $D_{\theta}\zeta(\theta,t)\colon\R^2\to\V$ linearna preslikava ranga 2
v vsaki točki $\theta=(\theta^1,\theta^2)\in\Theta$, ter da je $\zeta(\cdot,t)\colon\Theta\to\E$
injektivna. Pri danem $t\in I$ je $S(t)=\zeta(\Theta, t)$ ploskev
v $\E$ in $\zeta(\cdot, t)$ je njena regularna parametrizacija. Odslej bomo uporabljali
oznako $S(t)$ za $S(t)\cap B_t$ ter $I_c$ za podinterval v $I$, v katerem je $S(t)\cap B_t$ neprazen.

Dodatno naj velja, da pri vsakem $t\in I_c$ ploskev $S(t)$ razdeljuje materialno telo $B_t$
na disjunktna, med seboj nepovezana dela $B_t^+$ in $B_t^-$. Velja
\[ B_t^+\cup B_t^- = B_t\setminus S(t)\quad\mathrm{in}\quad \overline{B_t^+}\cap \overline{B_t^-}=S(t). \]
Vpeljimo še oznake
\[ \Omega^+=\{(x, t)\: ; \: x\in B_t^+,\ t\in I_c\},\quad\Omega^-=\{(x, t)\: ; \: x\in B_t^-,\ t\in I_c\} \]
\begin{equation*} \textrm{in}\quad\Sigma = \{(x, t)\: ; \: x\in S(t),\ t\in I_c\}. \end{equation*}

\begin{definicija}
	Pravimo, da je gibajoča se ploskev $S(t)$ \emph{singularna} glede na tenzorsko polje
	$f\colon\Omega\to\W$, če je $f$ zvezno na $\Omega^+$ in $\Omega^-$, a ni zvezno na $\Sigma$.
\end{definicija}
\begin{definicija}
	Naj bo $S(t)$ singularna glede na polje $f$. Naj bo $f|_{\Omega^+}$
	skrčitev $f$ na $\Omega^+$ in $f|_{\Omega^-}$ skrčitev $f$ na $\Omega^-$. Označimo z
	$f^+$ zvezno razširitev $f|_{\Omega^+}$ na $\overline{\Omega^+}=\Omega^+\cup\Sigma$ in z
	$f^-$ zvezno razširitev $f|_{\Omega^-}$ na $\overline{\Omega^-}=\Omega^-\cup\Sigma$. \emph{Skok} polja $f$
	je funkcija
	\[ \llbracket f\rrbracket\colon\Sigma\to\W,\qquad \llbracket f\rrbracket=f^+-f^-. \]
\end{definicija}
Kot opombo velja omeniti, da je tako definiran $\llbracket f\rrbracket$ zvezen na $\Sigma$,
če sta normi od $f^+$ in $f^-$ končni na $\Sigma$.
Singularne ploskve predstavljajo širjenje vala skozi prostor v določenem časovnem intervalu.

Označimo parcialne odvode parametrizacije $\zeta$ z
\[
	\zeta_{,\alpha}(\theta^1,\theta^2,t)=\frac{\partial\zeta}{\partial\theta^{\alpha}}(\theta^1,\theta^2,t),\ \alpha=1,2,\qquad
	\zeta_{,t}(\theta^1,\theta^2,t)=\frac{\partial\zeta}{\partial t}(\theta^1,\theta^2,t).
\]
Označimo z $\vek{n}(x,t)$ enotsko normalo na ploskev $S(t)$ v točki $x$, za katero predpostavimo,
da je usmerjena v območje $B_t^+$. Glede na parametrizacijo $\zeta$ je v točki $x=\zeta(\theta^1,\theta^2,t)$
enotska normala
\[
	\vek{n}(x,t)=
	\frac{\zeta_{,1}\times\zeta_{,2}}{\displaystyle\left\| \zeta_{,1}\times\zeta_{,2} \right\|}(\theta^1,\theta^2,t)
\]
in predpostaviti smemo, da je usmerjena v območje $B_t^+$. Če temu ni tako, lahko to
dosežemo z reparametrizacijo. Hitrost točke $(\theta^1,\theta^2)\in\Theta$, ki
ob času $t\in I$ zavzema prostorsko točko $x=\zeta(\theta^1,\theta^2,t)$, je
\[
	\vek{w}(x,t)=\zeta_{,t}(\theta^1,\theta^2,t).
\] 
V splošnem je $\vek{w}(x,t)\neq\vek{v}(x,t)$. Čeprav v obeh primerih $x$ predstavlja isto točko
v prostoru, jo ob času $t$ hkrati zavzemata ploskovna in materialna točka.
$\vek{w}(x,t)$ je hitrost ploskovne točke, $\vek{v}(x,t)$ pa hitrost materialne točke.

Parametrizaciji $\zeta$ pripada parametrizacija $\omega\colon\Theta\times I\to\E_{\r}$, definirana kot
\[ \omega(\theta,t)=\chi^{-1}\big(\zeta(\theta,t),t\big). \]
S $S_{\r}(t)$ označimo sliko $\omega(\Theta, t)$ oz.~kar $\omega(\Theta, t)\cap B$, kar
je ploskev, ki se s časom širi po referenčni konfiguraciji materialnega telesa, in
v časovnem intervalu $I_c$ razdeljuje materialno telo $B$ na disjunktna, med seboj nepovezana
dela. Vpeljimo še oznake
\[ \Omega_{\r}^+=\chi^{-1}(\Omega^+),\quad\Omega_{\r}^-=\chi^{-1}(\Omega^-),\quad\Sigma_{\r}=\chi^{-1}(\Sigma). \]
Za gibanje $\chi$ bomo predpostavili, da je zvezno na $B\times I_c$ ter da je razreda $C^2$ na $\Omega_{\r}^+$
ter $\Omega_{\r}^+$. \textcolor[rgb]{1,0,0}{Zunaj intervala $I_c$ je $\chi$ še vedno razreda $C^2$.}

Če je $f\colon B\to\W$ tenzorsko polje, zvezno na $\Omega_{\r}^+$ in $\Omega_{\r}^-$, ni pa zvezno na
$\Sigma_{\r}$, potem zvezno razširitev skrčitve $f|_{\Omega_{\r}^+}$ na $\overline{\Omega_{\r}^+}=\Omega_{\r}^+\cup\Sigma_{\r}$
označimo z $f^+$. Podobno definiramo še $f^-$. Skok polja $f$ je funkcija
\[ \llbracket f\rrbracket\colon\Sigma_{\r}\to\W,\qquad \llbracket f\rrbracket=f^+-f^-. \]

Hitrost ploskovne točke $\theta\in\Theta$, ki v referenčni konfiguraciji zavzema materialno točko
$X=\omega(\theta,t)$, je
\[
	\vek{w}_{\r}(X,t)=\frac{d}{dt}\omega(\theta, t)=\frac{d}{dt}\chi^{-1}\big(\zeta(\theta,t),t\big).
\]
Če uporabimo verižno pravilo za odvajanje in upoštevamo ? ter ?, dobimo
\begin{align}
	\vek{w}_{\r}&=(\nabla_{x}\chi^{-1})^{\pm}\zeta_{,t}+\left(\frac{\partial}{\partial t}\chi^{-1}\right)^{\pm} \nonumber \\
	&=(\ten{F}^{-1})^+(\vek{w}-\vek{v}^+) \label{e:w} \\
	&=(\ten{F}^{-1})^-(\vek{w}-\vek{v}^-). \nonumber
\end{align}
Čeprav je inverzno gibanje $\chi^{-1}$ zvezno v točkah hiperploskve $\Sigma$, njegov gradient in
parcialni odvod po času nista nujno. Zato smo dobili dve enačbi, iz katerih, če ju med seboj odštejemo, dobimo pogoj
\[ \llbracket \ten{F}^{-1}(\vek{w}-\vek{v}) \rrbracket = \vek{0}. \]


\subsection{Transportni izrek}


V tem razdelku bomo podali transportni izrek, ki je v bistvu posplošitev izreka o odvajanju
integrala s parametrom, ki ga poznamo iz matematične analize. S transportnim izrekom
bomo dobili enačbo za časovni odvod integrala po območju trenutne konfiguracije materialnega telesa.
Še prej pa potrebujemo formulo za materialni odvod determinante deformacijskega gradienta.

Najprej poiščimo odvod za determinanto $\det\colon\L(V)\to\R$, kjer naj bo tokrat $V$
realni $n$-razsežni vektorski prostor. Naj bo $\omega\colon V^n\to\R$ netrivialna
alternirajoča $n$-linearna forma. Spomnimo, za determinanto in sled velja
\begin{align*}
	&\omega(A\vek{u}_1,\dots,A\vek{u}_n)=(\det A)\,\omega(\vek{u}_1,\dots,\vek{u}_n), \\
	&\sum_{j=1}^n\omega(\vek{u}_1,\dots,A\vek{u}_j,\dots,\vek{u}_n)=(\tr A)\,\omega(\vek{u}_1,\dots,\vek{u}_n).
\end{align*}
Po definiciji krepkega odvoda na Banachovih prostorih velja za odvod determinante
\[ (\partial_{A}\det A)[S] = \det(A+S) - \det A + o(S). \]
%kjer je $o(S)$ količina, za katero velja \[ \lim_{\|S\|\to 0}\frac{|o(S)|}{\|S\|}=0. \]
Torej imamo
\begin{multline*}
	(\partial_{A}\det A)[S]\,\omega(\vek{u}_1,\dots,\vek{u}_n)=\\
	\omega\big((A+S)\vek{u}_1,\dots,(A+S)\vek{u}_n\big) -
	\omega(A\vek{u}_1,\dots,A\vek{u}_n) + o(S).
\end{multline*}
Nadalje uporabimo linearnost forme $\omega$, člene, v katerih nastopa $S$ vsaj dvakrat,
pa vključimo v izraz $o(S)$, in tako iz desne strani zadnje enakosti dobimo
\begin{align*}
	&= \sum_{j=1}^n\omega(A\vek{u}_1,\dots,S\vek{u}_j,\dots,A\vek{u}_n) + o(S) = \\
	&= \sum_{j=1}^n\omega(A\vek{u}_1,\dots,AA^{-1}S\vek{u}_j,\dots,A\vek{u}_n) + o(S) = \\
	&= (\det A)\sum_{j=1}^n\omega(\vek{u}_1,\dots,A^{-1}S\vek{u}_j,\dots,\vek{u}_n) + o(S) = \\
	&= (\det A\big)\tr(A^{-1}S)\,\omega(\vek{u}_1,\dots,\vek{u}_n) + o(S),
\end{align*}
iz česar sledi
\[
	(\partial_{A}\det A)[S] = (\det A)\tr(A^{-1}S) = (\det A)A^{-T}\cdot S=(\det A)A^{-T}[S],
\]
Pravilo za odvod determinante je torej
\[ \partial_{A}\det A=(\det A)A^{-T}. \]
Z uporabo verižnega pravila za odvajanje, pravkar izpeljane enakosti za odvod determinante
ter enačbe (\ref{e:L}) dobimo pravilo za materialni odvod determinante deformacijskega gradienta:
\begin{align}
	\dot{J}&=(\det\ten{F})\dot{}=(\partial_{\ten{F}}\det\ten{F})\big[\dot{\ten{F}}\big]=J\ten{F}^{-T}\cdot\dot{\ten{F}}=
	\nonumber \\ &= J\tr(\ten{F}^{-1}\dot{\ten{F}}) = J\tr(\grad\vek{v})=J\div\vek{v}. \label{e:dotJ}
\end{align}

\begin{izrek}[Transportni izrek] \label{i:transport}
	Naj bo $\mathcal{P}\subseteq\B$ del materialnega telesa v referenčni konfiguraciji in naj
	$\mathcal{P}_t=\chi(\mathcal{P},t)\subseteq\B$ označuje njegovo trenutno konfiguracijo ob času $t$.
	Naj bosta $\hat{f}\colon\mathcal{P}\to W$ in $\bar{f}\colon\mathcal{P}_t\to W$
	materialni oziroma prostorski opis neke fizikalne količine, ki sta razreda $C^1$. Potem velja
	\begin{equation*}
		\frac{d}{dt}\int_{\mathcal{P}_t}\bar{f}\,dv =
		\int_{\mathcal{P}_t}(\dot{\bar{f}}+\bar{f}\div\vek{v})\,dv.
	\end{equation*}
	Nadalje, če je območje $\mathcal{P}_t$ omejeno ter je njegov rob sestavljen iz končnega števila
	gladkih ploskev in če sta $\hat{f}$ in $\bar{f}$ skalarno ali pa vektorsko polje
	razreda $C^1$ na zaprtju $\overline{\mathcal{P}}$ oz.~$\overline{\mathcal{P}_t}$, potem velja
	\begin{equation*}
		\frac{d}{dt}\int_{\mathcal{P}_t}\bar{f}\,dv =
		\int_{\mathcal{P}_t}\frac{\partial\bar{f}}{\partial t}\,dv +
		\int_{\partial\mathcal{P}_t}(\vek{v}\cdot\vek{n})\bar{f}\,ds.
	\end{equation*}
\end{izrek}
\proof
	Z uporabo (\ref{e:dxdX})${}_3$ in (\ref{e:dotJ}) pridemo do
	\begin{align*}
		\frac{d}{dt}\int_{\mathcal{P}_t}\bar{f}\,dv &= \frac{d}{dt}\int_{\mathcal{P}}\hat{f}J\,dV =
		\int_{\mathcal{P}}\frac{d}{dt}(\hat{f}J)\,dV = \int_{\mathcal{P}}(\dot{\hat{f}}J+\hat{f}\dot{J})\,dV =\\
		&=\int_{\mathcal{P}}(\dot{\hat{f}}+\hat{f}\div\vek{v})J\,dV = \int_{\mathcal{P}_t}(\dot{\bar{f}}+\bar{f}\div\vek{v})\,dv.
	\end{align*}
	S čimer je dokazan prvi del izreka.
	Pri tem smo na drugem koraku smeli zamenjati vrstni red odvajanja in integriranja, ker se referenčna konfiguracija
	s časom ne spreminja. Formalno to utemeljimo tako, da odvod zapišemo kot limito diferenčnega kvocienta,
	dejstvo, da se v obeh integralih v diferenčnem kvocientu integrira po enakem območju, izkoristimo za uporabo
	linearnosti integrala in tako celoten diferenčni kvocient spravimo pod en sam integral, nazadnje pa še zamenjamo
	limito in integral, kar tudi smemo po Lebesguevem izreku o dominantni konvergenci, saj je $\dot{\hat{f}}$
	po predpostavki zvezen \textcolor[rgb]{1,0,0}{na kompaktnem območju}, torej omejen.
	
	Če sedaj na desni strani prve enačbe iz izreka razpišemo $\dot{\bar{f}}$ v skladu z enačbo (\ref{e:matodv}),
	nato pa upoštevamo tretjo oz.~četrto enačbo iz trditve \ref{t:divprop} ter nato še drugo oz.~tretjo enakost
	iz divergenčnega izreka \ref{i:divtheo}, odvisno glede na to, ali je $\bar{f}$ skalarno ali
	vektorsko polje, dobimo enačbo iz drugega dela izreka.
\endproof

\begin{izrek}[Transportni izrek]
	Naj bo $S(t)$ singularna ploskev glede na polje $f\colon\Omega\to\W$. Velja
	\begin{equation*}
		\frac{d}{dt}\int_B \hat{f}\,dV = \int_B \frac{\partial \hat{f}}{\partial t}\,dV -
		\int_{S_{\r}(t)} \llbracket\hat{f}\rrbracket(\vek{w}_{\r}\cdot\vek{N})\,dS\quad\textrm{in}
	\end{equation*}
	\[
		\frac{d}{dt}\int_{B_t} f\,dV = \int_{B_t} \frac{\partial f}{\partial t}\,dV +
		\int_{\partial B_t} f(\vek{v}\cdot\vek{n})\,dS - \int_{S(t)} \llbracket f\rrbracket(\vek{w}\cdot\vek{n})\,dS.
	\]
\end{izrek}
\proof
	Izrek bomo dokazali za primer, ko je gibanje singularne ploskve $S(t)$ nepovratno, tj.~parametrizacija
	$\omega$ je injektivna preslikava. Izrek velja tudi sicer, le dokaz je bolj zapleten.
	Najprej dokažimo prvo enakost. Opazimo, da velja
	\[
		\frac{d}{dt}\int_B \hat{f}(X,t)\,dV = \frac{d}{dt}\left(
		\int_{B^-(t)} \hat{f}(X,t)\,dV + \int_{B^+(t)} \hat{f}(X,t)\,dV \right).
	\]
	Posvetimo se le odvajanju prvega integrala, odvajanja drugega integrala poteka na enak način.
	Po definiciji odvoda je
	\begin{gather*}
		\frac{d}{dt}\int_{B^-(t)} \hat{f}(X,t)\,dV =\\ =\lim_{h\to 0} \frac{1}{h}\left(
		\int_{B^-(t+h)} \hat{f}(X,t+h)\,dV - \int_{B^-(t)} \hat{f}(X,t)\,dV \right) \\
		=\lim_{h\to 0} \frac{1}{h}\left( \int_{B^-(t+h)} \hat{f}(X,t+h)\,dV - \int_{B^-(t)} \hat{f}(X,t+h)\,dV \right) + \\
		+ \lim_{h\to 0} \frac{1}{h}\left( \int_{B^-(t)} \hat{f}(X,t+h)\,dV - \int_{B^-(t)} \hat{f}(X,t)\,dV \right) \\
		= \lim_{h\to 0} \frac{1}{h}\int_{B^-(t+h)\setminus B^-(t)} \hat{f}(X,t+h)\,dV +
		\int_{B^-(t)} \frac{\partial\hat{f}}{\partial t}(X,t)\,dV.
	\end{gather*}
	Območje $B^-(t)\setminus B^-(t+h)$ je po začetni predpostavki o nepovratnosti singularne ploskve prazno,
	zato smo integral po tem območju izpustili. Limito bomo računali po komponentah. Polje $\hat{f}$
	lahko namreč zapišemo v komponentni obliki glede na neko fiksno bazo $\{b_k\}_{k=1}^m$ prostora $\W$ kot
	\[ \hat{f}(X,t)=\sum_{k=1}^m \hat{f}_k(X,t)b_k. \]
	Zaradi začetne predpostavke lahko
	območje $B^-(t+h)\setminus B^-(t)$ regularno parametriziramo kar s skrčeno bijektivno parametrizacijo
	\[
		\omega(\theta,\tau)\colon\Theta_{\tau}\times [t,t+h]\to B^-(t+h)\setminus B^-(t),\quad
		\Theta_{\tau}=\{\theta\in \Theta\: ;\: \omega(\theta,\tau)\in B\},
	\]
	zato je
	\begin{multline*}
		F_k(t)=\lim_{h\to 0} \frac{1}{h}\int_{B^-(t+h)\setminus B^-(t)} \hat{f}_k(X,t+h)\,dV=\\
		=\lim_{h\to 0} \frac{1}{h}\int_{t}^{t+h}\iint_{\Theta_{\tau}}\hat{f}_k\big(\omega(\theta,\tau),t+h\big)
		\left|\frac{\partial\omega}{\partial\theta^1}\times\frac{\partial\omega}{\partial\theta^2}\cdot\frac{\partial\omega}{\partial\tau}\right|
		(\theta,\tau)\,d\theta^1 d\theta^2 d\tau.
	\end{multline*}
	Po Lagrangevem izreku o srednji vrednosti obstaja $\xi\in [t,t+h]$, da je
	\[
		F_k(t)=\lim_{h\to 0} \frac{1}{h}h\iint_{\Theta_{\xi}}\hat{f}_k\big(\omega(\theta,\xi),t+h\big)
		\left|\frac{\partial\omega}{\partial\theta^1}\times\frac{\partial\omega}{\partial\theta^2}\cdot\frac{\partial\omega}{\partial\tau}\right|
		(\theta,\xi)\,d\theta^1 d\theta^2.
	\]
	Ko gre $h$ proti $0$, gre $\xi$ proti $t$ in
	\[ \hat{f}_k\big(\omega(\theta,\xi),t+h\big)\to\hat{f}^-_k\big(\omega(\theta,t),t\big). \]
	Upoštevamo še
	\[
		\left|\frac{\partial\omega}{\partial\theta^1}\times\frac{\partial\omega}{\partial\theta^2}\cdot
		\frac{\partial\omega}{\partial\tau}\right|\,d\theta^1 d\theta^2=
		\vek{N}\cdot\vek{w}_{\r}
		\left\|\frac{\partial\omega}{\partial\theta^1}\times\frac{\partial\omega}{\partial\theta^2}\right\|\,d\theta^1 d\theta^2=
		\vek{N}\cdot\vek{w}_{\r}\,dS,
	\]
	pri čemer je skalarni produkt $\vek{w}_{\r}\cdot\vek{N}$ pozitiven, zato smo absolutno vrednost izspustili.
	Hitrost $\vek{w}_{\r}$ in normala $\vek{N}$ namreč zaradi začetne predpostavke vedno kažeta v isti polprostor v $\E$,
	določen s tangentno ravnino, katere normala je $\vek{N}$. Tako dobimo
	\[ F_k(t)=\int_{S_{\r}(t)}\hat{f}_k^-(\vek{w}_{\r}\cdot\vek{N})\,dS \]
	in če vsak $F_k(t)$ pomnožimo z $b_k$, nato pa tvorimo vsoto po vseh $k$, dobimo
	\[
		\lim_{h\to 0} \frac{1}{h}\int_{B^-(t+h)\setminus B^-(t)} \hat{f}(X,t+h)\,dV =
		\int_{S_{\r}(t)}\hat{f}^-(\vek{w}_{\r}\cdot\vek{N})\,dS.
	\]
	Dokazali smo
	\begin{equation} \label{e:vmrez}
		\frac{d}{dt}\int_{B^-(t)} \hat{f}\,dV =
		\int_{S_{\r}(t)}\hat{f}^-(\vek{w}_{\r}\cdot\vek{N})\,dS +
		\int_{B^-(t)} \frac{\partial\hat{f}}{\partial t}\,dV.
	\end{equation}
	Na popolnoma enak način dobimo še
	\begin{equation*}
		\frac{d}{dt}\int_{B^+(t)} \hat{f}\,dV =
		-\int_{S_{\r}(t)}\hat{f}^+(\vek{w}_{\r}\cdot\vek{N})\,dS +
		\int_{B^+(t)} \frac{\partial\hat{f}}{\partial t}\,dV.
	\end{equation*}
	Ploskev $S_{\r}(t)$ je pri tem orientirana nasprotno, kot v prejšnjem primeru,
	zato je njena normala $-\vek{N}$ in posledično je ploskovni integral v tem primeru
	nasprotno predznačen. Če obe dobljeni enačbi seštejemo, dobimo prvo enačbo
	iz izreka.
	
	Drugo enakost bomo dokazali za primer, ko je $f$ skalarno ali pa vektorsko polje.
	Tudi tu integral najprej razcepimo na vsoto dveh integralov,
	enega po območju $B_t^-$ in drugega po območju $B_t^+$. Zopet se posvetimo le odvajanju
	prvega integrala, saj pri računanju odvoda drugega integrala uporabimo povsem enake metode.
	Z uporabo (\ref{e:dxdX})${}_3$ in malo prej izpeljane enakosti (\ref{e:vmrez}) dobimo
	\begin{align}
		\frac{d}{dt}\int_{B_t^-}f\,dV&=\frac{d}{dt}\int_{B^-(t)}\hat{f}J\,dV \nonumber \\
		&=\int_{B^-(t)}\frac{\partial(\hat{f}J)}{\partial t}\,dV + \int_{S_{\r}(t)}\hat{f}^-J^-(\vek{w}_{\r}\cdot\vek{N})\,dS. \label{e:vr2}
	\end{align}
	Integrand v prvem integralu lahko s pomočjo enačbe (\ref{e:dotJ}) za odvod determinante $J$ razpišemo kot
	\[ \frac{\partial(\hat{f}J)}{\partial t}=\left(\frac{\partial\hat{f}}{\partial t}+\hat{f}\widehat{\div\vek{v}}\right)J, \]
	integral pa potem s pomočjo (\ref{e:dxdX}) in (\ref{e:matodv}) transformiramo nazaj v
	\[
		\int_{B^-(t)}\frac{\partial(\hat{f}J)}{\partial t}\,dV =
		\int_{B_t^-}\left(\frac{\partial f}{\partial t}+(\nabla_{x}f)[\vek{v}]+f\div\vek{v}\right)\,dV.
	\]
	Sedaj uporabimo tretjo oz.~četrto enačbo iz trditve \ref{t:divprop} ter nato še drugo oz.~tretjo enakost
	iz divergenčnega izreka \ref{i:divtheo}, odvisno od tega, ali je $f$ skalarno ali
	vektorsko polje, ter dobimo
	\begin{equation} \label{e:vr3}
		\int_{B^-(t)}\frac{\partial(\hat{f}J)}{\partial t}\,dV =
		\int_{B_t^-}\frac{\partial f}{\partial t}\,dV + \int_{\partial B_t^-}f^-(\vek{v}^-\cdot\vek{n})\,dS.
	\end{equation}
	Prevedimo še preostali integral:
	\begin{multline} \label{e:vr4}
		\int_{S_{\r}(t)}\hat{f}^-J^-(\vek{w}_{\r}\cdot\vek{N})\,dS =
		\int_{S_{\r}(t)}\hat{f}^-\big((\ten{F}^{-1})^-\ten{F}^-\vek{w}_{\r}\cdot J^-\vek{N}\big)\,dS= \\
		=\int_{S_{\r}(t)}\hat{f}^-\big(\ten{F}^-\vek{w}_{\r}\cdot J^-(\ten{F}^{-T})^-\vek{N}\big)\,dS
		=\int_{S(t)}f^-\big((\vek{w}-\vek{v}^-)\cdot\vek{n}\big)\,dS.
	\end{multline}
	Tu smo na zadnjem koraku uporabili (\ref{e:dxdX})${}_2$ in $(\ref{e:w})$. V enačbi (\ref{e:vr3})
	lahko ploskovni integral po območju $\partial B_t^-$ zapišemo kot vsoto dveh ploskovnih integralov,
	enega po ploskvi $\partial B_t^-\setminus S(t)$ in drugega po ploskvi $S(t)$.
	Iz (\ref{e:vr2}), (\ref{e:vr3}) in (\ref{e:vr4}) sedaj dobimo
	\[
		\frac{d}{dt}\int_{B_t^-}f\,dV=
		\int_{B_t^-}\frac{\partial f}{\partial t}\,dV + \int_{\partial B_t^-\setminus S(t)}f(\vek{v}\cdot\vek{n})\,dS +
		\int_{S(t)}f^-(\vek{w}\cdot\vek{n})\,dS.
	\]
	Na enak način pridemo do
	\[
		\frac{d}{dt}\int_{B_t^+}f\,dV=
		\int_{B_t^+}\frac{\partial f}{\partial t}\,dV + \int_{\partial B_t^+\setminus S(t)}f(\vek{v}\cdot\vek{n})\,dS -
		\int_{S(t)}f^+(\vek{w}\cdot\vek{n})\,dS.
	\]
	Pri tem je ploskev $S(t)$ orientirana nasprotno, kot v prejšnjem primeru, zato pride ploskovni integral po
	ploskvi $S(t)$ negativno predznačen. Če sedaj seštejemo dobljeni enakosti, dobimo drugo enakost iz izreka.
\endproof


\subsection{Zakon o ohranitvi mase}


Če je $\mathcal{P}\subseteq\B$ poljuben del materialnega telesa, ki v trenutni konfiguraciji pri času $t$
zavzema območje $\mathcal{P}_t=\chi(\mathcal{P},t)\subseteq\B_t$, potem je \emph{masa} tega déla telesa
ob času $t$ podana z integralom
\[ m(\mathcal{P},t)=\int_{\mathcal{P}_t}\rho\,dv, \]
kjer je
\[ \rho(\cdot,t)\colon\B_t\to(0,\infty) \]
pozitivno integrabilno skalarno polje, imenovano
\emph{masna gostota trenutne konfiguracije}. V klasični mehaniki se prepostavi naslednji zakon.
\begin{aksiom}[Zakon o ohranitvi mase]
	Masa katerega koli déla telesa $\mathcal{P}\subseteq\B$ se z gibanjem telesa ne spreminja, t.j.
	\[ \frac{d}{dt}m(\mathcal{P},t)=\frac{d}{dt}\int_{\mathcal{P}_t}\rho\,dv=0. \]
\end{aksiom}
V skladu z zakonom o ohranitvi mase pripada materialnemu telesu \emph{masna gostota referenčne konfiguracije}
\[ \rho_{\r}\colon\B\to(0,\infty), \]
tako da velja
\[ m(\mathcal{P},t)\equiv m(\mathcal{P})=\int_{\mathcal{P}}\rho_{\r}\,dV=\int_{\mathcal{P}_t}\rho\,dv. \]
Ker to velja za kateri koli del telesa $\mathcal{P}\subseteq\B$, lahko iz lastnosti (\ref{e:dxdX})${}_3$ in trditve \ref{t:oiz}
za masni gostoti $\rho_{\r}$ in $\rho$, če sta zvezni, dobimo zvezo
\[ \rho_{\r}(\vek{X})=J(\vek{X},t)\hat{\rho}(\vek{X},t). \]

\begin{izrek}[Lokalna oblika zakona o ohranitvi mase]
	\textcolor[rgb]{1,0,0}{Predpostavke o $\rho$.} Potem velja
	\begin{equation} \label{e:lozom} \dot{\bar{\rho}}+\rho\div\vek{v}=0. \end{equation}
\end{izrek}
\proof
	Sledi neposredno iz globalne oblike zakona o ohranitvi mase, če uporabimo transportni izrek \ref{i:transport}
	in nato trditev \ref{t:oiz}.
\endproof

\begin{izrek}
	Če je $f$ fizikalna količina materialnega telesa, podana na masno enoto, in je $\bar{f}\colon\B_t\to W$ njen
	prostorski opis, potem za vsak $\mathcal{P}\subseteq\B$ velja
	\[ \frac{d}{dt}\int_{\chi_t(\mathcal{P})}\bar{f}\rho\,dv=\int_{\chi_t(\mathcal{P})}\dot{\bar{f}}\rho\,dv. \]
\end{izrek}
\proof
	Iz transportnega izreka \ref{i:transport} dobimo
	\begin{align*}
		\frac{d}{dt}\int_{\chi_t(\mathcal{P})}\bar{f}\rho\,dv
		&= \int_{\chi_t(\mathcal{P})}\big((\bar{f}\rho)\dot{}+\bar{f}\rho\div\vek{v}\big)\,dv= \\
		&= \int_{\chi_t(\mathcal{P})}(\dot{\bar{f}}\rho+\bar{f}\dot{\rho}+\bar{f}\rho\div\vek{v})\,dv=\\
		&= \int_{\chi_t(\mathcal{P})}\dot{\bar{f}}\rho\,dv+
		\int_{\chi_t(\mathcal{P})}\bar{f}(\dot{\rho}+\rho\div\vek{v})\,dv,
	\end{align*}
	od koder sledi enakost iz izreka, če upoštevamo (\ref{e:lozom}).
\endproof
