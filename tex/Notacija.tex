\chapter*{Notacija} \thispagestyle{empty}


\begin{center}\begin{tabular}[h]{|c|p{11cm}|}
	\hline oznaka & pomen \\ \hline%\hline
	$A\times B$ & kartezični produkt množic $A$ in $B$ \\ %\hline
	$A^{\mathsf C}$ & komplement množice $A$ \\ %\hline
	$A\setminus B$ & razlika množic $A$ in $B$, $A\setminus B=A\cap B^{\mathsf C}$ \\ %\hline
	$f\circ g$ & kompozitum preslikav $f$ in $g$, $(f\circ g)(a)=f(g(a))$ \\ %\hline
	$f(A)$ & slika množice $A$ s preslikavo $f$, $f(A)=\{f(a)\;;\ a\in A\}$ \\ %\hline
	$f^{-1}$ & inverzna preslikava preslikave $f$ \\ %\hline
	$\L(V,W)$ & prostor vseh omejenih linearnih preslikav iz Banachovega \\ &
	prostora $V$ v Banachov prostor $W$ \\ %\hline
	$\L(V)$ & krajša oznaka za $\L(V,V)$ \\ %\hline
	$\langle u,v\rangle$ & skalarni produkt elementov $u$ in $v$ nekega vektorskega \\ &
	prostora s skalarnim produktom \\ %\hline
	$\|u\|$ & norma elementa $u$ nekega normiranega vektorskega prostora \\ %\hline
	$\V$ & trirazsežni evklidski vektorski prostor \\ %\hline
	$\vek{u}\times\vek{v}$ & vektorski produkt vektorjev $\vek{u},\vek{v}\in\V$ \\ %\hline
	$\vek{u}\otimes\vek{v}$ & tenzorski produkt vektorjev $\vek{u},\vek{v}\in\V$; \\
	& $\vek{u}\otimes\vek{v}\in\L(\V)$, $(\vek{u}\otimes\vek{v})\vek{w}=
	(\vek{v}\cdot\vek{w})\vek{u}$ \ $\forall\vek{u},\vek{v},\vek{w}\in\V$ \\ %\hline
	$\tr\ten{A}$ & sled linearne preslikave $\ten{A}\in\L(\V)$ \\ %\hline
	$\det\ten{A}$ & determinanta linearne preslikave $\ten{A}\in\L(\V)$ \\ %\hline
	$\ten{A}^{T}$ & transponirana linearna preslikava od $\ten{A}\in\L(\V)$; \\
	& $\langle\ten{A}\vek{u},\vek{v}\rangle=\langle\vek{u},\ten{A}^{T}\vek{v}\rangle$ \ $\forall\vek{u},\vek{v}\in\V$ \\ %\hline
	$\ten{A}^{-T}$ & $(\ten{A}^{-1})^{T}=(\ten{A}^{T})^{-1}$ \\ %\hline
	$\partial B$ & rob množice $B$, ki je podmnožica nekega metričnega prostora \\ %\hline
	$\overline{B}$ & zaprtje množice $B$ \\ %\hline
	$\topbot{\delta}{i}{j}$ & Kroneckerjev delta, $\topbot{\delta}{i}{j}=1$, če je $i=j$, in 0 sicer \\ \hline
\end{tabular}\end{center}

\textbf{Dogovor o seštevanju.} Če se v izrazu pojavi indeks natanko dvakrat, potem
se izraz sešteva po zalogi vrednosti tega indeksa. Privzeta zaloga vrednosti
vseh indeksov bo $\{1,2,3\}$, razen če ne bo navedeno drugače. Primer:
\[
	u^i\vek{e}_i=\sum_{i=1}^3 u^i\vek{e}_i,\qquad
	g_{ij}u^iv^j=\sum_{i=1}^3\sum_{j=1}^3g_{ij}u^iv^j.
\]